\documentclass[main]{subfiles}

\begin{document}

In this section, we adopt the following notation:
\begin{itemize}
\item $G=\GL_2(\mathbb R)^+$, $K=\SO(2)$, $\mathfrak g=\mathfrak{gl}_2(\mathbb R)$, $\{W,R,L,Z\}$ is a basis for $\mathfrak g_{\mathbb C}$
\item $V$ is an irreducible admissible $(\mathfrak g,K)$-module, then recall we have
\[V=\bigoplus_{m\in\mathbb Z}V(m),\quad V(m)=\{v\in V|Wv=mv\}=\{v\in V|kv=\chi_m(k)v,\forall k\in K\}\]
$\Sigma=\{m\in\mathbb Z|V(m)\neq0\}$ is the set of $K$-types
\end{itemize}

Recall
\begin{enumerate}
\item $\Delta=-\frac{1}{4}(W^2+2RL+2LR)=-LR-\frac{W}{2}(1+\frac{W}{2})=-RL+\frac{W}{2}(1-\frac{W}{2})$
\item $RV(m)\subseteq V(m+2)$, $LV(m)\subseteq V(m-2)$
\item $U\mathfrak g_{\mathbb C}=\bigoplus_{i>0}R^iA\oplus\bigoplus_{i\geq0}L^iA$, $A$ is the subalgebra generated by $W$ and $Z_{\mathfrak g}=\mathbb C[Z,\Delta]$
\end{enumerate}
We deduce
\begin{enumerate}
\item $\forall 0\neq x\in V(m)$, $V=\mathbb Cx\oplus\bigoplus_{n>0}\mathbb CR^nx\oplus\bigoplus_{n>0}\mathbb CL^nx$
\item $\dim V(m)\leq1$, $\Sigma$ has same parity
\item Let $\lambda$ be the eigenvalue of $\Delta$ on $V$, then $\forall x\in V(m)$, $LRx=(-\lambda-\frac{m}{2}(1+\frac{m}{2}))x$, $RLx=(-\lambda+\frac{m}{2}(1-\frac{m}{2}))x$. If $x\neq0,Rx=0$, then $\lambda=-\frac{m}{2}(1+\frac{m}{2})$, If $x\neq0,Lx=0$, then $\lambda=\frac{m}{2}(1-\frac{m}{2})$
\item Suppose $\lambda=\frac{n}{2}(1-\frac{n}{2})$, $n\in\mathbb Z$, $0\neq x\in V(m)$. If $Rx=0$, then $\frac{n}{2}(1-\frac{n}{2})=-\frac{m}{2}(1+\frac{m}{2})$ $\Rightarrow$ $m=-n$ or $m=n-2$
If $Lx=0$, then $\frac{n}{2}(1-\frac{n}{2})=\frac{m}{2}(1-\frac{m}{2})$ $\Rightarrow$ $m=n$ or $m=2-n$
\end{enumerate}

Consequence: irreducible admissible $(g,K)$ moduels are uniquely determined by infinitesimal character and $K$-types [Bump, Thm2.5.1,2.5.2]

Classification: Fix eigenvalue $\lambda$ of $\Delta$, $\mu$ of $Z$, parity $\epsilon\in\{0,1\}$
\begin{enumerate}
\item If $\lambda\notin\{\frac{n}{2}(1-\frac{n}{2})|n\equiv\epsilon\mod2\}$, then $\Sigma=\epsilon+2\mathbb Z$
\item If $\lambda=\frac{n}{2}(1-\frac{n}{2})$ for $n\in\mathbb Z_{n\geq1}$, $n\equiv\epsilon\mod2$. Then there are 3 possibilities: $\Sigma^+(n)=n+2\mathbb Z_{\geq0}$, $\Sigma^-(n)=-n-2\mathbb Z_{\geq0}$, $\Sigma^0(n)=\{n-2,\cdots,2-n\}$
\end{enumerate}

Parabolic induction: $B=NA\subseteq G$ standard Borel, $\chi:A\to \mathbb C^\times$ be a quasi-character(a group homomorphism), also viewed as $B\to\mathbb C^\times$
\[\Ind^G_B(\chi)=\{f\in C^\infty(G)|f(bg)=\chi(b)f(g)),\forall b\in B,g\in G\}\]
inner product $\langle f_1,f_2\rangle=\int_Kf_1(k)\overline{f_2(k)}dk$. Let $H_\chi$ be the Hilbert space completion of $\Ind^G_B\chi$, this a representation of $G$ induced by right regular action on $C^\infty(G)$. (Have to check $\langle r_gf,r_gf\rangle\leq c\langle f,f\rangle$, $\forall f\in\Ind^G_B\chi$, so $r_g$ is a bounded operator on $H_\chi$). We have a $G$-equivariant map
\begin{align*}
\Ind^G_B\chi_1\times\Ind^G_B\chi_2&\to\Ind^G_B\chi_1\chi_2 \\
(f_1,f_2)&\mapsto f_1f_2
\end{align*}
Let $\delta:A\to\mathbb C^\times$, $\delta\left(\begin{bmatrix}
a_1&0\\
0&a_2
\end{bmatrix}\right)=\dfrac{a_1}{a_2}$, there exists a nonzero $G$-equivariant bounded linear functional
\[\Ind^G_B\delta\to\mathbb C,f\mapsto\int_Kf(k)dk\]
Inducing $G$-equivariant pairing $\Ind^G_B\chi\delta^{\frac{1}{2}}\times\Ind^G_B\chi^{-1}\delta^{\frac{1}{2}}\to\mathbb C$

\begin{definition}
normalized induction $i^G_B\chi=\Ind^G_B\chi\delta^{\frac{1}{2}}$. So $i^G_B\chi$ is unitary with $G$-invariant product. When $\chi$ is unitary, $\chi^{-1}=\bar\chi$, so $i^G_B\chi$ is unitary with $G$-invariant inner product $\langle f_1,f_2\rangle=\int_Kf_1(k)\bar{f_2(k)}dk$
\end{definition}

If $s_1,s_2\in\mathbb C$, $\epsilon\in\{0,1\}$, and $\chi$ is defined by $\chi\left(\begin{bmatrix}
a_1&0\\
0&a_2
\end{bmatrix}\right)=|a_1|^{s_1}|a_2|^{s_2}\sgn(a_1)^\epsilon$, recall $a_1a_2>0$, deonte $H^\infty(s_1,s_2,\epsilon)=i^G_B\chi$, $H(s_1,s_2,\epsilon)$ its Hilbert space completion. Denote $s=\frac{1}{2}(s_1-s_2+1)$. Note $\forall f\in H^\infty(s_1,s_2,\epsilon)$, $f(-g)=(-1)^\epsilon f(g)$. $\forall m\in\epsilon+2\mathbb Z$, there is a unique $f_m\in H^\infty(s_1,s_2,\epsilon)$ such that
\[f_m\left(u\begin{bmatrix}
1&x\\
0&1
\end{bmatrix}\begin{bmatrix}
y^{\frac{1}{2}}&0\\
0&y^{-\frac{1}{2}}
\end{bmatrix}\begin{bmatrix}
\cos\theta&\sin\theta\\
-\sin\theta&\cos\theta
\end{bmatrix}\right)=u^{s_1+s_2}y^se^{im\theta}\]
$H(s_1,s_2,\epsilon)_{\mathrm{fin}}=\bigoplus_{m+\in\epsilon+2\mathbb Z}\mathbb Cf_m$, $(\mathfrak g,K)$-modules of $H(s_1,s_2,\epsilon)$ \\
Use formulas in HW1, one calculates
\[Wf_m=m\cdot f_m,Rf_m=(s+\frac{m}{2})f_{m+2},Lf_m=(s-\frac{m}{2})f_{m-2}\]
\[\Delta f_m=\lambda f_m,Zf_m=\mu f_m\]
where $\lambda=s(1-s)$, $\mu=s_1+s_2$, $s=\frac{1}{2}(s_1-s_2+1)$ \\
From this we get
\begin{enumerate}
\item If $s\notin\frac{\epsilon}{2}+\mathbb Z$, then $H(s_1,s_2,\epsilon)$ is irreducible. Denote its $(\mathfrak g,K)$-module by $P_\mu(\lambda,\epsilon)$, principal series
\item If $s=\frac{n}{2}$, $n\in\epsilon+2\mathbb Z$, $n\geq1$, then $H(s_1,s_2,\epsilon)$ has two irreducible subrepresentations \[H_+=\widehat{\bigoplus_{m\geq n}}\mathbb Cf_m,H_-=\widehat{\bigoplus_{m\leq-n}}\mathbb Cf_m\] denote these $(\mathfrak g,K)$-modules by $D^+_\mu(n)$, $D^-_\mu(n)$, and the set of $K$-types are $\Sigma^+(n)=n+2\mathbb Z_{\geq0}$, $\Sigma^-(n)=-n-2\mathbb Z_{\geq0}$. The quotient $H/H^+\oplus H^-$ is a finite dimensional $G$ representation (0 if $n=1$) with $K$-types $\Sigma^0(n)=\{m\in n+2\mathbb Z|2-n\leq m\leq n-2\}$
\item If $s=1-\frac{n}{2}$, $n\in\epsilon+2\mathbb Z$, $n>1$, then $H(s_1,s_2,\epsilon)$ has a finite dimensional irreducible subrepresentation $H^0=\bigoplus_{2-n\leq m\leq n-2,m\equiv n\mod 2}\mathbb Cf_m$ with set of $K$-type $\Sigma^0(n)$. $H/H^0=H^+\oplus H^-$ with $(\mathfrak g,K)$-module $D^+_\mu(n)\oplus D^-_\mu(n)$
\end{enumerate}
$D^\pm_\mu(n),n\geq2$ are called \textit{discrete series}. The limit of discrete series $D^\pm_\mu(1),D^\pm_\mu(2)$ are called \textit{Steinberg representations}
\[(\Ind^G_B1)_{\mathrm{fin}}/\mathbb C\cong D^+(2)\oplus D^-(2)\]
where $\mu=0$ \\
Irreducible unitary representations
\begin{enumerate}
\item Unitary principal series: $P_\mu(\lambda,\epsilon)=H(s_1,s_2,\epsilon)$, $s_1,s_2\in i\mathbb R$, $s=\frac{1}{2}(s_1-s_2+1)\in\frac{1}{2}+i\mathbb R$, $\mu=s_1+s_2\in i\mathbb R$, $\lambda=s(1-s)\in[\frac{1}{4},+\infty)$, $\epsilon\in\{0,1\}$. And if $s=\frac{1}{2}$, we require $\epsilon=0$
\item Complementary series: $P_\mu(\lambda,0)$, $\mu\in i\mathbb R$, $0<\lambda<\frac{1}{4}$, $s\in(0,1)$
\item Discrete series: $D^\pm_\mu(n)$, $n\in\mathbb Z_{\geq2}$, $\mu\in i\mathbb R$, $\lambda=\frac{n}{2}(1-\frac{n}{2})$
\item Limit of discrete series: $D^\pm_\mu(1)$, $n\in\mathbb Z_{\geq2}$, $\mu\in i\mathbb R$, $s=\frac{1}{2}$, $\lambda=\frac{1}{4}$, unitary since $D^+_\mu(1)\oplus D^-_\mu(1)\cong H(\frac{\mu}{2},\frac{\mu}{2},1)$
\item One dimensional representations: $g\mapsto (\det g)^\frac{\mu}{2}$, $\mu\in i\mathbb R$
\end{enumerate}
Among these, 1,3,4 are tempered. 3 is square integrable. Recall tempered/square integrable means one (equivalently any) nonzero matrix coefficient belong to $L^{2+\epsilon}(G/Z_G,\omega)$, $\forall \epsilon>0$/$L^2(G/Z_G,\omega)$, $\omega$ is an unitary central character. In particular, discrete series are subrepresentations of right regular $G$ representation on $L^2(G/Z_G,\omega)$ \\
Alternative realization of $H(s_1,s_2,\epsilon)$ (restricted to $\SL_2(\mathbb R)=G_1$). Recall $N\backslash\SL_2(\mathbb R)\cong \mathbb R^2\setminus\{(0,0)\}$, $N\cdot\begin{bmatrix}
a&b\\
c&d
\end{bmatrix}\mapsto(c,d)$
\[H^\infty(s_1,s_2,\epsilon)\cong\{f\in C^\infty(\mathbb R^2-\{(0,0)\})|f(\lambda x_1,\lambda x_2)=\sgn(\lambda)^\epsilon|\lambda|^{-2s}f(x_1,x_2)\}\]
$G_1$ action on RHS: $(g\cdots f)(x_1,x_2)=f((x_1,x_2)g)$, inner product
\[\langle f_1,f_2\rangle=\frac{1}{2\pi}\int_1^{2\pi}f_1\overline{f_2}(-\sin\theta,\cos\theta)d\theta\]
Suppose $-2s=n\in\mathbb Z_{\geq0}$, $\epsilon\equiv n\mod 2$, then get
\[H^\infty_n=\{\{f\in C^\infty(\mathbb R^2-\{(0,0)\})|f(\lambda x_1,\lambda x_2)=\lambda^nf(x_1,x_2)\}\]
With completion $H_n$, a representation of $G_1=\SL_2(\mathbb R)$. Let
\[f_{m,n}(x_1,x_2)=(x_1+ix_2)^{\frac{m+n}{2}}(x_1-ix_2)^{\frac{n-m}{2}},\forall m\in n+2\mathbb Z\]
$H^{\mathrm{fin}}_n=\bigoplus_{m\in n+2\mathbb Z}\mathbb Cf_{m,n}$ space of $K$-finite vectors in $H_n$ \\
Let $U_+=\mathbb{CP}^1\setminus\{-i\}\subseteq\mathcal H$, $U_-=\mathbb{CP}^1\setminus\{i\}\subseteq\mathcal H^-$, $U=U_+\cap U_-\supseteq\mathbb{RP}^1$, then

\[0\to\Gamma(\mathbb{CP}^1,\mathcal O(n))\to\Gamma(U,\mathcal O(n))\to H^1_{\{\pm i\}}(\mathbb{CP}^1,\mathcal O(n))\to H^1(\mathbb{CP}^1,\mathcal O(n))=0\]
$\Gamma(\mathbb{CP}^1,O(n))=\bigoplus_{-n\leq m\leq n,m\equiv n\mod2}\mathbb Cf_{m,n}$, finite dimensional representations of $G_1$, $\Gamma(U,O(n))=H^{\mathrm{fin}}_n$, $H^1_{\{\pm i\}}(\mathbb{CP}^1,\mathcal O(n))=D^+(n+2)\oplus D^-(n+2)$, $D^+(n+2)=H^1_{\{i\}}(\mathbb{CP}^1,\mathcal O(n))=\bigoplus_{m\geq n+2,m\equiv n\mod2}\mathbb Cf_{m,n}$, $D^-(n+2)=H^1_{\{-i\}}(\mathbb{CP}^1,\mathcal O(n))=\bigoplus_{m\leq -n-2,m\equiv n\mod2}\mathbb Cf_{m,n}$. All these are $(\mathfrak{sl}_2,K)$-modules \\
Now consider the case of $H_{-n}$, $n\in\mathbb Z_{>0}$, $\epsilon\equiv n\mod2$, $H^{\mathrm{fin}}_{-n}$ has basis $f_{m,-n}$. Cech sequence for $O(-n)$:
\[0=\Gamma(\mathbb{CP}^1,O(-n))\to\Gamma(U_+,O(-n))\oplus\Gamma(U_-,O(-n))\to\Gamma(U,O(-n))\to H^1(\mathbb{CP}^1,O(-n))\to0\]
\[\Gamma(U_+,O(-n))=\bigoplus_{m\leq-n,m\equiv n\mod2}\mathbb Cf_{m,-n}=D^-(n)\to\Gamma(U,O(-n))=H^{\mathrm{fin}}_n\]
\[\Gamma(U_-,O(-n))=\bigoplus_{m\geq n,m\equiv n\mod2}\mathbb Cf_{m,-n}=D^+(n)\to\Gamma(U,O(-n))=H^{\mathrm{fin}}_n\]
By Serre duality
\[H^1(\mathbb{CP}^1,O(-n))\cong H^0(\mathbb{CP}^1,O(n-2))^\vee=\bigoplus_{2-n\leq m\leq n-2,m\equiv n\mod2}\mathbb Cf_{m,-n}\]
Which is $0$ if $n=1$ \\
Using this description, get embedding
\[D^-(n)\to L^2_{\mathrm{hol}}(\mathcal H,n)=\left\{f \text{ holomorphic on } \mathcal H\middle|\int_{\mathcal H}|f(z)|^2(\Im z)^k\frac{dxdy}{y^2}<\infty\right\}\]
$f_{m,-n}\mapsto(z+i)^{\frac{m-n}{2}}(z-i)^{\frac{-m-n}{2}}$
$L^2_{\mathrm{hol}}(\mathcal H,n)$ is a unitary representation of $G_1=\SL_2(\mathbb R)$, $(g\cdot f)(z)=(f|_ng^t)(z)=j(g^t,z)^nf(g^t,z)$
Isometric embedding $L^2_\mathrm{hol}(\mathcal H,n)\to L^2(G_1)$, $f\mapsto\phi_f$, $\phi_f(g)=f(g^ti)j(g^t,i)^{-n}$


The case of $\tilde G=\GL_2(\mathbb R)$, $\tilde K=O(2)=K\sqcup K\eta$, $\eta=\begin{bmatrix}
-1&0\\
0&1
\end{bmatrix}$, $\ad(\eta)(k)=k^{-1}$, $\forall k\in K$, $\eta^2=1$, $\ad(\eta)(W)=-W$, $\ad(\eta)(R)=L$, $\ad(\eta)(L)=R$. Let $\rho$ be a representation of $G$, define another $G$-representation $\eta(\rho)$ by $\eta(\rho)(g)=\rho(\ad(\eta)g)$, then $\Ind^{\tilde G}_G\rho\cong\rho\oplus\eta(\rho)$

Suppose $\rho$ is irreducible, the following are equivalent
\begin{enumerate}
\item $\rho$ extends to an irreducible representation of $\tilde G$
\item $\Ind^{\tilde G}_G\rho$ is reducible $\tilde G$ representation
\item $\rho\cong\eta(\rho)$ as $G$ representation
\end{enumerate}
e.g. $\Ind^{\tilde G}_G\mathrm{triv}\cong \mathrm{triv}\oplus \sgn$ as $\tilde G$ representations

If these are satisfied, then there are 2 isomorphic classes of $\tilde G$ representation that extends $\rho$: $\tilde\rho$ and $\tilde\rho\otimes\sgn$ \\
If the irreducible representation $\rho$ does not extend to $\tilde G$ representation, then $\Ind^{\tilde G}_G\rho\cong(\Ind^{\tilde G}_G\rho)\otimes\sgn$ is an irreducible representation of $\tilde G$ that restricts to the $G$ representation $\rho\oplus\eta(\rho)$ \\
Note: An explicit $\tilde G$ isomorphism
\begin{align*}
\iota:\Ind^{\tilde G}_G\rho&\xrightarrow{\cong}(\Ind^{\tilde G}_G\rho)\otimes\sgn \\
(f:\tilde G\to V)&\mapsto\iota f(g)=\begin{cases}
f(g)&\text{if }g\in G \\
-f(g)&\text{if }g\in\eta G
\end{cases}
\end{align*}
Conversely, let $\sigma$ be an irreducible representation of $\tilde G$. If $\sigma|_G$ is irreducible, then $\sigma\cong\sigma\otimes\sgn$. If $\sigma|_G$ is reducible, then $\sigma\cong\sigma\otimes\sgn$ and $\sigma|_G\cong\rho\oplus\eta(\rho)$ for some irreducible representation $\rho$ of $G$

\begin{example}
Irreducible representations of $\tilde K$: triv, sgn, $\Ind^{\tilde K}_K\chi_m$, $0\neq m\in\mathbb Z$. For $\tilde G=\GL_2(\mathbb R)$, the principal series $P_\mu(\lambda,\epsilon)$ and finite dimensional representations of $G$ has 2 non-isomorphic extensions to $\tilde G$ representation and $(\mathfrak g,K)$-modules
\end{example}

$D^{\pm}_\mu(n)$, $(n\geq1)$ cannot be extended to $(\mathfrak g,\tilde K)$-module, or $\tilde G$ representation, $D^+_\mu(n)\oplus D^-_\mu(n)$ are irreducible $(\mathfrak g,\tilde K)$-modules, irreducible $\tilde G$ representations. Note that $D^-_\mu(n)=\eta(D^+_\mu(n))$

\begin{remark}
$\eta$ acts on $K$-types by negation, so $K$-type of $(\mathfrak g,\tilde K)$ modules are symmetric under negation
\end{remark}

Duality theorem of G-G-PS: $\Gamma\subseteq G$ is discrete subgroup, $\Vol(\Gamma\backslash\mathcal H)<\infty$, $-i\in\Gamma$, $\chi:\Gamma\to\mathbb C^\times$, $\omega:Z_G\to\mathbb C^\times$ such that $\chi(-1)=\omega(-1)$, $\omega(a)=a^\mu$, $\forall a\in Z_G^+$ (positive numbers), where $\mu\in i \mathbb R$
\begin{enumerate}[leftmargin=*]
\item $\forall n\in\mathbb Z_{\geq1}$ such that $\chi(-1)=(-1)^n$, we have canonical isomorphism \[S_n(\Gamma,\chi)=\Hom_{(\mathfrak g,K)}(D^+_\mu(n),\mathcal A_0(\Gamma\backslash G,\chi,\omega))=\Hom_{G}(D^+_\mu(n),\mathcal L^2_0(\Gamma\backslash G,\chi,\omega))\] Recall
\[S_n(\Gamma,\chi)=\{f:\mathcal H\to\mathbb C\text{ holomorphic, cuspidal }f|_n\gamma=\chi(\gamma)f,\forall \gamma\in\Gamma\}\]
The map sends $f$ to the $(\mathfrak g,K)$ module generated by 
\[\phi_f(g)=f(gi)\det(g)^{\frac{1}{2}}j(g,i)^{-n}\omega(\det(g)^{\frac{1}{2}}),\forall g\in G\]
Conversely, let $\phi\in\mathcal A_0(\Gamma\backslash G,\chi,\omega)$ be the image of a nonzero lowest weight vector in $D^+_\mu(n)$, $\forall z=gi\in\mathcal H,g\in G$, define \[f_\phi(z)=\phi(g)j(g,i)^n\det(g)^{-\frac{1}{2}}\omega(\det(g)^{-\frac{1}{2}})\] $\phi$ has weight $n$, $f_\phi$ is well-defined and $f_\phi|_n\gamma=\chi(\gamma)f_\phi$. $L\phi=0\iff f_\phi$ is holomorphic
\item $\forall n\in\mathbb Z_{\geq1}$ such that $\chi(-1)=(-1)^n$
\[S_n(\Gamma,\chi)^{\mathrm{anti}}=\Hom_{(\mathfrak g,K)}(D^-_\mu(n),\mathcal A_0(\Gamma\backslash G,\chi,\omega))=\Hom_{G}(D^-_\mu(n),\mathcal L^2_0(\Gamma\backslash G,\chi,\omega))\]
Use action of $\eta$ on $L_0^2$ and $\mathcal A_0$ $(\eta\phi)(g)=\phi(\ad(\eta)g)$. Composition
\[D^-_\mu(n)\to L_0^2(\Gamma\backslash G,\chi,\omega)\xrightarrow{\eta}L_0^2(\Gamma\backslash G,\chi,\omega)\]
defines a $G$-equivariant embedding
\[D^+_\mu(n)\cong \eta D^-_\mu(n)\to L_0^2(\Gamma\backslash G,\chi,\omega)\]
$\forall f\in S_k(\Gamma,\chi)$, let $\eta(f)(z)=f(-\bar z)$, then $\eta(f)\in S_k(\Gamma,\chi)^\mathrm{anti}$
\begin{center}
\begin{tikzcd}
{S_k(\Gamma,\chi)} \arrow[d, "\cong", "\eta"'] \arrow[r, hook] & {\mathcal A_0(\Gamma\backslash G,\chi,\omega)} \arrow[d, "\eta", "\cong"'] \\
{S_k(\Gamma,\chi)^\mathrm{anti}} \arrow[r, hook]      & {\mathcal A_0(\Gamma\backslash G,\chi,\omega)}                   \\
f \arrow[r,mapsto]                                           & \phi_f                                                          
\end{tikzcd}
\end{center}
\item Consider $P_\mu(\lambda,\epsilon)$, $\lambda=s(1-s)$ unitary principal series ($s\in\frac{1}{2}+i\mathbb R$) or complementary series ($s\in(0,1),\epsilon=0$)
\[\Hom_{(\mathfrak g,K)}(P_\mu(\lambda,\epsilon),\mathcal A_0(\Gamma\backslash,\chi,\omega))=W_s(\Gamma,\chi),\text{ "Maass wave forms"}\]
\[W_s(\Gamma,\chi)=\{f:\mathcal H\to\mathbb C\text{ real analytic, cuspidal }f(\gamma\cdot z)=\chi(\gamma)f(z),-y^2(\frac{\partial^2}{\partial x^2}+\frac{\partial^2}{\partial y^2})f=s(1-s)f\}\]
By Corollary \ref{A_0 decomposes into direct sum of irreducible admissible (g,K) modules with finite multiplicities}, the spaces $S_n(\Gamma,\chi)$, $W_s(\Gamma,\chi)$ are finite dimensional
\end{enumerate}

\begin{note}
If $s\in \frac{1}{2}+i\mathbb R$, then $\lambda\in[\frac{1}{4},+\infty)$ \\
If $s\in(0,1)$, then $\lambda\in(0,\frac{1}{4})$ ``exceptional eigenvalues''
\end{note}

\begin{conjecture}[Selberg]
If $\Gamma$ is a congruence subgroup of $SL_2(\mathbb Z)$, then $\Delta$ has no exceptional eigenvalues on $L_0^2(\Gamma\backslash \mathcal H,\chi)$, equivalently, all summands of $L_0^2(\Gamma\backslash G,\chi,\omega)$ are tempered (complementary series do not occur). Not true if $\Gamma$ is non-congruence subgroup of $\SL_2(\mathbb Z)$
\end{conjecture}

\end{document}