\documentclass[main]{subfiles}

\begin{document}

$p\nmid m$ imply $p$ unramified in $\mathbb Q(\zeta_m)$, $p\mathcal O_{\mathbb Q(\zeta_m)}=P_1^{e}\cdots P_g^{e}$, $f=[k(P_i):k(p)]$, $efg=[\mathbb Q(\zeta_m)$, $\mathbb Q]=\varphi(m)$. $e=1$, $f$ is the order of $p$ in $(\mathbb Z/m\mathbb Z)^\times$

\begin{lemma}
If $m=p^r$, $\mathbb Q(\zeta_m)/\mathbb Q$ is totally unramified at $p$, i.e. $p\mathcal O_{\mathbb Q(\zeta_m)}=P^{[\mathbb Q(\zeta_m):\mathbb Q]}$
\end{lemma}

\begin{proof}
Modulo $p$, $\Phi_m(X)=\frac{X^{p^r}-1}{X^{p^{r-1}-1}}=\frac{(X-1)^{p^r}}{(X-1)^{p^{r-1}}}=(X-1)^{\varphi(m)}$, thus $p\mathcal O_{\mathbb Q(\zeta_m)}=P^{\varphi(m)}$
\end{proof}

In general, $m=np^r$, $p\nmid n$, $\mathbb Q(\zeta_m)$ is the composite $\mathbb Q(\zeta_n)\mathbb Q(\zeta_{p^r})$, and $\mathbb Q(\zeta_n),\mathbb Q(\zeta_{p^r})$ are linearly disjoint over $\mathbb Q$ since $\varphi(m)=\varphi(n)\varphi(p^r)$

$p\nmid m$, $\mathbb Q(\zeta_m)_P/\mathbb Q_p\cong \mathbb Q_p(\zeta_m)/\mathbb Q_p$, here $\mathbb Q(\zeta_m)_P=\mathbb Q_p(\zeta_m)$ is the composite $\mathbb Q_p\mathbb Q(\zeta_m)$

$D_P(\mathbb Q(\zeta_m)/\mathbb Q)\cong\Gal(\mathbb Q_p(\zeta_m)/\mathbb Q_p)$, thus $\mathbb Q_p(\zeta_m)/\mathbb Q_p$ is unramified, and its degree is the order of $p$ in $\mathbb Z/m\mathbb Z^\times$. Similarly, $m=p^r, \mathbb Q_p(\zeta_m)/\mathbb Q_p$ is totally unramified, and its degree is $e=\varphi(p^r)$

\begin{fact}
$K$ local field, $E/K,F/K$ two finite Galois extensions. If $E/K$ is unramified of degree $f$ and $F/K$ is totally ramified of degree $e$, then $E,F$ are linear disjoint over $K$, $[E:K]=ef$, $e$ is ramification index, $f$ is residue extension degree
\end{fact}

$\mathbb Q_p(\zeta_m)$ is the composite $\mathbb Q_p(\zeta_n)\mathbb Q_p(\zeta_{p^r})$, $e=\varphi(p^r)$, $f$ is the order of $p$ in $\mathbb Z/m\mathbb Z^\times$. $[\mathbb Q_p(\zeta_m):\mathbb Q_p]=ef$. The Galois group is the direct product

\begin{center}
\begin{tikzcd}
\Gal(\mathbb Q_p(\zeta_m)/\mathbb Q_p) \arrow[r] & \Gal(\mathbb Q_p(\zeta_n)/\mathbb Q_p)\times\Gal(\mathbb Q_p(\zeta_{p^r})/\mathbb Q_p) \\
(\mathbb Z/m)^\times \arrow[r] \arrow[u, "\alpha_m"]      & (\mathbb Z/n)^\times\times(\mathbb Z/p^r)^\times \arrow[u]                                          
\end{tikzcd}
\end{center}

$\alpha_n:\Gal(\mathbb Q_p(\zeta_n),\mathbb Q_p)\to(\mathbb Z/n)^\times$ sends Frobenius element to $p$, and the subgroup generated by Frobenius element to the subgroup generated by $p$

$\alpha_{p^r}$ is an isomorphism because $[\mathbb Q_p(\zeta_{p^r}):\mathbb Q_p]=\varphi(p^r)=|(\mathbb Z/p^r)^\times|$

$\mathbb Q_p^\times=p^{\mathbb Z}\times\mathbb Z_p^\times$, $\mathbb Z_p^\times=\displaystyle\varprojlim_n(\mathbb Z/p^n)^\times$, $\mathbb Z_p^\times/(1+p^r\mathbb Z_p)\cong(\mathbb Z/p^r)^\times$

Define a map $j_m:\mathbb Q_p^\times\to (\mathbb Z/m)^\times$
\begin{center}
\begin{tikzcd}
\mathbb Q_p^\times \arrow[d, "j_m"'] & \cong & p^{\mathbb Z} \arrow[d, "p\mapsto p"] & \times & \mathbb Z_p^\times \arrow[d, "x\mapsto x^{-1}\mapsto\text{natural projection}"] \\
(\mathbb Z/m)^\times        & \cong & (\mathbb Z/n)^\times         & \times & (\mathbb Z/p^r)^\times                                 
\end{tikzcd}
\end{center}
\begin{center}
\begin{tikzcd}
\mathbb Q_p^\times \arrow[r] \arrow[rr, "\psi_m"', bend right] & (\mathbb Z/m)^\times & \Gal(\mathbb Q_p(\zeta_m)/\mathbb Q_p) \arrow[l]
\end{tikzcd}
\end{center}
$\psi_m$ is continuous and surjective \\
Suppose $m'\in \mathbb Z_{\geq1}$ divisible by $m$, then $\mathbb Q_p(\zeta_m)\subseteq \mathbb Q_p(\zeta_{m'})$
\begin{center}
\begin{tikzcd}
\mathbb Q_p^\times \arrow[r] \arrow[d,equal] & (\mathbb Z/m)^\times \arrow[d, two heads] & \Gal(\mathbb Q_p(\zeta_m)/\mathbb Q_p) \arrow[l] \arrow[d, two heads] \\
\mathbb Q_p^\times \arrow[r]           & (\mathbb Z/m')^\times                     & \Gal(\mathbb Q_p(\zeta_{m'})/\mathbb Q_p) \arrow[l]       
\end{tikzcd}
\end{center}
Take the inverse limit $\phi:\mathbb Q_p^\times\to\varprojlim_m\Gal(\mathbb Q_p(\zeta_m)/\mathbb Q_p)=\Gal(\cup \mathbb Q_p(\zeta_m)/\mathbb Q_p)=\Gal(\mathbb Q_p^{cyc}/\mathbb Q_p)$. The image is dense

\begin{theorem}[local Kronecker-Weber theorem for $\mathbb Q_p$]
Every finite abelian ext of $\mathbb Q_p$ is contained in some $\mathbb Q_p(\zeta_m)$. $\mathbb Q_p^{cyc}=\mathbb Q_p^{ab}$ is the maximal abelian extension(which is the union of all finite abelian extension) of $\mathbb Q_p$ in $\overline{\mathbb Q_p}$
\end{theorem}

One of the main goals of local CFT is: for every local field $K$, to construct a map $K^\times\to \Gal(K^{ab}/K)$(local Artin map) and study its behaviour

e.g. If $L/K$ is a finite abelian extension, if restrict to $L/K$(finite abelian extension, $\psi_L/K$), $\phi_L/K$ is surj and $\ker\psi_L/K=Im(N_L/K:L^\times\to K^\times)$

Can characterize which subgroups of $K^\times$ are of the form $\ker\psi_L/K$ for some $L$

Local-global relationship: We have local Artin map $\psi_p:\mathbb Q_p^\times\to\Gal(\mathbb Q_p^{ab}/\mathbb Q_p)$ and the global Artin map $\Psi_m:(\mathbb Z/m)^\times\to\Gal(\mathbb Q\zeta_m/\mathbb Q)$
\begin{center}
\begin{tikzcd}
(\mathbb Z/m)^\times \arrow[r, "\Psi_m"]           & \Gal(\mathbb Q(\zeta_m)/\mathbb Q)                                                         \\
\mathbb Q_p^\times \arrow[r, "{\psi_{p,m}}"] \arrow[u, "j_{m,p}"] & \Gal(\mathbb Q_p(\zeta_{m})/\mathbb Q_p)=D_p(\mathbb Q(\zeta_m)/\mathbb Q) \arrow[u, hook]
\end{tikzcd}
\end{center}

\begin{definition}[Chevalley(finite ideles)]
$\mathbb A_f^\times=\{(x_p)\in\prod_p\mathbb Q_p^\times|x_p\in \mathbb Z_p^\times\text{ for almost all }p\}$, ideles is $\mathbb A^\times=\mathbb R^\times\times\mathbb A_f^\times$
\end{definition}

$\mathbb Q^\times$ diagonally sits in $\mathbb A^\times$, i.e. $x\mapsto(x_\infty,x_2,x_3,x_5,\cdots)$. Idelic global Artin map $\Psi_m:\mathbb A_f^\times\to Gal(\mathbb Q(\zeta_m)/\mathbb Q),(x_p)\mapsto \psi_{\infty,p}\prod\psi_{p,m}(x_p)$ which is a finite product since $\mathbb Z_p^\times$ elements go to 0

Here $\psi_{\infty,p}(x)$ is identity if $x$ is positive, and $\sigma_\infty$ which is $-1\in\Gal(\mathbb Q(\zeta_m)/\mathbb Q)=(\mathbb Z/m)^\times$ if $x$ is negative. So
\[(x_\infty,x_f)\mapsto\begin{cases}
\Psi(x_f)& x_\infty>0\\
\sigma_\infty\Psi(x_f)& x_\infty<0
\end{cases}\]

\begin{exercise}
This is actually a map $A^\times/\mathbb Q^\times\to\Gal(\mathbb Q^{cyc}/\mathbb Q)=\Gal(\mathbb Q^{ab}/\mathbb Q)$
\end{exercise}

Global CFT: For any global field $K$, Artin map $\Psi:A_K^\times/K^\times\to\Gal(K^{ab}/K)$

Future plan:

review of profinite groups: allows us to talk about infinite Galois theory

State the main theorems in local CFT after reviewing some basics about local fields

Lubin-Tate theory: analogue of the local cyclotomic extensions, gives an explicit construction of $K^{ab}$ for a local field $K$

Group cohomology

Use Group cohomology to fully prove local CFT

Global CFT

\end{document}