\documentclass[12pt]{article}
\usepackage[left=2cm, right=2cm, top=2cm]{geometry}
\usepackage[utf8]{inputenc}
\usepackage{amsmath}
%\usepackage{bbold}
\usepackage{dsfont}
\usepackage{bbm}
\usepackage{amsfonts}
\usepackage{mathrsfs}
\usepackage{amssymb}
\usepackage{tikz}
\usetikzlibrary{cd}

\title{MATH673 homework3}
\author{Haoran Li}
\date{}

\setlength{\parindent}{0cm}

\begin{document}

\maketitle
\textbf{2.5.24} \par
Since the solution is given by d'Alembert's formula
\[
u(x,t)=\dfrac{1}{2}[g(x+t)-g(x-t)]+\dfrac{1}{2}\int_{x-t}^{x+t}h(y)dy
\]
and $g,h$ are compactly supported, $u(x,t)$ is also compactly supported for any fixed $t$, hence $k(t),p(t)$ make sense \par
\textbf{(a)} \par
\[
\dfrac{d}{dt}(p(t)+k(t))=\int_{-\infty}^{+\infty}u_tu_{tt}+u_xu_{xt}dx=\int_{-\infty}^{+\infty}u_tu_{xx}+u_xu_{xt}dx
=\int_{-\infty}^{+\infty}\dfrac{\partial}{\partial x}\left(u_tu_x\right) dx=u_tu_x\left|_{-\infty}^{+\infty}\right.=0
\]
Thus $k(t)+p(t)$ is a constant in $t$ \par
\textbf{(b)} \par
\[
\begin{aligned}
&u_t(x,t)=\dfrac{1}{2}[g'(x+t)-g'(x-t)]+\dfrac{1}{2}[h(x+t)+h(x-t)] \\
&u_x(x,t)=\dfrac{1}{2}[g'(x+t)+g'(x-t)]+\dfrac{1}{2}[h(x+t)-h(x-t)]
\end{aligned}
\]
Suppose $\mathrm{supp}\,g\cup\mathrm{supp}\,h\subset D(0,T)$, then consider $t>T$, one of $g'(x+t),g'(x-t)$ must be zero and one of $h(x+t),h(x-t)$ must be zero, thus $u_t^2=u_x^2$, $p(t)=k(t)$ \par
\textbf{Additional Problem} \par
\[
\begin{aligned}
|u(x,t)|
&=\left|\dfrac{1}{4\pi t}\int_{\partial B(x,t)}f(y)dS_y\right| \\
&=\left|\dfrac{t}{4\pi}\int_{S^2}f(x+t\omega)dS_\omega\right| \\
&=\left|-\dfrac{t}{4\pi}\int_{S^2}\int_t^{-\infty}\dfrac{d}{d\lambda}f(x+\lambda\omega)d\lambda dS_\omega\right| \\
&=\left|-\dfrac{t}{4\pi}\int_{S^2}\int_t^{-\infty}(\nabla f\cdot\omega)(x+\lambda\omega)d\lambda dS_\omega\right| \\
&\leq\dfrac{t}{4\pi}\int_{S^2}\int_t^{-\infty}\left|\nabla f\right||\omega|(x+\lambda\omega)d\lambda dS_\omega \\
&=\dfrac{t}{4\pi}\int_{S^2}\int_t^{-\infty}\left|\nabla f\right|(x+\lambda\omega)d\lambda dS_\omega \\
&=\dfrac{t}{4\pi}\int_{\partial B(x,\lambda)}\int_t^{-\infty}\left|\nabla f\right|(y)d\lambda dS_y \\
&\leq\dfrac{1}{4\pi t}\int_{\mathbb{R}^3}\left|\nabla f\right|dy \\
\end{aligned}
\]
\textbf{January 2007, Problem 2} \par
\textbf{(a)} \par
Define 
\[
E(t)=\dfrac{1}{2}\int_{\partial\Omega}u^2dS+\dfrac{1}{2}\int_{\Omega}\left\{u_t^2+|\nabla u|^2+V(x)u^2\right\}dx
\]
Then we have
\[
\begin{aligned}
E'(t)
&=\int_{\partial\Omega}uu_tdS+\int_{\Omega}\left\{u_tu_{tt}+\nabla u\cdot\nabla u_t+V(x)uu_t\right\}dx \\
&=\int_{\partial\Omega}uu_tdS+\int_{\Omega}\left\{u_t\left(\Delta u-V(x)u\right)+\nabla u\cdot\nabla u_t+V(x)uu_t\right\}dx \\
&=\int_{\partial\Omega}uu_tdS+\int_{\Omega}\left\{u_t\Delta u+\nabla u\cdot\nabla u_t\right\}dx \\
&=\int_{\partial\Omega}uu_tdS+\int_{\partial\Omega}u_t\dfrac{\partial u}{\partial n}dS \\
&=\int_{\partial\Omega}u_t\left(u+\dfrac{\partial u}{\partial n}\right)dS \\
&=0
\end{aligned}
\]
\textbf{(b)} \par
Uniqueness theorem: Suppose $u,v$ are two solutions to the equation, then $u=v$ \par
Consider $w=u-v$ which satisfies equation $w_{tt}-\Delta w+V(x)w=0,\,x\in\Omega,t>0$, with initial conditions $w(x,0)=0, w_t(x,0)=0$ and boundary conditions $w+\dfrac{\partial w}{\partial n}=0$ \par
Since $\displaystyle E(0)=\dfrac{1}{2}\int_{\partial\Omega}w^2(x,0)dS+\dfrac{1}{2}\int_{\Omega}\left\{w_t^2(x,0)+|\nabla w|^2(x,0)+V(x)w^2(x,0)\right\}dx=0$ and $E'(t)=0$, hence $E(t)=0$, since $V(x)\geq 0$, $u=0$ \par
\textbf{(c)} \par
As for $h\neq 0$, the result is the same as (b) \par
\textbf{August 2003, Problem 4} \par
Suppose there are two solutions $u,v$, define $w=u-v$, which would satisfies equation \par
\[w_{tt}-w_{xx}-w_{yyt}=0,\,(x,y)\in\Omega,\,t>0\]\[w(x,y,0)=0,\,w_{t}(x,y,0)=0\]\[w(0,y,t)=w(1,y,t)=0,\,w_y(x,0,t)=w_y(x,1,t)=0\]
Consider $\displaystyle E(t)=\dfrac{1}{2}\int_{\Omega}(w_t^2+w_x^2)dxdy$, since
\[w(0,y,t)=w(1,y,t)=0 \Rightarrow w_t(0,y,t)=w_t(1,y,t)=0\]\[w_y(x,0,t)=w_y(x,1,t)=0 \Rightarrow w_{yt}(x,0,t)=w_{yt}(x,1,t)=0\]
Thus we have
\[
\begin{aligned}
E'(t)
&=\int_{\Omega}(w_tw_{tt}+w_xw_{xt})dxdy \\
&=\int_{\Omega}(w_tw_{xx}+w_tw_{yyt}+w_xw_{xt})dxdy \\
&=\int_0^1\int_0^1(w_tw_x)_xdxdy+\int_0^1\int_0^1(w_tw_{yt})_ydydx-\int_{\Omega}w_{yt}^2dxdy \\
&\leq \int_0^1[w_tw_x(1,y,t)-w_tw_x(0,y,t)]dy+\int_0^1[w_tw_{yt}(x,1,t)-w_tw_{yt}(x,0,t)]dx \\
&=0
\end{aligned}
\]
Thus $u_x=0$, but $u(0,y,t)=0$, hence $u=0$ \par
\textbf{August 2003, Problem 5} \par
Since
\[
\begin{aligned}
u(0,t)^2
&=\left( \dfrac{1}{4\pi t}\int_{\partial B(0,t)}g(y)dS_y \right)^2 \\
&\leq \dfrac{1}{8\pi^2t^2}\left(\int_{\partial B(0,t)}g(y)^2dS_y\right)\left(\int_{\partial B(0,t)}1^2dS_y\right) \\
&=\dfrac{1}{4\pi}\int_{S^2}g(y)^2dS_y \\
\end{aligned}
\]
Thus
\[
\int_0^\infty u(0,t)^2dt\leq \int_0^\infty\dfrac{1}{4\pi}\int_{\partial B(0,t)}g(y)^2dS_ydt=\dfrac{1}{4\pi}\int_{\mathbb{R}^3}g(x)^2dx
\]
\textbf{August 2006, Problem 3} \par
\textbf{(a)} \par
Direct check to find $u_{tt}-\Delta u=0$ when $r\neq 0$, if $t<1$, then $\psi(t+r)=0$ on a neighborhood of $r=0$, thus $u(x,t)=0$ on this neighborhood, hence $u_{tt}-\Delta u=0$, if $t<1$ \par
\textbf{(b)} \par
The extended formula should be given by $\displaystyle u(x,t)=\dfrac{\psi(t+r)-\psi(t-r)}{r}=\int_{-1}^1\psi'(t+\lambda r)d\lambda $ \par
\textbf{(c)} \par
Since $\psi\in C^k$ by the formula of $\psi$ we showed that $u\in C^{k-1}$ \par
\textbf{(d)} \par
Since $\psi$ is compactly supported, so is $u$ \par
Define Energy to be $\displaystyle E(t)=\dfrac{1}{2}\int_{\mathbb{R}^3}\left(u_t^2+|\nabla u|^2\right)dx$, then 
\[
\begin{aligned}
E'(t)
&=\int_{\mathbb{R}^3}\left(u_tu_{tt}+\nabla u\cdot\nabla u_t\right)dx \\
&=\int_{\mathbb{R}^3}\left(u_t\Delta u+\nabla u\cdot\nabla u_t\right)dx \\
&=\int_{B(0,R)}\left(u_t\Delta u+\nabla u\cdot\nabla u_t\right)dx \\
&=\int_{\partial B(0,R)}u_t\dfrac{\partial u}{\partial n}dS \\
&=0
\end{aligned}
\]
Hence the energy is conserved \par
\textbf{August 2001, Problem 1} \par
\textbf{(a)} \par
$$ u(x,t)=\dfrac{1}{4\pi t}\int_{\partial B(x,t)}g(y)dS_y=\dfrac{t}{4\pi}\int_{S^2}g(x+t\omega)dS_\omega $$
but $|x+t\omega|\geq t|\omega|-|x|=t-|x|\geq a$, $g(x+t\omega)=0$, hence $u(x,t)=0$ \par
\textbf{(b)} \par
First we prove $g=0$, assume $g(x_1)>0$, then $\displaystyle u(x_0,|x_1-x_0|)=\dfrac{|x_1-x_0|}{4\pi}\int_{S^2}g(x_0+|x_1-x_0|\omega)dS_\omega >0$, since $g\in C(\mathbb{R}^3), g\geq 0$, that is a contradiction, thus $g=0$, $u(x,t)=0$ \par

\end{document}