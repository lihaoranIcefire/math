\documentclass{article}

\usepackage[a4paper, total={6in, 9in}]{geometry} %page size and margins
% \usepackage[math]{anttor}
\usepackage{ccfonts}
\usepackage{fontspec}
\usepackage{imakeidx}
\usepackage{hyperref}
\hypersetup{
colorlinks = true, %Allow color links
linkcolor = blue, %Setup the color for the links
citecolor = black, %Choose the color for citation
}
\usepackage{amssymb}
\usepackage{amsmath}
\usepackage{amsfonts}
\usepackage{mathrsfs}
\usepackage{mathtools}
\usepackage{amsthm}
\usepackage{tikz}
\usepackage{pgfplots} % drawing axis, addplots
\pgfplotsset{compat=newest} % ensure position and scaling compatibility
\usetikzlibrary{intersections} % handle intersections 
\usetikzlibrary{calc} % using calculations in tikz
\usepackage{tikz-cd}

%%%%%%%%%%%%%%%%%%%%%%%%%%%%%%%%%%%%%%%%%%%%%%%%%%%%%%%%%%%%%%%%%%%%%
\newtheorem{theorem}{Theorem}[subsection]
\newtheorem{corollary}[theorem]{Corollary}
\newtheorem{proposition}[theorem]{Proposition}
\newenvironment{problem}[2][Problem]{\begin{trivlist}
\item[\hskip \labelsep {\bfseries #1}\hskip \labelsep {\bfseries #2.}]}{\end{trivlist}}
%If you want to title your bold things something different just make another thing exactly like this but replace "problem" with the name of the thing you want, like theorem or lemma or whatever
\newenvironment{exercise}[2][Exercise]{\begin{trivlist}
\item[\hskip \labelsep {\bfseries #1}\hskip \labelsep {\bfseries #2.}]}{\end{trivlist}}

\theoremstyle{definition}
\newtheorem{example}[theorem]{Example}

\theoremstyle{remark}
\newtheorem*{remark}{Remark}

\theoremstyle{definition}
\newtheorem{definition}[theorem]{Definition}
\newtheorem{lemma}[theorem]{Lemma}

\def\CROP{0}

\title{MATH602 - HW7}
\author{Haoran Li}
\date{}

\setlength{\parindent}{0pt}

\makeindex[columns=2, title=Index, intoc] % Create the index

\begin{document}\sloppy % reduce overlong words

\maketitle
\begin{exercise}{\textbf{5.3.2}}
If $n\neq0$, the complex projective $n$-space $\mathbb CP^n$ is a simply connected manifold of dimension $2n$. As such $H_p(\mathbb CP^n)=0$ for $p>2n$. Given that there is a fibration $S^1\to S^{2n+1}\to\mathbb CP^n$, show that for $0\leq p\leq 2n$
\begin{align*}
H_p(\mathbb CP^n)\cong\begin{cases}
\mathbb Z &p\text{ even} \\
0 &p\text{ odd}
\end{cases}
\end{align*}
\end{exercise}

\begin{proof}
The $E^2$ page looks like
\begin{center}
\begin{tikzpicture}
\draw (-1,-1)--(11,-1)--(11,2)--(-1,2)--cycle;
\foreach \x in {0,1,...,5}
{

\node at (2*\x,0)[below] {$H_{\x}(\mathbb CP^n)$};
\node at (2*\x,1)[above] {$H_{\x}(\mathbb CP^n)$};
}
\foreach \x in {2,...,5}
{
\draw[red,->] (2*\x,0)--(2*\x-4,1);
}
\end{tikzpicture}
\end{center}
Thus we have
\[E^\infty_{p1}=E^3_{p1}=\mathrm{coker}(H_{p+2}(\mathbb CP^n)\to H_{p}(\mathbb CP^n))\]
\[E^\infty_{p0}=E^3_{p0}=\begin{cases}
\ker(H_p(\mathbb CP^n)\to H_{p-2}(\mathbb CP^n)) &p\geq2 \\
0 &p=0,1
\end{cases}\]
\[\bigoplus_{p=0}^{k}E^3_{p,k-p}=\bigoplus_{p=0}^{k}E^\infty_{p,k-p}=H_k(S^{2n+1})=\begin{cases}
\mathbb Z &k=0,2n+1 \\
0 &\text{otherwise}
\end{cases}\]
Since $H_0(\mathbb CP^n)=\mathbb Z$ and $H_k(S^{2n+1})=0$ for $k=2,\cdots,2n$, we know $H_1(\mathbb CP^n)=0$ and $H_{k}(\mathbb CP^n)\to H_{k-2}(\mathbb CP^n)$ are isomorphisms for $k=2,\cdots,2n$, hence for $0\leq p\leq2n$
\begin{align*}
H_p(\mathbb CP^n)\cong\begin{cases}
\mathbb Z &p\text{ even} \\
0 &p\text{ odd}
\end{cases}
\end{align*}
\end{proof}

\begin{exercise}{\textbf{5.4.1}}
Recall that the completion $\widehat C$ is a filtered complex. Show that $C/F_{p-k}C$ and $\widehat C/F_{p-k}\widehat C$ are naturally isomorphic
\end{exercise}

\begin{proof}
Fix $p$, we have exact sequence
\[0\to F_pC/F_{p-k}C\to C/F_{p-k}C\to C/F_pC\to 0\]
Since $F_pC/F_{p-k}C$ satisfies Mittag-Leffler condition, take limit we get exact sequence
\[0\to F_p\widehat{C}\to\widehat C\to C/F_pC\to\varprojlim_k\textstyle^1{F_pC/F_{p-k}C}=0\]
Thus $C/F_pC$ is naturally isomorphic to the cokernel $\widehat C/F_p\widehat{C}$
\end{proof}

\begin{exercise}{\textbf{5.4.2}}
Show that the spectral sequences for $C$, $\bigcup F_pC$, and $C/\bigcap F_pC$ are all isomorphic
\end{exercise}

\begin{proof}
The spectral sequence of $C$ and $\bigcup F_pC$ are isomorphic since they define the same $A^r_p=\{c\in F_pC|dc\in F_{p-r}C\}$ thus the same $Z^r_p,B^r_p,E^r_p$ \par
The spectral sequence of $C$ and $C/\bigcap F_pC$ are isomorphic since $\displaystyle\varprojlim_k\dfrac{C/\bigcap F_pC}{F_kC/\bigcap F_pC}\cong\varprojlim_k C/F_kC$, i.e. they have the same completion
\end{proof}

\begin{exercise}{\textbf{5.4.3}}(Shifting or D\'ecalage) Given a filtration $F$ on a chain complex $C$, define two new filtrations $\widetilde F$ and $\mathrm{Dec} F$ on $C$ by $\widetilde F_pC_n=F_{p-n}C_n$ and $(\mathrm{Dec}F)_pC_n=\{x\in F_{p+n}C_n|dx\in F_{p+n-1}C_{n-1}\}$. Show that the spectral sequences for these three filtrations are isomorphic after reindexing: $E^r_{pq}(F)\cong E^{r+1}_{2p+q,-p}(\widetilde F)$ for $r\geq0$, and $E^r_{pq}(F)\cong{} E^{r-1}_{-q,p+2q}(\mathrm{Dec} F)$ for $r\geq2$
\end{exercise}

\begin{proof}
$E^r_{pq}(F)\cong E^{r+1}_{2p+q,-p}(\widetilde F)$ for $r\geq0$ since
\[\widetilde A^{r+1}_{2p+q,-p}=\left\{x\in\widetilde F_{2p+q}C_{p+q}\middle|dx\in\widetilde F_{2p+q-r-1}C_{p+q-1}\right\}=\left\{x\in F_{p}C_{p+q}\middle|dx\in\widetilde F_{p-r}C_{p+q-1}\right\}=A^r_{p,q}\]
\begin{align*}
(\mathrm{Dec}A)^{r-1}_{-q,p+2q}&=\left\{x\in (\mathrm{Dec}F)_{-q}C_{p+q}\middle|dx\in(\mathrm{Dec}F)_{-q-r+1}C_{p+q-1}\right\} \\
&=\left\{x\in F_pC_{p+q}\middle|dx\in F_{p-1}C_{p+q-1},dx\in F_{p-r}C_{p+q-1},0=d^2x\in F_{p-r-1}C_{p+q-2}\right\} \\
&=\left\{x\in F_pC_{p+q}\middle|dx\in F_{p-r}C_{p+q-1}\right\} \\
&= A^r_{p,q}
\end{align*}
\end{proof}

\end{document}