\documentclass[main]{subfiles}

\begin{document}

\begin{customexercise}{A1.1}
Show that $\mathbb Z\subset \mathbb Q$ is epi in \textbf{Rings}. Show that $\mathbb Q\subset \mathbb R$ is epi in the category of Hausdorff topological spaces
\end{customexercise}

\begin{proof}
Suppose we have a commutative diagram of rings
\begin{center}
\begin{tikzcd}
\mathbb Z \arrow[r, "i", hook] & \mathbb Q \arrow[r, shift left, "f"] \arrow[r, shift right, "g"'] & R
\end{tikzcd}
\end{center}
Then $f(n)=g(n), \forall n\in\mathbb Z$, hence
\[f\left(\frac{1}{n}\right)=f\left(\frac{1}{n}\right)g(1)=f\left(\frac{1}{n}\right)g(n)g\left(\frac{1}{n}\right)=f\left(\frac{1}{n}\right)f(n)g\left(\frac{1}{n}\right)=f(1)g\left(\frac{1}{n}\right)=g\left(\frac{1}{n}\right)\]
Thus $f\left(\dfrac{m}{n}\right)=f(m)f\left(\dfrac{1}{n}\right)=g(m)g\left(\dfrac{1}{n}\right)=g\left(\dfrac{m}{n}\right)\Rightarrow f=g$ \par
Therefore, $\mathbb Z\hookrightarrow\mathbb Q$ is an epimorphism \par
Suppose we have a commutative diagram of Hausdorff spaces
\begin{center}
\begin{tikzcd}
\mathbb Q \arrow[r, "i", hook] & \mathbb R \arrow[r, shift left, "f"] \arrow[r, shift right, "g"'] & X
\end{tikzcd}
\end{center}
Suppose $f\neq g$, then there exists $r\in\mathbb R$ such that $x:=f(r)\neq g(r)=:y$. Since $X$ is Hausdorff, there are disjoint open neighborhoods $x\in U$, $y\in V$, then $r\in W:=f^{-1}(U)\cap g^{-1}(V)$ is an open neighborhood of $r$. Since $\mathbb Q\hookrightarrow\mathbb R$ is dense, there exists $q\in\mathbb Q\cap W$, we have $f(q)\in U, g(q)\in V$, $f(q)=g(q)$ which is a contradiction, hence $f=g$. Therefore, $\mathbb Q\hookrightarrow\mathbb R$ is an epimorphism
\end{proof}

\begin{customexercise}{A1.2}
In \textbf{Groups}, show that monics are just injective set maps, and kernels are monics whose image is a normal subgroup
\end{customexercise}

\begin{proof}
Consider a commutative diagram of groups
\begin{equation}\label{eq:1}
\begin{tikzcd}
K \arrow[r, shift left, "g"] \arrow[r, shift right, "h"'] & G \arrow[r, "f"] & H
\end{tikzcd}
\end{equation}
Given $f$ is an injective map, if $fg=fh$, then $f(g(k))=f(h(k))\Rightarrow g(k)=h(k)$, thus $f$ is a monomorphism \par
Conversely, given $f$ is a monomorphism. Suppose $f$ is not an injective map, then $f(g_1)= f(g_2)$ for some $g_1\neq g_2\in G$, if we take $K$ in \eqref{eq:1} to be the infinite cyclic group $\langle x\rangle$ generated by $x$ and $g(x)=g_1$, $h(x)=g_2$, then $g\neq h$ but $fg=fh$ which is a contradiction. Therefore $f$ is an injective map \par
Given $i:K\hookrightarrow G$ is a monomorphism and $N:=i(K)$ is a normal subgroup of $G$. $\pi i=0$ is the zero morphism where $\pi:G\to G/N$ is the quotient homomorphism, suppose $i':K'\to G$ is a homomorphism such that $\pi i'=0$, then $i'(K')\subseteq N=i(K)$, we can define $\phi:K'\to K$, $k'\mapsto i^{-1}i'(k')$, $\phi$ is a homomorphism since
\begin{align*}
i(i^{-1}i'(k_1')i^{-1}i'(k_2'))=i(i^{-1}i'(k_1'))i(i^{-1}i'(k_2'))=i'(k_1')i'(k_2')=i'(k_1'k_2') \\
\Rightarrow\phi(k_1'k_2')=i^{-1}i'(k_1'k_2')=i^{-1}i'(k_1')i^{-1}i'(k_2')=\phi(k_1')\phi(k_2')
\end{align*}
\begin{center}
\begin{tikzcd}
K \arrow[r, "i", hook]                         & G \arrow[r, "\pi", shift left] \arrow[r, "0"', shift right] & G/N \\
K' \arrow[ru, "i'"'] \arrow[u, "\phi", dashed] &                                                             &    
\end{tikzcd}
\end{center}
Conversely, given $i:K\hookrightarrow G$ is a kernel of $G\xrightarrow{\pi}H$. Suppose $f,g:M\to K$ such that $if=ig$, then $\pi if=\pi ig=0$, by universal property, $f=g$
\begin{center}
\begin{tikzcd}
M \arrow[d, "g"'] \arrow[rd, "ig"] &                                                             &   \\
K \arrow[r, "i", hook]             & G \arrow[r, "\pi", shift left] \arrow[r, "0"', shift right] & H \\
M \arrow[ru, "if"'] \arrow[u, "f"] &                                                             &  
\end{tikzcd}
\end{center}
Let $N$ be the kernel of $\pi$, since $\pi i=0$, $i(K)\subseteq N$, on the other hand, by universal property, there is a homomorphism $\phi:N\to K$ such that $i\phi=\iota$, where $\iota:N\hookrightarrow G$ is inclusion, thus $N=\iota(N)=\phi i(K)\subseteq i(K)$. Therefore $i(K)=N$ is a normal subgroup
\begin{center}
\begin{tikzcd}
K \arrow[r, "i", hook]                           & G \arrow[r, "\pi", shift left] \arrow[r, "0"', shift right] & H \\
N \arrow[ru, "\iota"'] \arrow[u, "\phi", dashed] &                                                             &  
\end{tikzcd}
\end{center}
\end{proof}

\begin{customexercise}{A1.3}(Pontrjagin duality)
Show that the category $\mathcal C$ of finite abelian groups is isomorphic to its opposite category $\mathcal C^{op}$, but that this fails for the category $\mathcal T$ of torsion abelian groups. We will see in exercise 6.11.4 that $\mathcal T^{op}$ is the category of profinite abelian groups
\end{customexercise}

\begin{proof}
Make the use of the following facts
\[Hom(\mathbb Z/n\mathbb Z,\mathbb Z/m\mathbb Z)\cong \mathbb Z/(n,m)\mathbb Z\]
And
\[\displaystyle Hom\left(\bigoplus_{j=1}^n\mathbb Z/n_j\mathbb Z,\bigoplus_{i=1}^m\mathbb Z/m_i\mathbb Z\right)\cong\bigoplus_{j=1}^n\bigoplus_{i=1}^m Hom\left(\mathbb Z/n_j\mathbb Z,\mathbb Z/m_i\mathbb Z\right)\]
We could think of elements in $\displaystyle Hom\left(\bigoplus_{j=1}^n\mathbb Z/n_j\mathbb Z,\bigoplus_{i=1}^m\mathbb Z/m_i\mathbb Z\right)$ as $m\times n$ matrices
\[
\begin{pmatrix}
k_{11} &\cdots & k_{1n} \\
\vdots &\ddots & \vdots \\
k_{m1} &\cdots & k_{mn}
\end{pmatrix}
\]
With $k_{ij}\in\mathbb Z/(n_j,m_i)\mathbb Z$, and composition can be thought of as matrix multiplication \par
The opposite category $\mathcal C^{op}$ can be thought of as a category with finite abelian groups as objects and homomorphism direction reversed as morphisms \par
Define $F:\mathcal C\to\mathcal C^{op}$, sending any object to itself and sending morphisms as follows
\[Hom\left(\bigoplus_{j=1}^n\mathbb Z/n_j\mathbb Z,\bigoplus_{i=1}^m\mathbb Z/m_i\mathbb Z\right)\to Hom_{\mathcal C^{op}}\left(\bigoplus_{j=1}^n\mathbb Z/n_j\mathbb Z,\bigoplus_{i=1}^m\mathbb Z/m_i\mathbb Z\right)\]
\[
\begin{pmatrix}
k_{11} &\cdots & k_{1n} \\
\vdots &\ddots & \vdots \\
k_{m1} &\cdots & k_{mn}
\end{pmatrix}
\mapsto
\begin{pmatrix}
k_{11} &\cdots & k_{m1} \\
\vdots &\ddots & \vdots \\
k_{1n} &\cdots & k_{mn}
\end{pmatrix}^{op}
\]
This is a functor since $F(I)=I^{op}$, $F(AB)=\left((AB)^T\right)^{op}=\left(B^TA^T\right)^{op}=\left(A^T\right)^{op}\left(B^T\right)^{op}=F(A)F(B)$ \par
Similarly, define $G:\mathcal C^{op}\to\mathcal C$, $G(X)=X,\forall X\in \mathcal C^{op}$, $G(A^{op})=A^T$, this is also a functor since $G(I^{op})=I$, $G(A^{op}B^{op})=G\left((BA)^{op}\right)=(BA)^T=A^TB^T=G(A)G(B)$ \par
Therefore $GF=1_{\mathcal C}$, $FG=1_{\mathcal C^{op}}$, $\mathcal C$ is isomorphic to $\mathcal C^{op}$ \par
Now let's show $\mathcal T$ is not equivalent to $\mathcal T^{op}$, suppose there are functors $F:\mathcal T\to\mathcal T^{op}$, $G:\mathcal T^{op}\to\mathcal T$ such that $GF\cong1_{\mathcal T}$, $FG\cong1_{\mathcal T^{op}}$ \par
\textbf{Claim I: }If $f:A\to B$ is a monomorphism in $\mathcal T$, then $f$ is an injective map, otherwise $0\neq\ker f\xhookrightarrow{i} A$ is a nonzero subgroup, but then $fi=f0$ which is a contradiction \par
\textbf{Claim II: }If $f:A\to B$ is a surjective map in $\mathcal T$, then $f$ is an epimorphism \par
\textbf{Claim III: }Let $Ab$ be the category of abelian groups, suppose $A_i\in ob\mathcal T$, the colimit of $A_1\to A_2\to\cdots$ in $Ab$ is $\oplus A_i/\sim$ which is also a torsion abelian group, it's clear that $\oplus A_i/\sim$ also satisfies universal property for colimits in $\mathcal T$, thus $\oplus A_i/\sim$ is also the colimit in $\mathcal T$, moreover, if $A_i\neq0$, $A_i\hookrightarrow A_j$ are monomorphisms thus injective, then $\oplus A_i/\sim$ is not zero \par
\textbf{Claim IV: }$0$ is obviously the zero object in $\mathcal T$ thus unique up to isomorphism \par
\textbf{Claim V: }The limit of $\mathbb Z/p^{n+1}\mathbb Z\xrightarrow{\mod p^n}\mathbb Z/p^n\mathbb Z$ in $\mathcal T$ is $0$ since for any torsion abelian group $A$, if diagram
\begin{center}
\begin{tikzcd}
\cdots & \mathbb Z/p^{n}\mathbb Z \arrow[l] & \mathbb Z/p^{n+1}\mathbb Z \arrow[l, "\mod p^n"'] & \cdots \arrow[l] \\
       &                                    & A \arrow[u, "f_{n+1}"'] \arrow[lu, "f_{n}"]        &                 
\end{tikzcd}
\end{center}
Commutes, suppose $0\neq a\in A$ is of order $k$ such that $f_n(a)\neq0$, since $0=f_m(ka)=kf_m(a)$ for any $m$, $k=p^l$ for some $l$, but $f_m(a)\equiv f(a)\mod p^n$ for any $m\geq n$, thus $p^lf_{n+l}(a)=0\Rightarrow f_n(a)=0$ which is a contradiction. Therefore $f_i=0$, giving the unique zero morphism $A\xrightarrow{0}0$ such that the following diagram commutes \par
\begin{center}
\begin{tikzcd}
\cdots & \mathbb Z/p^{n}\mathbb Z \arrow[l] & \mathbb Z/p^{n+1}\mathbb Z \arrow[l, "\mod p^n"']                                           & \cdots \arrow[l] \\
       &                                    & 0 \arrow[u, "0"'] \arrow[lu, "0"]                                                           &                  \\
       &                                    & A \arrow[luu, "f_n", bend left] \arrow[uu, "f_{n+1}"', bend right=49] \arrow[u, "0", dashed] &                 
\end{tikzcd}
\end{center}
\textbf{Claim VI: }$F$ is fully faithful, $F$ maps initial objects in $\mathcal T$ to final objects in $\mathcal T$ and maps final objects in $\mathcal T$ to initial objects in $\mathcal T$, $F(0)=0$, similarly, $G(0)=0$, thus $F(A)\neq0$ for any $A\neq0$, otherwise $A=GF(A)=0$, $F$ maps epimorphisms in $\mathcal T$ to monomorphisms in $\mathcal T$, $F$ maps limits in $\mathcal T$ to colimits in $\mathcal T$ \par
$\mathbb Z/p^{n+1}\mathbb Z\xrightarrow{\mod p^n}\mathbb Z/p^n\mathbb Z$ are surjective, $F$ map these to monomorphisms between nonzero objects with colimit $0$ which contradicts Claim III \par
Therefore, $\mathcal T$ is not equivalent to $\mathcal T^{op}$
\end{proof}

\begin{customexercise}{A1.4}
Show that 
\[Hom_{\mathcal C}\left(A,\prod C_i\right)\cong\prod_{i\in I}Hom_{\mathcal C}\left(A,C_i\right)\]
And that 
\[Hom_{\mathcal C}\left(\coprod C_i,A\right)\cong\prod_{i\in I}Hom_{\mathcal C}\left(C_i,A\right)\]
\end{customexercise}

\begin{proof}
Given $\phi\in Hom_{\mathcal C}\left(A,\prod C_i\right)$, compose with $p_i:\prod C_i\to C_i$, we have $\phi_i:=p_i\phi\in Hom_{\mathcal C}\left(A,C_i\right)$. Conversely, given $\phi_i\in Hom_{\mathcal C}\left(A,C_i\right),\forall i\in I$, by the universal property of product, there is a morphism $\phi\in Hom_{\mathcal C}\left(A,\prod C_i\right)$ such that $\phi_i=p_i\phi$ \par
Given $\phi\in Hom_{\mathcal C}\left(\coprod C_i,A\right)$, compose with $\iota_i:C_i\to\coprod C_i$, we have $\phi_i:=\phi\iota_i\in Hom_{\mathcal C}\left(C_i,A\right)$. Conversely, given $\phi_i\in Hom_{\mathcal C}\left(C_i,A\right),\forall i\in I$, by the universal property of coproduct, there is a morphism $\phi\in Hom_{\mathcal C}\left(\coprod C_i,A\right)$ such that $\phi_i=\phi\iota_i$
\end{proof}

\begin{customexercise}{A4.1}
Let $\mathcal A$ be an \textbf{Ab}-category and $f:B\to C$ a morphism. Show that: \par
1. $f$ is monic $\Leftrightarrow$ for every nonzero $e:A\to B$, $fe\neq0$ \par
2. $f$ is an epi $\Leftrightarrow$ for every nonzero $g:C\to D$, $gf\neq0$
\end{customexercise}

\begin{proof} \hfill
\textbf{1.} Given $f$ is a monomorphism. Suppose there is a nonzero $e:A\to B$, such that $fe=0$, since $f0=0$, $e=0$ which is a contradiction. Therefore, for every nonzero $e:A\to B$, $fe\neq0$. Conversely, given for every nonzero $e:A\to B$, $fe\neq0$. Suppose $e,e':A\to B$ are homomorphisms such that $fe=fe'$, then $f(e-e)=0\Rightarrow e-e'=0\Rightarrow e=e'$. Therefore, $f$ is a monomorphism \par
\textbf{2.} Given $f$ is an epimorphism. Suppose there is a nonzero $g:C\to D$, such that $gf=0$, since $0f=0$, $g=0$ which is a contradiction. Therefore, for every nonzero $g:C\to D$, $gf\neq0$. Conversely, given for every nonzero $g:C\to D$, $gf\neq0$. Suppose $g,g':C\to D$ are homomorphisms such that $gf=g'f$, then $(g-g')f=0\Rightarrow g-g'=0\Rightarrow g=g'$. Therefore, $f$ is an epimorphism
\end{proof}

\end{document}