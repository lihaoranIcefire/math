\documentclass[../main.tex]{subfiles}

\begin{document}

\begin{definition}
$Spin(n)$ is a subgroup of $Cl_n^{even}$ consisting of elements of norm $1$
\end{definition}

\begin{example}
$Spin(2)=\mathbb T$ is the circle group, Since $Cl_3^{even}\cong Cl_2\cong\mathbb H$, $SU(2)\cong Spin(3)$ are the corresponding subgroups
\end{example}

\begin{definition}
$Spin(n)$ naturally acts on $\mathbb R^n$ by $g\cdot v=gvg^{-1}$, then it is easy to see that $Spin(n)\to SO(n)$ is a double covering, if $M$ is a Riemannian manifold, it has a principal bundle \begin{tikzcd}
SO(n) \arrow[r] & P \arrow[d] \\
& M
\end{tikzcd} where $P_m$ are the oriented orthonormal frames on $T_mM$
A spin structure on $M$ is a lifting of $P$ to a principal bundle for $Spin(n)$, note this may or may not exist
\end{definition}

\begin{example}
$\mathbb CP^{2n}$ doesn't have a spin structure
\end{example}

\begin{proposition}
If spin structures exist, the set of such structures is acted on transitively by $H^1(M,\mathbb Z/2\mathbb Z)$
\end{proposition}

\begin{proof}
Consider the long exact sequence \par
\begin{center}
\begin{tikzcd}
H^1(M,\mathbb Z/2\mathbb Z) \arrow[r] & H^1(M,Spin(n))\arrow[r] &H^1(M,SO(n))\arrow[r,"w_2"]&H^2(M,\mathbb Z/2\mathbb Z)
\end{tikzcd}
\end{center}
\end{proof}

\begin{proposition}
If $n$ is even and $M$ has a spin structure, fix it, then since $Spin(n)\subseteq Cl_n^{even}\subseteq Cl_n\subseteq Cl_n\otimes\mathbb C\cong M^{2^{\frac{n}{2}}}(\mathbb C)$, and $Cl_n\otimes\mathbb C$ has a unique irreducible representation of dimension $2^{\frac{n}{2}}$, called the spin representation $\Delta$, so we get a (comple) spin vector bundle $S=P_{Spin(n)}\times_{Spin(n)}\Delta$
\end{proposition}

\begin{definition}
There is an elliptic operator acting on $S$ via $\displaystyle\sum_{j=1}^ne_j\cdot\nabla e_j$, where $e_1,\cdots,e_n$ is a local orthonormal frame, the symbol is $\sum e_j\xi_j$, this is called the Dirac operator $\slashed{D}$
\end{definition}

\begin{theorem}
Assume $M$ is a closed Riemannian spin manifold of dimension $n=2l$, $ind\slashed{D}=\langle \widehat{A}(M),[M]\rangle$, here $\widehat{A}(M)$ is a polynomial in Pontryagin classes defined as follows: suppose $TM\otimes\mathbb C\cong L_1\oplus\cdots\oplus L_l$ where $L_j$ are complex line bundle with Chern class $x_j$, then $\widehat{A}(M)=\displaystyle\prod_{j}\dfrac{x_j}{2\sinh\dfrac{x_j}{2}}$
\end{theorem}

\end{document}