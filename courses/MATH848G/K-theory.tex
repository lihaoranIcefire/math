\documentclass[../main.tex]{subfiles}

\begin{document}

\begin{definition}\label{Grothendieck group}
For commutative monoid $M$: there exists an abelian group $K$ and $i: M\to K$ having the universal property
\begin{tikzcd}
M \arrow[rd, "f"] \arrow[d, "i", hook] &   \\
K \arrow[r, "g", dashed]               & A
\end{tikzcd}
Where $A$ is an Abelian group \par
Concrete construction: consider the set of formal differences $M\times M\cong\{m-n|m,n\in M\}$ with addition $(m_1-n_1)+(m_2-n_2):=[(m_1+m_2)-(n_1+n_2)]$ which is an Abelian group with $0-0$ the identity and $n-m$ the inverse to $m-n$, let $K:=M\times M/\sim$, $m_1-n_1\sim m_2-n_2$ if $m_1+n_2+m=m_2+n_1+m$ for some $m\in M$(adding $m$ to make sure it is an equivalence relation), and $i:M\to K, m\mapsto m-0$ \par
Alternatively, consider the free Abelian group $F(M)$ generated by $M$, let $K:=F(M)/\sim$, $m+'n\sim(m+n)$, or mod the subgroup generated by $m+'n-'(m+n)$, here $+',-'$ are operations in $F(M)$
More generally, for a semigroup $S$, there exists a group $K$ and $i: S\to K$ having the universal property
\begin{tikzcd}
S \arrow[rd, "f"] \arrow[d, "i", hook] &   \\
K \arrow[r, "g", dashed]               & G
\end{tikzcd}
Where $G$ is a group \par
Concrete construction: consider the free group $F(S)$ generated by $S$, let $K:=F(S)/\sim$, $m*'n\sim(m*n)$, or mod the subgroup generated by $m*'n*'(m*n)^{-1}$, here $*', -^{-1}$  are operations in $F(M)$
\end{definition}

\begin{definition}(Alternative definition of Grothendieck group)
Let $R$ be a finite dimensional $k$ algebra(or more generally an Artinian ring), define Grothedieck group $G_0(R)$ to be the set of all finitely generated $R$ modules mod relation, if $0\to A\to B\to C\to0$ is exact, then $[B]=[A]+[C]$, namely, $G_0(R):=\{[M]\}$, where $[M]$ are equivalent classes of finitely generated $R$ modules
\end{definition}

\begin{definition}
Two vector bundles $E\to X$, $F\to X$ are stably isomorphic\index{Stably isomorphic} if $E\oplus\varepsilon^n\approx F\oplus\varepsilon^n$, denoted as $E\approx_s F$, we also denote $E\sim_s F$ if $E\oplus\varepsilon^n\approx F\oplus\varepsilon^m$ for some $n,m$
\end{definition}

\begin{remark}
Here stably isomorphic does not imply isomorphic, for example, $TS^2\approx_s\varepsilon^2$, since $\varepsilon^3\approx T^2\oplus NS^2\approx T^2\oplus\varepsilon^1$ whereas $TS^2$ is not trivial by the hairy ball theorem, and $NS^2\approx\varepsilon^1$ is trivial because it is very easy to find a nonvanishing global section
\end{remark}

\begin{definition}\index{Topologcial $K$-theory}\label{Topologcial $K$-theory}
Let $X$ be a topological space, then the Abelian group $\tilde{K}_{\mathbb F}(X)$ is defined to be the $\sim$-equivalent classes of $\mathbb F$-vector bundles over $X$, since all vector bundles over $X$ with $\oplus$ form a commutative monoid, we can define its Grothendieck group $K(X)$ to be the $K$-theory, as it turns out, $K(X)$ also has a commutative ring structure with $\otimes$, $(E_1-F_1)\otimes (E_2-F_2):=E_1\otimes E_2\oplus F_1\otimes F_2-E_1\otimes F_2\oplus F_1\otimes E_2$ \par
$K_{\mathbb F}(X,Y)$ is defined to be $\left\{E-F\in K_{\mathbb F}(X)\middle|\exists\text{isomorphism }E_Y\to F_Y\right\}$
Note suppose $X$ is compact Hausdorff, we can redefine the equivalence relation for the formal differences to be $E_1-F_1=E_2-F_2$ if $E_1\oplus F_2\approx_s E_2\oplus F_1$, because if $E_1\oplus F_2\oplus E\approx E_2\oplus F_1\oplus E$, for some $E$, then there exists $E'$ such that $E\oplus E'\approx\varepsilon^m$ for some $m$, and any $E-F=(E\oplus F')-\varepsilon^m$ for some $F'$ such that $F\oplus F'\approx\varepsilon^m$, thus we can define a map $K(X)\to\tilde{K}(X), E-\varepsilon^m\mapsto E$ which is obviously surjective, consider $K(X)\to K(*)\cong\mathbb Z$ by restricting a vector bundle to a point, and $K(*)\to K(X)$ by extending a vector space into a trivial vector bundle, we then get an exact sequence which splits \par
\begin{tikzcd}
0 \arrow[r] & K(*) \arrow[r] & K(X) \arrow[r] \arrow[l, bend right] & \tilde{K}(X) \arrow[r] & 0
\end{tikzcd}
Thus $K(X)\cong\tilde{K}(X)\oplus K(*)\cong\tilde{K}(X)\oplus\mathbb Z$ \par
Suppose $X$ is a locally compact Hausdorff space, think of its one point compactification $X^*$ as a pointed space, then $K(X)\cong\ker\left[K(X^*)\to K(*)\right]\cong\ker \tilde{K}(X^*)\cong K(X,*)$ \par
From this we have excision, if $Y\subseteq X$ are both compact Hausdorff, $\tilde{K}(X/Y)\cong\tilde{K}((X-Y)^*)\cong K(X-Y)\cong K(X,Y)$ 
\end{definition}

\begin{remark}
Normally $K(X)$ denote $K_{\mathbb C}(X)$, $KO(X)$ denote $K_{\mathbb R}(X)$
\end{remark}

\begin{theorem}
$K$-theory form an extraordinary cohomology theory
\end{theorem}

\begin{proof}

\end{proof}

\begin{definition}
Let $X$ be a locally compact, Hausdorff space, consider all the complexes
\[
\begin{tikzcd}
{\cdots} \arrow[r, "\alpha_{-2}"] & E_{-1} \arrow[r, "\alpha_{-1}"] & E_{0} \arrow[r, "\alpha_0"] & E_{1} \arrow[r, "\alpha_1"] & E_{2} \arrow[r, "\alpha_2"] & {\cdots}
\end{tikzcd}
\]
with only finitely many $E_j\neq0$, and exact off a compact set, give a semigroup structure by direct sum, then define $K(X):=$ all such complexes$/\sim$, where a complex is equivalent to the identity if it is chain homotopic to an exact sequence off a compact set, $K(X,A)$ means an extra condition that the complex in consideration are exact over $A$
\end{definition}

\begin{proposition}
The above two definitions of $K(X,A)$ coinside
\end{proposition}

\begin{proof}
If $X$ is compact, the every complex of vector bundles over $X$ will be exact off a compact set and homotopic to a complex with zero maps, then define Euler characteristic map\par\[ K(X)\to K(X), \left[0\to E_1\overset{\alpha_1}{\to}\cdots\to E_n\overset{\alpha_n}{\to}0\right]=\left[0\to E_1\overset{0}{\to}\cdots\to E_n\overset{0}{\to}0\right]\mapsto\displaystyle\sum_{k=0}^n(-1)^k[E_i] \]
which would give an isomorphism \par
If $X$ is locally compact and Hausdorff, $K(X)\cong\ker\left[K(X^*)\to K(*)\right]\cong K(X)$
\end{proof}

\begin{theorem}
Let $X$ be a connected compact oriented two dimensional manifold without boundary(oriented closed surface), every complex vector bundle $E$ is a sum of line bundles $[L_1]+\cdots+[L_n]-[L_{n+1}]-\cdots-[L_{n+m}]$, and line bundles are classified by Chern classes in $H^2(X,\mathbb Z)\cong\mathbb Z$, the rank and degree of $E$ are defined as $\mathrm{rank}E=n-m$, $\displaystyle\deg E=\sum_{i\leq n}\deg [L_i]-\sum_{i> n}\deg [L_i]$, Therefore, $K(X)\xrightarrow{(\mathrm{rank},\deg)}\mathbb Z\oplus\mathbb Z$ is an isomorphism
\end{theorem}

\end{document}