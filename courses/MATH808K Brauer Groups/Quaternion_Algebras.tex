\documentclass[a4paper,10pt]{article}
\usepackage{My_math_package}

\begin{document}

First let's assume $\Char k\neq 2$

\begin{definition}
$a,b\in k^\times$, the \textit{generalized quaternion algebra} is denoted as $(a,b)=k[i,j]/(i^2=a,j^2=b,ij=-ji)$, for $q=x+yi+zj+wij$, write $\bar q=x-yi-zj-wij$, and $\|q\|^2=q\bar q=\bar qq=x^2+ay^2+bz^2+abw^2$
\end{definition}

\begin{definition}
We say a $k$-algebra $A$ has \textit{division} if $A$ is a division ring. We say $A$ \textit{split} if $A\cong M_n(k)$ for some $n$. We call $f:A\xrightarrow{\cong}M_n(k)$ a splitting of $A$
\end{definition}

\begin{example}
$(-1,-1)/\mathbb R$ has divsion: $q^{-1}=\frac{\bar q}{\|q\|^2}=\frac{x-yi-zj-wij}{x^2+y^2+z^2+w^2}$ for $q=x+yi+zj+wij$. $(-1,-1)/\mathbb C$ splits with splitting $f:Q\to M_2(\mathbb C)$, $f(i)=\begin{bmatrix}-i&\\&i\end{bmatrix}$, $f(j)=\begin{bmatrix}&-1\\1&\end{bmatrix}$
\end{example}

\begin{proposition}
Suppose $Q=(a,b)/k$, then the following are equivalent
\begin{enumerate}
\item $Q$ split
\item $Q$ doesn't have division
\item There is a nontrivial solution over $k$ to $x^2-ay^2-bz^2+abw^2=0$
\item There is a solution over $k$ to either $a=x^2-by^2$ or $b=z^2-aw^2$
\end{enumerate}
\end{proposition}

\begin{remark}
$(-1,-1)/\bar k$ always split
\end{remark}

\begin{proof}
\begin{itemize}
\item (1)$\Rightarrow$(2): $E_{ij}$ has no inverse
\item (2)$\Rightarrow$(3): Assume not, then $q^{-1}=\frac{\bar q}{\|q\|^2}$ is defined
\item (3)$\Rightarrow$(4): 
\item (4)$\Rightarrow$(1): 
\end{itemize}
\end{proof}

\begin{lemma}
\begin{enumerate}
\item $(a,b)\cong(b,a)$
\item $(a,b)\cong(au^2,b)$ for all $u\in k^\times$
\item $(1,b)\cong M_2(k)$
\end{enumerate}
\end{lemma}

\begin{proof}
\begin{enumerate}
\item Switch $i,j$
\item Consider $i\mapsto ui$
\item $i\mapsto\begin{bmatrix}1&0\\0&-1\end{bmatrix}$, $j\mapsto\begin{bmatrix}0&b\\1&0\end{bmatrix}$
\end{enumerate}
\end{proof}

\begin{proposition}[Classification of finite division $k$-algebra of dimension 1,2,3,4]
\begin{enumerate}
\item $\dim_kA=1$: $A=k$
\item $\dim_kA=2$: $A=k1\oplus ki$, hence $A$ is commutative, therefore $A/k$ is a quadratic field extension
\item $\dim_kA=3$: Consider $Z(A)$, we have the following cases
\begin{enumerate}
\item $\dim_kZ(A)=3$: $A$ is commutative, hence $A/k$ is a cubic field extension
\item $\dim_kZ(A)=2$: we may assume $A=k1\oplus ki\oplus kj$ and $Z(A)=k1\oplus ki$, but then $ij=ji$ and $\dim_kZ(A)=3$
\item $\dim_kZ(A)=1$: we may assume $A=k1\oplus ki\oplus kj$, $Z(A)=k$ and $ij\neq ji$. Let's write $i^2=a+bi+cj$, then $ai+bi^2+cij=ii^2=i^2i=ai+bi^2+cji\Rightarrow c(ij-ji)=0\Rightarrow c=0$, hence $i^2=a+bi$. Then multiply by $i$ as a matrix would look like $\begin{bmatrix}0&a&x\\1&b&y\\0&0&z\end{bmatrix}=:M$ for some $x,y,z\in k$, so $\exists v\neq0$ such that $iv=zv$, but then $(i-z)v=0$ which is impossible
\end{enumerate}
\item $\dim_kA=4$: $A\cong(a,b)$ for some $a,b\in k^\times$
\end{enumerate}
\end{proposition}

\begin{lemma}
Let $D$ be a 4 dimensional, $k$-central division algebra, and assume there exists a $k$-subalgebra $E\subseteq A$ so that $E\cong k(\sqrt{a})$, $a\notin {k^\times}^2$, then $A\cong(a,b)$ for some $b\in k^\times$
\end{lemma}

If $\Char k=2$(note that $-1=1$), things are more complicated

\begin{definition}
$a\in k,b\in k^\times$, define $[a,b)=k[i,j]/(i^2+i=a,j^2=b,ij=ji+j)$, suppose $q=x+yi+zj+wij$, then $\bar q=x+y(1+i)+zj+wij=(x+y)+yi+zj+wij$, since the minimal polynomials of $i,j,ij$ are $x^2+x+a$ , $x^2+b$ and $x^2+ab$ [since $(ij)^2=ijij=i(ij+j)j=i^2j^2+ij^2=(i+a)b+ib=ab$]
\end{definition}

\begin{exercise}
Suppose $\Char k=2$, the following are equivalent
\begin{enumerate}
\item $[a,b)$ split
\item $[a,b)$ doesn't have division
\item $b$ is a norm from $k(\alpha)/k$ where $\alpha$ is a root of $x^2+x+a=0$
\item The conic $ax^2+by^2=z^2+zw$ has a $k$ point
\end{enumerate}
\end{exercise}

\end{document}