\documentclass[a4paper,10pt]{article}
\usepackage{My_math_package}

\begin{document}

$\Br(\bar k)=0$, $\Br(\mathbb F_q)=0$, $\Br(\mathbb R)=\mathbb Z/2\mathbb Z$ is generated by $(-1,-1)$, note that $\Br(\mathbb R)=\Br(\mathbb C/\mathbb R)=\mathbb R^\times/N_{\mathbb C/\mathbb R}(\mathbb C^\times)=\mathbb R^\times/\mathbb R_{>0}$, however $\Br(\mathbb Q)\neq 0$ and $\Br(\mathbb Q_p)\neq 0$

\begin{definition}
$A/k$ is a CSA, pick $E/k$ finite Galois splitting field for $A$ with splitting $f:A\otimes_kE\to M_n(E)$, then the following map
\[
A\cong A\otimes_kk\subseteq A\otimes_kE\xrightarrow[\cong]{f}M_n(E)\xrightarrow[]{\det}E
\]
is called the \textit{reduced norm} $\Nrd$ of $A$
\end{definition}

\begin{remark}
$\Nrd:A\to E$ is independent of the splitting, suppose $f,g:A\otimes_kE\xrightarrow[]{\cong}M_n(E)$ are splittings, then we have the commutative diagram
\begin{center}
\begin{tikzcd}
A \arrow[r, hook] \arrow[d,equal] & A\otimes_kE \arrow[d,equal] \arrow[r, "f", "\cong"'] & M_n(E) \arrow[d, "g\circ f^{-1}"] \arrow[r, "\det"] & E \arrow[d,equal] \\
A \arrow[r, hook]           & A\otimes_kE \arrow[r, "g", "\cong"']           & M_n(E) \arrow[r, "\det"]                            & E          
\end{tikzcd}
\end{center}
$\Nrd$ is independent of the choice of the splitting field $E$, \textcolor{blue}{for some reasoning about composite fields}
\end{remark}

\begin{remark}
$\Nrd:A\to k$ is multiplicative, i.e. $\Nrd(xy)=\Nrd(x)\Nrd(y)$, so $\Nrd:A^\times\to k^\times$ is a group homomorphism
\end{remark}

\begin{proposition}
$A/k$ is a CSA, $\deg A=n$, $N_{A/k}:A\to k$, $a\mapsto\det m_a$ is the \textit{field norm} map, $N_{A/k}=\Nrd^n$
\end{proposition}

\begin{proof}
Note that \textcolor{blue}{$N_{A/k}(a)=N_{A\otimes_kE/E}(a\otimes1)$} and that $\Nrd(a)=\Nrd(a\otimes1)$. So we may assume that $E=k$, if we write elementary matrices $E_{ij}$ as a basis of $M_n(k)$, so a matrix $M\in M_n(k)$ as an linear operator on $M_n(k)$ as the form
\[
A_M=\begin{bmatrix}
M&&\\
&\ddots&\\
&&M\\
\end{bmatrix}
\]
Hence $N_{A/k}(M)=\det(A_M)=(\det M)^n=\Nrd(M)^n$
\end{proof}

\begin{proposition}
$\Nrd$ is a homogeneous polynomial in $n^2$ variables(a basis of $A$ over $k$) of degree $n$ with $k$-coefficients
\end{proposition}

\begin{proof}
$a=\sum_{k=1}^{n^2}\lambda_kv_k$, so $\Nrd(a)=\det(f(a\otimes1))$
\end{proof}

\begin{definition}
$k$ is a quasi algebraically closed (or $C_1$) field if any homogeneous polynomial $f\in k[x_1,\cdots,x_n]$ with $\deg f=d<n$, there is some $a_1,\cdots,a_n$ not all zero with $f(a_1,\cdots,a_n)=0$
\end{definition}

\begin{proposition}
Algebraically closed fields are $C_1$
\end{proposition}

\begin{proof}
Write $f\in k[x_1,\cdots,x_n]$ as $\phi_m x_1^m+\cdots+\phi_0$ and then do induction on the variables
\end{proof}

\begin{lemma}
$L/k$ finite extension, $k$ is $C_1$, then so is $L$
\end{lemma}

\begin{proof}
Suppose $\{v_j\}_{j=1}^m$ is a $k$-basis of $L$, $f\in L[x_1,\cdots,x_n]$ is a degree $d<n$ homogeneous polynomial, write $x_i=\sum_{j=1}^mx_{ij}v_j$,
Then $f(x_1,\cdots,x_n)=g(x_{11},\cdots,x_{nm})$, note that products of $v_j$'s can written as the $k$-linear combinations of $v_j$'s, $g$ can be seen as a $k[x_{11},\cdots,x_{nm}]$-linear combination of $v_j$'s, view this as an linear operator on the $k$ vector space $L$, denote its determinant $h$, which is then a homogeneous polynomial of degree $md$ with coefficients in $k$, since $md<mn$, $h$ has non trivial solution, which says that $f$ also has a non-trivial solution.
\end{proof}

\begin{proposition}
If $k$ is $C_1$, then $\Br(k)=0$
\end{proposition}

\begin{proof}
If $k$ is finite, we already know that $\Br(k)=0$. \textcolor{blue}{We can actually show that finite fields are $C_1$} \\
Assume $k$ is infinite, $A/k$ is a central division algebra over $k$ of degree $n\geq1$, then by Theorem~\ref{} we know that $\Nrd:A\to k$ is a homogeneous polynomial of degree $n$ in $n^2$ variables with $k$-coefficients. If $n>1$, then $\Nrd(x)=0$ for some $0\neq x\in A$ which is impossible, so $n=1$, and $\Br(k)=0$
\end{proof}

\begin{theorem}[Tsen's theorem]
$k$ is algebraically closed, $C$ is a curve over $k$, then its function field $F=k(C)$ is $C_1$
\end{theorem}

\begin{remark}
This is equivalent of saying that $F$ is a finite extension of $k(t)$, e.g. $F=\mathbb C(t)$, $F=\mathbb C(t)(\sqrt{t(t-1)(t+1)})$
\end{remark}

\begin{proof}
By clearing out the denominators, we may assume $f$ is a degree $d$ homogeneous polynomial in $k[t][x_1,\cdots,x_n]$. Suppose $r$ is the maximal degree of the coefficients in $f$. If we write for some sufficiently large $N$
\[
x_i=\sum_{j=0}^Na_{ij}t^j
\]
Then we can rewrite $f$ as
\[
\sum_{l=0}^{dN+r}f_l(a_{10},\cdots,a_{nN})t^l
\]
Then $f_l$ are homogeneous polynomial of $a_{10},\cdots,a_{nN}$ of degree $d$, so $V(f_0,\cdots,f_{dN+r})$ is a closed subvariety of $\mathbb P^{n(N+1)-1}$, then we can apply Lemma~\ref{04/14/2022-13:40}, there exists $a_{10},\cdots,a_{nN}$ not all zero such that $f=0$, so $x_i$ would be a solution.
\end{proof}

\begin{lemma}\label{04/14/2022-13:40}
$W\subseteq\mathbb P^n$ is a positive dimensional closed subvariety, $f\neq0$ is a rational function on $\mathbb P^n$, then $W\cap V(f)\neq\varnothing$
\end{lemma}

\begin{proof}
This is equivalent to $W\subseteq D(f)$, we know that $D(f)$ is affine so we could assume $D(f)=\Spec A$, then a morphism $W\hookrightarrow D(f)$ corresponds to $A\to\Gamma(W,\mathcal O_W)=k$ since $W$ is proper and projective
\end{proof}

\begin{remark}
If $k=\overline{\mathbb F_p}$, $A/\mathbb F_q(C)$ is a CSA, then $A$ split over $k(C)$
\end{remark}

\end{document}