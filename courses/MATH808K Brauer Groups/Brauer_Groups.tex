\documentclass[a4paper,10pt]{article}
\usepackage{My_math_package}

\begin{document}

\begin{theorem}[Relative Skolem-Noether theorem]
$S$ is a CSA/k, $R\hookrightarrow S$ is a simple $k$-algebra, suppose $f,g:R\to S$ are are homomorphisms, there there exists automorphism $\alpha:S\to S$ such that $\alpha\circ f=g$
\end{theorem}




Consider $A$ is a CSA/k with $E$ being a maximal subfield where $E/k$ is finite Galois with Galois group $G$, then for each $\sigma\in G$, we can find $x_\sigma\in A^\times$ such that $x_\sigma a x_\sigma^{-1}=\sigma(a)$, $\forall a\in E$. If $y_\sigma$ is another choice, then $x_\sigma,y_\sigma$ differ by $E^\times$, hence this defines $x_\sigma x_\tau=\alpha_{\sigma,\tau}x_{\sigma\tau}$

$\alpha:G\times G\to E^\times$ is actually a 2-cocycle

\begin{definition}
Suppose $H$ is an abelian group with a $G$ action, $Z^2(G,H)$ consists of $\alpha:G\times G\to H$ such that $\alpha_{\rho\sigma,\tau}\cdot\alpha_{\rho,\sigma}=\rho(\alpha_{\sigma,\tau})\cdot\alpha_{\rho,\sigma\tau}$. $B^2(G,H)$ consists of $\gamma_{\sigma,\tau}=\frac{\sigma(f_\tau)\cdot f_\sigma}{f_{\sigma\tau}}$
\end{definition}

\begin{lemma}
Suppose $A$ is a finite dimensional $k$-algebra, $E/k\subseteq A$ is a finite Galois extension with Galois group $G$, and if $\{x_\sigma\}_{\sigma\in G}\subseteq A^\times$ such that $x_\sigma a x_\sigma^{-1}=\sigma(a)$, $\forall a\in E$. Then $x_\sigma$ are $E$-linear independent in $A$
\end{lemma}

\begin{proof}

\end{proof}

\begin{proposition}
Suppose $\alpha:G\times G\to E^\times$ is a 2-cocycle, There is a $k$-algebra $(E/k,\alpha):=\oplus Ex_\sigma$ with $x_\sigma x_\tau=\alpha_{\sigma,\tau}x_{\sigma\tau}$, $x_\sigma a x_\sigma^{-1}=\sigma(a)$, $\forall a\in E$. This is a CSA/k. Moreover, $(E/k,\alpha)\cong (E/k,\beta)$ if $\alpha\sim\beta$
\end{proposition}

\begin{proof}

\end{proof}

\begin{theorem}
Consider $\phi:\Br(E/k)\to H^2(G,E^\times)$
\end{theorem}

\begin{proof}
Consider the inverse $\phi^{-1}:H^2(G,E^\times)\to \Br(E/k)$, $[\alpha]\mapsto(E/k,\alpha)$
\end{proof}

\begin{definition}
Suppose $H$ is a group with a $G$-action, then
\[
H^0(G,H)=Z^0(G,H)=H^G=\{h\in H|\sigma(h)=h,\forall\sigma\in G\}
\]
If $H$ is not abelian, this is a pointed set
\end{definition}

\begin{remark}
A sequence of pointed set is $L\xrightarrow{\pi} M\xrightarrow{\rho}N$, and every element in $\ker\rho$ has an preimage in $L$. If $N=\{*\}$, then $\pi$ is surjective, but even if $L=\{*\}$, $\rho$ might be non-injective
\end{remark}

\begin{proposition}
Suppose $1\to A\to B\to C\to1$ is an exact sequence of groups with $G$-action, then there is the long exact sequence
\[
1\to H^0(G,A)\xrightarrow{(1)} H^0(G,B)\xrightarrow{(2)} H^0(G,C)\xrightarrow{\delta^1}H^1(G,A)\xrightarrow{(3)} H^1(G,B)\xrightarrow{(4)} H^1(G,C)
\]
Moreover, if $A$ is central in $B$, then $\delta^2$ exists and we can extend this with $\xrightarrow{\delta^2}H^2(G,A)\to H^2(G,B)\to H^2(G,C)$
\end{proposition}

\begin{proof}

\end{proof}

\begin{example}
\[
1\to E^\times\to \GL_n(E)\to\PGL_n(E)\to1
\]
Then we get $H^1(G,E^\times)\to H^1(G,\PGL_n(E))(\cong CSA/k)\xrightarrow{\delta^2}H^2(G,\times E)(\cong\Br(E/k))$
\end{example}

\begin{proposition}
\begin{enumerate}
\item For CSA/k $A,B$ with $\deg A=n$, $\deg B=m$, $\delta^2_{nm}([A\otimes_kB])=\delta_n^2([A])\delta_m^2([B])$
\item For CSA/k $A$ with $\deg A=n$, $\delta_n^2([A])=0\iff A\cong M_n(k)$
\item If $[E:k]=n$, then $\delta_n^2$ is surjective, and $\phi^{-1}\circ\delta_n^2([(E/k,\alpha)])=[(E/k,\alpha)^{\op}]$
\end{enumerate}
\end{proposition}

\begin{proposition}
Suppose $A$ is a CSA/k with $\ind A=d$, then $\ind A^{\otimes r}\mid\binom{d}{r}$, $r\leq d$. In particular, $A^{\otimes r}$ split, and $\Br(E/k)$ are torsion
\end{proposition}

\begin{proof}
Let $E/k$ be finite Galois extension that splits $A$, $G=\Gal(E/k)$, $V=E^{\oplus d}$, we get the following commutative diagram
\begin{center}
\begin{tikzcd}
1 \arrow[r] & E^\times \arrow[r] \arrow[d, "\lambda\mapsto\lambda^r"'] & \GL(V) \arrow[r] \arrow[d, "\pi"'] & \GL(V) \arrow[r] \arrow[d]   & 1 \\
1 \arrow[r] & E^\times \arrow[r]                                       & \GL(\bigwedge^rV) \arrow[r]        & \PGL(\bigwedge^rV) \arrow[r] & 1
\end{tikzcd}
\end{center}
Here $\pi(\phi)=\bigwedge^r\phi$, so we get exact squares

\end{proof}

\end{document}