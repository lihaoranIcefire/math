\documentclass[../main.tex]{subfiles}

\begin{document}

\begin{theorem}
$K\subseteq X$ compact with $\widehat K=K$. For all coherent sheaves $\mathcal E$ defined on a nbd of $K$
\begin{enumerate}
\item There exist sections of $\mathcal E$ on a nbd $U$ of $K$ that generate $\mathcal E_z$ for every $z\in U$
\item and if $\sigma$ is any other section of $\mathcal E$ on $U$, then $\sigma=\sum f_i\sigma_i$ for some $f_i\in A(U)$
\end{enumerate}
In terms of exact sequences, there exists on $U$ an exact sequence $O\to E\to0$ and the induced map  is exact
\end{theorem}

\begin{lemma}
$\widehat K=K$ as above, $f\in A(X)$. Suppose property holds for $K_a=\{z\in K|\Re(f(z))=a\}$ for all $z\in\mathbb R$. Then it holds for $K$
\end{lemma}

\begin{proof}
Define $K_{a,b}=\{z\in K|a\leq\Re(f(z))\leq b\}$, then $\widehat K_{a,b}=K_{a,b}$. Indeed, this condition is equivalent to saying $|e^{f(z)}|\leq e^b$, and $|e^{-f(z)}|\leq e^{-a}$. Suppose $z\in\widehat K_{a,b}$. If $z\in K$, then $|e^{f(z)}|\leq\|e^{f(z)}\|_{K_{a,b}}\leq e^b$, $|e^{-f(z)}|\leq\|e^{-f(z)}\|_{K_{a,b}}\leq e^{-a}$. If $z\notin K=\widehat K$, there is $g\in A(X)$ such that $|g(z)|\geq\|g\|_K\geq\|g\|_{K_{a,b}}$, $z\notin \widehat K_{a,b}$. Suppose $\beta=\sup$. If $\beta=+\infty$, then we are done. If not, by hypothesis we can find a nbd $U_1$ of $K_{\beta,\beta}$ and sections $\phi_1,\cdots,\phi_p$ generating $\mathcal E$ on $U_1$. Choose $\alpha<\beta<b$ such that $K_{a,b}\subseteq U_1$. We can find nbd $U_2$ of $K_{-\infty,a}$ and sections $\psi_1,\cdots,\psi_q$ generating $\mathcal E$ on $U_2$. Regard $\phi,\psi$ as column vectors, shrinking the sets if necessary, we may assume there are matrix valued holomorphic functions $A_1,A_2$ such that $\psi=A_1\phi$ and $\phi=A_2\psi$. In other words
\[\begin{bmatrix}
\phi\\
0
\end{bmatrix}=\begin{bmatrix}
1&0\\
-A_1&1
\end{bmatrix}\begin{bmatrix}
1&A_2\\
0&1
\end{bmatrix}\begin{bmatrix}
0\\
\psi
\end{bmatrix}\]
We wnat to find vectors $f_1,f_2$ of dimension $p+q$ such that
\[f_1\cdot\begin{bmatrix}
\phi\\
0
\end{bmatrix}=f_2\cdot\begin{bmatrix}
0\\
\psi
\end{bmatrix}\]
on the intersection $U_1\cap U_2$. Let
\[c_{21}=c_{12}^{-1}=\begin{bmatrix}
1&\\
A_2^T&1
\end{bmatrix}\begin{bmatrix}
1&-A_1^T\\
&1
\end{bmatrix}\]
Then $f_i$ satisfy $f_2=c_{21}f_1$. We may view $\{f_1,f_2\}$ as a section of rank $p+q$ holomorphic vector bundle over $U_1\cap U_2$ with transition function $c_{12}$. We may also suppose that $U_1\cap U_2$ is Stein. Since we have already show that global sections of vector bundles exist, we can find such a pair and this leads to a contradiction
\end{proof}

\begin{theorem}[Cartan Theorem A]
Let $\mathcal E\to X$ be a coherent analytic sheaf on a Stein manifold. Then for every $z\in X$, $\mathcal E_z$ is generated by the germs of global sections $\Gamma(X,\mathcal E)$
\end{theorem}

\begin{theorem}[Cartan Theorem B]
Let $\mathcal E\to X$ be a coherent analytic sheaf on a Stein manifold. Then $H^q(X,\mathcal E)=\{0\}$ for all $q\geq1$
\end{theorem}

\begin{remark}
For locally free sheaves, $H^q(X,\mathcal E)\cong H^{0,q}_{\bar\partial}(X,E)$, and we have proven this result using the solution to the $\bar\partial$ equation
\end{remark}

\begin{definition}
$\mathcal E\to X$ is called a \textit{fine sheaf}\index{Fine sheaf} if it admits a partition of unity for every locally finite cover
\end{definition}

\begin{proposition}
Suppose
\[0\to\Gamma(X,\mathcal S)\to\Gamma(X,\mathcal E)\to\Gamma(X,\mathcal Q)\]
is exact. If $\mathcal S$ is a fine sheaf, then is $\Gamma(X,\mathcal E)\to\Gamma(X,\mathcal Q)$ surjective
\end{proposition}

\begin{definition}
Cohomology is a funtor
\[H^\bullet:\{\text{sheaves on }X\}\to\{\text{abelian groups}\}\]
satisfying axioms
\begin{enumerate}
\item $H^0(X,\mathcal E)=\Gamma(X,\mathcal E)$
\item If $\mathcal E$ is a fine sheaf, then $H^p(X,\mathcal E)=0$ for $p\geq1$
\item If $0\to\mathcal S\to\mathcal E\to\mathcal Q\to0$ is a short exact sequence, then we have long exact sequence
\end{enumerate}
\end{definition}

\begin{proposition}
Let $\mathcal E\to X$ be a sheaf with a fine resolution, i.e. a long exact sequence
\[0\to\mathcal E\to\mathcal A^0\xrightarrow{d_0}\mathcal A^1\xrightarrow{d_1}\cdots\]
Then
\[H^p(X,\mathcal E)=\frac{\ker(\mathcal A^p(X)\xrightarrow{d_p}\mathcal A^{p+1}(X))}{\im(\mathcal A^{p-1}(X)\xrightarrow{d_{p-1}}\mathcal A^{p}(X))}\]
\end{proposition}

\begin{theorem}
Let $X$ be a complex manifold, and suppose that $H^1(X,\mathcal E)=0$ for every coherent analytic sheaf $\mathcal E\to X$. Then $X$ is Stein
\end{theorem}

\begin{proof}
$A(X)$ separates points: Let $\mathcal I_{x,y}$ be the sheaf of germs of holomorphic functions vanishing at $x$ and $y$. This is a coherent sheaf. From the exact sequence
\[0\to\mathcal I_{x,y}\to\mathcal O_X\to\mathcal O_X/\mathcal I_{x,y}\to0\]
We have by the assumption
\[H^0(X,\mathcal O_X)\to H^0(X,\mathcal O_X/\mathcal I_{x,y})\cong\mathcal C^{\oplus2}\to H^1(X,\mathcal I_{x,y})=0\]
Hence there is $f\in A(X)$ such that $f(x)=1$ and $f(y)=0$ \\
$A(X)$ gives local coordinates: Let $z_1,\cdots,z_n$ be local coordinates at $x$, then the germs of these generate $\mathcal I_x/\mathcal I_x^2$. we have exact sequence
\[0\to\mathcal I_x^2\to\mathcal I_x\to\mathcal I_x/\mathcal I_x^2\to0\]
From the vanishing of $H^1(X,\mathcal I_x^2)$, we may find functions $f_1,\cdots,f_n\in A(X)$, vanishing at $x$, $f_i=z_i$ modulo  \\
$X$ is holomorphically convex: $\{x_i\}$ is a discrete sequence, $\mathcal I_{\{x_i\}}$ is coherent. From the vanishing of $H^1(X,\mathcal O_X/\mathcal I_{\{x_i\}})$, there is $f\in A(X)$ such that $f(x_i)=i$
\end{proof}

\begin{remark}
Notice the similarity with the proof of the Kodaira embedding theorem
\end{remark}

\begin{theorem}[Cartan-Serre theorem]
$X$ is a compact complex manifold. Then for any coherent sheaf $\mathcal E\to X$, the cohomology groups $H^p(X,\mathcal E)$ are finite dimensional
\end{theorem}

\begin{proof}
Recall the Frechet space structure we have on sections of coherent sheaves. Coherent sheaves satisfy the Montel property: if $U\subset\subset V$, then the restriction map $H^0(V,\mathcal E)\to H^0(U,\mathcal E)$ is a compact operator. Now choose finite Stein open coverings $\{U_i\},\{V_i\}$, $V_i\subset\subset V_i$ such that
\[H^p(\mathcal U,\mathcal E)=H^p(\mathcal V,\mathcal E)=H^p(X,\mathcal E)\]
Then by the finiteness of the cover, the restriction map
\[\rho:C^p(\mathcal V.\mathcal E)\to C^p(\mathcal U,\mathcal E)\]
is compact. The coboundary map $\delta$ is clearly continuous, so $Z^p=\ker\delta$ is a closed subspace, hence a Frechet space. Now define
\[u:Z^p(\mathcal U,\mathcal E)\oplus C^{p-1}(\mathcal U,\mathcal E)\to Z^p(\mathcal U,\mathcal E)\]
by $u(\phi,\psi)=\rho(\phi)+\delta\psi$. Then $u$ is surjective. Apply Theorem \ref{Schwartz theorem} to $T=u$ and $S=-\rho$
\end{proof}

\begin{theorem}[Laurent Schwartz]\label{Schwartz theorem}
Let $E,F$ be Frechet spaces with $T:E\to F$ is surjective and $S:E\to F$ compact. Then $T+S$ has closed range and $F/\im(T+S)$ is finite dimensional
\end{theorem}

\begin{definition}
$X$ is a complex manifold. A subset $V\subseteq X$ is called an \textit{analytic subvariety}\index{analytic subvariety} of $X$ is for every point $x\in X$ there is neighborhood $U$ of $x$ and $f_1,\cdots,f_l\in A(U)$ such that $V\cap U=Z(f_1,\cdots,f_l)$, the common zero locus of $f_i$'s
\end{definition}

It follows that $V\subseteq X$ is closed. A complex submanifold of codimension $l$ is an analytic subvariety

The ideal sheaf $\mathcal I_V$ is coherent by Oka's theorem. $\mathcal O_V=\mathcal O_X/\mathcal I_V$

Fact: The maximum principle holds for holomorphic functions on subvarieties

Consequence: The only compact subvarieties of a Stein manifold are finite collections of points

\end{document}