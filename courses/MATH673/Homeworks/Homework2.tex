\documentclass[../main.tex]{subfiles}

\begin{document}

\begin{customproblem}\textbf{2.5.13}
\textbf{(a)} \par
We have $u_{t}=-\dfrac{x}{2t^{\frac{3}{2}}}v'$ and $u_{xx}=\dfrac{1}{t}v''$, thus $u_{t}-u_{xx}=0 \Leftrightarrow -\dfrac{x}{2t^{\frac{3}{2}}}v'-\dfrac{1}{t}v''=0 \Leftrightarrow v''+\dfrac{z}{2}v'=0$ \par
Multiply $e^{\frac{z^{2}}{4}}$ on both sides, we get $e^{\frac{z^{2}}{4}}v''+\dfrac{ze^{\frac{z^{2}}{4}}}{2}v'=\left(e^{\frac{z^{2}}{4}}v'\right)'=0$, Thus $e^{\frac{z^{2}}{4}}v'=c$ for some constant $c$, $v'=ce^{-\frac{z^{2}}{4}} \Rightarrow \displaystyle v=c\int_{0}^{z}e^{-\frac{s^{2}}{4}}ds+d$ for some other constant $d$ \par
\textbf{(b)} \par
According to (a), we have $u(x,t)=\displaystyle c\int_{0}^{\frac{x}{\sqrt{t}}}e^{-\frac{s^{2}}{4}}ds+d$, thus $u_{x}=\dfrac{c}{\sqrt{t}}e^{-\frac{x^{2}}{4t}}$, which again solve the heat equation, in order to obtain a fundamental solution, we also need $\lim_{t\downarrow 0}u_x(x,t)=u_x(x,0)=\delta_0$, thus we should have $1=\displaystyle \int_\mathbb{R}u_x(x,0)dx=\int_\mathbb{R}\lim_{t\downarrow 0}u_x(x,t)dx=\lim_{t\downarrow 0}\int_\mathbb{R}u_x(x,t)dx=\lim_{t\downarrow 0}\int_\mathbb{R}\dfrac{c}{\sqrt{t}}e^{-\frac{x^{2}}{4t}}dx=2c\sqrt{\pi}\Rightarrow \, c=\dfrac{1}{\sqrt{4\pi}}$
\end{customproblem}

\begin{customproblem}\textbf{2.5.14}
Consider $v=ue^{ct}$, we have $v_{t}-\Delta v=e^{ct}\left(u_{t}-\Delta u+cu\right)$, thus the initial value problem becomes \par
\[
\left\{\begin{matrix}
v_{t}-\Delta v=fe^{ct},\,\text{in}\,\mathbb{R}^{n}\times(0,\infty)\\ 
v=g,\,\text{on}\,\mathbb{R}^{n}\times\{t=0\}
\end{matrix}\right.
\]
Solve it to get
\[
v(x,t)=\int_{\mathbb{R}^n}\Phi(x-y,t)g(y)dy + \int_0^t\int_{\mathbb{R}^n}\Phi(x-y,t-s)f(y,s)e^{cs}dyds
\]
Thus
\[
u(x,t)=\int_{\mathbb{R}^n}\Phi(x-y,t)g(y)e^{-ct}dy + \int_0^t\int_{\mathbb{R}^n}\Phi(x-y,t-s)f(y,s)e^{-c(t-s)}dyds
\]
\end{customproblem}

\begin{customproblem}\textbf{January 2013, Problem 4}
\textbf{(a)} \par
Since $\displaystyle u(x,t)=\int_0^xu_x(y,t)dy+u(0,t)=\int_0^xu_x(y,t)dy$
$$\displaystyle |u(x,t)|^2=\left|\int_0^xu_x(y,t)dy\right|^2\leq\left(\int_0^1|u_x(y,t)|dy\right)^2\leq \left(\int_0^1|u_x(y,t)|^2dy\right)\left(\int_0^11^2dy\right)=\int_0^1|u_x(y,t)|^2dy$$
Hence $\displaystyle\sup_x|u(x,t)|^2\leq\int_0^1|u_x(y,t)|^2dy$ \par
Then we have
\[
\begin{aligned}
\int_0^1u^3dx&\leq\int_0^1|u^3|dx\leq \left(\int_0^1|u_x|^2dx\right)\int_0^1|u|dx \\
&\leq \left(\int_0^1|u_x|^2dx\right)\left(\int_0^1|u|^2dx\right)^\frac{1}{2}\left(\int_0^11^2dy\right)^\frac{1}{2} \\
&=\left(\int_0^1|u_x|^2dx\right)\left(\int_0^1|u|^2dx\right)^\frac{1}{2}
\end{aligned}
\]
Since $u$ is smooth, we have
\[
\begin{aligned}
\dfrac{d}{dt}\int_0^1|u|^2dx&=\int_0^1\dfrac{d}{dt}u^2dx=\int_0^12uu_tdx=2\int_0^1u(u_{xx}+cu^2)dx \\
&=2uu_{xx}|_0^1-2\int_0^1|u_x|^2dx+2c\int_0^1u^3dx \\
&\leq -2\int_0^1|u_x|^2dx+2c\left(\int_0^1|u_x|^2dx\right)\left(\int_0^1|u|^2dx\right)^\frac{1}{2} \\
&=-2\left(\int_0^1|u_x|^2dx\right)\left(1-c\left(\int_0^1|u|^2dx\right)^\frac{1}{2}\right)
\end{aligned}
\]
\textbf{(b)} \par
Since $u$ is smooth, so is $F(t):=\displaystyle\left(\int_0^1|u|^2dx\right)^\frac{1}{2}$, suppose $\exists t_0\in (0,\infty)$ such that $F(t_0)\geq \dfrac{1}{c^2}$, then $\varnothing\neq S:=\left\{t\in[0,t_0]\left|\right.F(t)=\dfrac{1}{c^2}\right\}$ is closed, let $0<t_1=\min_{t\in S}t$, then we have $F(t_1)=\dfrac{1}{c^2}$ and $F(t)<\dfrac{1}{c^2}$ for $t\in[0,t_1)$, using intermediate value theorem, we have $0<F(t_1)-F(0)=F'(\xi)t_1$, where $\xi\in(0,t_1)$, but according to (a), we have $\displaystyle F'(\xi)\leq -2\left(\int_0^1|u_x|^2dx\right)\left(1-cF(\xi)^\frac{1}{2}\right)\leq0$ which leads to a contradiction \par
\end{customproblem}

\begin{customproblem}\textbf{January 2012, Problem 2}
\textbf{(a)} \par
This is equivalent to solve ODE $
\left\{\begin{matrix}
w'+w^{3}=0,\,t>0\\ 
w(0)=c
\end{matrix}\right.
$, thus $w(t)=\dfrac{1}{\sqrt{2t+\frac{1}{c^{2}}}}$ \par
\textbf{(b)} \par
Assume otherwise, then $\exists (t_0,x_0)\in (0,\infty)\times (0,1)$ such that $(w-u)(t_0,x_0)<0$, suppose $w-u$ must attain the minimum of $[0,t_0]\times [0,1]$ at $(t_1,x_1)$, then $(w-u)(t_1,x_1)<0$, thus $0\leq w(t_1.x_1)<u(t_1.x_1)$, hence $w^3(t_1.x_1)<u^3(t_1.x_1)$, then we would have $0<u^3(t_1.x_1)-w^3(t_1.x_1)=(\frac{\partial}{\partial t}-\Delta)(w-u)(t_1,x_1)\geq 0$ which is a contradiction, hence $u(t,x)\leq \dfrac{1}{\sqrt{2t+\frac{1}{c^{2}}}}$ on $[0,\infty)\times[0,1]$
\end{customproblem}

\begin{customproblem}\textbf{August 2011, Problem 3}
\textbf{(a)} \par
Assume $\exists (x_0,t_0)\in\Omega\times(0,\infty)$ such that $u(x_0,t_0)<0$, suppose $u(x_1,t_1)=\min_{\bar{\Omega}_{t_0}}u<0$, since $|f'|\leq K$, $\displaystyle f(x)=\int_0^xf'dy\leq\int_0^xKdy=Kx$ if $x\geq0$, or $f(x)\geq Kx$ if $x<0$ \par
let $K_1>K$, then $
0\geq\left(\dfrac{\partial}{\partial t}-\Delta\right)\left(ue^{-K_1t}\right)
=e^{-K_1t}\left[\left(\dfrac{\partial}{\partial t}-\Delta\right)u-K_1u\right] 
=e^{-K_1t}\left[f(u)-K_1u\right]
=e^{-K_1t}\displaystyle\int_u^0\left[K_1-f'\right]dx>0$ at $(x_1,t_1)$ which is a contradiction \par
\textbf{(b)} \par
Assume $\exists (x_0,t_0)\in\Omega\times(0,\infty)$ such that $u(x_0,t_0)>Me^{Kt_0}\Rightarrow u(x_0,t_0)e^{-K_1t_0}>M$ for some $K_1>K$, suppose $u(x_1,t_1)e^{-K_1t_1}=\max_{\bar{\Omega}_{t_0}}ue^{-K_1t}>0$ \par
$0\leq\left(\dfrac{\partial}{\partial t}-\Delta\right)\left(ue^{-K_1t}\right)
=e^{-K_1t}\left[f(u)-K_1u\right]=e^{-K_1t}\displaystyle\int_0^u\left[f'-K_1\right]dx<0 $ at $(x_1,t_1)$ which is a contradiction
\end{customproblem}

\begin{customproblem}\textbf{August 2005, Problem 6}
Suppose $|f'|<K$, $u,w$ are both solutions, then $(u-w)e^{Kt}=0$ on $\Omega\times\{0\}\cup\partial\Omega\times(0,\infty)$, Assume $\exists (x_0,t_0)\in\Omega\times(0,\infty)$ such that $(u-w)(x_0,t_0)e^{Kt_0}<0$, suppose $(u-w)(x_1,t_1)e^{-Kt_1}=\min_{\bar{\Omega}_{t_0}}(u-w)e^{Kt}<0$, the we have
 $
0\geq\left(\dfrac{\partial}{\partial t}-\Delta\right)\left((u-w)e^{Kt}\right)
=e^{-Kt}\left[\left(f(w)-f(u)\right)+K(u-w)\right]
=e^{-Kt}\displaystyle\int_u^w\left[f'+K\right]dx>0 $ at $(x_1,t_1)$, which is a contradiction

\end{customproblem}

\begin{customproblem}\textbf{2.5.15}
Define $v(x,t)=u(x,t)-g(t)$ on $x\geq 0$, extend $v$ to $\{x<0\}$ by odd reflection, then we have $v(x,t)=-v(-x,t)$ on $x<0$, thus the initial boundary problem becomes
\[
\left\{\begin{matrix}
v_{t}-v_{xx}=-g'(t),\,\text{in }\mathbb{R}_{+}\times(0,\infty)\\ 
v_{t}-v_{xx}=g'(t),\,\text{in }\mathbb{R}_{-}\times(0,\infty)\\
v=0,\,\text{on }\mathbb{R}_{+}\times\{t=0\}\\ 
v=0,\,\text{on }\{x=0\}\times[0,\infty)
\end{matrix}\right.
\]
Define $f(x,t)=
\left\{\begin{matrix}
-g'(t),\,x\geq 0\\ 
g'(t),\,x<0
\end{matrix}\right.$, we have
\[
\begin{aligned}
v(x,t)
&=\int_{0}^{t}\dfrac{1}{\sqrt{4\pi(t-s)}}\int_{\mathbb{R}}e^{-\frac{(x-y)^{2}}{4(t-s)}}f(y,s)dyds \\
&=\int_{0}^{t}\dfrac{1}{\sqrt{4\pi(t-s)}}\left(\int_{-\infty}^{0}e^{-\frac{(x-y)^{2}}{4(t-s)}}g'(s)dy-\int_{0}^{\infty}e^{-\frac{(x-y)^{2}}{4(t-s)}}g'(s)dy\right)ds \\
&=\dfrac{1}{\sqrt{\pi}}\int_{0}^{t}\left(\int_{-\infty}^{-\frac{x}{\sqrt{4(t-s)}}}e^{-y^{2}}dy-\int_{-\frac{x}{\sqrt{4(t-s)}}}^{\infty}e^{-y^{2}}dy\right)g'(s)ds \\
&=\left.\dfrac{1}{\sqrt{\pi}}g(s)\left(\int_{-\infty}^{-\frac{x}{\sqrt{4(t-s)}}}e^{-y^{2}}dy-\int_{-\frac{x}{\sqrt{4(t-s)}}}^{\infty}e^{-y^{2}}dy\right)\right\vert_{0}^{t}+\dfrac{x}{\sqrt{4\pi}}\int_{0}^{t}\dfrac{1}{\left((t-s)\right)^{\frac{3}{2}}}e^{-\frac{x^{2}}{4(t-s)}}g(s)ds \\
&=-\mathrm{sgn}(x)\dfrac{1}{\sqrt{\pi}}g(t)\int_{-\infty}^{\infty}e^{-y^{2}}dy-\dfrac{x}{\sqrt{4\pi}}\int_{0}^{t}\dfrac{1}{\left((t-s)\right)^{\frac{3}{2}}}e^{-\frac{x^{2}}{4(t-s)}}g(s)ds \\
&=-\mathrm{sgn}(x)g(t)-\dfrac{x}{\sqrt{4\pi}}\int_{0}^{t}\dfrac{1}{\left((t-s)\right)^{\frac{3}{2}}}e^{-\frac{x^{2}}{4(t-s)}}g(s)ds 
\end{aligned}
\]
Where $\mathrm{sgn}(x)=\left\{\begin{matrix}
-1,\,x<0\\ 
0,\,x=0\\ 
1,\,x>0
\end{matrix}\right.$, thus $\displaystyle u(x,t)=\dfrac{x}{\sqrt{4\pi}}\int_{0}^{t}\dfrac{1}{\left((t-s)\right)^{\frac{3}{2}}}e^{-\frac{x^{2}}{4(t-s)}}g(s)ds$ is the solution
\end{customproblem}

\begin{customproblem}\textbf{2.5.17}
\textbf{a)} \par
Define $\displaystyle \phi(r):=\dfrac{1}{4r^{n}}\int\int_{E(0,0;r)}v(y,s)\dfrac{|y|^{2}}{s^{2}}dyds$, then
\[
\begin{aligned}
\phi'(r)
&=\dfrac{1}{r^{n+1}}\int\int_{E(0,0;r)}-nv_{s}\psi-\dfrac{n}{2s}\sum_{i=1}^{n}v_{y_{i}}y_{i}dyds \\
&\geq\dfrac{1}{r^{n+1}}\int\int_{E(0,0;r)}-n\Delta v\psi-\dfrac{n}{2s}\sum_{i=1}^{n}v_{y_{i}}y_{i}dyds \\
&=0
\end{aligned}
\]
Since $\displaystyle\phi(r)=v(0,0)+\dfrac{1}{4r^{n}}\int\int_{E(0,0;r)}\left(v(y,s)-v(0,0)\right)\dfrac{|y|^{2}}{s^{2}}dyds\rightarrow v(0,0), \,r\rightarrow 0$, $\displaystyle v(0,0)=\lim_{s\rightarrow 0}\phi(r)\leq\lim_{s\rightarrow r}\phi(s)=\phi(r)$ \par
Thus $\displaystyle v(x,t)\leq\dfrac{1}{4r^{n}}\int\int_{E(x,t;r)}v(y,s)\dfrac{|x-y|^{2}}{(t-s)^{2}}dyds$ \par
\textbf{b)} \par
Define $v_{\epsilon}(x,t)=v(x,t)+\epsilon|x|^{2}$, then $\left(\dfrac{\partial}{\partial t}-\Delta\right)v_{\epsilon}=\left(\dfrac{\partial}{\partial t}-\Delta\right)v-2n\epsilon<0$, $v_{\epsilon}$ can't attain its maximum at $(x_{0},t_{0})\in \overline{U_{T}}\setminus\Gamma_{T}$, otherwise $\dfrac{\partial}{\partial t}v_{\epsilon}(x_{0},t_{0})\geq 0, \Delta v_{\epsilon}(x_{0},t_{0})\leq 0$, then $\Delta v_{\epsilon}(x_{0},t_{0})\geq 0$, thus $\underset{\overline{U_{T}}}{\mathrm{max}}\left(v+\epsilon|x|^{2}\right)\leq\underset{\Gamma_{T}}{\mathrm{max}}\left(v+\epsilon|x|^{2}\right)\leq \underset{\Gamma_{T}}{\mathrm{max}} \,v+C\epsilon$ since $U$ is bounded, hence $\underset{\overline{U_{T}}}{\mathrm{max}}\,v=\underset{\Gamma_{T}}{\mathrm{max}} \,v$
\textbf{c)} \par
Since $\phi$ is convex, $\phi''(x)\geq 0$
\[
\left(\dfrac{\partial}{\partial t}-\Delta\right) v(x)=\left(\dfrac{\partial}{\partial t}-\Delta\right)\phi\left(v(x,t)\right)=\phi'(v)\left(\dfrac{\partial v}{\partial t}-\Delta\right)v-\phi''(v)|\nabla v|^{2}\leq 0
\]
Thus $v$ is a subsolution \par
\textbf{d)} \par
\[
\begin{aligned}
\left(\dfrac{\partial}{\partial t}-\Delta\right)v 
&= \left(\dfrac{\partial}{\partial t}-\Delta\right) \left(\left|\nabla u\right|^{2}+u_{t}^{2}\right)  \\
&= 2\nabla u \cdot \nabla u_{t}+2u_{t}u_{tt}-2\sum_{i,j}u_{x_{i}x_{j}}^{2}-2 \nabla u \cdot \nabla(\Delta u)-2|\nabla u_{t}|^{2}-2u_{t}(\Delta u)_{t} \\
&=2\nabla u \cdot \nabla (u_{t}-\Delta u)+2u_{t}(u_{t}-\Delta u)_{t}-2\sum_{i,j}u_{x_{i}x_{j}}^{2}-2|\nabla u_{t}|^{2} \\
&\leq 0
\end{aligned}
\]
Thus $v$ is a subsolution
\end{customproblem}

\end{document}