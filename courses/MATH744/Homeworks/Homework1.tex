\documentclass[main]{subfiles}

\begin{document}

\begin{exercise}
Let $G=O(2,\mathbb R)$ be the (real) orthogonal group in dimension 2
\begin{enumerate}[label=(\alph*),leftmargin=*]
\item Show that $\SO(2,\mathbb R)$ is isomorphic to the circle $S^1$
\item Show that $O(2,\mathbb R)$ has two connected components, and is not abelian
\item Show that $\{g\in O(2,\mathbb R)|\det g=-1\}$ is a single conjugacy class
\item Show there is an exact sequence $1\to \SO(2)\to O(2)\to\mathbb Z/2\mathbb Z\to1$. Is this sequence split
\end{enumerate}
\end{exercise}

\begin{solution}
\begin{enumerate}[label=(\alph*),leftmargin=*]
\item Suppose $g=\left( {\begin{array}{cc}
   a & b \\
   c & d \\
  \end{array} } \right) \in SO(2,\mathbb{R})$, then 
  $$\left( {\begin{array}{cc}
   1 & 0 \\
   0 & 1 \\
  \end{array} } \right)=\left( {\begin{array}{cc}
   a & b \\
   c & d \\
  \end{array} } \right)\left( {\begin{array}{cc}
   a & c \\
   b & d \\
  \end{array} } \right)=\left( {\begin{array}{cc}
   a^2+b^2 & ac+bd \\
   ac+bd & c^2+d^2 \\
  \end{array} } \right)\Rightarrow \begin{cases}
  a^2+b^2=1 \\
  ac+bd=0 \\
  c^2+d^2=1
  \end{cases}$$
 we can find $0\leq\theta-\varphi\leq2\pi$ such that $\begin{cases}
  a=\cos{\theta} \\
  b=\sin{\theta} \\
  c=\cos{\varphi} \\
  d=\sin{\varphi}
  \end{cases}$ \\
then $0=\cos{\theta}\cos{\varphi}+\sin{\theta}\sin{\varphi}=\cos{\theta-\varphi}$, hence $\theta-\varphi=\dfrac{\pi}{2}$ or $\dfrac{3\pi}{2}$ \par

hence if $\theta-\varphi=\dfrac{\pi}{2}$, $g=\left( {\begin{array}{cc}
   \cos{\theta} & \sin{\theta} \\
   \sin{\theta} & -\cos{\theta} \\
  \end{array} } \right)$, $\det{g}=1$, if $\theta-\varphi=\dfrac{3\pi}{2}$, $g=\left( {\begin{array}{cc}
   \cos{\theta} & \sin{\theta} \\
   -\sin{\theta} & \cos{\theta} \\
  \end{array} } \right)$, $\det{g}=1$, thus $\phi:S^1\rightarrow SO(2,\mathbb{R})$, $\theta\mapsto\left( {\begin{array}{cc}
   \cos{\theta} & \sin{\theta} \\
   -\sin{\theta} & \cos{\theta} \\
  \end{array} } \right)$ is an isomorphism
\item From the analysis in (a), all matrices of form $\left\{\left( {\begin{array}{cc}
   \cos{\theta} & \sin{\theta} \\
   \sin{\theta} & -\cos{\theta} \\
  \end{array} } \right)\right\}=O(2,\mathbb{R})\setminus SO(2,\mathbb{R})$ is path-connected, and all matrices of form $\left\{\left( {\begin{array}{cc}
   \cos{\theta} & \sin{\theta} \\
   -\sin{\theta} & \cos{\theta} \\
  \end{array} } \right)\right\}=SO(2,\mathbb{R})$ is also path-connected, but $O(2,\mathbb{R})$ is not path-connected since $\det:O(2,\mathbb{R})\rightarrow\mathbb{R}$ is a continuous function but the image is $\{\pm1\}$ is not connected, thus $O(2,\mathbb{R})$ has two connected components
\[\left( {\begin{array}{cc}
   \dfrac{\sqrt2}{2} & \dfrac{\sqrt2}{2} \\
   -\dfrac{\sqrt2}{2} & \dfrac{\sqrt2}{2} \\
  \end{array} } \right)\left( {\begin{array}{cc}
   1 & 0 \\
   0 & -1 \\
  \end{array} } \right)=\left( {\begin{array}{cc}
   \dfrac{\sqrt2}{2} & -\dfrac{\sqrt2}{2} \\
   -\dfrac{\sqrt2}{2} & -\dfrac{\sqrt2}{2} \\
  \end{array} } \right)\neq\left( {\begin{array}{cc}
   \dfrac{\sqrt2}{2} & \dfrac{\sqrt2}{2} \\
   \dfrac{\sqrt2}{2} & -\dfrac{\sqrt2}{2} \\
  \end{array} } \right)=\left( {\begin{array}{cc}
   1 & 0 \\
   0 & -1 \\
  \end{array} } \right)\left( {\begin{array}{cc}
   \dfrac{\sqrt2}{2} & \dfrac{\sqrt2}{2} \\
   -\dfrac{\sqrt2}{2} & \dfrac{\sqrt2}{2} \\
  \end{array} } \right)\]
hence $O(2,\mathbb{R})$ is not abelian 
\item $\left( {\begin{array}{cc}
   1 & 0 \\
   0 & -1 \\
  \end{array} } \right)\left( {\begin{array}{cc}
   \cos{\theta} & \sin{\theta} \\
   -\sin{\theta} & \cos{\theta} \\
  \end{array} } \right)=\left( {\begin{array}{cc}
   \cos{\theta} & \sin{\theta} \\
   \sin{\theta} & -\cos{\theta} \\
  \end{array} } \right)$
hence $O(2,\mathbb{R})\setminus SO(2,\mathbb{R})$ is a single conjugacy class 
\item Notice $\det:O(2,\mathbb{R})\rightarrow\mathbb{Z}/2\mathbb{Z}\cong\{\pm1,\text{multiplication}\}$ is a surjective group homomorphism with kernel $SO(2,\mathbb{R})$, hence
$$\begin{tikzcd}
1 \arrow[r] & {SO(2,\mathbb{R})} \arrow[r, hook] & {O(2,\mathbb{R})} \arrow[r, "\det"] & \mathbb{Z}/2\mathbb{Z} \arrow[r] & 1
\end{tikzcd}$$
is an exact sequence \\
There is a map $\mu:\mathbb{Z}/2\mathbb{Z}\rightarrow O(2,\mathbb{R})$, $\bar 1\mapsto\left( {\begin{array}{cc}
   1 & 0 \\
   0 & -1 \\
  \end{array} } \right)$ which is a homomorphism such that $\det\circ\mu=1$, hence the exact sequence splits \par
%This sequence does not split since both $SO(2,\mathbb{R})$ and $\mathbb{Z}/2\mathbb{Z}$ are abelian but $O(2,\mathbb{R})$ is not, which implies $O(2,\mathbb{R})\ncong SO(2,\mathbb{R})\times \mathbb{Z}/2\mathbb{Z}$
\end{enumerate}
\end{solution}

\begin{exercise}
\begin{enumerate}[label=(\alph*),leftmargin=*]

\end{enumerate}
\end{exercise}

\textbf{(a)} \par
Suppose $g=\left( {\begin{array}{cc}
   a & b \\
   c & d \\
  \end{array} } \right) \in SO(1,1)$, then 
  $$\left( {\begin{array}{cc}
   1 & 0 \\
   0 & 1 \\
  \end{array} } \right)=\left( {\begin{array}{cc}
   a & b \\
   c & d \\
  \end{array} } \right)\left( {\begin{array}{cc}
   1 & 0 \\
   0 & -1 \\
  \end{array} } \right)\left( {\begin{array}{cc}
   a & c \\
   b & d \\
  \end{array} } \right)=\left( {\begin{array}{cc}
   a^2+b^2 & ac+bd \\
   ac+bd & c^2+d^2 \\
  \end{array} } \right)\Rightarrow \begin{cases}
  a^2-b^2=1 \\
  ac-bd=0 \\
  c^2-d^2=-1
  \end{cases}$$
 we can find $x,y\in\mathbb{R}$ such that $\begin{cases}
  b=\sinh{x} \\
  c=\sinh{y}
  \end{cases}$, and $\begin{cases}
  a=\cosh{x} \\
  d=\cosh{y}
  \end{cases}$ or $\begin{cases}
  a=-\cosh{x} \\
  d=\cosh{y}
  \end{cases}$ or $\begin{cases}
  a=\cosh{x} \\
  d=-\cosh{y}
  \end{cases}$ or $\begin{cases}
  a=-\cosh{x} \\
  d=-\cosh{y}
  \end{cases}$ \par
then 
$$0=\cosh{x}\sinh{y}-\sinh{x}\cosh{y}=\sinh{(y-x)}\iff y=x, \text{or}$$
$$0=-\cosh{x}\sinh{y}-\sinh{x}\cosh{y}=-\sinh{(x+y)}\iff y=-x, \text{or}$$
$$0=\cosh{x}\sinh{y}+\sinh{x}\cosh{y}=\sinh{(x+y)}\iff y=-x, \text{or}$$
$$0=-\cosh{x}\sinh{y}+\sinh{x}\cosh{y}=\sinh{(x-y)}\iff y=x$$

then $\det{g}=1,-1,-1,1$ correspondingly, hence $g$ can only be $\left( {\begin{array}{cc}
   \cosh{x} & \sinh x \\
   \sinh{x} & \cosh{x} \\
  \end{array} } \right)$ or $\left( {\begin{array}{cc}
   -\cosh{x} & \sinh{x} \\
   \sinh{x} & -\cosh{x} \\
  \end{array} } \right)$, consider $\phi:\mathbb{R}^*\rightarrow SO(1,1)$, 
  $$x\mapsto\begin{cases}
  \left( {\begin{array}{cc}
   \cosh{\left(\ln{x}\right)} & \sinh{\left(\ln{x}\right)} \\
   \sinh{\left(\ln{x}\right)} & \cosh{\left(\ln{x}\right)} \\
  \end{array} } \right), x>0\\
  \left( {\begin{array}{cc}
   -\cosh{\left(\ln{\left(\dfrac{1}{-x}\right)}\right)} & \sinh{\left(\ln{\left(\dfrac{1}{-x}\right)}\right)} \\
   \sinh{\left(\ln{\left(\dfrac{1}{-x}\right)}\right)} & -\cosh{\left(\ln{\left(\dfrac{1}{-x}\right)}\right)} \\
  \end{array} } \right), x<0
  \end{cases}$$
is an isomorphism easily by checking 4 cases \par
\textbf{(b)} \par
Notice $\det:O(1,1)\rightarrow\mathbb{Z}/2\mathbb{Z}\cong\{\pm1,\text{multiplication}\}$ is a surjective group homomorphism with kernel $SO(1,1)$, hence $O(1,1)/SO(1,1)\cong\mathbb{Z}/2\mathbb{Z}$ \par
\textbf{(c)} \par
Notice $\phi$ from (a) is continuous, hence $O(1,1)^0=\left\{\left( {\begin{array}{cc}
   \cosh{x} & \sinh x \\
   \sinh{x} & \cosh{x} \\
  \end{array} } \right)\right\}$, and $\dfrac{x}{|x|}:\mathbb{R}^*\rightarrow\mathbb{Z}/2\mathbb{Z}\cong\{\pm1,\text{multiplication}\}$ is a surjective group homomorphism with kernel $\mathbb{R}_{>0}$, hence $SO(1,1)/O(1,1)^0\cong\mathbb{R}^*/\mathbb{R}_{>0}$, $\left|O(1,1)/O(1,1)^0\right|=\left[O(1,1):SO(1,1)\right]\left[SO(1,1):O(1,1)^0\right]=4$, hence it is isomorphic to either $\mathbb{Z}/4\mathbb{Z}$  or $\mathbb{Z}/2\mathbb{Z}\times \mathbb{Z}/2\mathbb{Z}$, but $-I,J,-J\in O(1,1)/O(1,1)^0$ are of order 2 and $(-I)J=-J,(-I)(-J)=J,J(-J)=-I$, thus the component group $O(1,1)/O(1,1)^0\cong\mathbb{Z}/2\mathbb{Z}\times \mathbb{Z}/2\mathbb{Z}$ \par

\begin{exercise}
\begin{enumerate}[label=(\alph*),leftmargin=*]

\end{enumerate}
\end{exercise}

\textbf{(a)} \par
Here $J\in GL(n,\mathbb{C})$, $G_J=\{g\in GL(n,\mathbb{C})|g^TJg=J\}=\{g\in GL(n,\mathbb{C})|g^T=Jg^{-1}J^{-1}\}$, $G_I=\{g\in GL(n,\mathbb{C})|g^Tg=I\}=\{g\in GL(n,\mathbb{C})|g^T=g^{-1}\}$, since $J$ is symmetric, $\exists P\in GL(n,\mathbb{C})$ such that $PJP^T=I$(corresponding to changing of basis), then we have $PG_JP^{-1}=G_I$ since $\forall g\in G_J$, we have $Jg^TJ^{-1}=g^{-1}$, then $(PgP^{-1})^{-1}=Pg^{-1}P^{-1}=PJg^TJ^{-1}P^{-1}=PJP^T(P^T)^{-1}g^TP^T(P^T)^{-1}J^{-1}P^{-1}=(PJP^T)(PgP^{-1})^T(PJP^T)^{-1}=(PgP^{-1})^T$, hence $PgP^{-1}\in G_I$, thus $G_J$ and $O(n,\mathbb{C})$ are conjugate in $GL(n,\mathbb{C})$ \par
\textbf{(b)} \par
Let $K=\left( {\begin{array}{cc}
    & 1 \\
   1 &  \\
  \end{array} } \right)$, then $\left( {\begin{array}{cc}
   a &  \\
    & b \\
  \end{array} } \right)\left( {\begin{array}{cc}
    & 1 \\
   1 &  \\
  \end{array} } \right)\left( {\begin{array}{cc}
   a &  \\
    & b \\
  \end{array} } \right)=\left( {\begin{array}{cc}
    & 1 \\
   1 &  \\
  \end{array} } \right)$ which gives $ab=1$, if $n$ is even, let $J=\left( {\begin{array}{ccc}
   K & & \\
    &\ddots & \\
     & & K\\
  \end{array} } \right)$, then the diagonal subgroup of $G_J$ is isomorphic to $\mathbb{C}^{*\frac{n}{2}}=\mathbb{C}^{*\floor{\frac{n}{2}}}$, if $n$ is odd, if $J=\left( {\begin{array}{cccc}
   K & & & \\
    &\ddots & & \\
     & & K &\\
     & &  &0\\
  \end{array} } \right)$, then the diagonal subgroup of $G_J$ is isomorphic to $\mathbb{C}^{*\floor{\frac{n}{2}}}$, if $J=\left( {\begin{array}{cccc}
   K & & & \\
    &\ddots & & \\
     & & K &\\
     & &  &1\\
  \end{array} } \right)$, then the diagonal subgroup of $G_J$ is isomorphic to $\mathbb{C}^{*\floor{\frac{n}{2}}}\times\mathbb{Z}/2\mathbb{Z}$ which contains a subgroup isomorphic to $\mathbb{C}^{*\floor{\frac{n}{2}}}$ \par
\textbf{(c)} \par
By (a) and (b), we know that $O(n,\mathbb{C})$ contains a subgroup isomorphic to $\mathbb{C}^{*\floor{\frac{n}{2}}}$ \par
\textbf{(d)} \par
If $A\in\rm{Lie}$, then $e^{tA}\in O(n,\mathbb{C})$, $e^{tA^T}=\left(e^{tA}\right)^T=\left(e^{tA}\right)^{-1}=e^{-tA}$, then $A^T=\dfrac{de^{tA^T}}{dt}\left.\right|_{t=0}=\dfrac{de^{-tA}}{dt}\left.\right|_{t=0}=-A$, hence $\rm{Lie}\left(O(n,\mathbb{C})\right)=\{A\in M_n(\mathbb{C})\|A^T=-A\}$, $\dim{\rm{Lie}\left(O(n,\mathbb{C})\right)}=\dfrac{n(n-1)}{2}$ \par

\begin{exercise}
\begin{enumerate}[label=(\alph*),leftmargin=*]

\end{enumerate}
\end{exercise}

\textbf{(a)} \par
$$
\begin{tikzcd}
1 \arrow[r] & {SL(n,\mathbb{C})} \arrow[r, hook] & {GL(n,\mathbb{C})} \arrow[r, "\det"] & \mathbb{C}^* \arrow[r] & 1
\end{tikzcd}
$$ is an exact sequence \par
\textbf{(b)} \par
$$
\begin{tikzcd}
1 \arrow[r] & \mathbb{C}^* \arrow[r, "a\mapsto aI", hook] & {GL(n,\mathbb{C})} \arrow[r, "q"] & {PGL(n,\mathbb{C})} \arrow[r] & 1
\end{tikzcd}
$$ is an exact sequence, where $q$ is the quotient map \par
\textbf{(c)} \par
$SL(n,\mathbb{C})\rightarrow GL(n,\mathbb{C})$ pass to $PSL(n,\mathbb{C})\rightarrow PGL(n,\mathbb{C})$ by quotient map, for $AZ\in PGL(n,\mathbb{C})$, $\dfrac{A}{\sqrt[n]{\det A}}Z\mapsto AZ$, hence the map is surjective, if $AZ\in PSL(n,\mathbb{C})$, and $A\in Z(PGL(n,\mathbb{C}))$, then $A=aI$, for some $a\neq0$, but since $A\in SL(n,\mathbb{C})$, $1=\det A$, $A\in Z(PSL(n,\mathbb{C}))$, hence the map is also injective \par
\textbf{(d)} \par
When $n$ is odd, the previous argument still works, namely $PSL(n,\mathbb{R})\cong PGL(n,\mathbb{R})$, when $n=2k$ is even, the injectivity still works, however, for $aI,A\in GL(n,\mathbb{R})$, $\det (aI\cdot A)=a^{2k}\det A$ has the same sign as $\det A$, thus there is an exact sequence 
$$
\begin{tikzcd}
1 \arrow[r] & {PSL(n,\mathbb{R})} \arrow[r, hook] & {PGL(n,\mathbb{R})} \arrow[r, "\det"] & \mathbb{Z}/2\mathbb{Z} \arrow[r] & 1
\end{tikzcd}
$$

\begin{exercise}
\begin{enumerate}[label=(\alph*),leftmargin=*]

\end{enumerate}
\end{exercise}

\textbf{(a)} \par
If $g=\left( {\begin{array}{cc}
   \alpha &\beta  \\
    \gamma & \xi \\
  \end{array} } \right)\in SU(2)$, then 
  $$
  I=gg^*=\left( {\begin{array}{cc}
   \alpha &\beta  \\
    \gamma & \xi \\
  \end{array} } \right)\left( {\begin{array}{cc}
   \bar\alpha &\bar\gamma  \\
    \bar\beta & \bar\xi \\
  \end{array} } \right)=\left( {\begin{array}{cc}
   |\alpha|^2+|\beta|^2 &\alpha\bar\gamma+\beta\bar\xi  \\
    \bar\alpha\gamma+\bar\beta\xi & |\gamma|^2+|\xi|^2 \\
  \end{array} } \right)
  $$
Hence $\begin{cases}
  |\alpha|^2+|\beta|^2=1 \\
  \alpha\bar\gamma+\beta\bar\xi=0 \\
  |\gamma|^2+|\xi|^2 \\
\end{cases}$, and also $1=\det g=\alpha\xi-\beta\gamma$ \par
If $\gamma=0$, then $|\alpha|=|\xi|=1$, $\beta=0$, $\alpha\xi=1\Rightarrow \xi=\bar\alpha$ \par
If $\gamma=0$, then $\alpha=-\dfrac{\beta\bar\xi}{\bar\gamma}$, $\beta=0$, then $1=-\dfrac{\beta|\xi|^2}{\bar\gamma}-\beta\gamma\Rightarrow-\bar\gamma=\beta$, hence $\alpha=\bar\xi$ \par
\textbf{(b)} \par
Since $S^3\subseteq \mathbb{C}^2=\{|\alpha|^2+|\beta|^2=1\}$ which is the same zero set, thus $SU(2)$ is topologically equivalent to $S^3$, and is therefore 3-dimensional, connected and simply connected \par
\textbf{(c)} \par
$$
\begin{tikzcd}
1 \arrow[r] & SU(2) \arrow[r, hook] & U(2) \arrow[r, "\det"] & S^1 \arrow[r] & 1
\end{tikzcd}
$$ is an exact sequence, there is a map $\mu:S^1\rightarrow U(2)$, $z\mapsto\left( {\begin{array}{cc}
   1 & 0 \\
   0 & z \\
  \end{array} } \right)$ which is a homomorphism such that $\det\circ\mu=1$, hence the exact sequence splits \par
%it doesn't split, if so, then $U(2)\cong SU(2)\times S^1$ which is not simply connected since $\pi_1(SU(2)\times S^1)\cong \pi_1(SU(2))\times\pi_(S^1)\cong\mathbb{Z}$ , on the other hand, $\forall U\in U(2)$, $\det U=e^{i\theta}$ for some $\theta$, then $U_t=e^{-\frac{it\theta}{2}}U$ is a deformation retraction from $U(2)$ onto $SU(2)$, hence $U(2)$ is simply-connected which is a contradiction\par
\textbf{(d)} \par
$$
\left( {\begin{array}{cc}
   \alpha &\beta  \\
    \gamma & \xi \\
  \end{array} } \right)+\left( {\begin{array}{cc}
   \bar\alpha &\bar\gamma  \\
    \bar\beta & \bar\xi \\
  \end{array} } \right)=\left( {\begin{array}{cc}
   \alpha+\bar\alpha &\beta+\bar\gamma  \\
    \gamma+\bar\beta & \xi+\bar\xi \\
  \end{array} } \right)=\left( {\begin{array}{cc}
   0 &0  \\
    0 & 0 \\
  \end{array} } \right)\Rightarrow \alpha+\bar\alpha=\xi+\bar\xi=\gamma+\bar\beta=0
$$
and $0=trg=\alpha+\xi$, thus $\xi=-\alpha\in i\mathbb{R}$ and $\gamma=-\bar\beta$, hence $W$ is a 3-dimensional real vector space of the form
$
\left\{\left( {\begin{array}{cc}
   ix &y+iz \\
  -y+iz & -ix \\
  \end{array} } \right)\left.\right|x,y,z\in\mathbb{R}\right\}
$
and $tr\left(\left( {\begin{array}{cc}
   ix_1 &y_1+iz_1 \\
  -y_1+iz_1 & -ix_1 \\
  \end{array} } \right)\left( {\begin{array}{cc}
   ix_2 &y_2+iz_2\\
  -y_2+iz_2 & -ix_2 \\
  \end{array} } \right)\right)=-2(x_1x_2+y_1y_2+z_1z_2)$ is a definite symmetric bilinear form \par

\textbf{(e)} \par
$\forall g\in SU(2)$, $g^*=g^{-1}$, $X,Y\in W$, $gXg^{-1}+(gXg^{-1})^*=gXg^{-1}+gX*g^*=g(X+X^*)g^*=0$, $tr(gXg^{-1})=tr(X)=0$, $(gXg^{-1},gYg^{-1})=tr(gXg^{-1}gYg^{-1})=tr(gXYg^{-1})=tr(XY)=(X,Y)$, hence $SU(2)$ acts on $W$ by conjugation, preserving the form \par
\textbf{(f)} \par
By calculation in (d), $\langle X,Y\rangle:=\frac{1}{2}tr(XY)$ will be the standard inner product in $\mathbb{R}^3$, and $SU(2)$ still acts on $W$ by conjugation, preserving the form, this can be regarded as a continuous homomorphism $\varphi:SU(2)\rightarrow O(3)$, since $SU(2)$ is connected by (b), hence $\varphi$ is actually a continuous homomorphism from $SU(2)$ to $SO(3)$, now compute its kernel
$$
\left( {\begin{array}{cc}
   ix &0  \\
    0 & -ix \\
  \end{array} } \right)=\left( {\begin{array}{cc}
   \alpha &\beta  \\
    -\bar\beta & \bar\alpha \\
  \end{array} } \right)\left( {\begin{array}{cc}
   ix &0  \\
    0 & -ix \\
  \end{array} } \right)\left( {\begin{array}{cc}
   \bar\alpha & -\beta  \\
    \bar\beta & \alpha \\
  \end{array} } \right)=ix\left( {\begin{array}{cc}
   |\alpha|^2-|\beta|^2 &-2\alpha\beta  \\
    -2\bar\alpha\bar\beta & |\beta|^2-|\alpha|^2 \\
  \end{array} } \right)\Rightarrow \begin{cases}
  \beta=0 \\
  |\alpha|^2=1
  \end{cases}
$$
and 
$$
\left( {\begin{array}{cc}
    & \gamma  \\
    -\bar\gamma &  \\
  \end{array} } \right)=\left( {\begin{array}{cc}
   \alpha &  \\
     & \bar\alpha \\
  \end{array} } \right)\left( {\begin{array}{cc}
    & \gamma  \\
    -\bar\gamma &  \\
  \end{array} } \right)\left( {\begin{array}{cc}
   \bar\alpha &   \\
     & \alpha \\
  \end{array} } \right)=\left( {\begin{array}{cc}
    & \alpha^2\gamma  \\
    -\bar\alpha^2\bar\gamma &  \\
  \end{array} } \right)\Rightarrow \alpha^2=1\Rightarrow \alpha=\pm1
$$
Thus we have the exact sequence
\begin{center}
\begin{tikzcd}
1 \arrow[r] & \mathbb{Z}/2\mathbb{Z} \arrow[r] & SU(2) \arrow[r, "\varphi"] & SO(3) \arrow[r] & 1
\end{tikzcd}
\end{center}

\begin{exercise}
Let $G=\GL(2,\mathbb F_q)$
\begin{enumerate}[label=(\alph*),leftmargin=*]
\item Show that $G$ acts transitively on $X=\mathbb F^2_2-\{(0,0)\}$
\item Show that the stabilizer of a point has order $q(q-1)$
\item Show there is a bijection $G/H\leftrightarrow X$
\item Show that $|\GL(2,\mathbb F_q)|=(q+1)q(q-1)^2$
\item What is the order of $\PGL(2,\mathbb F_q)|$? How about $\PGL(2,\mathbb F_5)$
\item \textit{Extra credit}: Show that $\PGL(2,\mathbb F_q)\cong A_5$
\end{enumerate}
\end{exercise}

\begin{solution}
\begin{enumerate}[label=(\alph*),leftmargin=*]
\item $\forall (x,y)\in X$, if $x=0$, then $\left( {\begin{array}{cc}
   0 & 1 \\
   y & 0 \\
  \end{array} } \right)\left( {\begin{array}{c}
   1 \\
   0 \\
  \end{array} } \right)=\left( {\begin{array}{c}
   0 \\
   y \\
  \end{array} } \right)$, if $x\neq0$, then $\left( {\begin{array}{cc}
   x & 0 \\
   y & 1 \\
  \end{array} } \right)\left( {\begin{array}{c}
   1 \\
   0 \\
  \end{array} } \right)=\left( {\begin{array}{c}
   x \\
   y \\
  \end{array} } \right)$, hence $\left( {\begin{array}{c}
   1 \\
   0 \\
  \end{array} } \right)$ can move to any point in $X$ by an action of $G$, hence $G$ acts transitively on $X$ 
\item If $\left( {\begin{array}{c}
   1 \\
   0 \\
  \end{array} } \right)=\left( {\begin{array}{cc}
   a & b \\
   c & d \\
  \end{array} } \right)\left( {\begin{array}{c}
   1 \\
   0 \\
  \end{array} } \right)=\left( {\begin{array}{c}
   a \\
   c \\
  \end{array} } \right)\Rightarrow\begin{cases}
  a=1 \\
  c=0
  \end{cases}$, and since $0\neq\det g=ad-bc=d$, hence $|H|=q(q-1)$ 
\item Consider $\varphi:G/H\rightarrow X$, $gH\mapsto g\left( {\begin{array}{c}
   1 \\
   0 \\
  \end{array} } \right)$ which is well-defined, and surjectivity follows from (a), injectivity follows from the fact if $g\left( {\begin{array}{c}
   1 \\
   0 \\
  \end{array} } \right)=g'\left( {\begin{array}{c}
   1 \\
   0 \\
  \end{array} } \right)$, then $g^{-1}g'\left( {\begin{array}{c}
   1 \\
   0 \\
  \end{array} } \right)=\left( {\begin{array}{c}
   1 \\
   0 \\
  \end{array} } \right)\Rightarrow g^{-1}g'\in H$, hence $\varphi$ is bijective 
\item Because of (c), $|G|=|G/H||H|=|X||H|=(q^2-1)q(q-1)=(q+1)q(q-1)^2$ 
\item Since $Z=\{aI|a\neq0\}$, $|Z|=q-1$, $|PGL(2,\mathbb{F}_q)|=|G|/|Z|=(q+1)q(q-1)$, in particular, $|PGL(2,\mathbb{F}_5)|=120$ 
\item From group action $GL(2,\mathbb{F}_4)$ on $\mathbb{F}_4^2\setminus\{(0,0)\}$, we can pass to a group action $PGL(2,\mathbb{F}_4)$ on $P\mathbb{F}_4$, this action is faithful since for $g=\left( {\begin{array}{cc}
   a & b \\
   c & d \\
  \end{array} } \right)$, if $c\neq0$, $g\left( {\begin{array}{c}
   1 \\
   0 \\
  \end{array} } \right)=\left( {\begin{array}{c}
   a \\
   c \\
  \end{array} } \right)\neq\left( {\begin{array}{c}
   1 \\
   0 \\
  \end{array} } \right)$ in $P\mathbb{F}_4$, if $b\neq0$,  $g\left( {\begin{array}{c}
   0 \\
   1 \\
  \end{array} } \right)=\left( {\begin{array}{c}
   b \\
   d \\
  \end{array} } \right)\neq\left( {\begin{array}{c}
   0 \\
   1 \\
  \end{array} } \right)$ in $P\mathbb{F}_4$, hence $b=c=0$, if $a\neq d$,  $g\left( {\begin{array}{c}
   1 \\
   1 \\
  \end{array} } \right)=\left( {\begin{array}{c}
   a \\
   d \\
  \end{array} } \right)\neq\left( {\begin{array}{c}
   1 \\
   1 \\
  \end{array} } \right)$ in $P\mathbb{F}_4$, hence $g\in Z$, thus we get a injective group $\begin{tikzcd}
{PGL(2,\mathbb{F}_4)} \arrow[r, hook] & S_5
\end{tikzcd}$ since $|P\mathbb{F}_4|=5$, but $|PGL(2,\mathbb{F}_4)|=60$, $|S_5|=120$, any subgroup of $S_5$ of order $60$ should be normal, intersection of normal subgroups is again a normal subgroup, but $A_5$ is a simple group, hence $A_5$ is the only subgroup of $S_5$ of order $60$, thus $PGL(2,\mathbb{F}_4)\cong A_5$ \par
Note that $G = \PGL(2, \mathbb F_5)$ has exactly 5 5-Sylow subgroups, and it acts by conjugation on this set of 5 elements. The resulting map $G \to S_5$ is injective, and has image $A_5$. Also $\PGL(2, \mathbb F_5)$ as on the projective plane over $\mathbb F_5$, which has 6 elements. This gives a homomorphism $G \to S_6$. Now $S_6$ has an outer automorphism (unique among symmetric groups), which takes an “obvious” $S_5$ subgroup (fixing one point of the 6 element set) to a subgroup not fixing any elements of the set. The even permutations in the image give $A_5$
\end{enumerate}
\end{solution}

\begin{exercise}
\begin{enumerate}[label=(\alph*),leftmargin=*]

\end{enumerate}
\end{exercise}

\begin{solution}
\begin{enumerate}[label=(\alph*),leftmargin=*]
\item By Jordan normal form theorem, $\forall g\in GL(n,\mathbb{C})$, $g$ can be decomposed uniquely as $SU=US$ where $S$ is semisimple and $U$ is unipotent, thus $\displaystyle\log U=\sum_{m=1}^{\infty}\frac{(-1)^{m+1}(U-I)^m}{m}$ is actually a finite sum, so is $e^{\log U}$, now consider $U(t)=I+t(U-I)$, when $t$ is small, $e^{\log U(t)}=U(t)$, and since $e^{\log U(t)}-U(t)$ is a matrix with entries polynomials in $t$, $e^{\log U(t)}=U(t), \forall t$, in particular, $e^{\log U}=U$, the general case follows by dividing $g$ into Jordan blocks 
\item The image of then exponential map from  $M_n(\mathbb{R})$ to $GL(n,\mathbb{R})$ is $\{A\in GL(n,\mathbb{R})|\det A|U>0, U \text{ is any A invariant space}\}$, notice that $\det e^{A|U}=e^{trA|U}>0$, $R_\theta=\left( {\begin{array}{cc}
   \cos\theta & \sin\theta \\
   -\sin\theta & \cos\theta \\
  \end{array} } \right)=\exp\left( {\begin{array}{cc}
    & \theta \\
   -\theta &  \\
  \end{array} } \right)$, for any $A\in GL(n,\mathbb{R})$ with $\det A>0$, $A$ can be written in real Jordan canonical form, and then a Jordan block can be written as $SU=US$, $U$ is unipotent, and $S$ is diagonalizable with real numbers or $R_\theta$'s, for negative diagonal numbers we can pair them up and notice $R_\pi=\left( {\begin{array}{cc}
   -1 &  \\
    & -1 \\
  \end{array} } \right)=\exp\left( {\begin{array}{cc}
    & \pi \\
   -\pi &  \\
  \end{array} } \right)$, thus $A$ is in the image of the exponential map
\end{enumerate}
\end{solution}

\end{document}