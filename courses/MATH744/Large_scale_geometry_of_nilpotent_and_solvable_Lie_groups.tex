\documentclass[main]{subfiles}

\begin{document}

Large scale is often captured by the notion of a quasi-isometry

\begin{definition}
A $(K,C)$ quasi isometry(QI) is $f:(X,d_X)\to(Y,d_Y)$ such that
\begin{enumerate}
\item $-C+\frac{1}{K}d_X(x_1,x_2)\leq d_Y(f(x_1),f(x_2))\leq Kd_X(x_1,x_2)+C$
\item $\forall y\in Y,\exists x\in X$ such that $d_Y(y,f(x))\leq C$
\end{enumerate}
\end{definition}

\begin{exercise}\hfill\\
\begin{enumerate}
\item Any map that is bounded distance (the difference with sup norm being finite) from a quasi-isometry is a quasi-isometry
\item The composition of quasi-isometries is a quasi-isometry
\item Every quasi-isometry has a coarse inverse (i.e if $f ∶ X \to Y$ is a quasi-isometry then there is $\bar f ∶ Y \to X$ such that $f\circ\bar f$ and $\bar f\circ f$ are bounded distance from the identity map)
\item Being quasi-isometric is an equivalence relation
\end{enumerate}
\end{exercise}

\begin{solution}
For any $y$, pick some $x\in X$ to be $\bar f(y)$ such that $d_Y(y,f\bar f(y))=d_Y(y,f(x))\leq C$
\begin{enumerate}
\item $d_X(\bar f(y_1),\bar f(y_2))\leq Kd_Y(f\bar f(y_1),f\bar f(y_2))+KC\leq Kd_Y(y_1,y_2)+3KC$. $d_Y(y_1,y_2)\leq d_Y(f\bar f(y_1),f\bar f(y_2))+d_Y(y_1,f\bar f(y_1))+d_Y(f\bar f(y_2),y_2)\leq d_Y(f\bar f(y_1),f\bar f(y_2))+2C\leq Kd_X(\bar f(y_1),\bar f(y_2))+3C$
\item $\forall x\in X$, $d_X(x,\bar ff(x))\leq Kd_Y(f(x),f\bar ff(x))+KC\leq 2KC$
\end{enumerate}
\end{solution}

\begin{note} \hfill \\
\begin{enumerate}
\item If $C=0$, then $f$ is biLipschitz
\item If $C=0,K=1$, then $f$ is just an isometry
\end{enumerate}
\end{note}

\begin{example}
$G$ is a finitely generated group, $S,S'$ are finite generating sets, then $(G,d_S)\overset{\text{QI}}{\simeq}(G,d_{S'})$, here $d_S$ is the word metric
\end{example}

\begin{exercise}
Any word metric on a finitely generated group is biLipschitz (hence quasi-isometric) to any other word metric on that group
\end{exercise}

\begin{exercise}
Any finite index subgroup of a finitely generated group is quasi-isometric to its supergroup
\end{exercise}

\begin{solution}
Say $[G:H]<\infty$, consider the left cosets $\{gH\}$, pick representatives $g_1H,\cdots,g_nH$
\end{solution}

\begin{example}[Fundamental lemma of geometric group theory]
If discrete group $G$ acts proper discontinuously and cocompactly on $(X,d_X)$ which is a proper geodesic metric space. Then $G$ with any word metric is QI to $X$, $X$ is called a model space for $G$
\end{example}

\begin{question}
\begin{itemize}
\item Which spaces/finitely generated groupos are QI to each other
\item Examples of (self) QI's
\item What properties are preserved by QI's
\end{itemize}
\end{question}

\begin{example}
$\mathbb R$ with euclidean metric is QI to $\mathbb R$ with $\max$ metric
\end{example}

\begin{example}
$f:\mathbb R\to\mathbb R$ is QI, then it preserves two "ends"
\end{example}

\begin{exercise}
Verify that $f ∶\mathbb R^2 \to\mathbb R^2$ defined by $f(r, \theta) = (r, \ln(1 + \theta))$ is a quasi-isometry
\end{exercise}

Focus on finitely generated nilpotent and solvable groups that have (connected and simply connected) Lie groups as model spaces

\begin{exercise}
Any Riemannian metric on $G$ give QI metric space
\end{exercise}

\begin{definition}
$\Gamma\leq G$ discrete is a uniform lattice if $G/\Gamma$ is compact
\end{definition}

\begin{exercise}
If $\Gamma$ is a uniform lattice in $G$, then $G$ is a model space for $\Gamma$
\end{exercise}

\begin{warning}
Not all Lie groups have uniform lattices!
\end{warning}

\begin{exercise}[Hard]
$\mathbb R\rtimes\mathbb R$ where $\mathbb R$ acts on $\mathbb R$ by $x\mapsto e^tx$
\begin{itemize}
\item Verify $\mathbb R \rtimes\mathbb R$ is solvable
\item Show that $\mathbb R \rtimes\mathbb R$ does not admit any uniform lattices
\item Find a metric on $\mathbb R \rtimes\mathbb R$ that makes it isometric to $\mathbb H^2$
\end{itemize}
\end{exercise}

\begin{fact}
\begin{itemize}
\item Any finitely generated nilpotent group is (virtually, i.e. has finite index subgroup) a lattice in a nilpotent Lie group
\item Any finitely generated polycyclic group is (virtually) a lattice in a solvable Lie group. For example, $BS(1,n)=\langle a,b|bab^{-1}=a^n\rangle$ is solvable but not polycyclic
\end{itemize}
\end{fact}

\begin{exercise}[Heisenberg group]
$H(\mathbb R)\begin{bmatrix}
1&x&z\\
&1&y\\
&&1
\end{bmatrix}$ with $x,y,z\in\mathbb R$, $H(\mathbb Z)$ is a uniform lattice and both are nilpotent 
\end{exercise}

\begin{example}[Sol]
$\mathbb R^2\rtimes\mathbb R$, $\mathbb R$ acts on $\mathbb R^2$ by $(x,y)\mapsto(e^tx,e^{-t}x)$ is a solvable group with lattice $\mathbb Z^2\rtimes\mathbb Z$, $\begin{bmatrix}
x\\
y
\end{bmatrix}\mapsto\begin{bmatrix}
2&1\\
1&1
\end{bmatrix}^n\begin{bmatrix}
x\\
y
\end{bmatrix}$
\end{example}

QI basics:

\begin{theorem}[Gromov]
Any finitely generated group $G$ that is QI to a nilpotent group is (virtually) nilpotent
\end{theorem}

\begin{note}[Ershler]
Exists finitely generated group $G$ that are QI to solvable groups that not virtually solvable
\end{note}

Open question: What about polycyclic groups?

\begin{theorem}[Eskin-Fisher-Whyte]
If $G$ is a finitely generated group QI to Sol then it is (virtually) a uniform lattice in Sol
\end{theorem}

\begin{fact}
$\exp:\mathscr N\to N$ is a diffeomorphism, assuming $N$ is connected and simply connected
\end{fact}

\begin{example}[Heisenberg group]
The Lie algebra has basis $\{e_1,e_2,e_3\}$, and $[e_1,e_2]=e_3$ is the only non-trivial bracket
\end{example}

\begin{exercise}
Show this is the Heisenberg group
\end{exercise}

Write $\mathscr N_i=[\mathscr N,\mathscr N_{i-1}]$, $\mathscr N=V_1\oplus V_2\oplus\cdots\oplus V_r$, where $V_i$ is a complement of $\mathscr N_i$ in $\mathscr N_{i-1}$, i.e. $V_i\cong\mathscr N_{i-1}/\mathscr N_{i}$, let $\pi_i:\mathscr N\to V_i$ be projection

\begin{theorem}[Ball-Box]
There exists $V_i$ and a left invariant Riemannian metric on $\mathscr N$ such that $V_i\perp V_j$ for all $i\neq j$ and the induced distance satisfies: There exists $a>1$ such that $\forall R\geq1$, $Box(\frac{R}{a})\subseteq B_1(R)\subseteq Box(aR)$, where $Box(R)=\{n\in N||\pi_i(n)|\leq R_i\}$
\end{theorem}

\begin{note}
$|n|_h=\sum_{i}|\pi_i(n)|^{\frac{1}{i}}$, $\frac{1}{d}d(0,n)\leq|n|_h\leq ard(0,n)$
\end{note}

Warning: $d_h(m,n)=|m^{-1}n|_h$ doesn't give a distance

QI of $N$

Affine maps $\mathbb R^n\rtimes\GL(n,\mathbb R)$ are QI's, generalize this, affine maps $N\rtimes\Aut(N)$ are QI's

\begin{theorem}[Xie]
Let $S_g:N\to N$, $S_g(n)=n+g(\pi_1(n))$ for some $g:V_1\to V_r$, then $S_g$ is QI iff $|g(x)-g(y)|\leq K|x-y|+C$. $S_g$ is bounded distance from an affine map iff $g$ is bounded distance from a linear map
\end{theorem}

\begin{proof}
If $|g(x)-g(y)|\leq K|x-y|+C$, then $S_g$ is a QI \\

Pretend $d_h$ is a distance
\begin{align*}
d_h(S_g(n_1),S_g(n_2))&=|(-n_2-g(\pi_1(n_2)))\star(n_1+g(\pi_1(n_2)))|_h \\
&\leq|(-n_2)| \\
&\leq d_h(n_1,n_2)+d_h(g(\pi_1(n_1))+g(\pi_1(n_2))) \\
&\leq d_h(n_1,n_2)+|Kd(n_1,n_2)+C|^{\frac{1}{r}}
\end{align*}
\end{proof}

See Xie for other examples of "exotic" QI's of Heisenberg group

QI between nilpotent Lie groups

\begin{definition}
The associated graded Lie group $\gr(N)$ as vector space is just $N$, but with bracket $[x,y]_\infty=\pi_{i+j}([x,y])$, for $x\in V_i,y\in V_j$, so that $[V_i,V_j]_\infty\subseteq V_{i+j}$
\end{definition}

\begin{theorem}[Pansu]
If $N_1,N_2$ are QI nilpotent Lie groups, then $\gr(N_1)\cong\gr(N_2)$
\end{theorem}

\begin{proof}
Asymptotic cone of $(N,d)$ is $(\gr(N),d_{cc})$, $(N,\frac{1}{t}d)\xrightarrow{t\to\infty}(\gr(N),d_{cc})$. And a QI indeces a biLipschitz map of asymptotic cones. Note that even if $N=\gr(N)$, $d_{cc}$ is not $d$, but rather Carnot-Cartheodory metric(a sub-Riemannian metric)
\[d_{cc}(p,q)=\inf_{\text{join }p\text{ to }q\text{, the tangent lines in translates of }V_1}\text{Length}(\gamma)\]
\end{proof}

Shalom,Sauer: cohomological invariants can be used to distinguish certain nilpotent Lie groups up to QI with the same associated gradded Lie groups

Isenrihc-Pallier-Tessera: Dehn functions of certain lattices can distinguish some nilpotent Lie groups up to QI

Open question: Which nilpotent Lie groups are QI. Smallest open case is actually 5

QI and geometry of solvable Lie gorups
\[1\to N\to S\to N'\to1\]
Let $N'=\mathbb R$, i.e. $S\cong N\rtimes\mathbb R$

\begin{example}
$\mathbb R\rtimes\mathbb R$, $ds^2=e^{-2t}dx^2+dt^2$ by letting $y=e^t$, this is the log model of $\mathbb H^2$. This is negatively curved

Sol$=\mathbb R^2\rtimes\mathbb R$, $d^2=e^{-2t}dx^2+e^{2t}dy^2+dt^2$, note that $(x,y_0,t)\overset{\Isom}{\cong}\mathbb H^2$, $(x_0,y,t)\overset{\Isom}{\cong}\mathbb H^2$. This is foliated by negatively curved Lie groups
\end{example}

\begin{question}
Which $N\times\mathbb R$ can be made negatively curved? Which $N\rtimes\mathbb R$ are foliated by negatively curved? Are there others? What do QI's look like? Which are QI's
\end{question}

Let $A:\mathscr N\to\mathscr N$ be a derivation, then $e^{tA}:N\to N$ is a one parameter family of automorphisms. $G_A=N\rtimes \mathbb R$ where action is $e^{tA}$

\begin{definition}
If the real part of eigenvalues of $A$ is positive, then $e^{tA}$ is a one parameter family of dilations(this is not a standard definition, a little more broad)
\end{definition}

We say $G_A$ is Heintze group if $e^{tA}$ is one parameter family of dilations. We say $G_A$ is a purely real if eigenvalues are real

\begin{theorem}[Heintze]
All negatively curved homogeneous spaces are Heintze groups, and all Heintze groups are negative curved homogeneous spaces
\end{theorem}

\begin{exercise}
Every Heintze group is QI to a purely real one. If $rA,sB$ have same real Jordan form, then $G_A\overset{QI}{\simeq}G_B$
\end{exercise}

Special Heintze groups:

$G_{I_n}=\mathbb R^n\rtimes\mathbb R\overset{\Isom}{\cong}\mathbb H^{n+1}$

$G_A=H_n\rtimes\mathbb R\overset{\Isom}{\cong}\mathbb{CH}^n$, $A=\begin{bmatrix}
1\\
&\ddots \\
&&1\\
&&&2
\end{bmatrix}$, $H_2=H$ is the Heisenberg group

There are $N$ such that $N\rtimes\mathbb R\overset{\Isom}{\cong}$ for any rank one symmetric space

Negative curvature give $\delta$ hyperbolic

Morse lemma: QI map geodesics to within a bounded distance of geodesics(use $\delta$ hyperbolicity)

Boundaries of $G_A$

For any $n\in N$, let $\gamma_n(s)=(n,s)$, this is a "vertical" geodesic choosing suitable metric

\[d_{G_A}(\gamma_n(s),\gamma_{n'}(s))=\begin{cases}
0&\text{as }s\to\infty \\
\infty&\text{as }s\to-\infty
\end{cases}\]
$\partial G_A=N\cup\{\infty\}\cong S^n$

\begin{fact}
QI's of $G_A$ induce homeomorphisms of the boundary(use Morse lemma)
\end{fact}

\begin{conjecture}[Pointed sphere - de Cornulier]
Unless $G_A\overset{\Isom}{\cong}$ rank 1 symmetric space, then any QI of $G_A$ preserves $\infty$
\end{conjecture}

\begin{remark}
True in many cases(Carraso, Xie)
\end{remark}

We will assume all QI's fix $\infty$

Parabolic visual boundary metric: $\partial_\infty G_A=N$. $d_A(n,n')=e^{t_0}$ is not always a metric, but it is a quasi-metric, i.e. $d_A(n_1,n_2)\leq M(d_A(n_1,n_3)+d_A(n_3,n_2))$

$(\mathbb R,d_{I_1})\simeq(\mathbb R,d_{Euc})$

$(\mathbb R^2,d_{\begin{bmatrix}
1\\
&2
\end{bmatrix}})\overset{biLipschitz}{\simeq}(\mathbb R),\max\{|\Delta x|,|\Delta y|^{\frac{1}{2}}\}$

\begin{exercise}
If two QI's (that fix $\infty$) are bounded distance apart, then they induce the same map on $(N,d_A)$
\end{exercise}

\begin{definition}
A QI of $G_A\cong N\rtimes\mathbb R$ is coarsely height respecting if it preserves level sets of $\mathbb R$ up to bounded distance, i.e. $f(N\times\{t_0\})$ is contained in a bounded neighborhood of $N\times\{t_0\}$
\end{definition}

\begin{conjecture}
All QI's of non-special Heintze groups are height respecting
\end{conjecture}

\begin{exercise}
Height respecting QI of $G_A$ induce biLipschitz maps of $(N,d_A)$
\end{exercise}

Conclusion: $QI(G_A)\simeq biLip(N,d_A)$

Sol-like group:
\[G_{A_1,A_2}=(N_1\times N_2)\rtimes\mathbb R\]
$\mathbb R$ acts on $N_1,N_2$ by $(n_1,n_2)\mapsto(e^{tA_1}n_1,e^{-tA_2}n_2)$, i.e. $G_{A_1},G_{A_2}$ are both Heintze groups

(EFW) Any QI of $G_{A_1,A_2}$ is up to isometry bounded distance from $(n_1,n_2,t)\to(f_1(n_1),f_2(n_2),t)$, $f_i\in biLip(N_i,d_{A_i})$

Other example: $H\rtimes\mathbb R=G_A$, $A=\begin{bmatrix}
1\\
&-1\\
&&&
\end{bmatrix}$, $[e^tx,e^{-t}y]=z$. Not much is known(open)

\begin{conjecture}[Eskin-Fisher-Whyte]
Let $E(S)$ be the group generated by exponentially distorted elements(exponential radical), then any QI coarsely preserves cosets of $E(S)$
\end{conjecture}

\begin{note}
In our examples $E(G_A)=N$
\end{note}

Other work:

Peng $\mathbb R^n\rtimes\mathbb R^m$

Healy-Pengitore $N\rtimes\mathbb R^n$ CAT0

Tullia Dymarz understand subgroups of $QI(G_A)$, used to show $G$ f.g., QI to $G_A$, then $G$ is virtually a lattice of some $G_{A'}$(QI rigidity)



\begin{exercise}[Hard]
 Any group that acts cocompactly and properly discontinously on a proper geodesic metric space X is finitely generated and quasi-isometric to that space
\end{exercise}

\begin{exercise}
Show that any two left invariant Riemannian metrics on a Lie group give metric spaces that are quasi-isometric
\end{exercise}

\begin{exercise}
A discrete subgroup of a Lie group that is cocompact (i.e $G/H$ compact ) is quasiisometric the Lie group itself (and is called a uniform lattice)
\end{exercise}

\begin{exercise}
There are coarsely dense subsets of $\mathbb Z^2$ that are not biLipschitz equivalent to $\mathbb Z^2$ (Burago-Kleiner)
\end{exercise}

\begin{exercise}
Are there uniform lattices in a nilpotent Lie group that are not biLipschitz equivalent?
\end{exercise}

\end{document}