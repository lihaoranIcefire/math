\documentclass{beamer}
\setbeamercovered{}

\usepackage{bookman}
\usepackage{newtxtext}
\usepackage{amssymb}
\usepackage{amsmath}
\usepackage{amsfonts}
\usepackage{mathrsfs}
\usepackage{mathtools}
\usepackage{hyperref}
\usepackage{pifont}
\usepackage{systeme}
\usepackage{subcaption}
\usepackage{bm} % Bold
\usepackage{bbm} % Can use \mathbbm{1}
\usepackage{ifthen} % Use if then else
\usepackage{shuffle} % Use shuffle product
\usepackage{minted} % For listing codes
\usepackage{tikz}
\usepackage{pgfplots}

\mode<presentation> {
    \usetheme{Madrid}
    \usecolortheme{beaver}
    \usefonttheme{professionalfonts} 
    \setbeamertemplate{navigation symbols}{}
    \setbeamertemplate{caption}[numbered]
}

\theoremstyle{definition}
\newtheorem{proposition}[theorem]{Proposition}
\newtheorem{construction}[theorem]{Construction}
\newtheorem{deduction}[theorem]{Deduction}
\newtheorem{exercise}[theorem]{Exercise}
\newtheorem{conjecture}[theorem]{Conjecture}

\theoremstyle{remark}
\newtheorem*{remark}{Remark}
\newtheorem*{recall}{Recall}
\newtheorem*{question}{Question}
\newtheorem*{answer}{Answer}

\newcounter{saveenumi}
\newcommand{\seti}{\setcounter{saveenumi}{\value{enumi}}}
\newcommand{\conti}{\setcounter{enumi}{\value{saveenumi}}}
\resetcounteronoverlays{saveenumi}

\AtBeginSection[]{
    \begin{frame}
    \centering
    \begin{beamercolorbox}[center, shadow=true, rounded=true]{title}
    \usebeamerfont{title}\insertsectionhead\par%
    \end{beamercolorbox}
    \end{frame}
}

\title{MATH240 - Introduction to Linear Algebra}
\author{Haoran Li}
\institute[UMD]{University of Maryland, College Park}
\date{Summer 2024}


\begin{document}

\maketitle

\section{Lecture 1 - System of linear equations}


\begin{frame}{Some Notations}\pause
\begin{itemize}
\item \textcolor{blue}{Natural numbers}: $\mathbb N=\{0,1,2,3,\cdots\}$.
\pause
\item \textcolor{blue}{Integers}: $\mathbb Z=\{\cdots,-2,-1,0,1,2,\cdots\}$.
\pause
\item \textcolor{blue}{Rational numbers}: $\mathbb Q=\left\{\frac{m}{n}\middle|m,n\in\mathbb Z,n\neq0\right\}$ is the set of fractions. Here $\in$ means \textcolor{blue}{belong to}.
\pause
\item \textcolor{blue}{Real numbers:} $\mathbb R$ is the set of numbers on the whole real number line. \pause It includes
\begin{itemize}
\item irrational numbers (like $\sqrt{2},\sqrt[3]{3}$) 
\item transcendental numbers (like $\pi,e$)
\end{itemize}
\pause
\item \textcolor{blue}{Complex numbers}: $\mathbb C=\left\{a+bi\middle|a,b\in\mathbb R\right\}$, $i=\sqrt{-1}$ is the imaginary number such that $i^2=-1$.
\pause
\item $\mathbb N\subsetneqq\mathbb Z\subsetneqq\mathbb Q\subsetneqq\mathbb R\subsetneqq\mathbb C$.
\pause
\item $\mathbb R^n=\{(r_1,r_2,\cdots,r_n)|r_1,r_2,\cdots,r_n\in\mathbb R\}$ is the set of all $n$-tuples of real numbers. \pause Geometrically, it is the $n$-dimensional Euclidean space. For example
\begin{itemize}
\item $\mathbb R^1=\mathbb R$ is a line
\item $\mathbb R^2$ is a plane
\item $\mathbb R^3$ is our usual three dimensional space
\end{itemize}
\end{itemize}
\end{frame}

\begin{frame}[t]
\begin{definition}
A \textcolor{blue}{linear equation}\index{linear equation} in the variables $x_1,x_2,x_3,\cdots,x_n$ is an equation that can be written in the form 
\begin{equation}\label{14:02-05/31/2022}
a_1x_1+a_2x_2+a_3x_3+\cdots+a_nx_n=b
\end{equation}
where the coefficients $a_1,a_2,a_3,\cdots,a_n$ and $b$ are real or complex numbers, usually known in advance.
\end{definition}
\pause
\begin{example}[Examples and non-examples of linear equations]\hfill
\begin{itemize}
\item $x_1+\frac{1}{2}x_2=2$ \pause\ding{51}
\item $\pi(x_1+x_2)-9.9x_3=e$ \pause\ding{51}. Because if we expand it, we got $\pi x_1+\pi x_2-9.9x_3=e$ in which case $a_1=\pi,a_2=\pi,a_3=-9.9,b=e$ as in the form of \eqref{14:02-05/31/2022}
\item $|x_2|-1=0$ \pause\ding{55}. Because this involves absolute values.
\item $x_1+x_2^2=9$ \pause\ding{55}. Because this is of degree two (quadratic equation).
\item $\sqrt{x_1}+\sqrt{x_2}=1$ \pause\ding{55}. Because this is of degree one half.
\end{itemize}
\end{example}
\end{frame}

\begin{frame}[t]
\begin{definition}
A \textcolor{blue}{system of linear equations} (or a \textcolor{blue}{linear system})\index{linear system} is a collection of one or more linear equations involving the same variables, say $x_1,x_2,x_3,\cdots,x_n$.
\begin{equation}\label{eq: general linear system}
\begin{cases}
\,\quad a_{11}x_1+a_{12}x_2+a_{13}x_3+\cdots+a_{1n}x_{n}=b_1 \\
\,\quad a_{21}x_1+a_{22}x_2+a_{23}x_3+\cdots+a_{2n}x_{n}=b_2 \\
\quad \,a_{31}x_1+a_{32}x_2+a_{33}x_3+\cdots+a_{3n}x_{n}=b_3 \\
\,\,\,\qquad\qquad\qquad\qquad\qquad\qquad\qquad\qquad\quad\vdots \\
a_{m1}x_1+a_{m2}x_2+a_{m3}x_3+\cdots+a_{mn}x_{n}=b_m
\end{cases}
\end{equation}
\end{definition}
\pause
\begin{example}
For $n=m=2$, \eqref{eq: general linear system} is just
\begin{equation}\label{eq: general 2 by 2 linear system}
\begin{cases}
a_{11}x_1+a_{12}x_2=b_1\\
a_{21}x_1+a_{22}x_2=b_2
\end{cases}
\end{equation}
\end{example}
\end{frame}

\begin{frame}[t]
\begin{example}\label{13:09-05/31/2022}
In a cage full of chickens and rabbits. The total number of heads is 10 and the total number of legs is 26. Calculate the number of chickens and rabbits.
\end{example}\pause
\begin{solution}
Let's assume the number of chickens and rabbits are $x_1$ and $x_2$, then we can write down a linear system
\begin{equation}\label{14:34-05/31/2022}
\systeme{x_1+x_2=10,2x_1+4x_2=26}
\end{equation}
\end{solution}
\end{frame}

\begin{frame}[t]
\begin{solution}
Let's solve the linear system \systeme{x_1+x_2=10, 2x_1+4x_2=26}\pause
\begin{enumerate}[I.]
\item Replace \texttt{Row 2} by \texttt{Row 2 - 2(Row 1)}, we get \systeme{x_1+x_2=10, 2x_2=6}
\pause
\item Divide \texttt{Row 2} by 2, we get \systeme{x_1+x_2=10, x_2=3}
\pause
\item Replace \texttt{Row 1} by \texttt{Row 1 - Row 2}, we have the solution \\
\systeme{x_1=7, x_2=3}
\end{enumerate}
\end{solution}\pause

\begin{remark}
This process is call the \textcolor{blue}{Gaussian elimination}\index{Gaussian elimination}
\end{remark}
\end{frame}

\begin{frame}[t]
\begin{definition}
\begin{itemize}
\item A \textcolor{blue}{solution} of the linear system~\eqref{eq: general linear system} is
\[
\begin{cases}
x_1=s_1\\
x_2=s_2\\
x_3=s_3\\
\quad\,\,\,\,\vdots\\
x_n=s_n
\end{cases}\qquad\text{$s_1,s_2,s_3,\cdots,s_n$ are numbers}
\]
which makes~\eqref{eq: general linear system} true. \pause
\item The set of all possible solutions is called the \textcolor{blue}{solution set} of the linear system.
\pause
\item \textcolor{blue}{To solve} a linear system is to find all its solutions.
\end{itemize}
\end{definition}
\end{frame}

\begin{frame}[t]{Geometric intepretation}\pause
\begin{example}
\systeme{x_1+x_2=10, 2x_1+4x_2=26} describe two lines in $\mathbb R^2$, and the solution set is the intersection.
\end{example}
\pause
\begin{question}
How many solutions does $\begin{cases}
a_{11}x_1+a_{12}x_2=b_1\\
a_{21}x_1+a_{22}x_2=b_2
\end{cases}$ have?
\end{question}
\pause
\begin{answer} It may have
\begin{itemize}
\item a \textit{unique} solution if these two lines \textit{intersect}.
\item (uncountably) \textit{infinitely many} solutions if these two lines \textit{overlap}.
\item \textit{no} solutions if these two lines are \textit{parallel} but not overlapping.
\end{itemize}
\end{answer}
\end{frame}

\begin{frame}[t]
\begin{example}\label{ex: number of solutions}
Compare the following three linear systems
\newpage
\begin{subequations}
\begin{table}
\begin{subtable}{.32\linewidth}
\begin{equation}\label{eq: unique solution}
\systeme{x_1+x_2=10,2x_1+4x_2=26}
\end{equation}
\end{subtable}
\begin{subtable}{.32\linewidth}
\begin{equation}\label{eq: no solutions}
\systeme{x_1+2x_2=10, 2x_1+4x_2=26}
\end{equation}
\end{subtable}
\begin{subtable}{.32\linewidth}
\begin{equation}\label{eq: inf solutions}
\systeme{x_1+2x_2=13, 2x_1+4x_2=26}
\end{equation}
\end{subtable}
\end{table}
\end{subequations}
\pause
\begin{itemize}
\item \eqref{eq: unique solution} has a unique solution \systeme*{x_1=7,x_2=3}
\pause
\item \eqref{eq: no solutions} has no solutions since the 1st equation contradicts the 2nd.
\pause
\item \eqref{eq: inf solutions} has infinitely many solutions since the 2nd equation is twice of the 1st.
\end{itemize}
\end{example}
\end{frame}

\begin{frame}[t]
\begin{figure}
\centering
\begin{subfigure}[b]{0.49\textwidth}
\centering
\scalebox{0.5}{
\begin{tikzpicture}[scale=0.1]
\pgfmathsetmacro{\A}{-30}
\pgfmathsetmacro{\B}{30}
\pgfmathsetmacro{\C}{-10}
\pgfmathsetmacro{\D}{30}
\coordinate (P) at (-10,20);
\coordinate (Q) at (-1,7);
\clip (\A,\C) rectangle (\B,\D);
\draw[help lines, color=gray!30, step=10, dashed] (\A,\C) grid (\B,\D);
\draw[->, color=gray!50] (\A,0)--(\B,0);
\draw[->, color=gray!50] (0,\C)--(0,\D);
\draw[thick, domain=\A:\B, samples=100, blue] plot ({\x}, {10-\x});
\draw[thick, domain=\A:\B, samples=100, red] plot ({\x}, {(13-\x)/2});
\node at (P)[above right] {\textcolor{blue}{$x_1+x_2=10$}};
\node at (Q)[below left] {\textcolor{red}{$2x_1+4x_2=26$}};
\end{tikzpicture}}
\caption{Two lines intersect}
\label{fig: two lines intersect}
\end{subfigure}
\hfill
\begin{subfigure}[b]{0.49\textwidth}
\centering
\scalebox{0.5}{
\begin{tikzpicture}[scale=0.3]
\pgfmathsetmacro{\A}{-10}
\pgfmathsetmacro{\B}{10}
\pgfmathsetmacro{\C}{-1}
\pgfmathsetmacro{\D}{15}
\coordinate (P) at (2,4);
\coordinate (Q) at (-1,7);
\clip (\A,\C) rectangle (\B,\D);
\draw[help lines, color=gray!30, step=3, dashed] (\A,\C) grid (\B,\D);
\draw[->, color=gray!50] (\A,0)--(\B,0);
\draw[->, color=gray!50] (0,\C)--(0,\D);
\draw[thick, domain=\A:\B, samples=100, blue] plot ({\x}, {(10-\x)/2});
\draw[thick, domain=\A:\B, samples=100, red] plot ({\x}, {(13-\x)/2});
\node at (P)[below left] {\textcolor{blue}{$x_1+2x_2=10$}};
\node at (Q)[above right] {\textcolor{red}{$2x_1+4x_2=26$}};
\end{tikzpicture}}
\caption{Two line parallel but not overlapping}
\label{fig: two line parallel but not overlapping}
\end{subfigure}
\hfill
\vspace{1mm}
\begin{subfigure}[b]{0.5\textwidth}
\centering
\scalebox{0.5}{
\begin{tikzpicture}[scale=0.3]
\pgfmathsetmacro{\A}{-10}
\pgfmathsetmacro{\B}{10}
\pgfmathsetmacro{\C}{-1}
\pgfmathsetmacro{\D}{15}
\pgfmathsetmacro{\E}{0.03}
\coordinate (Q) at (-1,7);
\clip (\A,\C) rectangle (\B,\D);
\draw[help lines, color=gray!30, step=3, dashed] (\A,\C) grid (\B,\D);
\draw[->, color=gray!50] (\A,0)--(\B,0);
\draw[->, color=gray!50] (0,\C)--(0,\D);
\draw[thick, domain=\A:\B, samples=100, blue] plot ({\x}, {(13-\x)/2-\E});
\draw[thick, domain=\A:\B, samples=100, red] plot ({\x}, {(13-\x)/2+\E});
\node at (Q)[below left] {\textcolor{blue}{$x_1+2x_2=13$}};
\node at (Q)[above right] {\textcolor{red}{$2x_1+4x_2=26$}};
\end{tikzpicture}}
\caption{Two line overlap}
\label{fig: two line overlap}
\end{subfigure}
\caption{Figure for Example~\ref{ex: number of solutions}}
\label{fig: geometric interpretation of 2 by 2 linear system}
\end{figure}
\end{frame}

\begin{frame}[t]{Generalization}
If we increase the number of equations, we get more lines, it might looks like
\pause
\begin{figure}
\centering
\begin{subfigure}[b]{0.2\textwidth}
\centering
\begin{tikzpicture}
\draw[help lines, color=gray!30, step=.4, dashed] (-1,-1) grid (1,1);
\draw(-1,0)--(1,0);
\draw(0,-1)--(0,1);
\draw(-1,-1)--(1,1);
\end{tikzpicture}
\end{subfigure}
\begin{subfigure}[b]{0.2\textwidth}
\centering
\begin{tikzpicture}[scale=0.5]
\draw[help lines, color=gray!30, step=.8, dashed] (-1,-1) grid (3,3);
\draw(-1,0)--(3,0);
\draw(0,-1)--(0,3);
\draw(-1,3)--(3,-1);
\end{tikzpicture}
\end{subfigure}
\begin{subfigure}[b]{0.2\textwidth}
\centering
\begin{tikzpicture}
\draw[help lines, color=gray!30, step=.4, dashed] (-1,-1) grid (1,1);
\draw(-1,0.5)--(1,0.5);
\draw(-1,-0.5)--(1,-0.5);
\draw(0,-1)--(0,1);
\end{tikzpicture}
\end{subfigure}
\begin{subfigure}[b]{0.2\textwidth}
\centering
\begin{tikzpicture}
\draw[help lines, color=gray!30, step=.4, dashed] (-1,-1) grid (1,1);
\draw(-1,0.5)--(1,0.5);
\draw(-1,-0.5)--(1,-0.5);
\draw(-1,0)--(1,0);
\end{tikzpicture}
\end{subfigure}
\end{figure}
\pause
If we increase the number of variables, we get
\begin{itemize}
\item $a_1x_1+a_2x_2+a_3x_3=b$ describes a plane in $\mathbb R^3$.
\pause
\item $a_1x_1+a_2x_2+a_3x_3+\cdots+a_nx_n=b$ describes a \textit{hyperplane} in $\mathbb R^n$.
\pause
\item Therefore the solution set of~\eqref{eq: general linear system} is the intersection of $m$ hyperplanes.
\end{itemize}
\end{frame}

\begin{frame}[t]
\begin{example}
Geometric interpretation of \systeme{x_1-3x_2+2x_3=0, -5x_1+12x_2-x_3=0}
\begin{center}
\begin{tikzpicture}
\definecolor{color1}{rgb}{255,0,0}
\definecolor{color2}{rgb}{0,0,255}
\definecolor{color3}{rgb}{255,0,255}
\def\XMAX{2.5};\def\YMAX{2.5};\def\ZMAX{4.5};
\begin{scope}[blend mode=multiply]
\draw [color1, fill=color1!20] plot (-2,-2,-2)--(-2,2,4)--(2,2,2)--(2,-2,-4)--cycle;
\node at (2,-2,-4) [below right]{$x_1-3x_2+2x_3=0$};
\draw [color2, fill=color2!20] plot (-2,-7/6,-4)--(-2,-1/2,4)--(2,7/6,4)--(2,1/2,-4)--cycle;
\node at (2,1/2,-4) [above]{$-5x_1+12x_2-x_3=0$};
\end{scope}
\draw [purple, ultra thick](-2,-6/7,-2/7)--(2,6/7,2/7);
\draw[->] (-\XMAX,0,0)--(\XMAX,0,0) node[right]{$x_1$};
\draw[->] (0,-\YMAX,0)--(0,\YMAX,0) node[above]{$x_2$};
\draw[->] (0,0,-\ZMAX)--(0,0,\ZMAX) node[below left]{$x_3$};
\end{tikzpicture}
\end{center}
\end{example}
\end{frame}

\begin{frame}[t]
\begin{remark}
It is geometrically clear that for a system of 2 equations in 3 variables, there are either no solutions or infinitely many, since two planes either intersects at a line, or overlap, or simply parallel.
\end{remark}
\end{frame}

\begin{frame}[t]
\begin{definition}
We say a linear system is \textcolor{blue}{consistent}\index{consistent} if it has solution(s), and \textcolor{blue}{inconsistent}\index{inconsistent} if it has none.
\end{definition}
\pause
\begin{example}
In the previous Example~\ref{ex: number of solutions}, \systeme{x_1+x_2=10,2x_1+4x_2=26} and \systeme{x_1+2x_2=13, 2x_1+4x_2=26} are consistent, while \systeme{x_1+2x_2=10, 2x_1+4x_2=26} is inconsistent
\end{example}
\end{frame}

\begin{frame}[t]{Exercise}
Try Gaussian elimination on the following linear systems
\begin{enumerate}
\item \systeme{x_1+5x_2 = 7,-2x_1 - 7x_2 = -5 }
\seti
\end{enumerate}
\end{frame}

\begin{frame}[t]{Exercise}
\begin{enumerate}
\conti
\item \systeme{2x_1+4x_2=-4, 5x_1+7x_2=11}
\seti
\end{enumerate}
\end{frame}

\begin{frame}[t]{Exercise}
\begin{enumerate}
\conti
\item \systeme{x_1-x_2+x_3=1, 2x_1-x_3=1,x_1+x_2+x_3=3}
\end{enumerate}
\end{frame}

\begin{frame}[t]{Exercise}
Find the point of intersection of the lines $x_1-5x_2=1$ and $3x_1-7x_2=5$.
\end{frame}

\begin{frame}[t]{Exercise}
For what values of $h$ and $k$ is the following system consistent?
\[
\systeme{2x_1-x_2=h, -6x_1+3x_2=k}
\]
\end{frame}


\end{document}