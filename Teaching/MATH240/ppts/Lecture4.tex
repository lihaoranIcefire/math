\documentclass{beamer}
\setbeamercovered{}

\usepackage{bookman}
\usepackage{newtxtext}
\usepackage{amssymb}
\usepackage{amsmath}
\usepackage{amsfonts}
\usepackage{mathrsfs}
\usepackage{mathtools}
\usepackage{hyperref}
\usepackage{pifont}
\usepackage{systeme}
\usepackage{subcaption}
\usepackage{bm} % Bold
\usepackage{bbm} % Can use \mathbbm{1}
\usepackage{ifthen} % Use if then else
\usepackage{shuffle} % Use shuffle product
\usepackage{minted} % For listing codes
\usepackage{tikz}
\usepackage{pgfplots}

\makeatletter
\newcommand*{\Relbarfill@}{\arrowfill@\Relbar\Relbar\Relbar}
% \newcommand*{\relbarfill@}{\arrowfill@\relbar-\relbar}
\newcommand*{\xequal}[2][]{\ext@arrow 0055\Relbarfill@{#1}{#2}}
% \newcommand*{\xdash}[2][]{\ext@arrow 0055\relbarfill@{#1}{#2}}
\makeatother

\newlength{\simlength}

\newcommand{\xsim}[1]{
\settowidth{\simlength}{$#1$}
\mathrel{\overset{#1}{\resizebox{\simlength}{\height}{\sim}}}
}

\mode<presentation> {
    \usetheme{Madrid}
    \usecolortheme{beaver}
    \usefonttheme{professionalfonts} 
    \setbeamertemplate{navigation symbols}{}
    \setbeamertemplate{caption}[numbered]
}

\theoremstyle{definition}
\newtheorem{proposition}[theorem]{Proposition}
\newtheorem{construction}[theorem]{Construction}
\newtheorem{deduction}[theorem]{Deduction}
\newtheorem{exercise}[theorem]{Exercise}
\newtheorem{conjecture}[theorem]{Conjecture}

\theoremstyle{remark}
\newtheorem*{remark}{Remark}
\newtheorem*{recall}{Recall}
\newtheorem*{question}{Question}
\newtheorem*{answer}{Answer}

\newcounter{saveenumi}
\newcommand{\seti}{\setcounter{saveenumi}{\value{enumi}}}
\newcommand{\conti}{\setcounter{enumi}{\value{saveenumi}}}
\DeclareMathOperator{\Span}{Span}
\resetcounteronoverlays{saveenumi}

\AtBeginSection[]{
    \begin{frame}
    \centering
    \begin{beamercolorbox}[center, shadow=true, rounded=true]{title}
    \usebeamerfont{title}\insertsectionhead\par%
    \end{beamercolorbox}
    \end{frame}
}

\title{MATH240 - Introduction to Linear Algebra}
\author{Haoran Li}
\institute[UMD]{University of Maryland, College Park}
\date{Summer 2024}


\begin{document}

\maketitle

\section{Lecture 4 - Matrix equations and linear independence}

\begin{frame}[t]
Recall a vector is a one column matrix, zero vector is the vector of zeros. For scalar (i.e. a number) $c$, and vectors $\mathbf a=\begin{bmatrix}
a_1\\a_2\\\vdots\\a_n
\end{bmatrix}$, $\mathbf b=\begin{bmatrix}
b_1\\b_2\\\vdots\\b_n
\end{bmatrix}$, we have\pause
\begin{itemize}
\item Addition $\mathbf a+\mathbf b=\begin{bmatrix}
a_1\\a_2\\\vdots\\a_n
\end{bmatrix}+\begin{bmatrix}
b_1\\b_2\\\vdots\\b_n
\end{bmatrix}=\begin{bmatrix}
a_1+b_1\\a_2+b_2\\\vdots\\a_n+b_n
\end{bmatrix}$\pause
\item Scalar multiplication $c \mathbf  a=c\begin{bmatrix}
a_1\\a_2\\\vdots\\a_n
\end{bmatrix}=\begin{bmatrix}
ca_1\\ca_2\\\vdots\\ca_n
\end{bmatrix}$
\end{itemize}
\pause
\begin{remark}
In handwritings, we use $\Vec{v}$ or $\overset{\rightharpoonup}{v}$ to denote a vector, while in printing materials we often use the math bold font $\mathbf v$.
\end{remark}
\end{frame}

\begin{frame}[t]
\begin{definition}
A \textcolor{blue}{vector equation}\index{vector equation} is of the form
\begin{equation}\label{09:29-06/03/2022}
x_1\mathbf a_1+x_2\mathbf a_2+\cdots+x_n\mathbf a_n=\mathbf b
\end{equation}\pause
We can also write the vector equation~\eqref{09:29-06/03/2022} as a \textcolor{blue}{matrix equation}\index{Matrix equation} with partitioned matrix
\[
x_1\mathbf a_1+x_2\mathbf a_2+\cdots+x_n\mathbf a_n=\begin{bmatrix}
\mathbf a_1&\mathbf a_2&\cdots&\mathbf a_n
\end{bmatrix}\begin{bmatrix}
x_1\\x_2\\\vdots\\x_n
\end{bmatrix}=A\mathbf x=\mathbf b
\]
Here $A=\begin{bmatrix}
\mathbf a_1&\mathbf a_2&\cdots&\mathbf a_n
\end{bmatrix}$ and $\mathbf x=\begin{bmatrix}
x_1\\x_2\\\vdots\\x_n
\end{bmatrix}$
\end{definition}
\end{frame}

\begin{frame}[t]
\begin{example}
\systeme{x_1+x_2=10, 2x_1+4x_2=26} can be written as a vector equation
\begin{center}
\scalebox{0.9}{
$\mathbf b=\begin{bmatrix}
10\\26
\end{bmatrix}\pause=\begin{bmatrix}
x_1+x_2\\2x_1+4x_2
\end{bmatrix}\pause=\begin{bmatrix}
x_1\\2x_1
\end{bmatrix}+\begin{bmatrix}
x_2\\4x_2
\end{bmatrix}\pause=x_1\begin{bmatrix}
1\\2
\end{bmatrix}+x_2\begin{bmatrix}
1\\4
\end{bmatrix}\pause=x_1\mathbf a_1+x_2\mathbf a_2$
}
\end{center}\pause
Or a matrix equation
\[
\mathbf b=\begin{bmatrix}
10\\26
\end{bmatrix}=x_1\mathbf a_1+x_2\mathbf a_2\pause=\begin{bmatrix}
\mathbf a_1&\mathbf a_2
\end{bmatrix}\begin{bmatrix}
x_1\\x_2
\end{bmatrix}\pause=\begin{bmatrix}
1&1\\2&4
\end{bmatrix}\begin{bmatrix}
x_1\\x_2
\end{bmatrix}\pause=A\mathbf x
\]
\end{example}
\end{frame}

\begin{frame}[t]
\begin{example}\label{16:17-06/03/2022}
The vector equation of \systeme*{x_1+x_3=1, 2x_1+x_3=2} is\pause
\[
x_1\begin{bmatrix}
1\\0
\end{bmatrix}+x_2\begin{bmatrix}
0\\2
\end{bmatrix}+x_3\begin{bmatrix}
1\\1
\end{bmatrix}=\begin{bmatrix}
1\\2
\end{bmatrix}
\]\pause
And the corresponding matrix equation is\pause
\[
\begin{bmatrix}
1&0&1\\
0&2&1
\end{bmatrix}\begin{bmatrix}
x_1\\x_2\\x_3
\end{bmatrix}=\begin{bmatrix}
1\\2
\end{bmatrix}
\]
\end{example}
\end{frame}

\begin{frame}[t]
\begin{definition}\hfill
\begin{itemize}
\item A \textcolor{blue}{linear combination}\index{linear combination} of $\mathbf a_1,\mathbf a_2,\cdots,\mathbf a_n$ is a sum $c_1\mathbf a_1+c_2\mathbf a_2+\cdots+c_n\mathbf a_n$ for some scalars $c_1,\cdots,c_n$.\pause
\item The \textcolor{blue}{span}\index{span} of $\{\mathbf v_1,\mathbf v_2,\cdots,\mathbf v_n\}$ is the set of all its linear combinations, which we denote $\Span\{\mathbf v_1, \mathbf v_2,\cdots,\mathbf v_n\}$.
\end{itemize}
\end{definition}
\pause
\begin{theorem}
$x_1\mathbf a_1+x_2\mathbf a_2+\cdots+x_n\mathbf a_n=\mathbf b$ has solution(s) $\iff\mathbf b$ is a linear combination of $\mathbf v_1, \mathbf v_2,\cdots,\mathbf v_n$ $\iff$ $\mathbf b\in\Span\{\mathbf v_1, \mathbf v_2,\cdots,\mathbf v_n\}$
\end{theorem}
\end{frame}

\begin{frame}[t]
\begin{exercise}
Let $\mathbf a_1=\begin{bmatrix}
1\\2\\1
\end{bmatrix}$, $\mathbf a_2=\begin{bmatrix}
-1\\0\\1
\end{bmatrix}$, $\mathbf a_3=\begin{bmatrix}
1\\-1\\-2
\end{bmatrix}$, $\mathbf b=\begin{bmatrix}
1\\1\\1
\end{bmatrix}$. Is $\mathbf b$ in $\Span\{\mathbf a_1, \mathbf a_2,\mathbf a_3\}$?
\end{exercise}
\pause
\begin{solution}
This is equivalent of asking if whether the vector equation $x_1\mathbf a_1+x_2\mathbf a_2+x_3\mathbf a_3=\mathbf b$ has solution(s)\pause, we find an REF of its augmented matrix
\begin{center}
\scalebox{0.8}{
$\begin{bmatrix}
1&-1&1&1\\
2&0&-1&1\\
1&1&-2&1\\
\end{bmatrix}\xsim{\substack{R2\rightarrow R2-2R1\\R3\rightarrow R3-R1}}\begin{bmatrix}
1&-1&1&1\\
0&2&-3&-1\\
0&2&-3&0\\
\end{bmatrix}\xsim{R3\rightarrow R3-R2}\begin{bmatrix}
1&-1&1&1\\
0&2&-3&-1\\
0&0&0&1\\
\end{bmatrix}$
}
\end{center}\pause
Since there is a pivot in the last column, the linear system is inconsistent, hence $\mathbf b\notin\Span\{\mathbf a_1, \mathbf a_2,\mathbf a_3\}$
\end{solution}
\end{frame}

\begin{frame}[t]
\begin{definition}
$\{\mathbf v_1,\cdots,\mathbf v_n\}$ is \textcolor{blue}{linearly dependent}\index{linearly dependent} if some $\mathbf v_i$ is in the span of the others (so it is somewhat redundant)\pause, or equivalently, if there is a non-trivial solution $c_1,\cdots,c_n$ (i.e. not all $c_i$'s are 0) to the vector equation
\begin{equation}\label{16:30-06/06/2022}
c_1\mathbf v_1+\cdots+c_n\mathbf v_n=\mathbf0
\end{equation}\pause
\eqref{16:30-06/06/2022} is referred to as a \textcolor{blue}{linear dependence}\index{linear dependence} between $\{\mathbf v_1,\cdots,\mathbf v_n\}$. If \eqref{16:30-06/06/2022} has only the trivial solution (i.e. $c_1,\cdots, c_n$ are all 0, which is of course always a solution), $\{\mathbf v_1,\cdots,\mathbf v_n\}$ is said to be \textcolor{blue}{linearly independent}\index{linearly independent}
\end{definition}
\pause
\begin{remark}
Equivalence between two different definitions of linear dependence
\begin{itemize}
\item If $\mathbf v_i=c_1\mathbf v_1+\cdots+c_n\mathbf v_n$, then $c_1\mathbf v_1+\cdots+(-1)\mathbf v_i+\cdots+c_n\mathbf v_n=\mathbf0$
\item If $c_1\mathbf v_1+\cdots+c_i\mathbf v_i+\cdots+c_n\mathbf v_n=\mathbf0$ and $c_i\neq0$ (since not all $c_i$'s are zero, we may assume some $c_i$ is nonzero), then $\mathbf v_i=-\frac{c_1}{c_i}\mathbf v_1-\cdots-\frac{c_{i-1}}{c_i}\mathbf v_{i-1}-\frac{c_{i+1}}{c_i}\mathbf v_{i+1}-\cdots-\frac{c_n}{c_i}\mathbf v_n$
\end{itemize}
\end{remark}
\end{frame}

\begin{frame}[t]
\begin{question}
How do we determine and find linear dependence of $\{\mathbf v_1,\cdots,\mathbf v_n\}$?
\end{question}
\pause
\begin{answer}
Let $A=\begin{bmatrix}
\mathbf v_1&\cdots&\mathbf v_n
\end{bmatrix}$, then non-trivial solutions to $A\mathbf x=x_1\mathbf v_1+\cdots+x_n\mathbf v_n=\mathbf0$ would be the linear dependences of $\{\mathbf v_1,\cdots,\mathbf v_n\}$. Therefore it is linearly independent if it has only the trivial(zero) solution.
\end{answer}
\pause
\begin{theorem} 
$\{\mathbf v_1,\cdots,\mathbf v_n\}$ is linearly independent $\iff A\mathbf x=\mathbf 0$ has only the trivial(zero) solution $\iff$ each column of the RREF of $A$ is a pivot column.
\end{theorem}
\pause
\begin{proof}
Consider the RREF of the augmented matrix $\begin{bmatrix}
A&\mathbf 0
\end{bmatrix}$, it is necessarily $\begin{bmatrix}
U&\mathbf 0
\end{bmatrix}$ for $A\mathbf x=\mathbf0$ to have only the trivial solution.
\end{proof}
\end{frame}

\begin{frame}[t]
\begin{example}\label{06:35-06/06/2022}
Suppose $\mathbf v_1=\begin{bmatrix}
1\\-1
\end{bmatrix}$, $\mathbf v_2=\begin{bmatrix}
1\\1
\end{bmatrix}$, $\mathbf e_1=\begin{bmatrix}
1\\0
\end{bmatrix}$, $\{\mathbf v_1,\mathbf v_2,\mathbf e_1\}$ is linearly dependent since
\[
\begin{bmatrix}
\mathbf v_1&\mathbf v_2&\mathbf e_1
\end{bmatrix}=\begin{bmatrix}
1&1&1\\
-1&1&0\\
\end{bmatrix}\sim\begin{bmatrix}
1&0&\frac{1}{2}\\
0&1&\frac{1}{2}\\
\end{bmatrix}
\]\pause
The solution to this augmented matrix would be $\begin{cases}
x_1=-\frac{1}{2}x_3\\
x_2=-\frac{1}{2}x_3\\
x_3\text{ is free}
\end{cases}$, by choosing any value nonzero value of $x_3$ (say 1) we get a linear dependence $-\frac{1}{2}\mathbf v_1-\frac{1}{2}\mathbf v_2+\mathbf e_1=\mathbf0$\pause. On the other hand, $\mathbf v_1,\mathbf v_2$ are linearly independent since
\[
\begin{bmatrix}
\mathbf v_1&\mathbf v_2
\end{bmatrix}=\begin{bmatrix}
1&1\\
-1&1
\end{bmatrix}\sim\begin{bmatrix}
1&0\\
0&1
\end{bmatrix}
\]
Where each column is a pivot column.
\end{example}
\end{frame}

\begin{frame}[t]{Exercise}
Write the system \systeme{8x_1-x_2=4, 5x_1+4x_2=1,x_1-3x_2=2} first as a vector equation and then as a matrix equation.
\end{frame}

\begin{frame}[t]{Exercise}
Let $\mathbf v_1=\begin{bmatrix}
0\\0\\-2
\end{bmatrix}$, $\mathbf v_2=\begin{bmatrix}
0\\-3\\8
\end{bmatrix}$, $\mathbf v_3=\begin{bmatrix}
4\\-1\\-5
\end{bmatrix}$.
\begin{itemize}
\item Does $\{\mathbf v_1,\mathbf v_2,\mathbf v_3\}$ span $\mathbb R^3$? Why or why not?\pause
\item Is $\{\mathbf v_1,\mathbf v_2,\mathbf v_3\}$ linearly independent? Why or why not?
\end{itemize}
\end{frame}

% \begin{frame}[t]
% \end{frame}

% \begin{frame}[t]
% \end{frame}

\end{document}