\documentclass{beamer}
\setbeamercovered{}

\usepackage{bookman}
\usepackage{newtxtext}
\usepackage{amssymb}
\usepackage{amsmath}
\usepackage{amsfonts}
\usepackage{mathrsfs}
\usepackage{mathtools}
\usepackage{hyperref}
\usepackage{pifont}
\usepackage{systeme}
\usepackage{subcaption}
\usepackage{bm} % Bold
\usepackage{bbm} % Can use \mathbbm{1}
\usepackage{ifthen} % Use if then else
\usepackage{shuffle} % Use shuffle product
\usepackage{minted} % For listing codes
\usepackage{tikz}
\usepackage{tikz-cd}
\usepackage{cancel}
\tikzcdset{ampersand replacement=\&}
\usetikzlibrary{calc}
\usepackage{pgfplots}

\makeatletter
\newcommand*{\Relbarfill@}{\arrowfill@\Relbar\Relbar\Relbar}
% \newcommand*{\relbarfill@}{\arrowfill@\relbar-\relbar}
\newcommand*{\xequal}[2][]{\ext@arrow 0055\Relbarfill@{#1}{#2}}
% \newcommand*{\xdash}[2][]{\ext@arrow 0055\relbarfill@{#1}{#2}}
\makeatother

\newlength{\simlength}

\newcommand{\xsim}[1]{
\settowidth{\simlength}{$#1$}
\mathrel{\overset{#1}{\resizebox{\simlength}{\height}{\sim}}}
}
\newcommand{\fatdot}{\mathrel{\raisebox{0.25ex}{\tikz\filldraw[black,x=2pt,y=2pt] (0,0) circle (0.5);}}}
\newcommand{\fatplus}{\mathrel{\raisebox{-0.1ex}{\tikz\filldraw[black,x=2pt,y=2pt,scale=1.4] (0.9,0)--(1.1,0)--(1.1,0.9)--(2,0.9)--(2,1.1)--(1.1,1.1)--(1.1,2)--(0.9,2)--(0.9,1.1)--(0,1.1)--(0,0.9)--(0.9,0.9)--cycle;}}}
\newcommand{\fatminus}{\raisebox{+0.4ex}{\tikz\filldraw[black,x=2pt,y=2pt,scale=1.4] (0,0.9)--(2,0.9)--(2,1.1)--(0,1.1)--cycle;}}

\DeclareMathOperator{\Nul}{Nul}
\DeclareMathOperator{\Col}{Col}
\DeclareMathOperator{\Row}{Row}
\DeclareMathOperator{\Rank}{Rank}
\DeclareMathOperator{\Range}{Range}
\DeclareMathOperator{\Ker}{Ker}
\DeclareMathOperator{\id}{id}
\DeclareMathOperator{\Proj}{Proj}

\mode<presentation> {
    \usetheme{Madrid}
    \usecolortheme{beaver}
    \usefonttheme{professionalfonts} 
    \setbeamertemplate{navigation symbols}{}
    \setbeamertemplate{caption}[numbered]
}

\theoremstyle{definition}
\newtheorem{proposition}[theorem]{Proposition}
\newtheorem{construction}[theorem]{Construction}
\newtheorem{deduction}[theorem]{Deduction}
\newtheorem{exercise}[theorem]{Exercise}
\newtheorem{conjecture}[theorem]{Conjecture}

\theoremstyle{remark}
\newtheorem*{remark}{Remark}
\newtheorem*{recall}{Recall}
\newtheorem*{question}{Question}
\newtheorem*{answer}{Answer}

\newcounter{saveenumi}
\newcommand{\seti}{\setcounter{saveenumi}{\value{enumi}}}
\newcommand{\conti}{\setcounter{enumi}{\value{saveenumi}}}
\DeclareMathOperator{\Span}{Span}
\resetcounteronoverlays{saveenumi}

\AtBeginSection[]{
    \begin{frame}
    \centering
    \begin{beamercolorbox}[center, shadow=true, rounded=true]{title}
    \usebeamerfont{title}\insertsectionhead\par%
    \end{beamercolorbox}
    \end{frame}
}

\title{MATH240 - Introduction to Linear Algebra}
\author{Haoran Li}
\institute[UMD]{University of Maryland, College Park}
\date{Summer 2023}


\begin{document}

\maketitle

\section{Lecture 15 - Complex eigenvalues}

Recall

\begin{frame}[t]
\begin{theorem}
Consider quadratic equation $at^2+bt+c=0$, the \textcolor{blue}{quadratic formula}\index{quadratic formula} reads $t=\frac{-b\pm\sqrt{\Delta}}{2a}$, where $\Delta=b^2-4ac$ is the discriminant.
\end{theorem}
\pause
Let's review some general stuff about complex numbers.

\begin{definition}
The set of complex numbers $\mathbb C$ is the defined to be $\{z=a+bi|a,b\in\mathbb R\}$, with $i=\sqrt{-1}$ so that $i^2=-1$\pause. We call $\operatorname{Re}z=a$ the \textcolor{blue}{real part}\index{real part} of $z$ and $\operatorname{Im}z=b$ to be the \textcolor{blue}{imaginary part}\index{imaginary part} of $z$\pause. We call $r=|z|=\sqrt{a^2+b^2}$ the \textcolor{blue}{modulus}\index{modulus} (or \textcolor{blue}{absolute value}) of $z$\pause, and the angle $\varphi$ between $z$ and the real axis the \textcolor{blue}{argument}\index{argument} of $z$.
\end{definition}
\end{frame}

\begin{frame}[t]
\begin{definition}
There is a natural identification between the complex numbers and the plane $\mathbb R^2$ via $z=a+bi\leftrightsquigarrow(a,b)$ (this is why the set of complex numbers is often called the complex plane). We can define
\begin{itemize}
\item Addition via $(a+bi)+(c+di)=(a+c)+(b+d)i$
\item Multiplication via $(a+bi)(c+di)=ac+adi+bci+bdi^2=(ac-bd)+(ad+bc)i$. 
\item \textcolor{blue}{Conjugate}\index{conjugate} as $\overline z=a-bi$.
\end{itemize}
\end{definition}
\pause
\begin{remark}
Through this identification it is easy to see that $a=r\cos\varphi$, $b=r\sin\varphi$, and we have \textcolor{blue}{Euler's formula}\index{Euler's formula} $z=re^{i\varphi}=r\cos\varphi=i\sin\varphi$.
\end{remark}
\end{frame}

\begin{frame}[t]
\begin{remark}
If we have a complex-valued matrix $A$, then the conjugation is defined entrywise (Note that column vectors are matrices). If we write a complex valued matrix $A=\operatorname{Re}A+i\operatorname{Im}A$\pause, then the conjugation would be $\overline A=\operatorname{Re}A-i\operatorname{Im}A$\pause. For example, if $A=\begin{bmatrix}
1+2i&3\\
-1-i&i
\end{bmatrix}=\begin{bmatrix}
1&3\\
-1&0
\end{bmatrix}+i\begin{bmatrix}
2&0\\
-1&1
\end{bmatrix}$\pause, then $\overline A=\begin{bmatrix}
1-2i&3\\
-1+i&-i
\end{bmatrix}=\begin{bmatrix}
1&3\\
-1&0
\end{bmatrix}-i\begin{bmatrix}
2&0\\
-1&1
\end{bmatrix}$\pause. It is easy to check that $\overline{A+B}=\overline A+\overline B$, $\overline{cA}=\overline c\overline A$, $\overline{AB}=\overline A\overline B$.
\end{remark}
\end{frame}

\begin{frame}[t]
\begin{question}
$A=\begin{bmatrix}
1&-2\\1&3
\end{bmatrix}$, can we diagonalize it?
\end{question}
\pause
\begin{answer}
First we compute its characteristic polynomial $t^2-4t+5$, an then we can solve the quadratic equation as follows
\[
\Delta=(-4)^2-4\cdot1\cdot5=-4,t=\frac{-(-4)\pm\sqrt{-4}}{2}=\frac{-(-4)\pm2i}{2}=2\pm i
\]
Hence the eigenvalues are $\lambda_1=2+i,\lambda_2=2-i$\pause. The eigenvector for $\lambda_1$ can be computed via
\[
\left[\begin{array}{c|c}
\lambda_1I-A&\mathbf0
\end{array}\right]\sim\left[\begin{array}{cc|c}
1&1-i&0\\
0&0&0
\end{array}\right]
\]
\end{answer}
\end{frame}

\begin{frame}[t]
\begin{answer}
So we get $\mathbf v_1=\begin{bmatrix}
-1+i\\1
\end{bmatrix}=\begin{bmatrix}
-1\\1
\end{bmatrix}+\begin{bmatrix}
i\\0
\end{bmatrix}=\begin{bmatrix}
-1\\1
\end{bmatrix}+\begin{bmatrix}
1\\0
\end{bmatrix}i=\operatorname{Re}\mathbf v_1+i\operatorname{Im}\mathbf v_1$\pause. The eigenvector for $\lambda_2$ can be computed via
\[
\left[\begin{array}{c|c}
\lambda_2I-A&\mathbf0
\end{array}\right]\sim\left[\begin{array}{cc|c}
1&1+i&0\\
0&0&0
\end{array}\right]
\]
So we get $\mathbf v_2=\begin{bmatrix}
-1-i\\1
\end{bmatrix}=\begin{bmatrix}
-1\\1
\end{bmatrix}+\begin{bmatrix}
-i\\0
\end{bmatrix}=\begin{bmatrix}
-1\\1
\end{bmatrix}+\begin{bmatrix}
-1\\0
\end{bmatrix}i=\operatorname{Re}\mathbf v_2+i\operatorname{Im}\mathbf v_2$\pause. Now we may realize that we have $\lambda_1$ and $\lambda_2$ are conjugate of each other. $\mathbf v_1$ and $\mathbf v_2$ are conjugate of each other.
\end{answer}
\end{frame}

\begin{frame}[t]
\begin{theorem}
In general, if $A$ is a 2 by 2 real-valued matrix with complex eigenvalues (characteristic polynomial have complex roots, no real roots), then they are conjugate of each other, we can write them as $\lambda=a-bi$ and $\overline\lambda=a+bi$\pause, suppose the eigenvector for $\lambda$ is $\mathbf v=\operatorname{Re}\mathbf v+i\operatorname{Im}\mathbf v$, then the eigenvector for $\overline\lambda$ will be $\overline{\mathbf v}=\operatorname{Re}\mathbf v-i\operatorname{Im}\mathbf v$\pause. If we write $P=\begin{bmatrix}
\operatorname{Re}\mathbf v&\operatorname{Im}\mathbf v
\end{bmatrix}$, and $C=\begin{bmatrix}
a&-b\\
b&a
\end{bmatrix}$, then $AP=PC$, therefore we get decomposition $A=PCP^{-1}$.
\end{theorem}
\end{frame}

\begin{frame}[t]
\begin{proof}
We have $A\mathbf v=\lambda\mathbf v$ (and hence by conjugation we have $A\overline{\mathbf v}=A\overline{\mathbf v}=\overline A\overline{\mathbf v}=\overline{A\mathbf v}=\overline{\lambda}\overline{\mathbf v}$, note that here $A$ is real-valued, so $\overline A=A$), we can rewrite this as
$
(A\operatorname{Re}\mathbf v)+i(A\operatorname{Im}\mathbf v)=A(\operatorname{Re}\mathbf v+i\operatorname{Im}\mathbf v)=A\mathbf v=\lambda\mathbf v=(a-bi)(\operatorname{Re}\mathbf v+i\operatorname{Im}\mathbf v)=(a\operatorname{Re}\mathbf v+b\operatorname{Im}\mathbf v)+i(a\operatorname{Im}\mathbf v-b\operatorname{Re}\mathbf v)
$\\
\pause
Looking at its real and imaginary parts we conclude that
\[
A\operatorname{Re}\mathbf v=a\operatorname{Re}\mathbf v+b\operatorname{Im}\mathbf v,\qquad A\operatorname{Im}\mathbf v=a\operatorname{Im}\mathbf v-b\operatorname{Re}\mathbf v
\]\pause
This is precisely $AP=PC$.
\end{proof}
\end{frame}

\begin{frame}[t]
\begin{remark}
Matrix $C$ is special in the following sense (it can be decomposed as a composition of rotation and scaling)
\[
C=\begin{bmatrix}
a&-b\\
b&a
\end{bmatrix}=\begin{bmatrix}
r\cos\varphi&-r\sin\varphi\\
r\sin\varphi&r\cos\varphi
\end{bmatrix}=r\begin{bmatrix}
\cos\varphi&-\sin\varphi\\
\sin\varphi&\cos\varphi
\end{bmatrix}
\]
Here we suppose $\overline\lambda=a+bi=re^{i\varphi}=r\cos\varphi+ir\sin\varphi$ so that $a=r\cos\varphi$, $b=r\sin\varphi$.
\end{remark}
\pause
\begin{example}
In the previous example, we could take $\lambda=2-i$ so that $a=2,b=1$, $\mathbf v=\begin{bmatrix}
-1\\1
\end{bmatrix}+i\begin{bmatrix}
-1\\0
\end{bmatrix}$ so that $\operatorname{Re}\mathbf v=\begin{bmatrix}
-1\\1
\end{bmatrix}$, $\operatorname{Im}\mathbf v=\begin{bmatrix}
-1\\0
\end{bmatrix}$, and we have the decomposition
\[
A=\begin{bmatrix}
1&-2\\
1&3
\end{bmatrix}=\begin{bmatrix}
-1&-1\\
1&0
\end{bmatrix}\begin{bmatrix}
2&-1\\
1&2
\end{bmatrix}\begin{bmatrix}
0&1\\
-1&-1
\end{bmatrix}=PCP^{-1}
\]
\end{example}
\end{frame}

\begin{frame}[t]{Exercise}
Suppose $A=\begin{bmatrix}
-1&-1\\5&-5
\end{bmatrix}$. Find an invertible matrix $P$ and a matrix $C$ of the form $\begin{bmatrix}
a&-b\\b&a
\end{bmatrix}$ such that $A=PCP^{-1}$
\end{frame}

% \begin{frame}[t]
% \end{frame}

% \begin{frame}[t]
% \end{frame}

\end{document}