\documentclass{beamer}
\setbeamercovered{}

\usepackage{bookman}
\usepackage{newtxtext}
\usepackage{amssymb}
\usepackage{amsmath}
\usepackage{amsfonts}
\usepackage{mathrsfs}
\usepackage{mathtools}
\usepackage{hyperref}
\usepackage{pifont}
\usepackage{systeme}
\usepackage{subcaption}
\usepackage{bm} % Bold
\usepackage{bbm} % Can use \mathbbm{1}
\usepackage{ifthen} % Use if then else
\usepackage{shuffle} % Use shuffle product
\usepackage{minted} % For listing codes
\usepackage{tikz}
\usetikzlibrary{calc}
\usepackage{pgfplots}

\makeatletter
\newcommand*{\Relbarfill@}{\arrowfill@\Relbar\Relbar\Relbar}
% \newcommand*{\relbarfill@}{\arrowfill@\relbar-\relbar}
\newcommand*{\xequal}[2][]{\ext@arrow 0055\Relbarfill@{#1}{#2}}
% \newcommand*{\xdash}[2][]{\ext@arrow 0055\relbarfill@{#1}{#2}}
\makeatother

\newlength{\simlength}

\newcommand{\xsim}[1]{
\settowidth{\simlength}{$#1$}
\mathrel{\overset{#1}{\resizebox{\simlength}{\height}{\sim}}}
}

\mode<presentation> {
    \usetheme{Madrid}
    \usecolortheme{beaver}
    \usefonttheme{professionalfonts} 
    \setbeamertemplate{navigation symbols}{}
    \setbeamertemplate{caption}[numbered]
}

\theoremstyle{definition}
\newtheorem{proposition}[theorem]{Proposition}
\newtheorem{construction}[theorem]{Construction}
\newtheorem{deduction}[theorem]{Deduction}
\newtheorem{exercise}[theorem]{Exercise}
\newtheorem{conjecture}[theorem]{Conjecture}

\theoremstyle{remark}
\newtheorem*{remark}{Remark}
\newtheorem*{recall}{Recall}
\newtheorem*{question}{Question}
\newtheorem*{answer}{Answer}

\newcounter{saveenumi}
\newcommand{\seti}{\setcounter{saveenumi}{\value{enumi}}}
\newcommand{\conti}{\setcounter{enumi}{\value{saveenumi}}}
\DeclareMathOperator{\Span}{Span}
\resetcounteronoverlays{saveenumi}

\AtBeginSection[]{
    \begin{frame}
    \centering
    \begin{beamercolorbox}[center, shadow=true, rounded=true]{title}
    \usebeamerfont{title}\insertsectionhead\par%
    \end{beamercolorbox}
    \end{frame}
}

\title{MATH240 - Introduction to Linear Algebra}
\author{Haoran Li}
\institute[UMD]{University of Maryland, College Park}
\date{Summer 2024}


\begin{document}

\maketitle

\section{Lecture 5 - Geometric interpretation of solutions of linear systems}

\begin{frame}[t]
We like to identify vectors in $M_{n\times1}(\mathbb R)$ with points in $\mathbb R^n$. And there are very nice geometric interpretation of vector additions and scalar multiplications.
\end{frame}

\begin{frame}[t]
\begin{example}[Vector-point correspondence in the case of $n=2$]
Let $\mathbf a=\begin{bmatrix}
1\\2
\end{bmatrix}$, $\mathbf b=\begin{bmatrix}
2\\1
\end{bmatrix}$, then $\mathbf a+\mathbf b=\begin{bmatrix}
3\\3
\end{bmatrix}$, $2\mathbf a=\begin{bmatrix}
2\\4
\end{bmatrix}$, $-\mathbf b=(-1)\mathbf b=\begin{bmatrix}
-2\\-1
\end{bmatrix}$, $\mathbf a-\mathbf b=\mathbf a+(-\mathbf b)=\begin{bmatrix}
-1\\1
\end{bmatrix}$.
\begin{overlayarea}{\textwidth}{0.9\textheight}
\only<1>{
\begin{center}
\begin{tikzpicture}[scale=0.65]
\draw[help lines, color=gray!30, dashed] (-4.5,-2.5) grid (4.5,4.5);
\draw[->, color=gray!80] (-4.5,0)--(4.5,0) node[right]{$x_1$};
\draw[->, color=gray!80] (0,-2.5)--(0,4.5) node[above]{$x_2$};
\coordinate (a) at (1,2); \node at (a)[above left]{$\mathbf a$}; \draw[->, thick] (0,0)--(a);
\coordinate (b) at (2,1); \node at (b)[right]{$\mathbf b$}; \draw[->, thick] (0,0)--(b);
\node at ($(a)+(b)$)[below right]{$\mathbf a+\mathbf b$}; \draw[->, thick] (0,0)--($(a)+(b)$);
\draw[dashed] (a)--($(a)+(b)$);
\draw[dashed] (b)--($(a)+(b)$);
\end{tikzpicture}
\end{center}
}\only<2>{
\begin{center}
\begin{tikzpicture}[scale=0.65]
\draw[help lines, color=gray!30, dashed] (-4.5,-2.5) grid (4.5,4.5);
\draw[->, color=gray!80] (-4.5,0)--(4.5,0) node[right]{$x_1$};
\draw[->, color=gray!80] (0,-2.5)--(0,4.5) node[above]{$x_2$};
\coordinate (a) at (1,2); \node at (a)[above left]{$\mathbf a$}; \draw[->, thick] (0,0)--(a);
\coordinate (b) at (2,1); \node at (b)[right]{$\mathbf b$}; \draw[->, thick] (0,0)--(b);
\node at ($2*(a)$)[above left]{$2\mathbf a$}; \draw[->, thick] (0,0)--($2*(a)$);
\node at ($(0,0)-(b)$)[below right]{$-\mathbf b$};
\draw[->, thick] (0,0)--($(0,0)-(b)$);
\end{tikzpicture}
\end{center}
}\only<3>{
\begin{center}
\begin{tikzpicture}[scale=0.65]
\draw[help lines, color=gray!30, dashed] (-4.5,-2.5) grid (4.5,4.5);
\draw[->, color=gray!80] (-4.5,0)--(4.5,0) node[right]{$x_1$};
\draw[->, color=gray!80] (0,-2.5)--(0,4.5) node[above]{$x_2$};
\coordinate (a) at (1,2); \node at (a)[above left]{$\mathbf a$}; \draw[->, thick] (0,0)--(a);
\coordinate (b) at (2,1); \node at (b)[right]{$\mathbf b$}; \draw[->, thick] (0,0)--(b);
\node at ($(0,0)-(b)$)[below right]{$-\mathbf b$};
\draw[->, thick] (0,0)--($(0,0)-(b)$);
\node at ($(a)+(b)$)[below right]{$\mathbf a+\mathbf b$}; \draw[->, thick] (0,0)--($(a)+(b)$);
\node at ($(a)-(b)$)[above left]{$\mathbf a-\mathbf b$}; \draw[->, thick] (0,0)--($(a)-(b)$);
\draw[dashed] ($(a)-(b)$)--($(a)+(b)$);
\draw[dashed] ($(a)-(b)$)--($(0,0)-(b)$);
\draw[dashed] (b)--($(a)+(b)$);
\end{tikzpicture}
\end{center}
}
\end{overlayarea}
\end{example}
\end{frame}

\begin{frame}[t]
\begin{theorem}\label{16:05-06/06/2022}
Let $A$ be an $m\times n$ matrix. Then the following statements are logically equivalent. That is, for a particular $A$, either they are all true statements or they are all false.
\begin{enumerate}
\item For each $\mathbf b$ in $\mathbb R^m$, the equation $A\mathbf x =\mathbf b$ has a solution.
\item Each $\mathbf b$ in $\mathbb R^m$ is a linear combination of the columns of $A$.
\item The columns of $A$ span $\mathbb R^m$.
\item $A$ has a pivot position in every row. (Equivalentely, in the last row)
\end{enumerate}
\end{theorem}
\end{frame}

\begin{frame}[t]
\begin{definition}
$\{\mathbf v_1,\mathbf v_2,\cdots,\mathbf v_n\}$ is said to be a basis for $\mathbb R^n$ if it is linearly independent and spans all of $\mathbb R^n$
\end{definition}
\pause
\begin{theorem}
Let $A=\begin{bmatrix}
\mathbf v_1&\mathbf v_2&\cdots&\mathbf v_n
\end{bmatrix}$, then $\{\mathbf v_1,\mathbf v_2,\cdots,\mathbf v_n\}$ forms basis of $\mathbb R^n\iff A\sim I_n$, in other words, each row and each column of $A$ has a pivot.
\end{theorem}
\end{frame}

\begin{frame}[t]
\begin{definition}\label{18:57-06/07/2022}
The \textcolor{blue}{standard basis}\index{standard basis} for $\mathbb R^n$ is the set of vectors $\{\mathbf e_1,\cdots,\mathbf e_n\}$, where 
\begin{tikzpicture}
\node at (0,0) {$\mathbf e_j=\begin{bmatrix}
0\\\vdots\\1\\\vdots\\0
\end{bmatrix}$};
\draw[->] (1.1,0)--(0.8,0);
\node at (1.1,0)[right] {$j$-th entry};
\end{tikzpicture}
\end{definition}
\pause
\begin{example}
$\left\{\mathbf e_1=\begin{bmatrix}
1\\0\\0
\end{bmatrix},\mathbf e_2=\begin{bmatrix}
0\\1\\0
\end{bmatrix},\mathbf e_3=\begin{bmatrix}
0\\0\\1
\end{bmatrix}\right\}$ is the standard basis for $\mathbb R^3$\pause, and
\scalebox{0.8}{
$\begin{bmatrix}
x_1\\x_2\\x_3
\end{bmatrix}=\begin{bmatrix}
x_1\\0\\0
\end{bmatrix}+\begin{bmatrix}
0\\x_2\\0
\end{bmatrix}+\begin{bmatrix}
0\\0\\x_3
\end{bmatrix}=\pause x_1\begin{bmatrix}
1\\0\\0
\end{bmatrix}+x_2\begin{bmatrix}
0\\1\\0
\end{bmatrix}+x_3\begin{bmatrix}
0\\0\\1
\end{bmatrix}=\pause x_1\mathbf e_1+x_2\mathbf e_2+x_3\mathbf e_3$
}
\end{example}
\end{frame}

\begin{frame}[t]
\begin{example}
Suppose $\mathbf v_1=\begin{bmatrix}
1\\-1
\end{bmatrix}$, $\mathbf v_2=\begin{bmatrix}
1\\1
\end{bmatrix}$, $\mathbf e_1=\begin{bmatrix}
1\\0
\end{bmatrix}$, $\mathbf e_2=\begin{bmatrix}
0\\1
\end{bmatrix}$
\begin{center}
\begin{tikzpicture}[scale=0.8]
\clip(-4.5,-2.5) rectangle (4.5,2.5);
\draw[help lines, color=gray!30, dashed] (-4.5,-2.5) grid (4.5,2.5);
\foreach \x in {-6,-4,-2,0,2,4,6} {\draw[help lines, color=purple!30, dashed] (-5+\x,-5)--++(10,10); \draw[help lines, color=purple!30, dashed] (-5+\x,5)--++(10,-10);}
\draw[->, color=gray!80] (-4.5,0)--(4.5,0) node[right]{$x_1$};
\draw[->, color=gray!80] (0,-2.5)--(0,2.5) node[above]{$x_2$};
\coordinate (a) at (1,-1); \node at (a)[below left]{\textcolor{purple}{$\mathbf v_1$}}; \draw[->, purple, thick] (0,0)--(a);
\coordinate (b) at (1,1); \node at (b)[below right]{\textcolor{purple}{$\mathbf v_2$}}; \draw[->, purple, thick] (0,0)--(b);
\node at (1,0)[above]{$\mathbf e_1$}; \draw[->, thick] (0,0)--(1,0);
\node at (0,1)[above]{$\mathbf e_2$}; \draw[->, thick] (0,0)--(0,1);
\node at ($2*(a)+(b)$)[below]{\textcolor{orange}{$\substack{2\mathbf v_1+\mathbf v_2\\=3\mathbf e_1-\mathbf e_2}$}}; \draw[->, orange, thick] (0,0)--($2*(a)+(b)$);
\end{tikzpicture}
\end{center}
$\Span\{\mathbf v_1,\mathbf v_2, \mathbf e_1\}$ and $\Span\{\mathbf v_1,\mathbf v_2\}$ are both the plane $\mathbb R^2$, $\mathbf e_1$ is in the span of $\{\mathbf v_1,\mathbf v_2\}$ because $\mathbf e_1=\frac{1}{2}\mathbf v_1+\frac{1}{2}\mathbf v_2$. The gray grids illustrate the span of $\{\mathbf e_1,\mathbf e_2\}$ and the purple grids illustrate the span of $\{\mathbf v_1,\mathbf v_2\}$.
\end{example}
\end{frame}

\begin{frame}[t]{Exercise}
Suppose $\mathbf a_1=\begin{bmatrix}
1\\-3\\0
\end{bmatrix}$, $\mathbf a_2=\begin{bmatrix}
2\\-1\\5
\end{bmatrix}$, $\mathbf a_3=\begin{bmatrix}
1\\2\\3
\end{bmatrix}$
\begin{itemize}
\item Determine whether $\{\mathbf a_1, \mathbf a_2, \mathbf a_3\}$ forms a basis for $\mathbb R^3$.\pause
\item Without performing elementary row operations, how many solutions does $A\mathbf x=\mathbf b$ have? Where $\mathbf b=\begin{bmatrix}
1\\45\\-9
\end{bmatrix}$, what about $\mathbf b=\begin{bmatrix}
0\\0\\0
\end{bmatrix}$.
\end{itemize}
\end{frame}

\begin{frame}[t]
\begin{example}
The solution set to \systeme{x_1-x_2+x_3=1, 2x_1-x_3=1} is $\begin{cases}
x_1=\frac{1}{2}x_3+\frac{1}{2}\\
x_2=\frac{3}{2}x_3-\frac{1}{2}\\
x_3\text{ is free}
\end{cases}$\pause, which can be written as \textcolor{blue}{parametric vector form}\index{parametric vector form}
\[
\begin{bmatrix}
x_1\\x_2\\x_3
\end{bmatrix}=\begin{bmatrix}
\frac{1}{2}x_3+\frac{1}{2}\\\frac{3}{2}x_3-\frac{1}{2}\\x_3
\end{bmatrix}=\pause\begin{bmatrix}
\frac{1}{2}x_3\\\frac{3}{2}x_3\\x_3
\end{bmatrix}+\begin{bmatrix}
\frac{1}{2}\\-\frac{1}{2}\\0
\end{bmatrix}=\pause x_3\begin{bmatrix}
\frac{1}{2}\\\frac{3}{2}\\1
\end{bmatrix}+\begin{bmatrix}
\frac{1}{2}\\-\frac{1}{2}\\0
\end{bmatrix}
\]
\end{example}
\end{frame}

\begin{frame}[t]
\begin{example}
The solution set to \systeme{x_1-2x_2-x_3 + 3x_4 = 0, -2x_1 + 4x_2 + 5x_3 - 5x_4 = 3, 3x_1 - 6x_2 - 4x_3 + 8x_4 = 2} is $\begin{cases}
x_1=2x_2+16\\
x_2\text{ is free}\\
x_3=\frac{5}{2}\\
x_4=-\frac{9}{2}
\end{cases}$\pause, which can be written as
\[
\begin{bmatrix}
x_1\\x_2\\x_3\\x_4
\end{bmatrix}=\begin{bmatrix}
2x_2+16\\x_2\\\frac{5}{2}\\-\frac{9}{2}
\end{bmatrix}=\pause\begin{bmatrix}
2x_2\\x_2\\0\\0
\end{bmatrix}+\begin{bmatrix}
16\\0\\\frac{5}{2}\\-\frac{9}{2}
\end{bmatrix}=\pause x_2\begin{bmatrix}
2\\1\\0\\0
\end{bmatrix}+\begin{bmatrix}
16\\0\\\frac{5}{2}\\-\frac{9}{2}
\end{bmatrix}
\]
\end{example}
\end{frame}

\begin{frame}[t]
\begin{exercise}
Suppose the augmented matrix of a linear system is equivalent to the following matrix
\[
\begin{bmatrix}
1&1&0&2&0&3\\
0&0&1&-2&0&2\\
0&0&0&0&1&1\\
\end{bmatrix}
\]
Write down the solution set in parametric vector form
\end{exercise}
\pause
\begin{solution}
\[
\systeme*{x_1+x_2+2x_4=3, x_3-2x_4=2,x_5=1}\Rightarrow\begin{cases}
x_1=3-x_2-2x_4\\ x_2\text{ is free}\\x_3=2+2x_4\\ x_4\text{ is free}\\ x_5=1
\end{cases}
\]
\end{solution}
\end{frame}

\begin{frame}[t]
\begin{solution}
\[
\systeme*{x_1+x_2+2x_4=3, x_3-2x_4=2,x_5=1}\Rightarrow\begin{cases}
x_1=3-x_2-2x_4\\ x_2\text{ is free}\\x_3=2+2x_4\\ x_4\text{ is free}\\ x_5=1
\end{cases}
\]
So the solution in parametric vector form would be
\begin{center}
\scalebox{0.8}{
$\begin{bmatrix}
x_1\\x_2\\x_3\\x_4\\x_5
\end{bmatrix}=\begin{bmatrix}
3-x_2-2x_4\\x_2\\2+2x_4\\x_4\\1
\end{bmatrix}=\pause\begin{bmatrix}
3\\0\\2\\0\\1
\end{bmatrix}+\begin{bmatrix}
-x_2\\x_2\\0\\0\\0
\end{bmatrix}+\begin{bmatrix}
-2x_4\\0\\2x_4\\x_4\\0
\end{bmatrix}=\pause\begin{bmatrix}
3\\0\\2\\0\\1
\end{bmatrix}+x_2\begin{bmatrix}
-1\\1\\0\\0\\0
\end{bmatrix}+x_4\begin{bmatrix}
-2\\0\\2\\1\\0
\end{bmatrix}$
}
\end{center}
\end{solution}
\end{frame}

\begin{frame}[t]
\begin{definition}
A linear system is \textcolor{blue}{homogeneous}\index{homogeneous} if it has matrix equation $A\mathbf x=\mathbf 0$ (note that this always have the zero solution, called the \textcolor{blue}{trivial solution}\index{trivial solution}).
\end{definition}
\pause
\begin{theorem}
Suppose $\begin{bmatrix}
A&\mathbf b
\end{bmatrix}\sim\begin{bmatrix}
U&\mathbf d
\end{bmatrix}$ is the RREF, then $U$ will be the RREF of $A$. The solutions of $A\mathbf x=\mathbf0$ and $A\mathbf x=\mathbf b$ differs by $\mathbf d$, i.e.
\[
\mathbf d+\{\textsf{solutions of }A\mathbf x=\mathbf0\}=\{\textsf{solutions of }A\mathbf x=\mathbf b\}
\]\pause
Geometrically speaking, the solution set of $A\mathbf x=\mathbf b$ is the hyperplane translated from the hyperplane through the origin (solution set of $A\mathbf x=\mathbf0$) by $\mathbf d$.
\end{theorem}
\end{frame}

\begin{frame}[t]
\begin{example}
$x_1-2x_2=2$ has augmented matrix $\begin{bmatrix}
1&-2&2
\end{bmatrix}$ which is already an RREF, we have solution set $\begin{cases}
x_1=2+2x_2\\
x_2\text{ is free}
\end{cases}$ which is $\begin{bmatrix}
2\\0
\end{bmatrix}+x_2\begin{bmatrix}
2\\1
\end{bmatrix}$ in parametric vector form\pause. Consider its corresponding homogeneous equation $x_1-2x_2=0$, we have solution set $\begin{cases}
x_1=2x_2\\
x_2\text{ is free}
\end{cases}$ which is $x_2\begin{bmatrix}
2\\1
\end{bmatrix}$ in parametric vector form.
\vspace{-5mm}
\begin{center}
\begin{tikzpicture}[scale=0.8]
\begin{scope}
\clip(-4.5,-2.5) rectangle (4.5,2.5);
\draw[help lines, color=gray!30, dashed] (-4.5,-2.5) grid (4.5,2.5);
\draw[orange] (-6,-3)--(6,3);
\draw[purple] (-6,-4)--(6,2);
\filldraw[blue] (2,0) node[below]{$\begin{bmatrix}
2\\0
\end{bmatrix}$} circle (1pt);
\draw[->,blue] (0,0)--(2,1) node[above]{$\begin{bmatrix}
2\\1
\end{bmatrix}$};
\draw[->,blue] (2,0)--++(2,1);
\node at (0,-1.3)[below right] {$x_1-2x_2=-2$};
\node at (0,0)[above left] {$x_1-2x_2=0$};
\end{scope}
\draw[->, color=gray!80] (-4.5,0)--(4.5,0) node[right]{$x_1$};
\draw[->, color=gray!80] (0,-2.5)--(0,2.5) node[above]{$x_2$};
\draw[dashed,blue] (0,0)--(2,0);
\draw[dashed,blue] (2,1)--(4,1);
\end{tikzpicture}
\end{center}
\end{example}
\end{frame}

\begin{frame}[t]
\begin{example}
Consider homogeneous linear system $3x_1+x_2-x_3=0$, and non-homogeneous linear system $3x_1+x_2-x_3=3$. The parametric vector form of the solution sets to both are\pause
\[
\begin{bmatrix}
x_1\\x_2\\x_3
\end{bmatrix}=x_2\begin{bmatrix}
-\frac{1}{3}\\1\\0
\end{bmatrix}+x_3\begin{bmatrix}
\frac{1}{3}\\0\\1
\end{bmatrix}
\]
\[
\begin{bmatrix}
x_1\\x_2\\x_3
\end{bmatrix}=x_2\begin{bmatrix}
-\frac{1}{3}\\1\\0
\end{bmatrix}+x_3\begin{bmatrix}
\frac{1}{3}\\0\\1
\end{bmatrix}+\begin{bmatrix}
1\\0\\0
\end{bmatrix}
\]
\end{example}
\end{frame}

\begin{frame}[t]
\begin{center}
\begin{tikzpicture}[scale=0.85]
\definecolor{color1}{rgb}{255,0,0}
\definecolor{color2}{rgb}{0,0,255}
\definecolor{color3}{rgb}{255,0,255}
\def\XMAX{3.5}
\def\YMAX{3.5}
\def\ZMAX{3.5}
\newcommand{\planeEq}[2]{##2/3-##1/3}
\newcommand{\planeEqq}[2]{##2/3-##1/3+1}
\begin{scope}[blend mode=multiply, rotate around x=-90, rotate around z=30]
\draw [color1, fill=color1!20] plot (\planeEq{-\YMAX}{-\ZMAX},-\YMAX,-\ZMAX,)node[below left]{$3x_1+x_2-x_3=0$}--(\planeEq{-\YMAX}{\ZMAX},-\YMAX,\ZMAX)--(\planeEq{\YMAX}{\ZMAX},\YMAX,\ZMAX)--(\planeEq{\YMAX}{-\ZMAX},\YMAX,-\ZMAX)--cycle;
\draw [color2, fill=color2!20] plot (\planeEqq{-\YMAX}{-\ZMAX},-\YMAX,-\ZMAX,)node[below right]{$3x_1+x_2-x_3=3$}--(\planeEqq{-\YMAX}{\ZMAX},-\YMAX,\ZMAX)--(\planeEqq{\YMAX}{\ZMAX},\YMAX,\ZMAX)--(\planeEqq{\YMAX}{-\ZMAX},\YMAX,-\ZMAX)--cycle;
\draw[->] (-\XMAX,0,0)--(\XMAX,0,0) node[right]{$x_1$};
\draw[->] (0,-\YMAX,0)--(0,\YMAX,0) node[above]{$x_2$};
\draw[->] (0,0,-\ZMAX)--(0,0,\ZMAX) node[below left]{$x_3$};
\coordinate (d) at (1,0,0);
\coordinate (o) at (0,0,0);
\coordinate (v1) at (-1/3,1,0);
\coordinate (v2) at (1/3,0,1);
\draw[->, ultra thick, color1] (o)--(v1)node[above left]{$\mathbf v_1$};
\draw[->, ultra thick, color1] (o)--(v2)node[above]{$\mathbf v_2$};
\draw[->, ultra thick, color2] ($(o)+(d)$)--($(d)+(v1)$)node[above right]{$\mathbf v_1$};
\draw[->, ultra thick, color2] ($(o)+(d)$)--($(d)+(v2)$)node[above]{$\mathbf v_2$};
\draw[->, dotted, ultra thick] (o)--(d)node[below]{$\mathbf d$};
\end{scope}
\end{tikzpicture}
\end{center}
\end{frame}

\end{document}