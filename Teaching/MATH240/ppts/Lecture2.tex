\documentclass{beamer}
\setbeamercovered{}

\usepackage{bookman}
\usepackage{newtxtext}
\usepackage{amssymb}
\usepackage{amsmath}
\usepackage{amsfonts}
\usepackage{mathrsfs}
\usepackage{mathtools}
\usepackage{hyperref}
\usepackage{pifont}
\usepackage{systeme}
\usepackage{subcaption}
\usepackage{bm} % Bold
\usepackage{bbm} % Can use \mathbbm{1}
\usepackage{ifthen} % Use if then else
\usepackage{shuffle} % Use shuffle product
\usepackage{minted} % For listing codes
\usepackage{tikz}
\usepackage{pgfplots}

\makeatletter
\newcommand*{\Relbarfill@}{\arrowfill@\Relbar\Relbar\Relbar}
% \newcommand*{\relbarfill@}{\arrowfill@\relbar-\relbar}
\newcommand*{\xequal}[2][]{\ext@arrow 0055\Relbarfill@{#1}{#2}}
% \newcommand*{\xdash}[2][]{\ext@arrow 0055\relbarfill@{#1}{#2}}
\makeatother

\newlength{\simlength}

\newcommand{\xsim}[1]{
\settowidth{\simlength}{$#1$}
\mathrel{\overset{#1}{\resizebox{\simlength}{\height}{\sim}}}
}

\mode<presentation> {
    \usetheme{Madrid}
    \usecolortheme{beaver}
    \usefonttheme{professionalfonts} 
    \setbeamertemplate{navigation symbols}{}
    \setbeamertemplate{caption}[numbered]
}

\theoremstyle{definition}
\newtheorem{proposition}[theorem]{Proposition}
\newtheorem{construction}[theorem]{Construction}
\newtheorem{deduction}[theorem]{Deduction}
\newtheorem{exercise}[theorem]{Exercise}
\newtheorem{conjecture}[theorem]{Conjecture}

\theoremstyle{remark}
\newtheorem*{remark}{Remark}
\newtheorem*{recall}{Recall}
\newtheorem*{question}{Question}
\newtheorem*{answer}{Answer}

\newcounter{saveenumi}
\newcommand{\seti}{\setcounter{saveenumi}{\value{enumi}}}
\newcommand{\conti}{\setcounter{enumi}{\value{saveenumi}}}
\resetcounteronoverlays{saveenumi}

\AtBeginSection[]{
    \begin{frame}
    \centering
    \begin{beamercolorbox}[center, shadow=true, rounded=true]{title}
    \usebeamerfont{title}\insertsectionhead\par%
    \end{beamercolorbox}
    \end{frame}
}

\title{MATH240 - Introduction to Linear Algebra}
\author{Haoran Li}
\institute[UMD]{University of Maryland, College Park}
\date{Summer 2024}


\begin{document}

\maketitle

\section{Lecture 2 - Matrices and row echelon form}

\begin{frame}[t]
\begin{definition}
A $m$ by $n$ (or $m\times n$) \textcolor{blue}{matrix}\index{matrix} is a rectangular array of numbers with $m$ rows and $n$ columns
\begin{equation}\label{eq: general m by n matrix}
\begin{bmatrix}
a_{11}&a_{12}&\cdots&a_{1n}\\
a_{21}&a_{22}&\cdots&a_{2n}\\
\vdots&\vdots&&\vdots\\
a_{m1}&a_{m2}&\cdots&a_{mn}
\end{bmatrix}
\end{equation}
We use the \textcolor{blue}{$(i,j)$-th entry} to mean the entry on the $i$-th row and $j$-column (i.e. $a_{ij}$).
\end{definition}
\end{frame}

\begin{frame}[t]
\begin{definition}
A matrix is
\begin{itemize}
\item a \textcolor{blue}{zero matrix}\index{zero matrix} is a matrix with all entries zeros.
\vspace{-3mm}
\[
\begin{bmatrix}
0&0&\cdots&0\\
0&0&\cdots&0\\
\vdots&\vdots&&\vdots\\
0&0&\cdots&0
\end{bmatrix}
\]
\pause
\item a \textcolor{blue}{square matrix}\index{square matrix} is a matrix with the same number of rows and columns, i.e. $m=n$.
\vspace{-3mm}
\[
\begin{bmatrix}
a_{11}&a_{12}&\cdots&a_{1n}\\
a_{21}&a_{22}&\cdots&a_{2n}\\
\vdots&\vdots&&\vdots\\
a_{n1}&a_{n2}&\cdots&a_{nn}
\end{bmatrix}
\]
\pause
\item a \textcolor{blue}{row vector}\index{row vector} if it only has one row, i.e. $m=1$.
\vspace{-3mm}
\[
\begin{bmatrix}
a_{1}&a_{2}&\cdots&a_{n}
\end{bmatrix}
\]
\end{itemize}
\end{definition}
\end{frame}

\begin{frame}[t]
\begin{definition}
\begin{itemize}
\item a \textcolor{blue}{(column) vector}\index{vector} if it only has one column, i.e. $n=1$.
\[
\begin{bmatrix}
a_{1}\\a_{2}\\\vdots\\a_{m}
\end{bmatrix}
\]
\pause
\item the \textcolor{blue}{identity matrix}\index{identity matrix} if it is a square matrix with diagonal elements 1's, and 0's otherwise. Here the diagonal are the $(i,i)$-th entries.
\[
\begin{bmatrix}
1&0&\cdots&0\\
0&1&\cdots&0\\
\vdots&\vdots&&\vdots\\
0&0&\cdots&1
\end{bmatrix}
\]
\end{itemize}
\end{definition}
\end{frame}

\begin{frame}[t]
\begin{definition}
Soon we will be getting tired of writing all these equations in the linear system
\[
\begin{cases}
\,\quad a_{11}x_1+a_{12}x_2+a_{13}x_3+\cdots+a_{1n}x_{n}=b_1 \\
\,\quad a_{21}x_1+a_{22}x_2+a_{23}x_3+\cdots+a_{2n}x_{n}=b_2 \\
\quad \,a_{31}x_1+a_{32}x_2+a_{33}x_3+\cdots+a_{3n}x_{n}=b_3 \\
\,\,\,\qquad\qquad\qquad\qquad\qquad\qquad\qquad\qquad\quad\vdots \\
a_{m1}x_1+a_{m2}x_2+a_{m3}x_3+\cdots+a_{mn}x_{n}=b_m
\end{cases}
\]
Instead we write down its \textcolor{blue}{augmented matrix}\index{augmented matrix}, which is obtained by omitting $x_i$'s, pluses, and equal signs.
\[
\begin{bmatrix}
a_{11}&a_{12}&a_{13}&\cdots&a_{1n}&b_1\\
a_{21}&a_{22}&a_{23}&\cdots&a_{2n}&b_2\\
a_{31}&a_{32}&a_{33}&\cdots&a_{1n}&b_3\\
\vdots&\vdots&\vdots&&\vdots\\
a_{m1}&a_{m2}&a_{m3}&\cdots&a_{mn}&b_m
\end{bmatrix}
\]
\end{definition}
\end{frame}

\begin{frame}[t]
\begin{definition}
If we delete the last column, we will get the \textcolor{blue}{coefficient matrix}\index{coefficient matrix}
\[
\begin{bmatrix}
a_{11}&a_{12}&a_{13}&\cdots&a_{1n}\\
a_{21}&a_{22}&a_{23}&\cdots&a_{2n}\\
a_{31}&a_{32}&a_{33}&\cdots&a_{1n}\\
\vdots&\vdots&\vdots&&\vdots\\
a_{m1}&a_{m2}&a_{m3}&\cdots&a_{mn}
\end{bmatrix}
\]
\end{definition}
\end{frame}

\begin{frame}[t]
\begin{example}\hfill
\begin{itemize}
\item For \systeme{x_1+x_2=10, 2x_1+4x_2=26}, its augmented matrix and coefficient matrix are
\[
\begin{bmatrix}
1&1&10\\
2&4&26
\end{bmatrix},\begin{bmatrix}
1&1\\
2&4
\end{bmatrix}
\]
\pause
\item For \systeme{x_1-x_2+x_3=1, 2x_1-x_3=1,x_1+x_2+x_3=3}, its augmented matrix and coefficient matrix are
\[
\begin{bmatrix}
1&-1&1&1\\
2&0&-1&1\\
1&1&1&3
\end{bmatrix},\begin{bmatrix}
1&-1&1\\
2&0&-1\\
1&1&1
\end{bmatrix}
\]
\pause
\item In general, a linear system of $m$ equations in $n$ variables has a $m$ by $(n+1)$ augmented matrix and a $m$ by $n$ coefficient matrix.
\end{itemize}
\end{example}
\end{frame}

\begin{frame}[t]
\begin{definition}
Inspired by Gaussian elimination, we define the following three \textcolor{blue}{elementary row operations}\index{elementary row operations}
\pause
\begin{itemize}
\item \textcolor{blue}{Replacement}\index{Row replacement}: Replace one row by the sum of itself and a multiple of another row.
\pause
\item \textcolor{blue}{Interchange}\index{Row interchange}: Interchange two rows.
\pause
\item \textcolor{blue}{Scaling}\index{Row scaling}: Multiply all entries in a row by a \textit{nonzero} constant.
\end{itemize}
\pause
We say matrices $A,B$ are \textcolor{blue}{row equivalent}\index{row equivalence} ($A\sim B$) if $B$ can obtained by applying a sequence of elementary row operations to $A$ (or vise versa).
\end{definition}
\end{frame}

\begin{frame}[t]
\begin{example}
Let's rewrite the process of Gaussian elimination in solving \systeme{x_1+x_2=10, 2x_1+4x_2=26}
\begin{align*}
&\begin{bmatrix}
1&1&10\\
2&4&26
\end{bmatrix}\xsim{R2\rightarrow R2-2R1}
\begin{bmatrix}
1&1&10\\
0&2&6
\end{bmatrix}\xsim{R2\rightarrow\frac{R2}{2}}
\begin{bmatrix}
1&1&10\\
0&1&3
\end{bmatrix}\\
&\xsim{R1\rightarrow R1-R2}
\begin{bmatrix}
1&0&7\\
0&1&3
\end{bmatrix}
\end{align*}
\end{example}
\end{frame}

\begin{frame}[t]
\begin{example}\label{18:10-06/01/2022} 
Solve \systeme{x_1-x_2+x_3=1, 2x_1-x_3=1,x_1+x_2+x_3=3} with augmented matrix.\\

\resizebox{\textwidth}{!}{
\begin{minipage}{\textwidth}
\begin{align*}
&\begin{bmatrix}
1&-1&1&1\\
2&0&-1&1\\
1&1&1&3\\
\end{bmatrix}\xsim{\substack{R2\rightarrow R2-2R1\\R3\rightarrow R3-R1}}\begin{bmatrix}
1&-1&1&1\\
0&2&-3&-1\\
0&2&0&2\\
\end{bmatrix}\xsim{R3\rightarrow R3-R2}\begin{bmatrix}
1&-1&1&1\\
0&2&-3&-1\\
0&0&3&3\\
\end{bmatrix}\\
&\xsim{R3\to\frac{R3}{3}}\begin{bmatrix}
1&-1&1&1\\
0&2&-3&-1\\
0&0&1&1\\
\end{bmatrix}\xsim{\substack{R1\rightarrow R1-R3\\R2\rightarrow R2+3R3}}\begin{bmatrix}
1&-1&0&0\\
0&2&0&2\\
0&0&1&1\\
\end{bmatrix}\\
&\xsim{R2\to\frac{R2}{2}}\begin{bmatrix}
1&-1&0&0\\
0&1&0&1\\
0&0&1&1\\
\end{bmatrix}\xsim{R1\rightarrow R1+R2}\begin{bmatrix}
1&0&0&1\\
0&1&0&1\\
0&0&1&1\\
\end{bmatrix}
\end{align*}
\end{minipage}
}\\

This gives the unique solution \systeme*{x_1=1, x_2=1, x_3=1}
\end{example}
\end{frame}

\begin{frame}[t]
\begin{definition}\hfill
\begin{itemize}
\item A \textcolor{blue}{leading entry}\index{leading entry} of a row refers to the leftmost nonzero entry (in a nonzero row).
\pause
\item A matrix is of \textcolor{blue}{row echelon form (REF)}\index{row echelon form (REF)} if it is of a ``staircase shape".
\[
\overset{\text{REF}}{
\begin{bmatrix}
\blacksquare&*&*&*&*&*&*&*\\
0&0&\blacksquare&*&*&*&*&*\\
0&0&0&0&\blacksquare&*&*&*\\
0&0&0&0&0&\blacksquare&*&*\\
\end{bmatrix}
}
\]
$\blacksquare$ are the leading entries, $*$ are some unknown numbers.
\pause
\item The leading entries of an REF matrix are called \textcolor{blue}{pivots}\index{pivot}.
\pause
\item The position of pivots are called \textcolor{blue}{pivot positions}\index{pivot positions}.
\pause
\item The column where pivots are in are called \textcolor{blue}{pivot columns}\index{pivot columns}.
\end{itemize}
\end{definition}
\end{frame}

\begin{frame}[t]
\begin{definition}
\begin{itemize}
\item An REF of \textcolor{blue}{reduced row echelon form (RREF)}\index{reduced row echelon form, RREF} if all its pivots are 1's and in each pivot column, every entry except the pivot are 0's.
\end{itemize}
\[
\overset{\text{RREF}}{\begin{bmatrix}
1&*&0&*&0&0&*&*\\
0&0&1&*&0&0&*&*\\
0&0&0&0&1&0&*&*\\
0&0&0&0&0&1&*&*\\
\end{bmatrix}
}
\]
\end{definition}
\pause
\begin{example}
In Example~\ref{18:10-06/01/2022}, $\begin{bmatrix}
1&-1&1&1\\
0&2&-3&-1\\
0&2&0&2\\
\end{bmatrix}$ is not an REF.
\pause
$\begin{bmatrix}
1&-1&1&1\\
0&2&-3&-1\\
0&0&3&3\\
\end{bmatrix}$ is an REF, but not an RREF.
\pause
$\begin{bmatrix}
1&0&0&1\\
0&1&0&1\\
0&0&1&1\\
\end{bmatrix}$ is an RREF.
\end{example}
\end{frame}

\begin{frame}[t]
\begin{theorem}\label{18:46-06/08/2022}
Every matrix is row equivalent to some REF matrix (which is not in general unique), but it is row equivalent to some unique RREF matrix.
\end{theorem}
\pause
\begin{example}
In Example~\ref{18:10-06/01/2022}, $\begin{bmatrix}
1&-1&1&1\\
0&2&-3&-1\\
0&0&3&3\\
\end{bmatrix}$ is an REF that is row equivalent to the original matrix $\begin{bmatrix}
1&-1&1&1\\
2&0&-1&1\\
1&1&1&3\\
\end{bmatrix}$, and $\begin{bmatrix}
1&0&0&1\\
0&1&0&1\\
0&0&1&1\\
\end{bmatrix}$ is its unique row equivalent RREF.
\end{example}
\end{frame}

\begin{frame}[t]
\begin{example}\label{ex: no solution system}
Solve \systeme{x_1-x_2+x_3=1, 2x_1-x_3=1,x_1+x_2-2x_3=1} with augmented matrix.
\pause
\begin{align*}
&\begin{bmatrix}
1&-1&1&1\\
2&0&-1&1\\
1&1&-2&1\\
\end{bmatrix}\xsim{\substack{R2\rightarrow R2-2R1\\R3\rightarrow R3-R1}}\begin{bmatrix}
1&-1&1&1\\
0&2&-3&-1\\
0&2&-3&0\\
\end{bmatrix}\\
&\xsim{R3\rightarrow R3-R2}\begin{bmatrix}
1&-1&1&1\\
0&2&-3&-1\\
0&0&0&1\\
\end{bmatrix}
\end{align*}
\pause
You might notice that the last row represents $0x_1+0x_2+0x_3=1$, this is a contradiction, therefore the linear system is inconsistent.
\end{example}
\end{frame}

\begin{frame}[t]
\begin{example}\label{09:47-06/02/2022}
Solve \systeme{x_1-x_2+x_3=1, 2x_1-x_3=1}, with augmented matrix.

\begin{align*}
&\begin{bmatrix}
1&-1&1&1\\
2&0&-1&1\\
\end{bmatrix}\xsim{R2\rightarrow R2-2R1}\begin{bmatrix}
1&-1&1&1\\
0&2&-3&-1\\
\end{bmatrix}\\
&\xsim{R2\to\frac{R2}{2}}\begin{bmatrix}
1&-1&1&1\\
0&1&-\frac{3}{2}&-\frac{1}{2}\\
\end{bmatrix}\xsim{R1\rightarrow R1+R2}\begin{bmatrix}
1&0&-\frac{1}{2}&\frac{1}{2}\\
0&1&-\frac{3}{2}&-\frac{1}{2}\\
\end{bmatrix}
\end{align*}
\pause
This gives the solution set
\begin{equation}
\systeme{x_1-\frac{1}{2}x_3=\frac{1}{2}, x_2-\frac{3}{2}x_3=\frac{1}{2}}\Rightarrow\systeme*{x_1=\frac{1}{2}x_3+\frac{1}{2}, x_2=\frac{3}{2}x_3-\frac{1}{2}}
\end{equation}
\end{example}
\end{frame}

\begin{frame}[t]
Let's formalize these as \textcolor{blue}{row reduction algorithm}\index{row reduction algorithm}
\pause
\begin{enumerate}
\item Begin with the leftmost nonzero column. This is a pivot column. The pivot position should be at the top. 
\pause
\item Select a nonzero entry in the pivot column as a pivot. If necessary, interchange rows to move this entry into the pivot position.
\pause
\item Use row replacement operations to create zeros in all positions below the pivot. 
\pause
\item Cover (or ignore) the rows containing the pivot positions. Apply Steps 1-3 to the rows that remains. Repeat the process until you are left with an REF.
\pause
\item Beginning with the rightmost pivot and working upward and to the left, create zeros above each pivot. If a pivot is not 1, make it 1 by a scaling operation. 
\end{enumerate}
\pause
Steps 1-4 are call \textcolor{blue}{forward phase}, after which you get an REF.
\pause
Step 5 is called \textcolor{blue}{backward phase}, after which you get the RREF.
\end{frame}

\begin{frame}[t]
\begin{example}
Consider the augmented matrix $\begin{bmatrix}
0&-3&-6&4&9\\
-1&-2&-1&3&1\\
-2&-3&0&3&-1\\
1&4&5&-9&-7\\
\end{bmatrix}$
\begin{itemize}\item Forward phase\end{itemize}
\begin{overlayarea}{\textwidth}{0.5\textheight}
\only<1>{
\[
\begin{bmatrix}
0&-3&-6&4&9\\
-1&-2&-1&3&1\\
-2&-3&0&3&-1\\
1&4&5&-9&-7\\
\end{bmatrix}\xsim{\substack{R1\leftrightarrow R4\\\text{Step 1,2}}}\begin{bmatrix}
1&4&5&-9&-7\\
-1&-2&-1&3&1\\
-2&-3&0&3&-1\\
0&-3&-6&4&9\\
\end{bmatrix}
\]
}\only<2>{
\[
\begin{bmatrix}
1&4&5&-9&-7\\
-1&-2&-1&3&1\\
-2&-3&0&3&-1\\
0&-3&-6&4&9\\
\end{bmatrix}\xsim{\substack{R2\rightarrow R2+R1\\R3\rightarrow R3+2R1\\\text{Step 3}}}\begin{bmatrix}
1&4&5&-9&-7\\
0&2&4&-6&-6\\
0&5&10&-15&-15\\
0&-3&-6&4&9\\
\end{bmatrix}
\]
}\only<3>{
\[
\begin{bmatrix}
1&4&5&-9&-7\\
0&2&4&-6&-6\\
0&5&10&-15&-15\\
0&-3&-6&4&9\\
\end{bmatrix}\xsim{\substack{R3\rightarrow R3-\frac{5}{2}R2\\R4\rightarrow R4+\frac{3}{2}R2\\\text{Step 4,1,2,3}}}\begin{bmatrix}
1&4&5&-9&-7\\
0&2&4&-6&-6\\
0&0&0&0&0\\
0&0&0&-5&0\\
\end{bmatrix}
\]
}\only<4>{
\[
\begin{bmatrix}
1&4&5&-9&-7\\
0&2&4&-6&-6\\
0&0&0&0&0\\
0&0&0&-5&0\\
\end{bmatrix}\xsim{\substack{R3\leftrightarrow R4\\\text{Step 4,1}}}\begin{bmatrix}
1&4&5&-9&-7\\
0&2&4&-6&-6\\
0&0&0&-5&0\\
0&0&0&0&0\\
\end{bmatrix}
\]
}
\end{overlayarea}
\end{example}
\end{frame}

\begin{frame}[t]
\begin{example}
\begin{itemize}\item Backward phase\end{itemize}
\begin{align*}
&\begin{bmatrix}
1&4&5&-9&-7\\
0&2&4&-6&-6\\
0&0&0&-5&0\\
0&0&0&0&0\\
\end{bmatrix}\xsim{\substack{R3\rightarrow R3/(-5)\\\text{Step 5}}}\begin{bmatrix}
1&4&5&-9&-7\\
0&2&4&-6&-6\\
0&0&0&1&0\\
0&0&0&0&0\\
\end{bmatrix}\\
&\xsim{\substack{R1\rightarrow R1+9R3\\R2\rightarrow R2+6R3\\\text{Step 5}}}\begin{bmatrix}
1&4&5&0&-7\\
0&2&4&0&-6\\
0&0&0&1&0\\
0&0&0&0&0\\
\end{bmatrix}\xsim{\substack{R2\rightarrow R2/2\\\text{Step 5}}}\begin{bmatrix}
1&4&5&0&-7\\
0&1&2&0&-3\\
0&0&0&1&0\\
0&0&0&0&0\\
\end{bmatrix}\\
&\xsim{\substack{R1\rightarrow R1-4R2\\\text{Step 5}}}\begin{bmatrix}
1&0&-3&0&5\\
0&1&2&0&-3\\
0&0&0&1&0\\
0&0&0&0&0\\
\end{bmatrix}
\end{align*}
\end{example}
\end{frame}

\begin{frame}[t]
\begin{definition}
The variables corresponding to pivot columns in a matrix are called \textcolor{blue}{basic variables}\index{basic variable}, 
the other variables are called \textcolor{blue}{free variables}\index{free variable}. 
In a solution set, basic variables are expressed in terms of free variables, and a free variable can take any value.
\end{definition}
\pause
\begin{example}
In Example~\ref{09:47-06/02/2022}, $x_1,x_2$ are basic variables and $x_3$ is a free variable. And we formally write our solution set as $\begin{cases}
x_1=\frac{1}{2}x_3+\frac{1}{2} \\
x_2=\frac{3}{2}x_3-\frac{1}{2} \\
x_3\text{ is free}
\end{cases}$
\end{example}
\end{frame}

\begin{frame}[t]
\begin{exercise}\label{08:47-06/06/2022}
Find the general solution of the system \systeme{x_1-2x_2-x_3 + 3x_4 = 0, -2x_1 + 4x_2 + 5x_3 - 5x_4 = 3, 3x_1 - 6x_2 - 4x_3 + 8x_4 = 2}
\end{exercise}
\end{frame}

\begin{frame}[t]
\begin{solution}
\resizebox{\textwidth}{!}{
\begin{minipage}{\textwidth}
\begin{align*}
&\begin{bmatrix}
1&-2&-1&3&0\\
-2&4&5&-5&3\\
3&-6&-4&8&2\\
\end{bmatrix}\xsim{\substack{R2\rightarrow R2+2R1\\R3\rightarrow R3-3R1}}\begin{bmatrix}
1&-2&-1&3&0\\
0&0&3&1&3\\
0&0&-1&-1&2\\
\end{bmatrix}\xsim{R3\to(-1)\cdot R3}\begin{bmatrix}
1&-2&-1&3&0\\
0&0&3&1&3\\
0&0&1&1&-2\\
\end{bmatrix}\\
&\xsim{R2\rightarrow R2-3R3}\begin{bmatrix}
1&-2&-1&3&0\\
0&0&0&-2&9\\
0&0&1&1&-2\\
\end{bmatrix}\xsim{R2\leftrightarrow R3}\begin{bmatrix}
1&-2&-1&3&0\\
0&0&1&1&-2\\
0&0&0&-2&9\\
\end{bmatrix}\\
&\xsim{R3\to\frac{R3}{-2}}\begin{bmatrix}
1&-2&-1&3&0\\
0&0&1&1&-2\\
0&0&0&1&-\frac{9}{2}\\
\end{bmatrix}\xsim{\substack{R2\rightarrow R2-R3\\R1\rightarrow R1-3R3}}\begin{bmatrix}
1&-2&-1&0&\frac{27}{2}\\
0&0&1&0&\frac{5}{2}\\
0&0&0&1&-\frac{9}{2}\\
\end{bmatrix}\xsim{R1\rightarrow R1+R2}\begin{bmatrix}
1&-2&0&0&16\\
0&0&1&0&\frac{5}{2}\\
0&0&0&1&-\frac{9}{2}\\
\end{bmatrix}
\end{align*}
\end{minipage}
}
\pause
\newline\newline\newline
Write this as solution set, we get
\[
\systeme*{x_1-2x_2=16, x_3=\frac{5}{2}, x_4=-\frac{9}{2}}\Rightarrow
\begin{cases}
x_1=2x_2+16\\
x_2\text{ is free}\\
x_3=\frac{5}{2}\\
x_4=-\frac{9}{2}
\end{cases}
\]
\end{solution}
\end{frame}

\begin{frame}[t]
\begin{theorem}\label{14:16-06/03/2022}
Suppose the augmented matrix of a linear system is $\begin{bmatrix}
A&\mathbf b
\end{bmatrix}$, and its RREF is $\begin{bmatrix}
U&\mathbf d
\end{bmatrix}$, then the linear system has
\pause
\begin{itemize}
\item no solutions $\iff \mathbf d$ is a pivot column, i.e. contains a pivot.
\pause
\item has solutions $\iff\mathbf d$ is not a pivot column
\pause
\begin{itemize}
\item a unique solution $\iff$ every column of $U$ is a pivot column.
\pause
\item infinitely many solutions $\iff$ some columns of $U$ is not a pivot column.
\end{itemize}
\end{itemize}
\end{theorem}
\end{frame}

\begin{frame}[t]
\begin{example}\hfill
\begin{itemize}
\item In Example~\ref{ex: no solution system}, the linear system has no solutions since
\[
\begin{bmatrix}
A&\mathbf b
\end{bmatrix}=
\begin{bmatrix}
1&-1&1&1\\
2&0&-1&1\\
1&1&-2&1\\
\end{bmatrix}\sim\begin{bmatrix}
1&-1&1&1\\
0&2&-3&-1\\
0&0&0&1\\
\end{bmatrix}=\begin{bmatrix}
U&\mathbf d
\end{bmatrix}
\]
\item In Example~\ref{18:10-06/01/2022}, the linear system has a unique solution since
\[
\begin{bmatrix}
A&\mathbf b
\end{bmatrix}=\begin{bmatrix}
1&−1 &1& 1\\
2&0 &−1 &1\\
1&1&1 &3
\end{bmatrix}\sim\begin{bmatrix}
1&0&0&1\\
0&1&0&1\\
0&0&1&1\\
\end{bmatrix}=\begin{bmatrix}
U&\mathbf d
\end{bmatrix}
\]
\item In Example~\ref{08:47-06/06/2022}, the linear system has infinitely many solutions since
\[
\begin{bmatrix}
A&\mathbf b
\end{bmatrix}=\begin{bmatrix}
1&-2&-1&3&0\\
-2&4&5&-5&3\\
3&-6&-4&8&2\\
\end{bmatrix}\sim\begin{bmatrix}
1&-2&0&0&16\\
0&0&1&0&\frac{5}{2}\\
0&0&0&1&-\frac{9}{2}\\
\end{bmatrix}=\begin{bmatrix}
U&\mathbf d
\end{bmatrix}
\]
\end{itemize}
\end{example}
\end{frame}

\begin{frame}[t]{Exercise}
Find the general solutions of the system with given augmented matrix, name the pivot columns, pivot positions, basic and free variables.
\pause
\newline\newline\newline
$\begin{bmatrix}
0&1&-6&5\\
1&-2&7&-4\\
\end{bmatrix}$
\end{frame}

\begin{frame}[t]{Exercise}
$\begin{bmatrix}
1&-7&0&6&5\\
0&0&1&-2&-3\\
-1&7&-4&2&7
\end{bmatrix}$
\end{frame}

\begin{frame}[t]
\begin{question}
How does the size of the augmented matrix affect the solution set?
\end{question}
\end{frame}

\end{document}