\documentclass{beamer}
\setbeamercovered{}

\usepackage{bookman}
\usepackage{newtxtext}
\usepackage{amssymb}
\usepackage{amsmath}
\usepackage{amsfonts}
\usepackage{mathrsfs}
\usepackage{mathtools}
\usepackage{hyperref}
\usepackage{pifont}
\usepackage{systeme}
\usepackage{subcaption}
\usepackage{bm} % Bold
\usepackage{bbm} % Can use \mathbbm{1}
\usepackage{ifthen} % Use if then else
\usepackage{shuffle} % Use shuffle product
\usepackage{minted} % For listing codes
\usepackage{tikz}
\usepackage{pgfplots}

\makeatletter
\newcommand*{\Relbarfill@}{\arrowfill@\Relbar\Relbar\Relbar}
% \newcommand*{\relbarfill@}{\arrowfill@\relbar-\relbar}
\newcommand*{\xequal}[2][]{\ext@arrow 0055\Relbarfill@{#1}{#2}}
% \newcommand*{\xdash}[2][]{\ext@arrow 0055\relbarfill@{#1}{#2}}
\makeatother

\newlength{\simlength}

\newcommand{\xsim}[1]{
\settowidth{\simlength}{$#1$}
\mathrel{\overset{#1}{\resizebox{\simlength}{\height}{\sim}}}
}

\mode<presentation> {
    \usetheme{Madrid}
    \usecolortheme{beaver}
    \usefonttheme{professionalfonts} 
    \setbeamertemplate{navigation symbols}{}
    \setbeamertemplate{caption}[numbered]
}

\theoremstyle{definition}
\newtheorem{proposition}[theorem]{Proposition}
\newtheorem{construction}[theorem]{Construction}
\newtheorem{deduction}[theorem]{Deduction}
\newtheorem{exercise}[theorem]{Exercise}
\newtheorem{conjecture}[theorem]{Conjecture}

\theoremstyle{remark}
\newtheorem*{remark}{Remark}
\newtheorem*{recall}{Recall}
\newtheorem*{question}{Question}
\newtheorem*{answer}{Answer}

\newcounter{saveenumi}
\newcommand{\seti}{\setcounter{saveenumi}{\value{enumi}}}
\newcommand{\conti}{\setcounter{enumi}{\value{saveenumi}}}
\resetcounteronoverlays{saveenumi}

\AtBeginSection[]{
    \begin{frame}
    \centering
    \begin{beamercolorbox}[center, shadow=true, rounded=true]{title}
    \usebeamerfont{title}\insertsectionhead\par%
    \end{beamercolorbox}
    \end{frame}
}

\title{MATH240 - Introduction to Linear Algebra}
\author{Haoran Li}
\institute[UMD]{University of Maryland, College Park}
\date{Summer 2024}


\begin{document}

\maketitle

\section{Lecture 3 - Matrix algebra}

\begin{frame}[t]
\begin{definition}
Let's use $M_{m\times n}(\mathbb R)$ to denote the set of all (real-valued) $m$ by $n$ matrices.
\end{definition}
\pause
\begin{definition}
Suppose $A,B$ are $m\times n$ matrices, $c$ is a scalar (i.e. a number), then we can define\pause
\begin{itemize}
\item Addition
\begin{align*}
&\begin{bmatrix}
a_{11}&a_{12}&\cdots&a_{1n}\\
a_{21}&a_{22}&\cdots&a_{2n}\\
\vdots&\vdots&&\vdots\\
a_{m1}&a_{m2}&\cdots&a_{mn}
\end{bmatrix}+\begin{bmatrix}
b_{11}&b_{12}&\cdots&b_{1n}\\
b_{21}&b_{22}&\cdots&b_{2n}\\
\vdots&\vdots&&\vdots\\
b_{m1}&b_{m2}&\cdots&b_{mn}
\end{bmatrix}\\
&=\begin{bmatrix}
a_{11}+b_{11}&a_{12}+b_{12}&\cdots&a_{1n}+b_{1n}\\
a_{21}+b_{21}&a_{22}+b_{22}&\cdots&a_{2n}+b_{2n}\\
\vdots&\vdots&&\vdots\\
a_{m1}+b_{m1}&a_{m2}+b_{m2}&\cdots&a_{mn}+b_{mn}
\end{bmatrix}
\end{align*}
\end{itemize}
\end{definition}
\end{frame}

\begin{frame}[t]
\begin{definition}
\begin{itemize}
\item Scalar multiplication
\[
c\begin{bmatrix}
a_{11}&a_{12}&\cdots&a_{1n}\\
a_{21}&a_{22}&\cdots&a_{2n}\\
\vdots&\vdots&&\vdots\\
a_{m1}&a_{m2}&\cdots&a_{mn}
\end{bmatrix}=\begin{bmatrix}
ca_{11}&ca_{12}&\cdots&ca_{1n}\\
ca_{21}&ca_{22}&\cdots&ca_{2n}\\
\vdots&\vdots&&\vdots\\
ca_{m1}&ca_{m2}&\cdots&ca_{mn}
\end{bmatrix}
\]
\end{itemize}
\end{definition}
\pause
\begin{example}\hfill
\begin{itemize}
\item $\begin{bmatrix}
1&2\\
3&4
\end{bmatrix}+\begin{bmatrix}
4&3\\
2&1
\end{bmatrix}=\begin{bmatrix}
5&5\\
5&5
\end{bmatrix}$\pause
\item $2\begin{bmatrix}
1&2\\
3&4
\end{bmatrix}=\begin{bmatrix}
2&4\\
6&8
\end{bmatrix}$
\end{itemize}
\end{example}
\end{frame}

\begin{frame}[t]
\begin{definition}
Suppose $A$ is a $m\times n$ matrix, and $B$ is a $n\times p$ matrix, we can define \textcolor{blue}{matrix multiplication}\index{matrix multiplication} $AB$ to be the $m\times p$ matrix, computed via the \textcolor{blue}{row-column rule}\index{rule-column rule}:\pause
The $(i,j)$-entry is to multiply the $i$-row and $j$-th column
\[
\begin{bmatrix}
\\
\\
\\
a_{i1}&a_{i2}&\cdots &a_{in}\\
\\
\\
\end{bmatrix}\begin{bmatrix}
&&b_{1j}&&&\\
&&b_{2j}&&&\\
&&\vdots&&&\\
&&b_{nj}&&&\\
\end{bmatrix}=\begin{bmatrix}
&&&&&\\
&&&&&\\
&&&&&\\
&&\blacksquare&&\\
&&&&&\\
&&&&&
\end{bmatrix}
\]
Where the $(i,j)$-entry $\blacksquare=a_{i1}b_{1j}+a_{i2}b_{2j}+\cdots+a_{in}b_{nj}$.
\pause
If $A$ is a square matrix, then we could define matrix power $A^k$ to be simply $\overbrace{AA\cdots A}^{k\text{ times}}$
\end{definition}
\end{frame}

\begin{frame}[t]
\begin{example}\hfill
\begin{itemize}
\item 
\scalebox{0.8}{
$\begin{bmatrix}
a_{11}&a_{12}&a_{13}\\
a_{21}&a_{22}&a_{23}
\end{bmatrix}
\begin{bmatrix}
b_{11}&b_{12}\\
b_{21}&b_{22}\\
b_{31}&b_{32}
\end{bmatrix}=\begin{bmatrix}
a_{11}b_{11}+a_{12}b_{21}+a_{13}b_{31} & a_{11}b_{12}+a_{12}b_{22}+a_{13}b_{32}\\
a_{21}b_{11}+a_{22}b_{21}+a_{23}b_{31} & a_{21}b_{12}+a_{22}b_{22}+a_{23}b_{32}
\end{bmatrix}$
}\pause
\item 
\scalebox{0.8}{
$\begin{bmatrix}
1&2&2\\
2&1&1
\end{bmatrix}\begin{bmatrix}
3&1\\
1&2\\
2&3
\end{bmatrix}=\begin{bmatrix}
1\cdot3+2\cdot1+2\cdot2&1\cdot1+2\cdot2+2\cdot3\\
2\cdot3+1\cdot1+1\cdot2&2\cdot1+1\cdot2+1\cdot3
\end{bmatrix}=\begin{bmatrix}
9&11\\
9&7
\end{bmatrix}$
}\pause
\item
\scalebox{0.8}{
$\begin{bmatrix}
a_{11}&a_{12}&a_{13}\\
a_{21}&a_{22}&a_{23}
\end{bmatrix}
\begin{bmatrix}
x_1\\
x_2\\
x_3
\end{bmatrix}=\begin{bmatrix}
a_{11}x_1+a_{12}x_2+a_{13}x_3 \\
a_{21}x_1+a_{22}x_2+a_{23}x_3
\end{bmatrix}=x_1\begin{bmatrix}a_{11}\\a_{21}\end{bmatrix}+x_1\begin{bmatrix}a_{12}\\a_{22}\end{bmatrix}+x_1\begin{bmatrix}a_{13}\\a_{23}\end{bmatrix}$
}\pause
\item
\scalebox{0.8}{
$\begin{bmatrix}
0&1\\
0&0
\end{bmatrix}\begin{bmatrix}
1&0\\
0&0
\end{bmatrix}=\begin{bmatrix}
0\cdot1+1\cdot0&0\cdot0+1\cdot0\\
0\cdot1+0\cdot0&0\cdot0+0\cdot0
\end{bmatrix}=\begin{bmatrix}
0&0\\
0&0
\end{bmatrix}$
}\pause
\item
\scalebox{0.8}{
$\begin{bmatrix}
1&0\\
0&0
\end{bmatrix}\begin{bmatrix}
0&1\\
0&0
\end{bmatrix}=\begin{bmatrix}
1\cdot0+0\cdot0&1\cdot1+0\cdot0\\
0\cdot0+0\cdot0&0\cdot1+0\cdot0
\end{bmatrix}=\begin{bmatrix}
0&1\\
0&0
\end{bmatrix}$
}
\end{itemize}
\end{example}
\end{frame}

\begin{frame}[t]
\begin{example}
\begin{itemize}
\item 
\[
\begin{bmatrix}
a_{11}&a_{12}&a_{13}\\
a_{21}&a_{22}&a_{23}\\
a_{31}&a_{32}&a_{33}\\
\end{bmatrix}\begin{bmatrix}
1&0&0\\
0&1&0\\
0&0&1
\end{bmatrix}=\begin{bmatrix}
a_{11}&a_{12}&a_{13}\\
a_{21}&a_{22}&a_{23}\\
a_{31}&a_{32}&a_{33}\\
\end{bmatrix}
\]\pause
\item 
\[
\begin{bmatrix}
1&0&0\\
0&1&0\\
0&0&1
\end{bmatrix}\begin{bmatrix}
a_{11}&a_{12}&a_{13}\\
a_{21}&a_{22}&a_{23}\\
a_{31}&a_{32}&a_{33}\\
\end{bmatrix}=\begin{bmatrix}
a_{11}&a_{12}&a_{13}\\
a_{21}&a_{22}&a_{23}\\
a_{31}&a_{32}&a_{33}\\
\end{bmatrix}
\]
\end{itemize}
\end{example}
\end{frame}

\begin{frame}[t]
\begin{example}\label{ex: elementary matrices}\hfill
\begin{itemize}
\item 
\scalebox{0.6}{
$\begin{bmatrix}
1&0&-3\\
0&1&0\\
0&0&1
\end{bmatrix}\begin{bmatrix}
a_{11}&a_{12}&a_{13}\\
a_{21}&a_{22}&a_{23}\\
a_{31}&a_{32}&a_{33}\\
\end{bmatrix}=\begin{bmatrix}
a_{11}-3a_{31}&a_{12}-3a_{32}&a_{13}-3a_{33}\\
a_{21}&a_{22}&a_{23}\\
a_{31}&a_{32}&a_{33}\\
\end{bmatrix}$
}\pause
\item 
\scalebox{0.6}{
$\begin{bmatrix}
a_{11}&a_{12}&a_{13}\\
a_{21}&a_{22}&a_{23}\\
a_{31}&a_{32}&a_{33}
\end{bmatrix}\begin{bmatrix}
1&0&0\\
0&1&0\\
2&0&1
\end{bmatrix}=\begin{bmatrix}
a_{11}+2a_{13}&a_{12}&a_{13}\\
a_{21}+2a_{23}&a_{22}&a_{23}\\
a_{31}+2a_{33}&a_{32}&a_{33}\\
\end{bmatrix}$
}\pause
\item 
\scalebox{0.6}{
$\begin{bmatrix}
1&0&0\\
0&2&0\\
0&0&1
\end{bmatrix}\begin{bmatrix}
a_{11}&a_{12}&a_{13}\\
a_{21}&a_{22}&a_{23}\\
a_{31}&a_{32}&a_{33}
\end{bmatrix}=\begin{bmatrix}
a_{11}&a_{12}&a_{13}\\
2a_{21}&2a_{22}&2a_{23}\\
a_{31}&a_{32}&a_{33}
\end{bmatrix}$
}\pause
\item 
\scalebox{0.6}{
$\begin{bmatrix}
a_{11}&a_{12}&a_{13}\\
a_{21}&a_{22}&a_{23}\\
a_{31}&a_{32}&a_{33}
\end{bmatrix}\begin{bmatrix}
1&0&0\\
0&3&0\\
0&0&1
\end{bmatrix}=\begin{bmatrix}
a_{11}&3a_{12}&a_{13}\\
a_{21}&3a_{22}&a_{23}\\
a_{31}&3a_{32}&a_{33}
\end{bmatrix}$
}\pause
\item 
\scalebox{0.6}{
$\begin{bmatrix}
0&0&1\\
0&1&0\\
1&0&0
\end{bmatrix}\begin{bmatrix}
a_{11}&a_{12}&a_{13}\\
a_{21}&a_{22}&a_{23}\\
a_{31}&a_{32}&a_{33}
\end{bmatrix}=\begin{bmatrix}
a_{31}&a_{32}&a_{33}\\
a_{21}&a_{22}&a_{23}\\
a_{11}&a_{12}&a_{13}
\end{bmatrix}$
}\pause
\item 
\scalebox{0.6}{
$\begin{bmatrix}
a_{11}&a_{12}&a_{13}\\
a_{21}&a_{22}&a_{23}\\
a_{31}&a_{32}&a_{33}
\end{bmatrix}\begin{bmatrix}
0&1&0\\
1&0&0\\
0&0&1
\end{bmatrix}=\begin{bmatrix}
a_{12}&a_{11}&a_{13}\\
a_{22}&a_{21}&a_{23}\\
a_{32}&a_{31}&a_{33}
\end{bmatrix}$
}
\end{itemize}
\end{example}
\end{frame}

\begin{frame}[t]{Exercise}
Suppose $A=\begin{bmatrix}
1&1&1\\
2&2&1\\
2&1&1
\end{bmatrix}, B=\begin{bmatrix}
1&1&1\\
2&2&1\\
2&1&1
\end{bmatrix}$, computes matrix multiplication $AB$
\end{frame}

\begin{frame}[t]
Suppose $A,B,C,D$ are matrices, $c$ is a scalar, $0$ is the zero matrix, $I$ is the identity matrix. we have the following facts\pause
\begin{enumerate}
\item Matrix multiplication is generally \textit{NOT commutative}, i.e. $AB\neq BA$\pause
\item Matrix multiplication is \textit{associative}, i.e. the order of multiplication doesn't matter, in other words $(AB)C=A(BC)$, so it makes sense to write successive multiplication $A_1A_2A_3\cdots A_n$\pause
\item Scalar multiplication and matrix multiplication commutes, $A(cB)=c(AB)=(cA)B$. so it makes sense to write $cA_1A_2A_3\cdots A_n$\pause
\item Matrix multiplication is \textit{distributive} over addition, i.e. $A(B+C)=AB+AC$, $(A+B)C=AC+BC$\pause
\item Zero matrix and identity matrix acts as 0 and 1, i.e. $A+0=0+A=A$, $A0=0A=0$, $IA=AI=A$\pause
\item Even if $A\neq0$, $B\neq0$, $AB$ could still be $0$\pause
\item $AB=AC$ does NOT imply $B=C$
\end{enumerate}
\pause
\begin{remark}
Some of the properties of matrices are really similar to that of numbers, so we dub this the name of \textit{matrix algebra}
\end{remark}
\end{frame}

\begin{frame}[t]
\begin{definition}
$A$ is a \textcolor{blue}{partitioned} (or \textcolor{blue}{block}) matrix\index{partitioned matrix} if is divided into smaller submatrix by some horizontal and vertical lines. And the submatrices are the blocks
\begin{center}
\scalebox{0.8}{
$\left[\begin{array}{c|c|c}
A_{11}&A_{12}&A_{13}\\
\hline
A_{21}&A_{22}&A_{23}\\
\hline
A_{31}&A_{32}&A_{33}\\
\end{array}\right]=
\left[\begin{array}{cc|cccc|c}
a_{11}&a_{12}&a_{13}&a_{14}&a_{15}&a_{16}&a_{17}\\
a_{21}&a_{22}&a_{23}&a_{24}&a_{25}&a_{26}&a_{27}\\
\hline
a_{31}&a_{32}&a_{33}&a_{34}&a_{35}&a_{36}&a_{37}\\
\hline
a_{41}&a_{42}&a_{43}&a_{44}&a_{45}&a_{46}&a_{47}\\
a_{51}&a_{52}&a_{53}&a_{54}&a_{55}&a_{56}&a_{57}\\
a_{61}&a_{62}&a_{63}&a_{64}&a_{65}&a_{66}&a_{67}\\
\end{array}
\right]$
}
\end{center}
Here the blocks are
\begin{center}
\scalebox{0.8}{
$A_{11}=\begin{bmatrix}
a_{11}&a_{12}\\
a_{21}&a_{22}
\end{bmatrix}$,
$A_{12}=\begin{bmatrix}
a_{13}&a_{14}&a_{15}&a_{16}\\
a_{23}&a_{24}&a_{25}&a_{26}
\end{bmatrix}$,
$A_{13}=\begin{bmatrix}a_{17}\\ a_{27}\end{bmatrix}$
}
\end{center}
\begin{center}
\scalebox{0.8}{
$A_{21}=\begin{bmatrix}a_{31}&a_{32}\end{bmatrix}$,
$A_{22}=\begin{bmatrix}a_{33}&a_{34}&a_{35}&a_{36}\end{bmatrix}$,
$A_{23}=\begin{bmatrix}a_{37}\end{bmatrix}$
}
\end{center}
\begin{center}
\scalebox{0.8}{
$A_{31}=\begin{bmatrix}
a_{41}&a_{42}\\
a_{51}&a_{52}\\
a_{61}&a_{62}
\end{bmatrix}$,
$A_{32}=\begin{bmatrix}
a_{43}&a_{44}&a_{45}&a_{46}\\
a_{53}&a_{54}&a_{55}&a_{56}\\
a_{63}&a_{64}&a_{65}&a_{66}
\end{bmatrix}$,
$A_{33}=\begin{bmatrix}a_{47}\\ a_{57}\\ a_{67}\end{bmatrix}$
}
\end{center}
\end{definition}
\end{frame}

\begin{frame}[t]
\begin{example}
Consider $\left[\begin{array}{c|c}
A_{11}&A_{12}\\
\hline
A_{21}&A_{22}
\end{array}\right]=
\left[\begin{array}{cc|c}
1&1&1\\
2&2&1\\
\hline
2&1&1
\end{array}\right]$, $\left[\begin{array}{c|c}
B_{11}&B_{12}\\
\hline
B_{21}&B_{22}
\end{array}\right]=
\left[\begin{array}{c|cc}
1&1&1\\
2&2&1\\
\hline
2&1&1
\end{array}\right]$\pause, then
\[
A_{11}B_{11}+A_{12}B_{21}=\begin{bmatrix}
1&1\\
2&2
\end{bmatrix}\begin{bmatrix}
1\\2
\end{bmatrix}+\begin{bmatrix}
1\\1
\end{bmatrix}\begin{bmatrix}
2
\end{bmatrix}=\pause\begin{bmatrix}
5\\8
\end{bmatrix}
\]
\[
A_{11}B_{12}+A_{12}B_{22}=\begin{bmatrix}
1&1\\
2&2
\end{bmatrix}\begin{bmatrix}
1&1\\
2&1
\end{bmatrix}+\begin{bmatrix}
1\\1
\end{bmatrix}\begin{bmatrix}
1&1
\end{bmatrix}=\pause\begin{bmatrix}
4&3\\7&5
\end{bmatrix}
\]
\[
A_{21}B_{11}+A_{22}B_{21}=\begin{bmatrix}
2&1
\end{bmatrix}\begin{bmatrix}
1\\2
\end{bmatrix}+\begin{bmatrix}
1
\end{bmatrix}\begin{bmatrix}
2
\end{bmatrix}=\pause\begin{bmatrix}
6
\end{bmatrix}
\]
\[
A_{21}B_{12}+A_{22}B_{22}=\begin{bmatrix}
2&1
\end{bmatrix}\begin{bmatrix}
1&1\\
2&1
\end{bmatrix}+\begin{bmatrix}
1
\end{bmatrix}\begin{bmatrix}
1&1
\end{bmatrix}=\pause\begin{bmatrix}
5&4
\end{bmatrix}
\]
\begin{center}
\vspace{-10mm}
\scalebox{0.8}{
$\left[\begin{array}{c|c}
A_{11}&A_{12}\\
\hline
A_{21}&A_{22}
\end{array}\right]\left[\begin{array}{c|c}
B_{11}&B_{12}\\
\hline
B_{21}&B_{22}
\end{array}\right]=\left[\begin{array}{c|c}
A_{11}B_{11}+A_{12}B_{21}&A_{11}B_{12}+A_{12}B_{22}\\
\hline
A_{21}B_{11}+A_{22}B_{21}&A_{21}B_{12}+A_{22}B_{22}
\end{array}\right]=\pause\left[\begin{array}{c|cc}
5    & 4&     3\\
8 &    7  &   5\\
\hline
6&     5   &  4
\end{array}\right]$
}
\end{center}
\end{example}
\end{frame}

\begin{frame}[t]
\begin{fact}
Suppose $A=\begin{bmatrix}
A_{11}&A_{12}&\cdots& A_{1q}\\
A_{21}&A_{22}&\cdots& A_{2q}\\
\vdots&\vdots&&\vdots\\
A_{p1}&A_{p2}&\cdots& A_{pq}\\
\end{bmatrix}$, $B=\begin{bmatrix}
B_{11}&B_{12}&\cdots& B_{1r}\\
B_{21}&B_{22}&\cdots& B_{2r}\\
\vdots&\vdots&&\vdots\\
B_{q1}&B_{q2}&\cdots& B_{qr}\\
\end{bmatrix}$ are partitioned matrices, and the number of columns of $A_{1k}$ is equal to the number of rows of $B_{k1}$ (so that all submatrices multiplications make sense).\pause
Then the usual row-column rule still WORKS!!! By treating submatrices as if they are numbers.
\[
AB=C=\begin{bmatrix}
C_{11}&C_{12}&\cdots& C_{1r}\\
C_{21}&C_{22}&\cdots& C_{2r}\\
\vdots&\vdots&&\vdots\\
C_{p1}&C_{p2}&\cdots& C_{pr}\\
\end{bmatrix}
\]
\[
C_{ij}=A_{i1}B_{1j}+A_{i2}B_{2j}+\cdots+A_{iq}B_{qj}
\]
\end{fact}
\end{frame}

\begin{frame}[t]
\begin{example}
Suppose $3\times3$ matrix $A$ can be partitioned into $\begin{bmatrix}
R1\\R2\\R3
\end{bmatrix}$ or $\begin{bmatrix}
\mathbf a_1&\mathbf a_2&\mathbf a_3
\end{bmatrix}$, then Example~\ref{ex: elementary matrices} reads\pause
\begin{enumerate}
\item $E=\begin{bmatrix}
1&0&-3\\
0&1&0\\
0&0&1
\end{bmatrix}$, $EA=\begin{bmatrix}
1&0&-3\\
0&1&0\\
0&0&1
\end{bmatrix}\begin{bmatrix}
R1\\R2\\R3
\end{bmatrix}=\pause\begin{bmatrix}
R1-3R3\\R2\\R3
\end{bmatrix}$, $EA$ acts as subtracting 3 times row 3 from row 1.\pause
\item $E=\begin{bmatrix}
1&0&0\\
0&1&0\\
2&0&1
\end{bmatrix}$,
\scalebox{0.8}{
$AE=\begin{bmatrix}
\mathbf a_1&\mathbf a_2&\mathbf a_3
\end{bmatrix}\begin{bmatrix}
1&0&0\\
0&1&0\\
2&0&1
\end{bmatrix}=\begin{bmatrix}
\mathbf a_1+2\mathbf a_3&\mathbf a_2&\mathbf a_3
\end{bmatrix}$
}, $AE$ acts as adding 2 times column 3 to column 1.\pause
\item $E=\begin{bmatrix}
1&0&0\\
0&2&0\\
0&0&1
\end{bmatrix}$, $EA=\begin{bmatrix}
1&0&0\\
0&2&0\\
0&0&1
\end{bmatrix}\begin{bmatrix}
R1\\R2\\R3
\end{bmatrix}=\begin{bmatrix}
R1\\2R2\\R3
\end{bmatrix}$, $EA$ acts as scaling the third row by 2.
\seti
\end{enumerate}
\end{example}
\end{frame}

\begin{frame}[t]
\begin{example}
\begin{enumerate}
\conti
\item $E=\begin{bmatrix}
1&0&0\\
0&3&0\\
0&0&1
\end{bmatrix}$, $AE=\begin{bmatrix}
\mathbf a_1&\mathbf a_2&\mathbf a_3
\end{bmatrix}\begin{bmatrix}
1&0&0\\
0&3&0\\
0&0&1
\end{bmatrix}=\pause\begin{bmatrix}
\mathbf a_1&3\mathbf a_2&\mathbf a_3
\end{bmatrix}$, $AE$ acts as scaling the third column by 3.\pause
\item $E=\begin{bmatrix}
0&0&1\\
0&1&0\\
1&0&0
\end{bmatrix}$, $EA=\begin{bmatrix}
0&0&1\\
0&1&0\\
1&0&0
\end{bmatrix}\begin{bmatrix}
R1\\R2\\R3
\end{bmatrix}=\pause\begin{bmatrix}
R3\\R2\\R1
\end{bmatrix}$, $EA$ acts as interchanging row 1 and row 3.\pause
\item $E=\begin{bmatrix}
0&1&0\\
1&0&0\\
0&0&1
\end{bmatrix}$, $AE=\begin{bmatrix}
\mathbf a_1&\mathbf a_2&\mathbf a_3
\end{bmatrix}\begin{bmatrix}
0&1&0\\
1&0&0\\
0&0&1
\end{bmatrix}=\pause\begin{bmatrix}
\mathbf a_2&\mathbf a_1&\mathbf a_3
\end{bmatrix}$, $AE$ acts as interchanging column 1 and column 2.
\end{enumerate}
\end{example}
\pause
\begin{definition}
Matrices $E$ in the previous example is called \textcolor{blue}{elementary matrices}\index{Elementary matrices}. They describe row and column elementary operations.
\end{definition}
\end{frame}

\begin{frame}[t]{Exercise}
If $A$ is a 4 by 5 matrix, what is the elementary matrix $E$ that acts as replacing the fourth row by adding twice of the second row.
\end{frame}

\begin{frame}[t]{Exercise}
Suppose $B=\begin{bmatrix}
\mathbf b_1&\mathbf b_2&\cdots&\mathbf b_n
\end{bmatrix}$, show that $AB=\begin{bmatrix}
A\mathbf b_1&A\mathbf b_2&\cdots&A\mathbf b_n
\end{bmatrix}$
\end{frame}

\begin{frame}[t]{Exercise}
Verify that $A^2=I_2$ where $A=\begin{bmatrix}
1&0\\3&-1
\end{bmatrix}$, and use partitioned matrices to show that $M^2=I_4$, where $M=\begin{bmatrix}
1&0&0&0\\
3&-1&0&0\\
1&0&-1&0\\
0&1&-3&1
\end{bmatrix}$
\end{frame}

% \begin{frame}[t]
% \end{frame}

% \begin{frame}[t]
% \end{frame}

\end{document}