\subsection{Sylvester matrix and resultant}

Next, we will talk about B\'ezout's Theorem

\begin{definition}\label{definition 19}
For $ p,q \in \Poly(\mathbb{F}) $, Assume
$$ p(x) = a_{0} + a_{1}x + a_{2}x^{2} + \cdots + a_{m}x^{m} $$
$$ q(x) = b_{0} + b_{1}x + b_{2}x^{2} + \cdots + b_{n}x^{n} $$
Define its Sylvester matrix $ S(p,q) $ to be a $ (m+n) \times (m+n) $ matrix
$$
S(p,q) = 
\left[
\begin{matrix}
 a_{0}  &     0     &  \cdots  &     0     &  b_{0}  &     0     &  \cdots  &     0       \\
 a_{1}  &   a_{0}   &  \cdots  &     0     &  b_{1}  &   b_{0}   &  \cdots  &     0       \\
 a_{2}  &   a_{1}   &  \ddots  &     0     &  b_{2}  &   b_{1}   &  \ddots  &     0       \\
\vdots  &  \vdots   &  \ddots  &   a_{0}   & \vdots  &  \vdots   &  \ddots  &   b_{0}     \\
 a_{m}  &  a_{m-1}  &  \cdots  &  \vdots   &  b_{n}  &  b_{n-1}  &  \cdots  &  \vdots     \\
  0     &   a_{m}   &  \ddots  &  \vdots   &   0     &   b_{n}   &  \ddots  &  \vdots     \\
\vdots  &  \vdots   &  \ddots  &  a_{m-1}  & \vdots  &  \vdots   &  \ddots  &  b_{n-1}    \\
  0     &     0     &  \cdots  &   a_{m}   &   0     &     0     &  \cdots  &   b_{n}     \\
\end{matrix}
\right]
$$
Define the resultant $ \Res(p,q) $ of $ p,q $ to be $ \det S(p,q) $, and using $ (p,q) $ for the greatest 
common divisor of $ p,q $ \par
Define $ \Coe(p) $ to be the vector of coefficients
$$ \Coe(p) = \left[ a_{0}, \cdots, a_{n} \right]^{\mathrm{T}} $$

\end{definition}

\begin{theorem}\label{theorem 20}
$ \Res(p,q) = 0 $ if and only if $ p,q $ has common roots.
\end{theorem}

\begin{proof}
Since
$$ (p,q) = d \Leftrightarrow sp + tq = d $$
For unique $ s,t \in \Poly(\mathbb{F}) $ with $ \mathrm{deg}s < \mathrm{deg}q - \mathrm{deg}d, 
\mathrm{deg}t < \mathrm{deg}p - \mathrm{deg}d $, this result is also known as B\'ezout's Theorem \par
Moreover
$$ sp + tq = 0, \text{for some $ s,t $ with $ \mathrm{deg}s < \mathrm{deg}q, \mathrm{deg}t < \mathrm{deg}p $} 
\Leftrightarrow (p,q) = d \neq 1 $$
Where $ \Rightarrow $ is true because of the uniqueness of B\'ezout's Theorem, and $ \Leftarrow $ is true because we could take $ s = q/d, t = -p/d $ \par
Suppose
$$ s(x) = c_{0} + c_{1}x + c_{2}x^{2} + \cdots + c_{k}x^{k} $$
$$ t(x) = d_{0} + d_{1}x + d_{2}x^{2} + \cdots + d_{l}x^{l} $$
Then
$$ Coe(sp + tq) = S(p,q) \left[ \begin{matrix} Coe(p) \\ Coe(q) \end{matrix} \right] $$
Hence
$$ sp + tq = 0, \text{for some $ s,t $ with $ \mathrm{deg}s < \mathrm{deg}q, \mathrm{deg}t < \mathrm{deg}p $} 
\Leftrightarrow Res(p,q) = 0 $$
\end{proof}

Now we introduce a version of Fundamental Theorem of Algebra in $ \mathbb{CP}^{1} $

\begin{theorem}\label{theorem 21}
Suppose $ p \in \Poly(\mathbb{C}^{2}) $ is a homogenous polynomial of degree $ n $ then it can be factored 
into linear factors.
\end{theorem}

\begin{proof}
Assume
$$ p(x,y) = a_{0}x^{n} + a_{1}x^{n-1}y + \cdots + a_{n-1}xy^{n-1} + a_{n}y^{n} $$
Define
$$ q(z) = a_{0} + a_{1}z + \cdots + a_{n}z^{n} $$
Then by the usual Fundamental Theorem of Algebra on $ \mathbb{C} $, hence $ q(z) $ can be factored into linear factors
$$ q(z) = C(z - \lambda_{1}) \cdots (z - \lambda_{n}) $$
Therefore we have 
$$ p(x,y) = C(y - \lambda_{1}x) \cdots (y - \lambda_{n}x) $$
\end{proof}

\subsection{B\'ezout's Theorem}

The next result is formally known as B\'ezout's Theorem

\begin{theorem}\label{theorem 22}
Suppose two curves in $ \mathbb{C}^{2} $ defined by polynomials $ p,q \in \Poly(\mathbb{C}^{2}) $ with 
$ (p,q) = 1 $ and $ \mathrm{deg}p = m, \mathrm{deg}q = n $, then the number of intersections of the two curves 
is no more than $ mn $
\end{theorem}

\begin{remark}
$ (p,q) = 1 $ means that $ p $ and $ q $ don't have common factors, otherwise $ p,q $ would have infinitely many common roots
\end{remark}

\begin{proof}
Suppose $ p,q $ have more than $ mn $ common roots, say, $ mn + 1 $ common roots $ P_{1}, \cdots, P_{mn+1} $, than after an affine transformation, we could assume that neither two of lines $ OP_{i} $ will overlap \par
Suppose the homogenization of $ p,q $ are
$$ p(x,y,z) = \sum_{i+j+k=m} c_{ijk}x^{i}y^{j}z^{k} = p_{0}(x,y)z^{m}+p_{1}(x,y)z^{m-1}+\cdots+p_{m}(x,y) $$
$$ q(x,y,z) = \sum_{i+j+k=n} d_{ijk}x^{i}y^{j}z^{k} = q_{0}(x,y)z^{n}+q_{1}(x,y)z^{n-1}+\cdots+q_{n}(x,y) $$
Where $ [x,y,z] $ is the homogenous coordinate, and $ p_{i}(x,y) $ are homogenous polynomial of degree $ i $ \par
We could consider $ \Res(p,q) $ which would be a homogenous polynomial of degree $ mn $, and the proof of this fact we defer to the next lemma. And $ \Res(p,q) \neq 0 $, otherwise $ f,g $ would have common roots in $ k(x,y)[z] $, by Gauss's lemma, $ f,g $ would have common factors in $ k[x,y,z] $ \par
According to Theorem \ref{theorem 21}, $ \Res(p,q) $ could be factored into $ mn $ linear factors, then each $ P_{i} $ should lie on different linear factor, which is a contradiction
\end{proof}

\begin{lemma}\label{lemma 23}
$\Res(p,q) $ is a homogenous polynomial of degree $ mn $
\end{lemma}

\begin{proof}
we would prove that each term in the $ (m+n)! $ terms of the expansion of Sylvester matrix $ S(p,q) $ is a homogenous polynomial of degree $ mn $assume one term is
$$ \pm S_{i_{1},1} \cdots S_{i_{n},n} S_{i_{n+1},n+1} \cdots S_{i_{m+n},m+n} $$
Where $ (i_{1} \cdots i_{m+n}) \in S_{m+n} $ is a permutation of $ {1, \cdots, m+n} $, if we want this term not to be zero, we need further to assume
$$ j \leq i_{j} \leq j+m, j = 1, \cdots, n $$
$$ j \leq i_{n+j} \leq j+n, j = 1, \cdots, m $$
Due to the arranging of $ S(p,q) $, we have the degree $ d $ of this term satisfies
$$ \sum_{k=1}^{m+n} k = \sum_{j=1}^{m+n} i_{j} = d + \sum_{k=1}^{n} k + \sum_{k=1}^{m} k $$
Thus we have $ d = mn $
\end{proof}