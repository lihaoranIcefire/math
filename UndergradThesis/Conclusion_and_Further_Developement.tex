B\'ezout's theorem could be generalized to higher dimensional projectivce space with the inequality replaced 
with equality \par
Roth's theorem could be generalized to the following statement: \par
If $ \alpha $ is an irrational algebraic number, then
$$ |\alpha-\xi| < \dfrac{1}{H(\xi)^{\kappa}} $$
has only finitely many solutions, where height $ H(\xi) $ is the maximum of the absolute values of the coefficients of its primitive minimal polynomial, and $ \kappa > 2\mathrm{deg}(\xi) $, and infinitely many solutions if $ \kappa \leq 2\mathrm{deg}(\xi) $ \par
There are also generalizations by LeVeque to arbitratry algebraic number field, and by Schmidt to higher dimension which called Schmidt's subspace theorem, and an analogue to $p$-adic metric Roth's theorem \par
Also, Serge Lang made a stronger conjecture, that if $ \alpha $ is an irrational algebraic number, then
$$ \left| \alpha-\dfrac{p}{q} \right| < \dfrac{1}{q^{2}\log(q)^{1+\epsilon}} $$
has only finitely many solutions \par
The proofs in this thesis, they often don't construct a desired polynomial concretely, rather they use a 
probabilistic method, or pegion hole's principle by certifying only the existence of such a polynomial \par
I have to say I'm astonished over and over again by how some simple yet powerful structure hidden behind could 
link and shed light on so many interesting questions