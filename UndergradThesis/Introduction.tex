Polynomial, a simple notion that everybody think they know everything about it, but in fact we are often so 
shocked that we hardly know anything about it, especially when the degree or its coefficients changes. Yet it 
has so many beautiful and elegant structures hidden behind it, driving generations of generations of brilliant 
mathematicians to explore its wonderful mestery \par
This thesis mainly focuses on only some problems concerning a few combinatorial and algebraic structures of 
polynomials which otherwise would be very hard to solve. For example, the finite field Kakeya problem has been 
tried in many ways without succeeding, however it could be easily solved by some arguments in linear algebra. 
So is many other problems in this thesis, all the proofs are quite elementary, involving only some linear or 
abstract algebra, including the famous Roth's theorem. In essence, all the proofs are based on some simple 
algebraic structures(vector spaces, rings, etc.) of polynomials and how they behave on the underlying fields
(especially vanishing property).