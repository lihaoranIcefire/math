Zhao in~\cite{Zhao_MultipleZetaFunctionsMultiplePolylogarithmsAndTheirSpecialValues} gave formulas for the monodromy of multiple logarithms $\Li_{1,\cdots,1}$, by interpreting them as iterated integrals over $S_d(\mathbb C)$ (see Section~\ref{sec: Zhao's calc on monodromy of multiple logarithms}). He claimed that the monodromy of a general multiple polylogarithm can be computed by taking the limit of variation mixed Hodge structures of multiple logarithms, but gave neither an explicit formula nor an algorithm.

In this chapter, we present a fresh viewpoint by interpreting multiple polylogarithms as iterated integrals on $\mathbb P^1-\{0,1,\infty\}$. Using Proposition~\ref{prop: realization of iterated integrals}, we show that when a multiple polylogarithm undergoes a monodromy, it corresponds to a continuous deformation of the integration path, providing an alternative geometric interpretation. This perspective naturally give rise to explicit formulas, which are also implemented by the author using Mathematica.

\section{Deformation of integration paths of iterated integrals under monodromy}

Recall from Section~\ref{sec: Zhao's calc on monodromy of multiple logarithms} that $\nu_i$, $\nu_{j,k}$ are loops around divisors $x_i=0$ and $x_j\cdots x_k=1$ in $S_d(\mathbb C)$, and that $\mathcal M_{\nu_i}$, $\mathcal M_{\nu_{j,k}}$ denote the monodromy operators on iterated integrals if we think of them as multi-valued functions on $S_d(\mathbb C)$.

Recall the realization map $\Re_{\mathbb I}:\mathbb I^{\Symb}\to\mathcal O(\widetilde S_d(\mathbb C))$ from Definition~\ref{def: Re_I}. Without loss of generality, we fix some choice of $a_i=(x_i\cdots x_d)^{-1}$ with $0<x_i<1$, and paths $\gamma$ from $a_i$ to $a_j$ such that the image of $\gamma$ is below the real axis $\{\operatorname{Im} z=0\}$. Since monodromies of $I(a_{i_0};a_{i_1},\cdots,a_{i_n};a_{i_{n+1}})$ can be converted into deformations of the integration path of $I_\gamma(a_{i_0};a_{i_1},\cdots,a_{i_n};a_{i_{n+1}})$, as discussed in Remark~\ref{rmk: gamma1 = gamma0 => same integral} and Remark~\ref{rmk: gamma1 = gamma0 => same reg integral}, we are able to give a complete formulation of monodromies of realizations of iterated integrals in $\mathbb I^{\Symb}(d)$.

\subsection{Deformation of integration path under $\mathcal M_{\nu_{i_0}}$}

If we fix a path from $a$ to $b$, then the homotopy classes of paths in $\mathbb C-\{a_i\}$ from $a$ to $b$ can be identified with $\pi_1(\mathbb C-\{a_i\})$, so we could write any path from $a$ to $b$ as a loop in $\pi_1(\mathbb C-\{a_i\})$.

\begin{theorem}\label{thm: monodromy on iterated integrals I}\hfill
\begin{enumerate}[i.]
\item If $1\leq i\leq i_0<j\leq d+1$, then
\begin{equation}\label{eq: M_{i0}I(a_i;...;a_j)}
\mathcal M_{\nu_{i_0}}I(a_i;\cdots;a_j)=I_{\sigma_{i_0+1}\cdots\sigma_{j-1}\sigma_0}(a_i;\cdots;a_j)
\end{equation}
\item If $1\leq i\leq i_0$, then
\begin{equation}\label{eq: M_{i0}I(0;...;a_i)}
\begin{aligned}
\mathcal M_{\nu_{i_0}}I(0;\cdots;a_i)&=I_{\sigma_0^{-1}\sigma_d^{-1}\cdots\sigma_{i_0+1}^{-1}}(0;\cdots;a_i)
\end{aligned}
\end{equation}
\end{enumerate}
\end{theorem}

\begin{proof}
We choose $\nu_{i_0}$ to be a very small counterclockwise circle around $0$, causing $a_{1},\cdots,a_{i_0}$ to move around $0$ in 
clockwise, concentric circles.

\begin{figure}[H]
\centering
\begin{subfigure}[b]{0.48\textwidth}
\ContinuedFloat
\centering
\begin{tikzpicture}[scale=0.5]
\foreach \x/\y/\p/\labelpos/\labeltext in {0/0/o/above/{$0$}, 1/0/aj/above/{$a_j$}, 3/0/aip1/below/{$a_{i_0+1}$}, 4/0/a0i0/above/{$a_{i_0}$}, 6/0/ai0/above/{$a_i$}, 7/0/a10/above/{$a_1$}, 0/-4/a0i/below/{}, 0/-6/ai/below/{}, 0/-7/a1/below/{}}{
    \coordinate (\p) at (\x,\y);
    \draw[fill] (\p) circle (0.08);
    \node at (\p)[\labelpos] {\labeltext};
}
\node at (2,0) {$\cdots$};
\node at (5,0) {$\cdots$};
\node at (0,-5) {$\vdots$};
\draw[->-=0.5, blue, dashed] (a0i0) arc (0:-90:4);
\draw[->-=0.5, blue, dashed] (ai0) arc (0:-90:6);
\draw[->-=0.5, blue, dashed] (a10) arc (0:-90:7);
\draw[->-=0.5, opacity=0.3] (ai0) to [curve through={(2,-5.3)..(0,-5.7)..(-.5,-3)}] (aj);
\node at (2,-5.3)[above, opacity=0.3] {$\gamma_0$};
\draw[->-=0.5] (ai) to [curve through={(-5.5,0)..(-4,4)..(-3,4.8)..(0,5.5)..(0,3)..(-1,0)}] (aj);
\node at (-5.5,0)[right] {$\gamma_{1/3}$};
\end{tikzpicture}
\caption{$I_{\gamma_0}(a_0;\cdots;a_j)$ to $I_{\gamma_{1/3}}(a_0;\cdots;a_j)$}
\label{fig: M_{v_i0}I(a_i;...;a_j), gamma_0 -> gamma_{1/3}}
\end{subfigure}
\begin{subfigure}[b]{0.48\textwidth}
\centering
\begin{tikzpicture}[scale=0.5]
\foreach \x/\y/\p/\labelpos/\labeltext in {0/0/o/above/{$0$}, 1/0/aj/above/{$a_j$}, 3/0/aip1/below/{$a_{i_0+1}$}, 0/-4/a0i0/right/{$a_{i_0}$}, 0/-6/ai0/right/{$a_i$}, 0/-7/a10/right/{$a_1$}, 0/4/a0i/above/{}, 0/6/ai/above/{}, 0/7/a1/above/{}}{
    \coordinate (\p) at (\x,\y);
    \draw[fill] (\p) circle (0.08);
    \node at (\p)[\labelpos] {\labeltext};
}
\node at (2,0) {$\cdots$};
\node at (0,-5) {$\vdots$};
\node at (0,5) {$\vdots$};
\draw[->-=0.5, blue, dashed] (a0i0) arc (270:90:4);
\draw[->-=0.5, blue, dashed] (ai0) arc (270:90:6);
\draw[->-=0.5, blue, dashed] (a10) arc (270:90:7);
\draw[->-=0.5, opacity=0.3] (ai0) to [curve through={(-5.5,0)..(-4,4)..(-3,4.8)..(0,5.5)..(0,3)..(-1,0)}] (aj);
\node at (-5.5,0)[right, opacity=0.3] {$\gamma_{\frac{1}{3}}$};
\draw[->-=0.5] (ai) to [curve through={(4,4.2)..(5.5,0)..(4,-1)..(3.7,0)..(2,1)..(-.8,0)..(0,-.4)}] (aj);
\node at (2,1)[above] {$\gamma_{2/3}$};
\end{tikzpicture}
\caption{$I_{\gamma_{1/3}}(a_0;\cdots;a_j)$ to $I_{\gamma_{2/3}}(a_0;\cdots;a_j)$}
\label{fig: M_{v_i0}I(a_i;...;a_j), gamma_{1/3} -> gamma_{2/3}}
\end{subfigure}
\begin{subfigure}[b]{0.48\textwidth}
\ContinuedFloat
\centering
\begin{tikzpicture}[scale=0.8]
\foreach \x/\y/\p/\labelpos/\labeltext in {0/0/o/above/{$0$}, 1/0/aj/above/{$a_j$}, 3/0/aip1/below/{$a_{i_0+1}$}, 0/4/a0i0/below/{$a_{i_0}$}, 0/6/ai0/below/{$a_i$}, 0/7/a10/below/{$a_1$}, 4/0/a0i/below/{}, 6/0/ai/below right/{}, 7/0/a1/below/{}}{
    \coordinate (\p) at (\x,\y);
    \draw[fill] (\p) circle (0.08);
    \node at (\p)[\labelpos] {\labeltext};
}
\node at (2,0) {$\cdots$};
\node at (0,5) {$\vdots$};
\node at (5,0) {$\cdots$};
\draw[->-=0.5, blue, dashed] (a0i0) arc (90:0:4);
\draw[->-=0.5, blue, dashed] (ai0) arc (90:0:6);
\draw[->-=0.5, blue, dashed] (a10) arc (90:0:7);
\draw[->-=0.5, opacity=0.3] (ai0) to [curve through={(4,4.2)..(5.5,0)..(4,-1)..(3.7,0)..(2,1)..(-.8,0)..(0,-.4)}] (aj);
\node at (2,1)[above, opacity=0.3] {$\gamma_{3/4}$};
\draw[->-=0.5] (ai) to [curve through={(5,-1.5)..(4,-1.5)..(3.3,0)..(2,.5)..(-.5,0)..(0,-.3)}] (aj);
\node at (4,-1.5)[below] {$\gamma_1$};
\end{tikzpicture}
\caption{$I_{\gamma_{2/3}}(a_0;\cdots;a_j)$ to $I_{\gamma_{1}}(a_0;\cdots;a_j)$}
\label{fig: M_{v_i0}I(a_i;...;a_j), gamma_{2/3} -> gamma_1}
\end{subfigure}
\caption{Deformation of $I(a_0;\cdots;a_j)$}
\label{fig: M_{v_i0}I(a_i;...;a_j)}
\end{figure}

To justify~\eqref{eq: M_{i0}I(a_i;...;a_j)}, first we deform $\gamma_0$ to $\gamma_{1/3}$ as $a_{1},\cdots,a_{i_0}$ moves clockwise by $\pi/2$. This is shown in Figure~\ref{fig: M_{v_i0}I(a_i;...;a_j), gamma_0 -> gamma_{1/3}}, where the faint path is $\gamma_0$, while the dashed paths are traces of $a_1,\cdots,a_{i_0}$. Then we deform $\gamma_{1/3}$ to $\gamma_{2/3}$ as $a_{1},\cdots,a_{i_0}$ move clockwise by $\pi$. This is illustrated in Figure~\ref{fig: M_{v_i0}I(a_i;...;a_j), gamma_{1/3} -> gamma_{2/3}}. Lastly we deform $\gamma_{2/3}$ to $\gamma_{1}$ as $a_{1},\cdots,a_{i_0}$ move clockwise by $\pi/2$, and back to where they started. This is presented in Figure~\ref{fig: M_{v_i0}I(a_i;...;a_j), gamma_{2/3} -> gamma_1}.

\eqref{eq: M_{i0}I(0;...;a_i)} is similar, and the deformation from $\gamma_0$ to $\gamma_1$ is shown in Figure~\ref{fig: M_{v_i0}I(0;...;a_i), gamma_0 -> gamma_1}.

\begin{figure}[H]
\centering
\begin{tikzpicture}[scale=1]
\foreach \x/\y/\p/\labelpos/\labeltext in {0/0/o/above/{$0$}, 2/0/ai0p1/below/{$a_{i_0+1}$}, 3/0/ai0/above/{$a_{i_0}$}, 5/0/ai/above/{$a_i$}, 7/0/a1/above/{$a_1$}}{
    \coordinate (\p) at (\x,\y);
    \draw[fill] (\p) circle (0.04);
    \node at (\p)[\labelpos] {\labeltext};
}
\node at (1,0) {$\cdots$};
\node at (4,0) {$\cdots$};
\node at (6,0) {$\cdots$};
\node at (0,.7)[above] {$\gamma_1$};
\node at (2,-1)[below, opacity=0.3] {$\gamma_0$};
\draw[-] (o)--(.2,0);
\draw[->-=0.5, opacity=0.3] (.2,0) to [curve through={(2,-1)}] (ai);
\draw[->-=0.5] (.2,0) to [curve through={(-.5,0)..(0,.7)..(2.3,.2)..(2.7,-.2)}] (ai);
\end{tikzpicture}
\caption{Deformation from $I_{\gamma_0}(0;\cdots;a_i)$ to $I_{\gamma_1}(0;\cdots;a_i)$}
\label{fig: M_{v_i0}I(0;...;a_i), gamma_0 -> gamma_1}
\end{figure}
\end{proof}

\subsection{Deformation of integration path under $\mathcal M_{\nu_{i_0},\nu_{j_0}}$}

\begin{theorem}\label{thm: monodromy on iterated integrals II}\hfill
\begin{enumerate}[i.]
\item 
\begin{equation}\label{eq: M_{i0,j0}I(a_{i0};...;a_{j0+1})}
\begin{aligned}
\mathcal M_{\nu_{i_0,j_0}}I(a_{i_0}&;\cdots;a_{j_0+1})\\
&=I_{\sigma_{i_0+1}\cdots\sigma_{j_0}\sigma_{j_0+1}^{-1}\cdots\sigma_{i_0}^{-1}\sigma_{j_0}^{-1}\cdots\sigma_{i_0+1}^{-1}\sigma_{i_0}\cdots\sigma_{j_0}}(a_{i_0};\cdots;a_{j_0+1})\\
&=I_{\sigma_{i_0+1}\cdots\sigma_{j_0}\sigma_{j_0}^{-1}\cdots\sigma_{i_0+1}^{-1}\sigma_{j_0}^{-1}\cdots\sigma_{i_0+1}^{-1}\sigma_{i_0+1}\cdots\sigma_{j_0}}(a_{i_0};\cdots;a_{j_0+1})\\
&=I(a_{i_0};\cdots;a_{j_0+1})
\end{aligned}
\end{equation}
\item If $j_0+1<j\leq d+1$, then
\begin{equation}\label{eq: M_{i0,j0}I(a_{i0};...;a_j)}
\begin{aligned}
\mathcal M_{\nu_{i_0,j_0}}I(a_{i_0};\cdots;a_j)&=I_{\sigma_{i_0+1}\cdots\sigma_{j_0}\sigma_{j_0+1}^{-1}\cdots\sigma_{i_0}^{-1}}(a_{i_0};\cdots;a_j)\\
&=I_{\sigma_{i_0+1}\cdots\sigma_{j_0}\sigma_{j_0+1}^{-1}\cdots\sigma_{i_0+1}^{-1}}(a_{i_0};\cdots;a_j)\\
\end{aligned}
\end{equation}
\item If $1\leq i<i_0$, then
\begin{equation}\label{eq: M_{i0,j0}I(a_i;...;a_{j0+1})}
\begin{aligned}
\mathcal M_{\nu_{i_0,j_0}}I(a_i;\cdots;a_{j_0+1})&=I_{\sigma_{j_0}^{-1}\cdots\sigma_{i_0+1}^{-1}\sigma_{i_0}\cdots\sigma_{j_0}}(a_i;\cdots;a_{j_0+1})\\
% &=\begin{cases}
% I_{\sigma_{i_0}\cdots\sigma_{j_0}}(a_i;\cdots;a_{j_0+1}),\quad\text{if $a_i$ is ascending}\\
% I_{\sigma_{j_0}^{-1}\cdots\sigma_{i_0+1}^{-1}\sigma_{i_0}}(a_i;\cdots;a_{j_0+1}),\quad\text{if $a_i$ is descending}
% \end{cases}
\end{aligned}
\end{equation}
\item 
\begin{equation}\label{eq: M_{i0,j0}I(0;...;a_{i0})}
\begin{aligned}
\mathcal M_{\nu_{i_0,j_0}}I(0;\cdots;a_{i_0})&=I_{\sigma_{i_0}\cdots\sigma_{j_0}\sigma_{j_0+1}^{-1}\cdots\sigma_{i_0+1}^{-1}}(0;\cdots;a_{i_0})\\
&=I_{\sigma_{i_0+1}\cdots\sigma_{j_0}\sigma_{j_0+1}^{-1}\cdots\sigma_{i_0+1}^{-1}}(0;\cdots;a_{i_0})\\
% &=\begin{cases}
% I_{\sigma_{i_0+1}\cdots\sigma_{j_0}\sigma_{j_0+1}^{-1}}(0;\cdots;a_{i_0}),\quad\text{if $a_i$ is ascending}\\
% I_{\sigma_{j_0+1}^{-1}\cdots\sigma_{i_0+1}^{-1}}(0;\cdots;a_{i_0}),\quad\text{if $a_i$ is descending}
% \end{cases}
\end{aligned}
\end{equation}
\item 
\begin{equation}\label{eq: M_{i0,j0}I(0;...;a_{j0+1})}
\begin{aligned}
\mathcal M_{\nu_{i_0,j_0}}I(0;\cdots;a_{j_0+1})&=I_{\sigma_{j_0}^{-1}\cdots\sigma_{i_0+1}^{-1}\sigma_{i_0}\cdots\sigma_{j_0}}(0;\cdots;a_{j_0+1})\\
% &=\begin{cases}
% I_{\sigma_{i_0}\cdots\sigma_{j_0}}(0;\cdots;a_{j_0+1}),\quad\text{if $a_i$ is ascending}\\
% I_{\sigma_{j_0}^{-1}\cdots\sigma_{i_0+1}^{-1}\sigma_{i_0}}(0;\cdots;a_{j_0+1}),\quad\text{if $a_i$ is descending}
% \end{cases}
\end{aligned}
\end{equation}
\end{enumerate}
\end{theorem}

\begin{proof}
We choose $\nu_{i_0j_0}$ to be the loop in $S_d(\mathbb C)$ where $x_{i_0}$ traces through the path $\alpha\varepsilon\alpha^{-1}$ (see Figure~\ref{fig: choice of nu_{i0j0}}) and $\{x_j\}_{j\neq i_0}$ stay still. Here $\alpha$ is an arc and $\varepsilon$ is a sufficiently small circle around $(x_{i_0+1}\cdots x_{j_0})^{-1}$.

\begin{figure}[H]
\centering
\begin{tikzpicture}[scale=1]
\draw[fill] (-2,0) circle (0.03);
\node at (-2,0)[above]{$x_{i_0}$};
\draw[fill] (2,0) circle (0.03);
\node at (2,0.2)[above]{$(x_{i_0+1}\cdots x_{j_0})^{-1}$};
\draw[->-=0.5] (-2,0) arc (-180:-5:2);
\node at (0,-2)[above]{$\alpha$};
\draw[->-=0.5] (2.1744775495,0) arc (0:360:0.1744775495);
\node at (2.1744775495,0)[right]{$\varepsilon$};
\end{tikzpicture}
\caption{Choice of $\nu_{i_0,j_0}$}
\label{fig: choice of nu_{i0j0}}
\end{figure}

To justify~\eqref{eq: M_{i0,j0}I(a_{i0};...;a_{j0+1})}, first we deform $\gamma_0$ into $\gamma_{1/3}$. As shown in Figure~\ref{fig: M_{v_i0j0}I(a_i0;...;a_{j0+1}), gamma_0 -> gamma_{1/3}}. The faint path is $\gamma_0$, while the dashed path is the trace of $a_{i_0}$.

\begin{figure}[H]
\centering
\begin{subfigure}[b]{0.8\textwidth}
\centering
\begin{tikzpicture}[scale=1]
\def\r{0.5229344565}
\def\X{0.04557674096}
\def\Y{0.520944533}
\foreach \x/\y/\p/\labelpos/\labeltext in {0/0/aj0p1/right/{$a_{j_0+1}$}, 2/0/aj0/right/{$a_{j_0}$}, 6/0/ai00/below/{$a_{i_0}$}, \X/\Y/ai0'/above right/{}, \X/-\Y/ai0/right/{}}{
    \coordinate (\p) at (\x,\y);
    \draw[fill] (\p) circle (0.04);
    \node at (\p)[\labelpos] {\labeltext};
}
\node at (4,0) {$\cdots$};
\draw[->-=0.5, blue, dashed] (ai00) arc (0:170:3);
\draw[->-=0.5, blue, dashed] (ai0') arc (85:275:\r);
\draw[->-=0.5, opacity=0.3] (ai00) to [curve through={(2,-.5)..(1,-.4)}](aj0p1);
\node at (4,-.5)[below, opacity=0.3] {$\gamma_0$};
\draw[->-=0.5] (ai0) to [curve through={(-2,0)..(2,4)..(8,0)..(2,-1)}](aj0p1);
\node at (8,0)[left] {$\gamma_{1/3}$};
\end{tikzpicture}
\caption{Deformation from $I_{\gamma_0}(a_{i_0};\cdots;a_{j_0+1})$ to $I_{\gamma_{1/3}}(a_{i_0};\cdots;a_{j_0+1})$}
\label{fig: M_{v_i0j0}I(a_i0;...;a_{j0+1}), gamma_0 -> gamma_{1/3}}
\end{subfigure}
\begin{subfigure}[b]{0.8\textwidth}
\ContinuedFloat
\centering
\begin{tikzpicture}[scale=1]
\def\r{0.5229344565}
\def\X{0.04557674096}
\def\Y{0.520944533}
\foreach \x/\y/\p/\labelpos/\labeltext in {0/0/aj0p1/above right/{$a_{j_0+1}$}, 2/0/aj0/right/{$a_{j_0}$}, \X/-\Y/ai0/below right/{$a_{i_0}$}}{
    \coordinate (\p) at (\x,\y);
    \draw[fill] (\p) circle (0.04);
    \node at (\p)[\labelpos] {\labeltext};
}
\node at (4,0) {$\cdots$};
\draw[->-=0.5, opacity=0.3] (ai0) to [curve through={(-2,0)..(2,4)..(8,0)..(2,-.7)}](aj0p1);
\node at (8,0)[left, opacity=0.3] {$\gamma_{1/3}$};
\draw[->-=0.5] (ai0) to [curve through={(-1.5,0)..(2,3.7)..(7,0)..(4,-1)..(2,-.5)..(1.5,0)..(5.5,0)..(6,-.5)..(6.5,0)..(2,3.3)..(-1,1)}](aj0p1);
\node at (4,-1)[above] {$\gamma_{2/3}$};
\end{tikzpicture}
\caption{Deformation from $I_{\gamma_{1/3}}(a_{i_0};\cdots;a_{j_0+1})$ to $I_{\gamma_{2/3}}(a_{i_0};\cdots;a_{j_0+1})$}
\label{fig: M_{v_i0j0}I(a_i0;...;a_{j0+1}), gamma_{1/3} -> gamma_{2/3}}
\end{subfigure}
\begin{subfigure}[b]{0.8\textwidth}
\ContinuedFloat
\centering
\begin{tikzpicture}[scale=1]
\def\r{0.5229344565}
\def\X{0.04557674096}
\def\Y{0.520944533}
\foreach \x/\y/\p/\labelpos/\labeltext in {0/0/aj0p1/below left/{$a_{j_0+1}$}, 2/0/aj0/right/{$a_{j_0}$}, \X/-\Y/ai00/below right/{$a_{i_0}$}, \X/\Y/ai0'/above right/{}, 6/0/ai0/below/{}}{
    \coordinate (\p) at (\x,\y);
    \draw[fill] (\p) circle (0.04);
    \node at (\p)[\labelpos] {\labeltext};
}
\node at (4,0) {$\cdots$};
\draw[->-=0.5, blue, dashed] (ai00) arc (-85:85:\r);
\draw[->-=0.5, blue, dashed] (ai0') arc (170:0:3);
\draw[->-=0.5, opacity=0.3] (ai00) to [curve through={(-1.5,0)..(2,3.7)..(7,0)..(4,-1)..(2,-.5)..(1.5,0)..(5.5,0)..(6,-.5)..(6.5,0)..(2,3.3)..(-1,1)}](aj0p1);
\node at (4,-1)[above, opacity=0.3] {$\gamma_{2/3}$};
\draw[->-=0.5] (ai0) to [curve through={(3,2.5)..(2,2)..(1,0)..(.5,-.5)}](ai00);
\node at (2,2)[left] {$\gamma'$};
\end{tikzpicture}
\caption{Deformation from $I_{\gamma_{2/3}}(a_{i_0};\cdots;a_{j_0+1})$ to $I_{\gamma_1}(a_{i_0};\cdots;a_{j_0+1})$}
\label{fig: M_{v_i0j0}I(a_i0;...;a_{j0+1}), gamma_{2/3} -> gamma_1}
\end{subfigure}

\caption{Deformation of $I(a_{i_0};\cdots;a_{j_0+1})$}
\label{fig: M_{v_i0j0}I(a_i0;...;a_{j0+1})}
\end{figure}

Notice that we didn't take the traces of $a_1,\cdots, a_{i_0-1}$ into account, simply because the iterated integral $I(a_{i_0;\cdots;a_{j_0+1}})$ is not affected by them. Next, we deform $\gamma_{1/3}$ into $\gamma_{2/3}$. This is illustrated in Figure~\ref{fig: M_{v_i0j0}I(a_i0;...;a_{j0+1}), gamma_{1/3} -> gamma_{2/3}}. Lastly, we can stretch $\gamma_{2/3}$ into $\gamma_1=\gamma'\gamma_{2/3}$. This is presented in Figure~\ref{fig: M_{v_i0j0}I(a_i0;...;a_{j0+1}), gamma_{2/3} -> gamma_1}. This proves the first equality in~\eqref{eq: M_{i0,j0}I(a_{i0};...;a_{j0+1})}. The second equality holds since $a_{i_0}$ and $a_{j_0+1}$ are the endpoints, so the monodromy $\sigma_{i_0}$, $\sigma_{j_0+1}$ are trivial. The last equality holds because $\sigma$'s cancel off each other.

The deformations in \eqref{eq: M_{i0,j0}I(a_{i0};...;a_j)}, \eqref{eq: M_{i0,j0}I(a_i;...;a_{j0+1})}, \eqref{eq: M_{i0,j0}I(0;...;a_{i0})} and~\eqref{eq: M_{i0,j0}I(0;...;a_{j0+1})} are similar. The eventual paths $\gamma_1$ are presented in Figure~\ref{fig: deformations for the rest}.

\begin{figure}[H]
\centering
% First row, left
\begin{subfigure}[b]{0.45\textwidth}
\centering
\begin{tikzpicture}[scale=1]
\foreach \x/\y/\p/\labelpos/\labeltext in {0/0/aj/above/{$a_{j}$}, 2/0/aj0p1/below/{$a_{j_0+1}$}, 3/0/aj0/below/{$a_{j_0}$}, 5/0/ai0/below/{$a_{i_0}$}}{
\coordinate (\p) at (\x,\y);
\draw[fill] (\p) circle (0.04);
\node at (\p)[\labelpos] {\labeltext};
}
\node at (1,0) {$\cdots$};
\node at (4,0) {$\cdots$};
\draw[->-=0.5, opacity=0.3] (ai0) to [curve through={(2,-1)}](aj);
\node at (2,-1)[above, opacity=0.3] {$\gamma_0$};
\draw[->-=0.5] (ai0) to [curve through={($(aj0p1)+(.2,0)$)..($(aj0p1)+(0,-.2)$)..($(aj0p1)+(-.2,0)$)..(4,1)..($(ai0)+(.5,0)$)..(4,-1)..(2,-1.2)}](aj);
\node at (4,1)[above] {$\gamma_1$};
\end{tikzpicture}
\caption{Deformation of $I(a_{i_0};\cdots;a_j)$}
\label{fig: M_{i0,j0}I(a_{i0};...;a_j)}
\end{subfigure}
% First row, right
\begin{subfigure}[b]{0.45\textwidth}
\centering
\begin{tikzpicture}[scale=1]
\foreach \x/\y/\p/\labelpos/\labeltext in {0/0/aj0p1/above left/{$a_{j_0+1}$}, 1/0/aj0/below right/{$a_{j_0}$}, 3/0/ai0/above/{$a_{i_0}$}, 5/0/ai/below/{$a_i$}}{
    \coordinate (\p) at (\x,\y);
    \draw[fill] (\p) circle (0.04);
    \node at (\p)[\labelpos] {\labeltext};
}
\node at (2,0) {$\cdots$};
\node at (4,0) {$\cdots$};
\draw[->-=0.5, opacity=0.3] (ai) to [curve through={(2,-1)}](aj0p1);
\node at (2,-1)[above, opacity=0.3] {$\gamma_0$};
\draw[->-=0.5] (ai) to [curve through={(3,-.5)..(2,-.5)..(.7,0)..(2,.3)..($(ai0)+(-.2,0)$)..($(ai0)+(0,-.2)$)..($(ai0)+(.2,0)$)..($(ai0)+(0,.7)$)..(2,.7)..(1,.5)}](aj0p1);
\node at (1,.5)[above] {$\gamma_1$};
\end{tikzpicture}
\caption{Deformation of $I(a_i;\cdots;a_{j_0+1})$}
\label{fig: M_{i0,j0}I(a_i;...;a_{j0+1})}
\end{subfigure}
\vspace{1cm} % Add space between rows
% Second row, left
\begin{subfigure}[b]{0.45\textwidth}
\centering
\begin{tikzpicture}[scale=1]
\foreach \x/\y/\p/\labelpos/\labeltext in {0/0/o/above/{$0$}, 2/0/aj0p1/below/{$a_{j_0+1}$}, 3/0/aj0/below/{$a_{j_0}$}, 5/0/ai0/below/{$a_{i_0}$}}{
    \coordinate (\p) at (\x,\y);
    \draw[fill] (\p) circle (0.04);
    \node at (\p)[\labelpos] {\labeltext};
}
\node at (1,0) {$\cdots$};
\node at (4,0) {$\cdots$};
\draw[-] (o)--(.2,0);
\draw[->-=0.5, opacity=0.3] (.2,0) to [curve through={(2,-1)}](ai0);
\node at (2,-1)[above, opacity=0.3] {$\gamma_0$};
\draw[->-=0.5] (.2,0) to [curve through={(2,-1.2)..(4,-1)..($(ai0)+(.5,0)$)..(4,1)..($(aj0p1)+(-.2,0)$)..($(aj0p1)+(0,-.2)$)..($(aj0p1)+(.2,0)$)}](ai0);
\node at (4,1)[above] {$\gamma_1$};
\end{tikzpicture}
\caption{Deformation of $I(0;\cdots;a_{i_0})$}
\label{fig: M_{i0,j0}I(0;...;a_i0)}
\end{subfigure}
% Second row, right
\begin{subfigure}[b]{0.45\textwidth}
\centering
\begin{tikzpicture}[scale=1]
\foreach \x/\y/\p/\labelpos/\labeltext in {0/0/o/above/{$0$}, 2/0/aj0p1/above/{$a_{j_0+1}$}, 3/0/aj0/below/{$a_{j_0}$}, 5/0/ai0/below right/{$a_{i_0}$}}{
    \coordinate (\p) at (\x,\y);
    \draw[fill] (\p) circle (0.04);
    \node at (\p)[\labelpos] {\labeltext};
}
\node at (1,0) {$\cdots$};
\node at (4,0) {$\cdots$};
\draw[-] (o)--(.2,0);
\draw[->-=0.5, opacity=0.3] (.2,0) to [curve through={(1,-.2)}](aj0p1);
\node at (1,-.2)[below, opacity=0.3] {$\gamma_0$};
\draw[->-=0.5] (.2,0) to [curve through={(1,-.7)..(2.5,0)..(3,.2)..(4.8,0)..(5,-.2)..(5.2,0)..(3,.5)}](aj0p1);
\node at (3,.5)[above] {$\gamma_1$};
\end{tikzpicture}
\caption{Deformation of $I(0;\cdots;a_{j_0+1})$}
\label{fig: M_{i0,j0}I(0;...;a_{j0+1})}
\end{subfigure}

\caption{Deformations for \eqref{eq: M_{i0,j0}I(a_{i0};...;a_j)}, \eqref{eq: M_{i0,j0}I(a_i;...;a_{j0+1})}, \eqref{eq: M_{i0,j0}I(0;...;a_{i0})} and~\eqref{eq: M_{i0,j0}I(0;...;a_{j0+1})}}
\label{fig: deformations for the rest}
\end{figure}

\end{proof}

\section{Computation of monodromy matrices}

\subsection{Monodromies of iterated integrals}

In this section, let us assume $\{a_i\}_i\subseteq\mathbb C$, and $\sigma_p$ are the loops such that $\displaystyle\int_{\sigma_q}d\log(z-a_q)=2\pi i\delta_{pq}$ where $\delta$ is the Kronecker delta. The following Lemma is the key to the calculation of the monodromies matrices.

\begin{lemma}\label{lem: monodromy of Iterated integrals}(Corollary 2.6 in \cite{Goncharov_MultiplePolylogarithmsAndMixedTateMotives}, Proposition 6.3 in~\cite{FrancisBrown_SingleValuedHyperlogarithmsAndUnipotentDifferentialEquations})
Suppose $\gamma=\gamma_1'\gamma_2'$ is a path from $a_0$ to $a_{n+1}$ and $\gamma'=\gamma_1'\sigma\gamma_2'$, $a_i\neq a$ for $1\leq i\leq n$, then
\begin{equation}\label{Equation for monodromy}
I_{\gamma'}(a_0;\cdots,\overbrace{a,\cdots,a}^k,\cdots;a_{n+1})-I_{\gamma}(a_0;\cdots,\overbrace{a,\cdots,a}^k,\cdots;a_{n+1})
\end{equation}
is equal to
\begin{equation}\label{Equation for monodromy2}
\sum_{\substack{p+q+r=k\\r\geq1}}I_{\gamma_1}(a_0;\cdots,\overbrace{a,\cdots,a}^p;a)I_{\sigma}(a;\overbrace{a,\cdots,a}^r;a)I_{\gamma_2}(a;\overbrace{a,\cdots,a}^q,\cdots;a_{n+1})
\end{equation}
Which is equal to
\begin{equation}\label{eq: equation for monodromy}
\sum_{\substack{p+q+r=k\\r\geq1}}\frac{(2\pi i)^{r}}{(r)!}I_{\gamma_1}(a_0;\cdots,\overbrace{a,\cdots,a}^p;a)I_{\gamma_2}(a;\overbrace{a,\cdots,a}^q,\cdots;a_{n+1})
\end{equation}
\begin{figure}[H]
\centering
\begin{tikzpicture}[scale=2]
% \mygrid{(-3,-3)}{(3,3)}{(0,0)}
\coordinate (s) at (-2,0); \coordinate (t) at (2,0); \coordinate (a) at (0,1); \coordinate (b) at (0,0.9);
\draw[fill] (s) circle (0.02); \draw[fill] (t) circle (0.02); \draw[fill] (a) circle (0.02); \draw[fill] (b) circle (0.02);
\draw[->-=0.2,->-=0.8] (s) to [curve through={(-1,0.6)..(b)..(1,0.6)}] (t);
\draw[->-=0.2,->-=0.8,red] (s) to [curve through={(-1,0.7)}] (a);
\draw[->-=0.2,->-=0.8,blue] (a) to [curve through={(1,0.7)}] (t);
\draw[->-=0.5] (b) to [curve through={(0,1.2)}] (b);
\node at (s)[left] {$a_0$};
\node at (t)[right] {$a_{n+1}$};
\node at (a)[above] {$a$};
\node at (0,1.2)[above] {$\sigma$};
\node at (b)[below] {$a'$};
\node[red] at (-1.4,0.5)[above] {$\gamma_1$};
\node[blue] at (1.4,0.5)[above] {$\gamma_2$};
\node at (-1.4,0.4)[below] {$\gamma_1'$};
\node at (1.4,0.4)[below] {$\gamma_2'$};
\end{tikzpicture}
\caption{Monodromy of $I(a_0;\cdots,a,\cdots,a,\cdots;a_{n+1})$ at $a$}
\label{fig: monodromies of iterated integrals}
\end{figure}
\end{lemma}

\begin{proof}
Let's write $\omega_p=\dfrac{dt}{t-a_p}$ and $\omega=\dfrac{dt}{t-a}$, then \eqref{eq: equation for monodromy} can be written as
\begin{align*}
&\int_{\gamma_1'\sigma\gamma_2'}\omega_1\cdots\overbrace{\omega\cdots\omega}^k\cdots\omega_n-\int_{\gamma_1'\gamma_2'}\omega_1\cdots\overbrace{\omega\cdots\omega}^k\cdots\omega_n\\
&=\sum_{\substack{p+q+r=k\\r\geq1}}\int_{\gamma_1'}\omega_1\cdots\overbrace{\omega\cdots\omega}^p\int_{\sigma}\overbrace{\omega\cdots\omega}^r\int_{\gamma_2'}\overbrace{\omega\cdots\omega}^q\cdots\omega_n\\
&+\sum_i\sum_{q+r=k}\int_{\gamma_1'}\omega_1\cdots\omega_i\int_{\sigma}\omega_{i+1}\cdots\overbrace{\omega\cdots\omega}^r\int_{\gamma_2'}\overbrace{\omega\cdots\omega}^q\cdots\omega_n\\
&+\sum_j\sum_{p+r=k}\int_{\gamma_1'}\omega_1\cdots\overbrace{\omega\cdots\omega}^p\int_{\sigma}\overbrace{\omega\cdots\omega}^r\cdots\omega_j\int_{\gamma_2'}\omega_{j+1}\cdots\omega_n
\end{align*}
When $a'$ approaches $a$, we may take the limit of $\gamma_1',\gamma_2'$ to be $\gamma_1,\gamma_2$ respectively. If we choose $\sigma$ to be $a+\epsilon e^{i\theta}$, and let $\epsilon\to0$ we have
\begin{align*}
\int_{\sigma}\overbrace{\omega\cdots\omega}^r&=\int_0^{2\pi}\overbrace{\frac{\epsilon ie^{i\theta}d\theta}{\epsilon e^{i\theta}}\cdots\frac{\epsilon ie^{i\theta}d\theta}{\epsilon e^{i\theta}}}^r=i^r\int_0^{2\pi}\overbrace{d\theta\cdots d\theta}^r=\frac{(2\pi i)^r}{r!}
\end{align*}
which explains the coefficients in the first sum. To prove that the second and third sum vanish as $\epsilon\to0$, note that
\begin{align*}
\left|\int_{\sigma}\omega_{i+1}\cdots\overbrace{\omega\cdots\omega}^r\right|&=\left|\int_{0}^{2\pi}\frac{\epsilon ie^{i\theta}d\theta}{a-a_{i+1}+\epsilon e^{i\theta}}\cdots\overbrace{\frac{\epsilon ie^{i\theta}d\theta}{\epsilon e^{i\theta}}\cdots\frac{\epsilon ie^{i\theta}d\theta}{\epsilon e^{i\theta}}}^r\right|\\
&\leq\epsilon\cdot\frac{1}{|a-a_{i+1}|-\epsilon}\cdots\left|\int_0^{2\pi}\overbrace{\frac{\epsilon ie^{i\theta}d\theta}{\epsilon e^{i\theta}}\cdots\frac{\epsilon ie^{i\theta}d\theta}{\epsilon e^{i\theta}}}^r\right|\leq\epsilon\cdot C
\end{align*}
and similarly $\displaystyle\left|\int_{\sigma}\overbrace{\omega\cdots\omega}^r\cdots\omega_j\right|\leq\epsilon\cdot C$. One might argue that $\displaystyle\int_{\gamma_2'}\overbrace{\omega\cdots\omega}^q\cdots\omega_n$ is singular, but we know from Proposition~\ref{prop: I_epsilon = O(log^m(epsilon))} that it is $O(\log^q\epsilon)$. Therefore we have proved \eqref{eq: equation for monodromy}.
\end{proof}

\begin{remark}
If we replace $\sigma$ with $\sigma^{-1}$, then
\[
\int_{\sigma^{-1}}\overbrace{\omega\cdots\omega}^r=(-1)^r\int_{\sigma}\overbrace{\omega\cdots\omega}^r=\frac{(-2\pi i)^r}{r!}
\]
and~\eqref{Equation for monodromy2} becomes
\begin{equation}\label{eq: equation for monodromy with sign}
\sum_{\substack{p+q+r=k\\r\geq1}}\frac{(-2\pi i)^{r}}{(r)!}I_{\gamma_1}(a_0;\cdots,\overbrace{a,\cdots,a}^p;a)I_{\gamma_2}(a;\overbrace{a,\cdots,a}^q,\cdots;a_{n+1})
\end{equation}
\end{remark}

When $\gamma$, $\gamma_1$, $\gamma_2$ are fixed choices, without ambiguity, we omit the mention of the integration path $\gamma$, and write only its monodromy loops. For instance, we simplify $I_{\gamma'}$ to $I_\sigma$ and $I_\gamma$, $I_{\gamma_1}$, $I_{\gamma_2}$ to $I$. In addition, for degenerates $I(a;\cdots;a)$, we pick the trivial path so that they evaluate to zero.

If we remove the condition $a_i\neq a$, for $1\leq i\leq n$, and deploy the notation in Definition~\ref{def: I^w}. Lemma~\ref{lem: monodromy of Iterated integrals} can be easily generalized.

\begin{corollary}\label{cor: monodromy of iterated integral}
\begin{equation}\label{eq: monodromy for iterated integral in general - one loop}
I_{\sigma_p^{\epsilon}}(a_{i_0};a_{i_1},\cdots,a_{i_n};a_{i_{n+1}})=\sum_{k=0}^\infty\frac{(2\pi i\epsilon)^k}{k!} I^{\sigma_p^k}(a_{i_0};a_{i_1},\cdots,a_{i_n};a_{i_{n+1}})
\end{equation}
Here $\epsilon=\pm1$ is the sign, $I(a;a)=1$ and degenerates vanish. For a product of loops, we have
\begin{equation}\label{eq: monodromy for iterated integral in general}
I_{\sigma_{p_{1}}^{\epsilon_1}\cdots \sigma_{p_{m}}^{\epsilon_m}}(a_{i_0};\cdots;a_{i_{n+1}})=\sum_{k_1,\cdots,k_l\geq0}\prod_{r=1}^{m}\frac{(2\pi i\epsilon_r)^{k_r}}{k_r!}I^{\sigma_{p_{1}}^{k_1}\cdots\sigma_{p_{m}}^{k_m}}(a_{i_0};\cdots;a_{i_{n+1}})
\end{equation}
Where $\epsilon_r=\pm1$. By Proposition~\ref{prop: basic properties of iterated integrals}, ii., We also deduce that
\begin{equation}\label{eq: monodromy for iterated integral in general - path reversed}
\begin{aligned}
I_{\sigma_{p_m}^{\epsilon_m}\cdots\sigma_{p_1}^{\epsilon_1}}&(a_{i_{n+1}};\cdots;a_{i_0})=\sum_{k_1,\cdots,k_l\geq0}\prod_{r=1}^{m}\frac{(2\pi i\epsilon_r)^{k_r}}{k_r!}I^{\sigma_{p_m}^{k_m}\cdots\sigma_{p_1}^{k_1}}(a_{i_{n+1}};\cdots;a_{i_0})\\
&=\sum_{k_1,\cdots,k_l\geq0}\prod_{r=1}^{m}\frac{(2\pi i\epsilon_r)^{k_r}}{k_r!}(-1)^{n-k_m-\cdots-k_1}I^{\sigma_{p_1}^{k_1}\cdots\sigma_{p_m}^{k_m}}(a_{i_0};\cdots;a_{i_{n+1}})\\
&=(-1)^n\sum_{k_1,\cdots,k_l\geq0}\prod_{r=1}^{m}\frac{(-2\pi i\epsilon_r)^{k_r}}{k_r!}I^{\sigma_{p_1}^{k_1}\cdots\sigma_{p_m}^{k_m}}(a_{i_0};\cdots;a_{i_{n+1}})\\
&=(-1)^nI_{\sigma_{p_1}^{-\epsilon_1}\cdots\sigma_{p_m}^{-\epsilon_m}}(a_{i_0};\cdots;a_{i_{n+1}})
\end{aligned}
\end{equation}
\end{corollary}

\begin{remark}
If $p_r=p_{r+1}$ and $\epsilon_r=-\epsilon_{r+1}$, then $\sigma_{p_r}^{\epsilon_r}\sigma_{p_{r+1}}^{\epsilon_{r+1}}=1$ may cancel each other.
\end{remark}

\begin{proof}
This is just applying Lemma~\ref{lem: monodromy of Iterated integrals} repeatedly.
\end{proof}

\subsection{Computation of monodromy matrices}

Recall from Proposition~\ref{prop: structure of variation matrix}, that
\begin{equation}\label{eq: general entry in V^I}
(-1)^{l-k}I^{\sigma_{i_1}\sigma_0^{m_{i_1}-1}\cdots\sigma_{i_k}\sigma_0^{m_{i_k}-1}}(0;a_{j_1},0^{p_{j_1}-1},\cdots,a_{j_l},0^{p_{j_l}-1};1)
\end{equation}
is the complementary entry of $(-1)^kI(0;a_{i_1},0^{m_{i_1}-1},\cdots,a_{i_k},0^{m_{i_k}-1};1)$ with respect to $(-1)^lI(0;a_{j_1},0^{p_{j_1}-1},\cdots,a_{j_l},0^{p_{j_l}-1};1)$. Now we can describe a concrete algorithm for computing monodromy matrices. For this, we only need to compute the entry corresponding to~\eqref{eq: general entry in V^I} in the monodromy matrix.

First we discuss the monodromy matrix for $\mathcal M_{\nu_{i_0}}$.

\begin{theorem}
Suppose $i_r\leq i_0<i_{r+1}$, and denote $w_0=\sigma_{i_1}\sigma_0^{m_{i_1}-1}\cdots\sigma_{i_k}\sigma_0^{m_{i_k}-1}$, we have
\begin{equation}
\mathcal M_{\nu_{i_0}}I^{w_0}(0;a_{j_1},0^{p_{j_1}-1},\cdots,a_{j_l},0^{p_{j_l}-1};1)=\sum_{w}M^{(i_0)}_{w,w_0}I^{w}(0;a_{j_1},0^{p_{j_1}-1},\cdots,a_{j_l},0^{p_{j_l}-1};1)
\end{equation}
With constant coefficient
\begin{equation}
M^{(i_0)}_{w,w_0}=\sum_{\substack{\theta_\alpha\in\{0,1\}\\q_0,q_r\geq0}}\frac{(-2\pi i)^{q_0}}{q_0!}(2\pi i)^{\theta_{i_0+1}+\cdots+\theta_{i_{r+1}-1}}\frac{(2\pi i)^{q_r}}{q_r!}
\end{equation}
for $w=\sigma_0^{q_0}\sigma_{i_1}\sigma_0^{m_{i_1}-1}\cdots\sigma_{i_r}\sigma_{i_0+1}^{\theta_{i_0+1}}\cdots\sigma_{i_{r+1}-1}^{\theta_{i_{r+1}-1}}\sigma_0^{m_{i_r}-1+q_r}\cdots\sigma_{i_k}\sigma_0^{m_{i_k}-1}$, and otherwise zero. It is then straightfoward to see that $M^{(i_0)}=\left\{(-1)^{k+\theta_{i_0+1}+\cdots+\theta_{i_{r+1}-1}}M^{(i_0)}_{w,v}\right\}_{w,v}$ defines precisely the monodromy matrix for the operator $\mathcal M_{\nu_{i_0}}$.
\end{theorem}

\begin{proof}
First notice that
\begin{multline}\label{eq: depcomposition of an entry in the variation matrix}
I^{\sigma_{i_1}\sigma_0^{m_{i_1}-1}\cdots\sigma_{i_k}\sigma_0^{m_{i_k}-1}}(0;a_{j_1},0^{p_{j_1}-1},\cdots,a_{j_l},0^{p_{j_l}-1};1)\\=I(0;\cdots;a_{i_1})\left(\prod_{t=1}^{k}I^{\sigma_0^{m_{i_t}-1}}(a_{i_t};\cdots;a_{i_{t+1}})\right)
\end{multline}
And if $m_{i_t}>1$,
\begin{equation}
I^{\sigma_0^{m_{i_t}-1}}(a_{i_t};\cdots;a_{i_{t+1}})=\sum I(a_{i_t};\cdots;0)I(0;\cdots;a_{i_{t+1}})
\end{equation}
Thanks to Corollary~\ref{cor: monodromy of iterated integral} and Theorem~\ref{thm: monodromy on iterated integrals I}, we have
\begin{equation}
\mathcal M_{\nu_{i_0}}I(0;\cdots;a_{i_1})=\sum_{q_0\geq0}\frac{(-2\pi i)^{q_0}}{q_0!}I^{\sigma_0^{q_0}}(0;\cdots;a_{i_1})
\end{equation}
If $m_{i_r}=1$,
\begin{multline}
\mathcal M_{\nu_{i_0}}I(a_{i_r};\cdots;a_{i_{r+1}})\\=\sum_{\substack{\theta_{i_0+1},\cdots,\theta_{i_{r+1}-1}\in\{0,1\}\\q_r\geq0}}(2\pi i)^{\theta_{i_0+1}+\cdots+\theta_{i_{r+1}-1}}\frac{(2\pi i)^{q_r}}{q_r!}I^{\sigma_{i_0+1}^{\theta_{i_0+1}}\cdots\sigma_{i_{r+1}-1}^{\theta_{i_{r+1}-1}}\sigma_0^{q_r}}(a_{i_r};\cdots;a_{i_{r+1}})
\end{multline}
And if $m_{i_r}>1$,
\begin{equation}
\begin{aligned}
\mathcal M_{\nu_{i_0}}&I^{\sigma_0^{m_{i_r}-1}}(a_{i_r};\cdots;a_{i_{r+1}})=\sum\mathcal M_{\nu_{i_0}}I(a_{i_r};\cdots;0)\mathcal M_{\nu_{i_0}}I(0;\cdots;a_{i_{r+1}})\\
&=\sum_{\substack{\theta_\alpha\in\{0,1\}\\q_r\geq0}}(2\pi i)^{\theta_{i_0+1}+\cdots+\theta_{i_{r+1}-1}}\frac{(2\pi i)^{q_r}}{q_r!}I^{\sigma_{i_0+1}^{\theta_{i_0+1}}\cdots\sigma_{i_{r+1}-1}^{\theta_{i_{r+1}-1}}\sigma_0^{q_r}}(a_{i_r};\cdots;0)I(0;\cdots;a_{i_{r+1}})\\
&=\sum_{\substack{\theta_\alpha\in\{0,1\}\\q_r\geq0}}(2\pi i)^{\theta_{i_0+1}+\cdots+\theta_{i_{r+1}-1}}\frac{(2\pi i)^{q_r}}{q_r!}I^{\sigma_{i_0+1}^{\theta_{i_0+1}}\cdots\sigma_{i_{r+1}-1}^{\theta_{i_{r+1}-1}}\sigma_0^{m_{i_r}-1+q_r}}(a_{i_r};\cdots;a_{i_{r+1}})
\end{aligned}
\end{equation}
Note that for the second equality~\eqref{eq: monodromy for iterated integral in general - path reversed} is used. If $m_{i_t}>1$ and $t<r$,
\begin{equation}
\begin{aligned}
\mathcal M_{\nu_{i_0}}I^{\sigma_0^{m_{i_t}-1}}(a_{i_t};\cdots;a_{i_{t+1}})&=\sum\mathcal M_{\nu_{i_0}}I(a_{i_t};\cdots;0)\mathcal M_{\nu_{i_0}}I(0;\cdots;a_{i_{t+1}})\\
&=\sum_{x,y\geq0}\frac{(2\pi i)^x}{x!}I^{\sigma_0^x}(a_{i_t};\cdots;0)\frac{(-2\pi i)^y}{y!}I^{\sigma_0^y}(0;\cdots;a_{i_{t+1}})\\
&=\sum_{q_t\geq0}(2\pi i)^{q_t}\sum_{x+y=q_t}\frac{(-1)^y}{x!y!}I^{\sigma_0^x}(a_{i_t};\cdots;0)I^{\sigma_0^y}(0;\cdots;a_{i_{t+1}})\\
&=0
\end{aligned}
\end{equation}
To summarize, we have
\begin{equation}
\begin{aligned}
\mathcal M_{\nu_{i_0}}&I^{\sigma_{i_1}\sigma_0^{m_{i_1}-1}\cdots\sigma_{i_k}\sigma_0^{m_{i_k}-1}}(0;a_{j_1},0^{p_{j_1}-1},\cdots,a_{j_l},0^{p_{j_l}-1};1)\\
&=\sum_{\substack{\theta_\alpha\in\{0,1\}\\q_0,q_r\geq0}}\frac{(-2\pi i)^{q_0}}{q_0!}(2\pi i)^{\theta_{i_0+1}+\cdots+\theta_{i_{r+1}-1}}\frac{(2\pi i)^{q_r}}{q_r!}\\
&I^{\sigma_0^{q_0}\sigma_{i_1}\sigma_0^{m_{i_1}-1}\cdots\sigma_{i_r}\sigma_{i_0+1}^{\theta_{i_0+1}}\cdots\sigma_{i_{r+1}-1}^{\theta_{i_{r+1}-1}}\sigma_0^{m_{i_r}-1+q_r}\cdots\sigma_{i_k}\sigma_0^{m_{i_k}-1}}(0;a_{j_1},0^{p_{j_1}-1},\cdots,a_{j_l},0^{p_{j_l}-1};1)
\end{aligned}
\end{equation}
\end{proof}

Next we discuss the monodromy matrix for $\mathcal M_{\nu_{i_0,j_0}}$.

\begin{theorem}
\begin{equation}
\mathcal M_{\nu_{i_0,j_0}}I^{w_0}(0;a_{j_1},0^{p_{j_1}-1},\cdots,a_{j_l},0^{p_{j_l}-1};1)=\sum_{w}M^{(i_0,j_0)}_{w,w_0}I^{w}(0;a_{j_1},0^{p_{j_1}-1},\cdots,a_{j_l},0^{p_{j_l}-1};1)
\end{equation}
With constant coefficient $M^{(i_0,j_0)}_{w,w_0}$ being
\begin{equation}
(-1)^\delta(2\pi i)^{\delta(\theta_{i_0+1}+\cdots+\theta_{j_0}+1)}
\end{equation}
for $w=\sigma_{i_1}\sigma_0^{m_{i_1}-1}\cdots\sigma_{i_r}\sigma_{i_0+1}^{\delta\theta_{i_0+1}}\cdots\sigma_{j_0}^{\delta\theta_{j_0}}\sigma_{j_0+1}^\delta\sigma_0^{m_{i_r}-1}\cdots\sigma_{i_k}\sigma_0^{m_{i_k}-1}$, $i_r=i_0<j_0+1<i_{r+1}$, and
\begin{equation}
(2\pi i)^{\delta(1+\theta_{i_0+1}+\cdots+\theta_{j_0})}
\end{equation}
for $w=\sigma_{i_1}\sigma_0^{m_{i_1}-1}\cdots\sigma_{i_r}\sigma_0^{m_{i_r}-1}\sigma_{i_0}^\delta\sigma_{i_0+1}^{\delta\theta_{i_0+1}}\cdots\sigma_{j_0}^{\delta\theta_{j_0}}\cdots\sigma_{i_k}\sigma_0^{m_{i_k}-1}$, $i_r<i_0<j_0+1=i_{r+1}$. It is then straightforward to see that $M^{(i_0,j_0)}=\left\{(-1)^{k+\delta(1+\theta_{i_0+1}+\cdots+\theta_{j_0})}M^{(i_0,j_0)}_{w,v}\right\}_{w,v}$ defines precisely the monodromy matrix for the operator $\mathcal M_{\nu_{i_0,j_0}}$.
\end{theorem}

\begin{proof}
Again with the help of Corollary~\ref{cor: monodromy of iterated integral} and Theorem~\ref{thm: monodromy on iterated integrals II}, it is not difficult to show that
\begin{equation}
\begin{aligned}
\mathcal M_{\nu_{i_0,j_0}}I(a_{i_0};\cdots;a_j)=\sum_{\theta_\alpha,\delta\in\{0,1\}}(-1)^\delta(2\pi i)^{\delta(\theta_{i_0+1}+\cdots+\theta_{j_0}+1)}I^{\sigma_{i_0+1}^{\delta\theta_{i_0+1}}\cdots\sigma_{j_0}^{\delta\theta_{j_0}}\sigma_{j_0+1}^\delta}(a_{i_0};\cdots;a_j)
\end{aligned}
\end{equation}
\begin{equation}
\begin{aligned}
\mathcal M_{\nu_{i_0,j_0}}I(a_{i_0};\cdots;0)=\sum_{\theta_\alpha,\delta\in\{0,1\}}(-1)^\delta(-2\pi i)^{\delta(\theta_{i_0+1}+\cdots+\theta_{j_0}+1)}I^{\sigma_{i_0+1}^{\delta\theta_{i_0+1}}\cdots\sigma_{j_0}^{\delta\theta_{j_0}}\sigma_{j_0+1}^\delta}(a_{i_0};\cdots;0)
\end{aligned}
\end{equation}
\begin{equation}
\begin{aligned}
\mathcal M_{\nu_{i_0,j_0}}I(a_i;\cdots;a_{j_0+1})=\sum_{\theta_\alpha,\delta\in\{0,1\}}(2\pi i)^{\delta(1+\theta_{i_0+1}+\cdots+\theta_{j_0})}I^{\sigma_{i_0}^\delta\sigma_{i_0+1}^{\delta\theta_{i_0+1}}\cdots\sigma_{j_0}^{\delta\theta_{j_0}}}(a_i;\cdots;a_{j_0+1})
\end{aligned}
\end{equation}
\begin{equation}
\begin{aligned}
\mathcal M_{\nu_{i_0,j_0}}I(0;\cdots;a_{j_0+1})=\sum_{\theta_\alpha,\delta\in\{0,1\}}(2\pi i)^{\delta(1+\theta_{i_0+1}+\cdots+\theta_{j_0})}I^{\sigma_{i_0}^\delta\sigma_{i_0+1}^{\delta\theta_{i_0+1}}\cdots\sigma_{j_0}^{\delta\theta_{j_0}}}(0;\cdots;a_{j_0+1})
\end{aligned}
\end{equation}
If $i_0\leq i_r<i_{r+1}\leq j_0+1$ or $\{i_0,j_0+1\}\cap\{i_r\}_{r=1}^k=\emptyset$, $\mathcal M_{\nu_{i_0,j_0}}$ acts trivially.

If $i_r=i_0<j_0+1<i_{r+1}$,
\begin{multline}
\mathcal M_{\nu_{i_0,j_0}}I^{\sigma_0^{m_{i_r}-1}}(a_{i_0};\cdots;a_{i_{r+1}})\\
=\sum_{\theta_\alpha,\delta\in\{0,1\}}(-1)^\delta(2\pi i)^{\delta(\theta_{i_0+1}+\cdots+\theta_{j_0}+1)}I^{\sigma_{i_0+1}^{\delta\theta_{i_0+1}}\cdots\sigma_{j_0}^{\delta\theta_{j_0}}\sigma_{j_0+1}^\delta\sigma_0^{m_{i_r}-1}}(a_{i_0};\cdots;a_{i_{r+1}})
\end{multline}
therefore
\begin{equation}
\begin{aligned}
&\mathcal M_{\nu_{i_0,j_0}}I^{\sigma_{i_1}\sigma_0^{m_{i_1}-1}\cdots\sigma_{i_k}\sigma_0^{m_{i_k}-1}}(0;a_{j_1},0^{p_{j_1}-1},\cdots,a_{j_l},0^{p_{j_l}-1};1)\\
&=\sum_{\theta_\alpha,\delta\in\{0,1\}}(-1)^\delta(2\pi i)^{\delta(\theta_{i_0+1}+\cdots+\theta_{j_0}+1)}\\
&I^{\sigma_{i_1}\sigma_0^{m_{i_1}-1}\cdots\sigma_{i_r}\sigma_{i_0+1}^{\delta\theta_{i_0+1}}\cdots\sigma_{j_0}^{\delta\theta_{j_0}}\sigma_{j_0+1}^\delta\sigma_0^{m_{i_r}-1}\cdots\sigma_{i_k}\sigma_0^{m_{i_k}-1}}(0;a_{j_1},0^{p_{j_1}-1},\cdots,a_{j_l},0^{p_{j_l}-1};1)
\end{aligned}
\end{equation}
If $i_r<i_0<j_0+1=i_{r+1}$,
\begin{multline}
\mathcal M_{\nu_{i_0,j_0}}I^{\sigma_0^{m_{i_r}-1}}(a_{i_r};\cdots;a_{j_0+1})\\
=\sum_{\theta_\alpha,\delta\in\{0,1\}}(2\pi i)^{\delta(1+\theta_{i_0+1}+\cdots+\theta_{j_0})}I^{\sigma_0^{m_{i_r}-1}\sigma_{i_0}^\delta\sigma_{i_0+1}^{\delta\theta_{i_0+1}}\cdots\sigma_{j_0}^{\delta\theta_{j_0}}}(a_{i_r};\cdots;a_{j_0+1})
\end{multline}
therefore,
\begin{equation}
\begin{aligned}
&\mathcal M_{\nu_{i_0,j_0}}I^{\sigma_{i_1}\sigma_0^{m_{i_1}-1}\cdots\sigma_{i_k}\sigma_0^{m_{i_k}-1}}(0;a_{j_1},0^{p_{j_1}-1},\cdots,a_{j_l},0^{p_{j_l}-1};1)\\
&=\sum_{\theta_\alpha,\delta\in\{0,1\}}(2\pi i)^{\delta(1+\theta_{i_0+1}+\cdots+\theta_{j_0})}\\
&I^{\sigma_{i_1}\sigma_0^{m_{i_1}-1}\cdots\sigma_{i_r}\sigma_0^{m_{i_r}-1}\sigma_{i_0}^\delta\sigma_{i_0+1}^{\delta\theta_{i_0+1}}\cdots\sigma_{j_0}^{\delta\theta_{j_0}}\cdots\sigma_{i_k}\sigma_0^{m_{i_k}-1}}(0;a_{j_1},0^{p_{j_1}-1},\cdots,a_{j_l},0^{p_{j_l}-1};1)
\end{aligned}
\end{equation}
\end{proof}

\begin{example}
The monodromy matrix of $\Li_{3,1}(x_1,x_2)$ for $\nu_{1}$, $\nu_{2}$, $\nu_{1,1}$, $\nu_{2,2}$, $\nu_{1,2}$ are respectively
\[
M_{1}=\begin{bmatrix}
 1 & 0 & 0 & 0 & 0 & 0 & 0 & 0 \\
 0 & 1 & 0 & 0 & 0 & 0 & 0 & 0 \\
 0 & 0 & 1 & 0 & 0 & 0 & 0 & 0 \\
 0 & 0 & -1 & 1 & 0 & 0 & 0 & 0 \\
 0 & 0 & 1 & 0 & 1 & 0 & 0 & 0 \\
 0 & 0 & 0 & 1 & 0 & 1 & 0 & 0 \\
 0 & 0 & \frac{1}{2} & 0 & 0 & 0 & 1 & 0 \\
 0 & 0 & 0 & \frac{1}{2} & 0 & 0 & 0 & 1 \\
\end{bmatrix},\quad M_2=\begin{bmatrix}
 1 & 0 & 0 & 0 & 0 & 0 & 0 & 0 \\
 0 & 1 & 0 & 0 & 0 & 0 & 0 & 0 \\
 0 & 0 & 1 & 0 & 0 & 0 & 0 & 0 \\
 0 & 0 & 0 & 1 & 0 & 0 & 0 & 0 \\
 0 & 0 & 1 & 0 & 1 & 0 & 0 & 0 \\
 0 & 0 & 0 & 0 & 0 & 1 & 0 & 0 \\
 0 & 0 & \frac{1}{2} & 0 & 0 & 0 & 1 & 0 \\
 0 & 0 & 0 & 0 & 0 & 0 & 0 & 1 \\
\end{bmatrix}
\]
\[
M_{1,1}=\begin{bmatrix}
 1 & 0 & 0 & 0 & 0 & 0 & 0 & 0 \\
 0 & 1 & 0 & 0 & 0 & 0 & 0 & 0 \\
 0 & 0 & 1 & 0 & 0 & 0 & 0 & 0 \\
 0 & -1 & 1 & 1 & 0 & 0 & 0 & 0 \\
 0 & 0 & 0 & 0 & 1 & 0 & 0 & 0 \\
 0 & 0 & 0 & 0 & 0 & 1 & 0 & 0 \\
 0 & 0 & 0 & 0 & 0 & 0 & 1 & 0 \\
 0 & 0 & 0 & 0 & 0 & 0 & 0 & 1 \\
\end{bmatrix},\quad M_{2,2}=\begin{bmatrix}
 1 & 0 & 0 & 0 & 0 & 0 & 0 & 0 \\
 -1 & 1 & 0 & 0 & 0 & 0 & 0 & 0 \\
 0 & 0 & 1 & 0 & 0 & 0 & 0 & 0 \\
 0 & 0 & -1 & 1 & 0 & 0 & 0 & 0 \\
 0 & 0 & 0 & 0 & 1 & 0 & 0 & 0 \\
 0 & 0 & 0 & 0 & -1 & 1 & 0 & 0 \\
 0 & 0 & 0 & 0 & 0 & 0 & 1 & 0 \\
 0 & 0 & 0 & 0 & 0 & 0 & -1 & 1 \\
\end{bmatrix}
\]
\[
M_{1,2}=\begin{bmatrix}
 1 & 0 & 0 & 0 & 0 & 0 & 0 & 0 \\
 0 & 1 & 0 & 0 & 0 & 0 & 0 & 0 \\
 -1 & 0 & 1 & 0 & 0 & 0 & 0 & 0 \\
 1 & 0 & 0 & 1 & 0 & 0 & 0 & 0 \\
 0 & 0 & 0 & 0 & 1 & 0 & 0 & 0 \\
 0 & 0 & 0 & 0 & 0 & 1 & 0 & 0 \\
 0 & 0 & 0 & 0 & 0 & 0 & 1 & 0 \\
 0 & 0 & 0 & 0 & 0 & 0 & 0 & 1 \\
\end{bmatrix}
\]
\end{example}