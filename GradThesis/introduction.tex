\section{Background and motivation}

Polylogarithms, denoted as $\Li_n(z)=\sum\limits_{k=1}^\infty\dfrac{x^k}{k^n}$, are a collection of multi-valued holomorphic functions on $\mathbb C-\{0,1\}$, extended through analytic continuation. They are named polylogarithms because $\Li_1(z)=-\log(1-z)$, thereby generalizing  the natural logarithm. One could also define their single-valued variants $\mathcal L_n(z)$ (see Definition~\ref{def: single valued polylogarithm}). Polylogarithms satisfy various functional equations, including the classical five-term relation
\begin{equation}\label{eq: five term relation}
\mathcal L_2(x)-\mathcal L_2(y)+\mathcal L_2\left(\frac{y}{x}\right)-\mathcal L_2\left(\frac{1-x^{-1}}{1-y^{-1}}\right)+\mathcal L_2\left(\frac{1-x}{1-y}\right)=0
\end{equation}
Polylogarithms play a significant role 
in several areas of mathematics and physics. They appear in formulas for scattering amplitudes in quantum field theory~\cite{golden_clusterPolylogarithmsForScattering}. Moveover, $\mathcal L_2$ shows up in the formula for volumes of hyperbolic $3$-simplices~\cite{Zagier_TheDilogarithm}, and Cheeger-Chern-Simons class for $\SL(2,\mathbb C)$~\cite{zickert_dilogarithmCCCclass}.

Polylogarithms naturally give rise to a variation of mixed Hodge structures over $\mathbb P^1-\{0,1,\infty\}$~\cite{Deligne_InterpretationMotiviqueDeLaConjectureDeZagierReliantPolylogarithmesEtRegulateurs}. Beilinson and Deligne constructed a matrix (see~\ref{eq: Deligne's variation matrix}) composed of polylogarithms, which we shall refer to as the variation matrix. The variation matrix is the fundamental matrix to a linear differential equation defined on $\mathbb P^1-\{0,1,\infty\}$. They calculated its monodromy matrices, and showed that filtrations on its columns define a variation of mixed Hodge structures.
% In addition to this, they conjectured that $\mathcal L_n$ corresponds to the regulator $K_{2n-1}(F)\to\mathbb C/(2\pi i)^n\mathbb Q$. This has since been proven for $n=2,3$.

Later, Goncharov made connections between polylogarithms and the conjectural category of mixed Tate motives over a number field $F$~\cite{GoncharovMotivicGalois}. In particular, he constructed groups $\mathcal B_n(F)=\mathbb Z[\mathbb P^1_F]/R_n(F)$, where each generator $[a]_n$ can be viewed as representing $\mathcal L_n(a)$, and $R_n(F)$ is a subgroup of $\mathbb Z[\mathbb P_F^1]$ defined inductively, and may be thought of as generated by functional relations for $\mathcal L_n$. For example, $R_2(F)$ is widely believed to be generated by all five-term relations
\begin{equation}
[x]_2-[y]_2+\left[\frac{y}{x}\right]_2-\left[\frac{1-x^{-1}}{1-y^{-1}}\right]_2+\left[\frac{1-x}{1-y}\right]_2
\end{equation}
% given this assumption, $\mathcal B_2(F)$ will be the Bloch group. 
which corresponds to~\eqref{eq: five term relation}. These $\mathcal B$ groups fit into a chain complex $\Gamma(F,n)$, which reads: (See~\eqref{eq: differentials for Bloch complex} for the definitions of the differentials)
\begin{equation}\label{eq: Bloch complex}
\mathcal B_n(F)\xrightarrow{\delta_n}\mathcal B_{n-1}(F)\otimes F^\times\xrightarrow{\delta_{n-1}}\mathcal B_{n-2}(F)\otimes \textstyle\bigwedge^2F^\times\to\cdots\xrightarrow{\delta_2}\textstyle\bigwedge^nF^\times
\end{equation}
Goncharov conjectured that the $i$-th cohomology group of $\Gamma(F,n)$ is rationally isomorphic to the $i$-th motivic cohomology group $H^i_{\mathcal M}(F,\mathbb Q(n))$. Notice that this is only true rationally, and it fails to be true integrally even for $n=2$.

There is also a multivariate analog of polylogarithm, called multiple polylogarithms, introduced by Goncharov in~\cite{Goncharov_GaloisSymmetriesOfFundamentalGroupoidsAndNoncommutativeGeometry}. The multiple polylogarithm $\Li_{n_1,\cdots,n_d}(x_1,\cdots,x_d)$ is defined as the power series
\[
\sum_{0<k_1<\cdots<k_d}\frac{x_1^{k_1}\cdots x_d^{k_d}}{k_1^{n_1}\cdots k_d^{n_d}}
\]
for $|x_i|<1$. We refer to $d$ and $|\mathbf n|=n_1+\cdots+n_d$ as its \textit{depth} and \textit{weight}. Note that multiple polylogarithms in depth 1 are the classical polylogarithms. Extended by analytic continuation, they are multi-valued holomorphic functions on a subvariety $S_d(\mathbb C)$ of $\mathbb C^d$ by removing the locus of $\{x_i=0\}_i$ and
$\{x_jx_{j+1}\cdots x_k=1\}_{j\leq k}$ (see Definition~\ref{eq: Sd}). It is tempting to ask the following questions:

\begin{question}\label{Fundamental questions}\hfill
\begin{enumerate}[i.]
\item Do the multiple polylogarithms also define variations of mixed Hodge structures encoded by multiple polylogarithms? And how do we compute the monodromies of multiple polylogarithms?
\item Is there an analog of Goncharov's Bloch complex~\eqref{eq: Bloch complex} with generators representing multiple polylogarithms, which computes the rational motivic cohomology?
\end{enumerate}
\end{question}

The first question is partially addressed by Zhao~\cite{Zhao_MultipleZetaFunctionsMultiplePolylogarithmsAndTheirSpecialValues} in the special case of multiple logarithms, which are multiple polylogarithms with all indices equal to one. The complex in the second question is supposed to be quasi-isomorphic to the Chevalley-Eilenberg complex of the motivic Lie coalgebra associated to the motivic Galois group~\cite{GoncharovMotivicGalois}. Goncharov and Rudenko constructed such complexes up to weight 4 using motivic correlators (see~\cite{GoncharovRudenko}). In contrast, our approach avoids motivic correlators, relying simply on symbolic generators of multiple polylogarithms, making the complexes applicable to arbitrary weights (see~\cite{ZDHZ_TheLieCoalgebraOfMultiplePolylogarithms}). Later, Rudenko and Matveiakin extended Goncharov and Rukendo's work to handle all weights as well (see~\cite{Rudenko_ClusterPolylogarithmsIQuadrangularPolylogarithms}). 

In~\cite{Zickert_HolomorphicPolylogarithmsAndBlochComplexes}, Zickert considered lifted polylogarithms $\widehat{\mathcal L}_n$ as functions from $\widehat{\mathbb C}$ to $\mathbb C/\frac{(2\pi i)^n}{(n-1)!}\mathbb Z$, where $\widehat{\mathbb C}$ is the universal abelian cover of $\mathbb P^1-\{0,1,\infty\}$, modeled by
\begin{equation}\label{eq: CHat}
\widehat{\mathbb C}=\left\{(u,v)\in\mathbb C^2\middle|e^u+e^v=1\right\}
\end{equation}
here $u,v$ should be thought of as $\log(x)$ and $\log(1-x)$, respectively. The term ``lifted'' refers to constructions over $\widehat{\mathbb C}$. Goncharov previously argued that $\mathcal L_n$, when viewed as a map $\mathcal B_n(\mathbb C)\to\mathbb R$ by taking $[a]_n$ to $\mathcal L_n(a)$, should make the following diagram commutes
\begin{center}
\begin{tikzcd}\label{cd: cycle map and L}
{H^1_{\mathcal M}(\mathbb C,\mathbb Z(n))} \arrow[d, "\cong_{\mathbb Q}"] \arrow[r, "b_n"]                             & \mathbb C/(2\pi i)^n\mathbb Z \arrow[d, "\operatorname{ReIm}"] \\
\ker(\mathcal B_n(\mathbb C)\xrightarrow{\delta_n}\mathcal B_{n-1}\otimes\mathbb C^\times) \arrow[r, "\mathcal L_n"] & \mathbb R
\end{tikzcd}
\end{center}
here $b_n$ is the cycle map~\eqref{eq: cycle map}, the left is the conjectural rational isomorphism, and $\operatorname{ReIm}$ is $\operatorname{Re}$ when $n$ is odd and $\operatorname{Im}$ when $n$ is even. Zickert then constructed the lifted Bloch complex $\widehat\Gamma(\mathbb C,n)$ by defining $\widehat{\mathcal B}_n(\widehat{\mathbb C})$ groups, whose generators are thought to be $\widehat{\mathcal L}_n$, and conjectured its $i$-th cohomology group is isomorphic to the $i$-th integral motivic cohomology group $H^i_{\mathcal M}(\mathbb C,\mathbb Z(n))$, rather than just $H^i_{\mathcal M}(\mathbb C,\mathbb Q(n))$. He believe that $\widehat{\mathcal L}_n$ would correspond to the cycle map $b_n$ in~\eqref{cd: cycle map and L}.

Moreover, Zickert proved (unpublished) that $\widehat{\mathcal L}_n$ introduces a lifted variation of mixed Hodge structures over $\widehat{\mathbb C}$. And he also defined the lifted Bloch complex $\widehat\Gamma(F,n)$ for a field $F$ satisfying certain necessary conditions (see~\ref{eq: lifted Bloch complex}), and he conjectured its $i$-th cohomology group is isomorphic to the $i$-th integral motivic cohomology group $H^i_{\mathcal M}(F,\mathbb Z(n))$. This leads to several natural questions:

\begin{question}\label{Fundamental questions LHat}\hfill
\begin{enumerate}[i.]
\item Can we define lifted multiple polylogarithms $\widehat{\mathcal L}_{n_1,\cdots,n_d}$ on $\widehat S_d(\mathbb C)$, the universal abelian cover of the domain $S_d(\mathbb C)$ of $\Li_{n_1,\cdots,n_d}$?
\item Do these lifted multiple polylogarithms yield a lifted variation of mixed Hodge structures?
\item Is it possible to construct a lifted motivic complex with generators representing lifted multiple polylogarithms that computes integral (rather than rational) motivic cohomology?
\end{enumerate}
\end{question}

In the thesis, we will try to answer both Question~\ref{Fundamental questions} and Question~\ref{Fundamental questions LHat}. The treatment of Question~\ref{Fundamental questions} ii. constitutes the joint work with Greenberg, Kaufman and Zickert~\cite{ZDHZ_TheLieCoalgebraOfMultiplePolylogarithms}, and is discussed in Chapter 3. Meanwhile, Question~\ref{Fundamental questions LHat} iii., Question~\ref{Fundamental questions LHat} i. and ii. are addressed in the joint work with Greenberg, Kaufman and Zickert~\cite{ZDHZ_HopfAlgebrasOfMultiplePolylogarithmsAndHolomorphicOneForms}, and are covered in Chapter 4. Monodromy calculations for multiple polylogarithms related to Question~\ref{Fundamental questions LHat} iii. is the independent work of the author, which is detailed in Chapter 5.

\section{Structure of the paper and main results}

In Chapter 2, we review fundamental concepts and well-known theorems that set the stage for the discussions in subsequent Chapters. We begin by examining key concepts such as iterated integrals, Hopf algebras, Lie coalgebras and variations of mixed Hodge structures. In particular, we discuss the regularization of singular iterated integrals, where shuffle algebras and Lyndon words play a crucial role. After establishing these foundations, we formally define multiple polylogarithms, explore their key properties, and introduce a coproduct structure on the Hopf algebra of iterated integrals. Additionally, we cover basic definitions and facts regarding connections on vector bundles, and variations of mixed Hodge structures, which prepare us for the discussions in Chapter 4. Lastly, we briefly touch on the connections between multiple polylogarithms and motivic cohomology, further motivating our research.

% In this chapter, we construct two complexes $\bigwedge^\bullet\mathbb L^{\Symb}$, $\bigwedge^\bullet\widehat{\mathbb L}^{\Symb}$, and define a map $w$ from $\bigwedge^\bullet\widehat{\mathbb L}^{\Symb}$ to de Rham complex $\Omega^\bullet$. These two complexes serve as precursors to the motivic complex and the lifted motivic complex that we sought after in Question~\ref{Fundamental questions} ii and Question~\ref{Fundamental questions LHat} iii. And $w$ associates generators with differential one-forms. Zickert conjectured that after sheafifying the complexes over a manifold, $w$ induces a morphism from its integral motivic cohomology to its singular cohomology.

Chapter 3 is joint work with Zachary Greenberg, Dani Kaufman and Christian K. Zickert~\cite{ZDHZ_TheLieCoalgebraOfMultiplePolylogarithms}. Firstly, we define a Hopf algebra $\mathbb H^{\Symb}$ generated by generators $[x_1,\cdots,x_d]_{n_1,\cdots,n_d}$ that represent $\Li_{n_1,\cdots,n_d}(x_1,\cdots,x_d)$. Unlike previous constructions by Goncharov and others, we do not impose shuffle or inversion relations on the generators. This explains the use of the superscript ``Symb'', indicating that the generators are symbolic, with no relations imposed. We also equip $\mathbb H^{\Symb}$ with a natural coproduct derived from Goncharov's coproduct on iterated integrals. Next, we associate differential one-forms to elements in $\mathbb H^{\Symb}$. These one-forms live in $\Omega^1_{\mathbb Q[\{u_i,v_{j,k}\}]/\mathbb Q}$, where $u_i$, $v_{j,k}$ represent $\log(x_i)$, $\log(1-x_j\cdots x_k)$, which are coordinates on $\widehat{S}_d(\mathbb C)$ (see~\eqref{eq: SdHat}). For instance, the one-form associated to $[x_1]_2$ is $\frac{1}{2}(u_1dv_{1,1}-v_{1,1}du_1)$. The one-forms offer a new approach to exploring functional relations in addition to another widely used tool, the symbols of multiple polylogarithms~\cite{CharltonDuhrGangl_SingleValuedPolylogs}. Additionally, we consider several useful Hopf algebras modeled on $\mathbb H^{\Symb}$, such as a sheaf of Hopf algebras $\widehat{\mathbb H}^{\Symb}_M$ over a complex manifold $M$, by sheafifying morphisms from open subsets of $M$ to $\widehat S_d(\mathbb C)$. These various Hopf algebras exhibit similar structural properties. To capture the essence of these structures, we introduce the notion of a contraction system and use it to justify all constructions simultaneously. Finally, by taking quotient of $\mathbb H^{\Symb}$ by products (see Theorem~\ref{thm: H modulo products is Lie coalgebra}), we obtain a Lie coalgebra $\mathbb L^{\Symb}$, whose Chevalley-Eilenberg complex $\bigwedge^\bullet\mathbb L^{\Symb}$ maps to the de Rham complex by sending an elements to its one-form, resulting in a commutative diagram.

\begin{theorem*}
Suppose $\Omega^\bullet_{\mathbb Q[\{u_i,v_{j,k}\}]/\mathbb Q}$ is the de Rham complex, and $w$ is the map that takes an element to its one-form, then the following diagram commutes:
\begin{center}
\begin{tikzcd}\label{cd: L complex => de Rham complex}
% \widehat{\mathbb L}^{\Symb} \arrow[d, "w"] \arrow[r, "\delta"] & \textstyle\bigwedge^2\widehat{\mathbb L}^{\Symb} \arrow[d, "w\wedge w"] \arrow[r, "1\wedge\delta-\delta\wedge1"] & \textstyle\bigwedge^3\widehat{\mathbb L}^{\Symb} \arrow[d, "w\wedge w\wedge w"] \arrow[r] & \cdots \\
% \Omega^1 \arrow[r, "d"]                       & \Omega^2 \arrow[r, "d"]                                                                             & \Omega^3 \arrow[r, "d"]                                                  & \cdots
\mathbb L^{\Symb} \arrow[d, "w"] \arrow[r, ""] & \textstyle\bigwedge^2\mathbb L^{\Symb} \arrow[d, "w\wedge w"] \arrow[r, ""] & \textstyle\bigwedge^3\mathbb L^{\Symb} \arrow[d, "w\wedge w\wedge w"] \arrow[r] & \cdots \\
\Omega^1 \arrow[r, "d"]                       & \Omega^2 \arrow[r, "d"]                                                                             & \Omega^3 \arrow[r, "d"]                                                  & \cdots
\end{tikzcd}
\end{center}
\end{theorem*}

A sheaf of graded Lie coalgebra, denoted $\widehat{\mathbb L}_M$, should exist by taking the quotient of $\widehat{\mathbb H}^{\Symb}_M$ by the products and functional relations on its stalks over $M$. Moreover, we conjecture the existence of a chain map $\bigwedge^\bullet\widehat{\mathbb L}_M\to\Omega^\bullet$, induced by the theorem above, which could establish a connection between motivic cohomology and singular cohomology.

\begin{conjecture}
The following diagram commutes
\begin{center}
\begin{tikzcd}
{H^i(M,(\bigwedge^*\widehat{\mathbb L}_M)_n)} \arrow[r]              & {H^i(M,\Omega^*)} \arrow[d, "\cong"] \\
{H^i_{\mathcal M}(M,\mathbb Z(n))} \arrow[u, dashed] \arrow[r] & {H^i(M,\mathbb C)}
\end{tikzcd}
\end{center}
Here subscript $n$ means degree $n$ part (see Definition~\ref{def: graded Lie coalgebra}), for example, if $n=4$, then $(\bigwedge^*\widehat{\mathbb L}_M)_4$ reads
\[
(\widehat{\mathbb L}_M)_4\to(\widehat{\mathbb L}_M)_3\otimes(\widehat{\mathbb L}_M)_1\oplus(\widehat{\mathbb L}_M)_2\wedge(\widehat{\mathbb L}_M)_2\to(\widehat{\mathbb L}_M)_2\otimes\textstyle\bigwedge^2(\widehat{\mathbb L}_M)_1\to\textstyle\bigwedge^4(\widehat{\mathbb L}_M)_1
\]
where $(\widehat{\mathbb L}_M)_k$ is the degree $k$ part of $\widehat{\mathbb L}_M$. And $H^i(M,(\bigwedge^*\widehat{\mathbb L}_M)_n)$ is the hypercohomology of $M$ with coefficients in the complex $(\bigwedge^*\widehat{\mathbb L}_M)_n$. The top arrow is induced by the chain map $\bigwedge^\bullet\widehat{\mathbb L}_M\to\Omega^\bullet$. The bottom arrow represents the realization functor from integral motivic cohomology to singular cohomology, while the right arrow is the isomorphism between de Rham cohomology and singular cohomology~\cite{Grothendieck_OnTheDeRhamCohomologyOfAlgebraicVarieties}. The nature of the left arrow is currently unclear and still needs to be determined.
\end{conjecture}

Chapter 4 is partially joint work with Zachary Greenberg, Dani Kaufman and Christian K. Zickert~\cite{ZDHZ_HopfAlgebrasOfMultiplePolylogarithmsAndHolomorphicOneForms}. In this chapter, we aim to address Question~\ref{Fundamental questions LHat} i, ii. Firstly, we introduce a generalized concept of variation matrices, which are naturally decomposed into submatrices of different weights. For example, $V^{\mathbb H}$ is the variation matrix entirely composed of generators in $\mathbb H$. We also define a realization map $\Re$, which ``realizes'' these generators as actual multiple polylogarithms, for instance, $\Re([x_1,\cdots,x_d]_{n_1,\cdots,n_d})=\Li_{n_1,\cdots,n_d}(x_1,\cdots,x_d)$. When $\Re$ is applied to $V^{\mathbb H}$, we obtain variation matrices previously considered by Deligne and Zhao, which we denote as $V$. Next, we show that the columns of $V$ generate the sections of a trivial vector bundle over $S_d(\mathbb C)$ with the flat connection $\nabla=d-\omega$, where $\omega$ is simply the differential of $V$. Additionally, we explain how filtrations on columns of $V$ define a variation of mixed Hodge structures over $S_d(\mathbb C)$. Similarly, we define lifted multiple polylogarithms $\widehat{\mathcal L}_{n_1,\cdots,n_d}$, and construct a lifted variation matrix $\widehat V$ consisting of lifted multiple polylogarithms. Columns of $\widehat V$ generate the sections of a trivial vector bundle over $\widehat S_d(\mathbb C)$ with a lifted flat connection $\widehat\nabla=d-\widehat\omega$, where $\widehat\omega$ is composed of one-forms associated to multiple polylogarithms. And filtrations on columns of $\widehat V$ define a lifted variation of mixed Hodge structures over $\widehat S_d(\mathbb C)$. Finally, we briefly discuss how to use variation matrices to re-interpret a result by Zickert~\cite{Zickert_HolomorphicPolylogarithmsAndBlochComplexes}, and explore their connection to the recursion formulas in Greenberg's Thesis~\cite{ZackThesis}.

Chapter 5 is the independent work of the author. This chapter contains the proof of the rationality of monodromy of multiple polylogarithms, and explicit formulas and algorithms for calculations of monodromy matrices. Zhao cited a general result in~\cite{DeligneGoncharov_GroupesFondamentauxMotiviquesDeTateMixte} for the existence of rational monodromies, but only gave a closed formulas for monodromies of multiple logarithms $\Li_{1,\cdots,1}$. We propose a new approach to this problem. By carefully deforming the integration path of multiple polylogarithms, we break the problem into canonical subcases, and derive an explicit formula and an algorithm for computing monodromy matrices for an arbitrary multiple polylogarithm. Our construction guarantees that the resulting monodromies are always rational. Furthermore, the algorithm has been implemented in Mathematica by the author.