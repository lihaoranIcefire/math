\section{Two new Hopf algebras of multiple polylogarithms}

In this section, we give the definitions of two new purely symbolic Hopf algebras $\overline{\mathbb H}^{\Symb}$ and $\mathbb H^{\Symb}$ with generators representing multiple polylogarithms. $\overline{\mathbb H}^{\Symb}$ is the smallest Hopf algebra that allows~\eqref{eq:GoncharovCoproductFormula}, a purely symbolic representation of the coproduct for multiple polylogarithms. $\mathbb H^{\Symb}$ is the smallest Hopf algebra that admits a differential and has no relations among its generators, this is explained in Lemma~\ref{lem: df=0 => f=const}. This is worth noting that this differs from Goncharov and others various definitions of Hopf algebra of multiple polylogarithms, which typically assumes some polylogarithmic relations among its generators.

We will use the following shorthand throughout this chapter
\[
x_{i\to j}:=\prod_{i\leq k<j}x_k,\quad x^{-1}_{i\to j}:=\prod_{i\leq k<j}x_k^{-1},\quad 0^n:=\overbrace{0,\cdots,0}^k
\]

\subsection{Definition of $\overline{\mathbb H}^{\Symb}$}

\begin{definition}
$\overline{\mathbb H}^{\Symb}$ is the free $\mathbb Q$-algebra generated by two types of symbols:
\begin{align*}
\text{regular symbols:}&\quad [x_{i}]_0, [x_{i_1\to i_2}, x_{i_2\to i_3},\cdots,x_{i_d\to i_{d+1}}]_{n_1,n_2,\cdots,n_d}\\
\text{inverted symbols:}&\quad [x_{i_d\to i_{d+1}}^{-1},x_{i_{d-1}\to i_{d}}^{-1},\cdots,x_{i_1\to i_2}^{-1}]_{n_d,n_{d-1},\cdots,n_1}
\end{align*}
where, $i,i_k, n_k\in\mathbb Z_{\geq1}$ and $1\leq i_1<i_2<\cdots<i_{d+1}$.
\end{definition}

One should think of $[x_{i}]_0, [x_{i_1\to i_2},\cdots,x_{i_d\to i_{d+1}}]_{n_1,\cdots,n_d}, [x_{i_d\to i_{d+1}}^{-1},\cdots,x_{i_1\to i_2}^{-1}]_{n_d,\cdots,n_1}$ as $\log(x_i)$, $\Li_{n_1,\cdots,n_d}(x_{i_1\to i_2},\cdots,x_{i_d\to i_{d+1}})$ and $\Li_{n_d,\cdots,n_1}(x_{i_d\to i_{d+1}}^{-1},\cdots,x_{i_1\to i_2}^{-1})$ respectively.

If we associate a symbol with weight $n_1+\cdots+n_d$ (1 for $[x_i]_0$), then $\overline{\mathbb H}^{\Symb}$ is graded with
\begin{itemize}
\item $\overline{\mathbb H}^{\Symb}_0=\mathbb Q$
\item $\overline{\mathbb H}^{\Symb}_1=\mathbb Q\{[x_i]_0, [x_{i\to j}]_1\}$
\item $\overline{\mathbb H}^{\Symb}_{\geq2}=\mathbb Q\{\text{symbols with weights }\geq2\}$
\end{itemize}

For convenience, we also define
\[
[x_{i\to j}]_0:=\sum_{i\leq k<j}[x_k]_0,\quad [x_{i\to j}^{-1}]_0:=-\sum_{i\leq k<j}[x_k]_0
\]

Inspired by Goncharov~\cite{Goncharov_GaloisSymmetriesOfFundamentalGroupoidsAndNoncommutativeGeometry}. One could define a coproduct on $\overline{\mathbb H}^{\Symb}$. First denote the formal power series
\begin{equation}
\begin{aligned}
[\mathbf y|\mathbf t]=[y_1,\dots,y_d|t_1,\dots,t_d]&=\sum_{n_i\geq 1}[y_1,\dots,y_d]_{n_1,\dots,n_d}t_1^{n_1-1}\dots t_d^{n_d-1}
\end{aligned}
\end{equation}
Where $[y_1,\dots,y_d]_{n_1,\dots,n_d}$ are regular symbols with $y_k=x_{i_k\to i_{k+1}}$ for convenience. Then the coproduct formula is given by $\Delta([x_{i}]_0)=[x_{i}]_0\otimes 1+1\otimes [x_{i}]_0$, $\Delta([x_{i\to j}^{\pm}]_1)=[x_{i\to j}^{\pm}]_1\otimes 1+1\otimes [x_{i\to j}^{\pm}]_1$ and
\begin{equation}\label{eq:GoncharovCoproductFormula}
\begin{aligned}
\Delta([\mathbf y|\mathbf t])=\sum&[y_{i_1\to i_2},\dots,y_{i_k\to i_{k+1}}|t_{j_1},\dots ,t_{j_k}]\bigotimes\prod_{\alpha=0}^k(-1)^{j_\alpha-i_\alpha}\exp([y_{i_\alpha\to i_{\alpha+1}}]_0t_{j_\alpha})\\
&[y_{j_\alpha-1}^{-1},y_{j_\alpha-2}^{-1},\dots,y_{i_\alpha}^{-1}|t_{j_\alpha}-t_{j_\alpha-1},\dots, t_{j_\alpha}-t_{i_\alpha}]\\
&[y_{j_\alpha+1},y_{j_\alpha+2},\dots,y_{i_{\alpha+1}-1}|t_{j_\alpha+1}-t_{j_\alpha},\dots,t_{i_{\alpha+1}-1}-t_{j_\alpha}].\\
\end{aligned}
\end{equation}
The sum is over all instances of $1=i_0\leq j_0<i_1\leq j_1<\dots <i_k\leq j_k<i_{k+1}=d+1$, and by definition we have $y_{i\to j}=\prod_{r=i}^{j-1}y_r$.

\begin{theorem}\label{thm: Hbar is a Hopf algebra}
$\overline{\mathbb H}^{\Symb}$ forms a graded Hopf algebra.
\end{theorem}

\begin{proof}
We justify the coassociativity of the coproduct in Corollary~\ref{cor: Hbar is coassociative}. The counit and antipode is defined by Lemma~\ref{lem: kernel of counit is H_{>0}} and~Theorem~\ref{thm: uniqueness of antipode}.
\end{proof}

\begin{example}
\begin{multline}
\Delta([x_1,x_2]_{1,1})=1\otimes[x_1,x_2]_{1,1}+[x_2]_1\otimes[x_1]_1-[x_1x_2]_1\otimes[x_1^{-1}]_1\\
+[x_1x_2]_1\otimes[x_2]_1+[x_1,x_2]_{1,1}\otimes1
\end{multline}
\begin{multline}
\Delta([x_1,x_2]_{2,1})=1\otimes [x_1,x_2]_{2,1}+[x_1,x_2]_{2,1}\otimes 1-[x_1,x_2]_{1,1}\otimes [x_2]_0\\
+[x_1,x_2]_{1,1}\otimes[x_1 x_2]_0+[x_2]_1\otimes[x_1]_2+[x_1x_2]_1\otimes [x_1^{-1}]_2\\
-[x_1x_2]_1\otimes[x_2]_2+[x_1x_2]_2\otimes [x_2]_1+[x_1x_2]_1\otimes\left([x_2]_1[x_1 x_2]_0\right)
\end{multline}
\end{example}

\subsection{Definition of $\mathbb H^{\Symb}$}

The following is inspired by Goncharov's inversion~\eqref{eq: Goncharov's inversion formula} for multiple polylogarithms. It is natural to consider a map $\overline{\INV}$ that converts inverted symbols into regular symbols. Let $\overline{\mathbb H}^{\Symb}[\pi i]$ denote $\overline{\mathbb H}^{\Symb}$ adjoined with a free generator $\pi i$.

\begin{definition}
 $\overline{\INV}:\overline{\mathbb H}^{\Symb}\to\overline{\mathbb H}^{\Symb}[\pi i]$ fixes regular symbols and acts on inverted symbols inductively by
\begin{equation}\label{eq: INVbar}
\begin{aligned}
&\overline{\INV}([x_d^{-1},\cdots,x_1^{-1}]_{n_d,\cdots,n_1})=-\sum_{r=0}^{d-1}\overline{\INV}[x_r^{-1},\cdots,x_1^{-1}]_{n_r,\cdots,n_1}[x_{r+1},\cdots,x_d]_{n_{r+1},\cdots,n_d}-\\
&\sum_{r=1}^d\sum_{m_1+\cdots+m_d=n_r}(-1)^{1+n_r+\cdots+n_d+m_{r+1}+\cdots+m_d}\prod_{\substack{1\leq i\leq d\\i\neq r}}\binom{n_i+m_i-1}{n_i-1}B_{m_r}\left(\frac{[x_1\cdots x_d]_0}{2\pi i}\right)\\
&\frac{(2\pi i)^{m_r}}{m_r!}\overline{\INV}[x_{r-1}^{-1},\cdots,x_1^{-1}]_{n_{r-1}+m_{r-1},\cdots,n_1+m_1}[x_{r+1},\cdots,x_d]_{n_{r+1}+m_{r+1},\cdots,n_d+m_d}
\end{aligned}
\end{equation}
\end{definition}

In depth 1 and 2, \eqref{eq: Goncharov inversion depth 1} and~\eqref{eq: Goncharov inversion depth 1} correspond to

\begin{equation}\label{eq: Depth1Inv}
[x]_n+(-1)^{n}[x^{-1}]_n=-\frac{(2\pi i)^n}{n!}B_n\left(\frac{[x]_0}{2\pi i}\right),
\end{equation}
\begin{equation}\label{eq: Depth2Inv}
\begin{aligned}
[x_1,x_2]_{n_1,n_2}+(-1)^{n_1+n_2}[x_2^{-1},x_1^{-1}]_{n_2,n_1}+(-1)^{n_1}[x_1^{-1}]_{n_1}[x_2]_{n_2}\\
+\sum_{p+q=n_1}\frac{(2\pi i)^p}{p!}(-1)^q\binom{q+n_2-1}{n_2-1}B_p\left(\frac{[x_1x_2]_0}{2\pi i}\right)[x_2]_{q+n_2}\\
+\sum_{p+q=n_2}\frac{(2\pi i)^p}{p!}(-1)^{n_1}\binom{n_1+q-1}{n_1-1}B_p\left(\frac{[x_1x_2]_0}{2\pi i}\right)[x_1^{-1}]_{n_1+q}=0,
\end{aligned}
\end{equation}

It is difficult to work with powers of $\pi i$, so we also define $\INV$ as $\overline{\INV}$ modulo $\pi i$.

\begin{definition}
The map $\INV:\overline{\mathbb H}^{\Symb}\to\overline{\mathbb H}^{\Symb}$ fixes regular symbols and acts on inverted symbols inductively by
\begin{equation}\label{eq: INV map}
\begin{aligned}
&\INV([y_d^{-1},\cdots,y_1^{-1}|-t_d,\cdots,-t_1])\\
&=\sum_{j=0}^{d-1}(-1)^{d-1+j}\INV([y_j^{-1},\cdots,y_1^{-1}|-t_j,\cdots,-t_1])[y_{j+1},\cdots,y_d|t_{j+1},\cdots,t_d]\\
&+\sum_{j=1}^d\frac{(-1)^{d-1+j}}{t_j}\INV([y_{j-1}^{-1},\cdots,y_1^{-1}|-t_{j-1},\cdots,-t_1])[y_{j+1},\cdots,y_d|t_{j+1},\cdots,t_d]\\
&+\sum_{j=1}^d\bigg(\frac{(-1)^{d+j}}{t_j}\INV([y_{j-1}^{-1},\cdots,y_1^{-1}|t_j-t_{j-1},\cdots,t_j-t_1])\\
&\qquad\qquad\exp([y_{1\to d+1}]_0t_j)[y_{j+1},\cdots,y_d|t_{j+1}-t_{j},\cdots,t_d-t_j]\bigg)
\end{aligned}
\end{equation}
with the induction starting with $\INV([y^{-1}|-t])=[y|t]+\dfrac{\exp([y]_0t)-1}{t}$.
\end{definition}

\begin{example}
Corresponding to~\eqref{eq: Depth1Inv} and~\eqref{eq: Depth2Inv}, we have
\[
\INV([x_1^{-1}]_n)=(-1)^{n-1}[x_1]_n+\dfrac{(-1)^{n-1}}{n!}[x_1]^n,
\]
\begin{multline}
\INV([x_2^{-1},x_1^{-1}]_{1,1})=-[x_1,x_2]_{1,1}+[x_1]_1[x_2]_1-[x_1]_2+[x_2]_2+[x_2]_1[x_1]_0\\
+[x_1]_1[x_1 x_2]_0-[x_2]_1[x_1 x_2]_0-\frac{1}{2}[x_1]_0^2+[x_1 x_2]_0[x_1]_0
\end{multline}
\end{example}

\begin{definition}
$\mathbb H^{\Symb}$ is the subalgebra of $\overline{\mathbb H}^{\Symb}$ generated solely by regular symbols $[x_{i}]_0$, $[x_{i_1\to i_2},\cdots,x_{i_d\to i_{d+1}}]_{n_1,\cdots,n_d}$.
\end{definition}

\begin{theorem}\label{thm: H is a Hopf algebra}
$\mathbb H^{\Symb}$ forms a graded Hopf algebra, with coproduct $\Delta_{\mathbb H}:\mathbb H^{\Symb}\to\mathbb H^{\Symb}\otimes\mathbb H^{\Symb}$ as the composition of the coproduct $\Delta_{\overline{\mathbb H}}$ of $\overline{\mathbb H}^{\Symb}$ and $\INV$, to be precise, $\Delta_{\mathbb H}=(1\otimes\INV)\circ\Delta_{\overline{\mathbb H}}$
\end{theorem}

\begin{proof}
This is precisely Theorem~\ref{thm: coproduct commutes with INV}, combined with Corollary~\ref{cor: Hbar is coassociative}.
\end{proof}

\begin{example}
\begin{multline}
\Delta([x_1,x_2]_{1,1})=1\otimes[x_1,x_2]_{1,1}+[x_2]_1\otimes[x_1]_1-[x_1x_2]_1\otimes([x_1]_1+[x_1]_0)\\
+[x_1x_2]_1\otimes[x_2]_1+[x_1,x_2]_{1,1}\otimes1
\end{multline}
\begin{multline}
\Delta([x_1,x_2]_{2,1})=1\otimes [x_1,x_2]_{2,1}+[x_1,x_2]_{2,1}\otimes 1-[x_1,x_2]_{1,1}\otimes [x_2]_0\\
+[x_1,x_2]_{1,1}\otimes[x_1 x_2]_0+[x_2]_1\otimes[x_1]_2+[x_1x_2]_1\otimes\left(-[x_1]_2-\frac{1}{2}[x_1]_0^2\right)\\
-[x_1x_2]_1\otimes[x_2]_2+[x_1x_2]_2\otimes [x_2]_1+[x_1x_2]_1\otimes\left([x_2]_1[x_1 x_2]_0\right)
\end{multline}
\end{example}

\subsection{Orderings on $\mathbb H^{\Symb}$}\label{sec: multiple polylog ordering}

To order the terms in the coproduct formula~\ref{eq:GoncharovCoproductFormula} and in turn the variation matrix in chapter 4, we define a total ordering on the generators of $\mathbb H^{\Symb}$. Consider a total ordering on $\mathbb Z_{\geq0}^\infty=\bigcup_{\ell}\mathbb Z_{\geq0}^\ell$ where $\mathbf k\prec\mathbf l$ if
\begin{itemize}
\item $\|\mathbf k\|<\|\mathbf l\|$
\item or if $\|\mathbf k\|=\|\mathbf l\|$ and $\dim(\mathbf k)<\dim(\mathbf l)$
\item or if $\|\mathbf k\|=\|\mathbf l\|$ and $\dim(\mathbf k)=\dim(\mathbf l)$ and the rightmost nonzero entry of $\mathbf l-\mathbf k$ is negative.
\end{itemize}
Here $\|\mathbf k\|=\sum k_i$ is the weight for any $\mathbf k=(k_1,\cdots,k_\ell)\in\mathbb Z_{\geq0}^\infty$.

\begin{definition}\label{def: order on polylog}
We can impose a total ordering on the set of regular symbols by identifying it with $\mathbb Z^\infty_{\geq0}$ via
\[
[x_{i_1\to i_2},\cdots,x_{i_d\to i_{d+1}}]_{n_1,\cdots,n_d}\leftrightsquigarrow(0^{i_1-1},n_1,0^{i_2-i_1-1},n_2,\cdots,0^{i_d-i_{d-1}-1},n_d,0^{i_{d+1}-i_d-1})
\]
Here $0^i$ means a tuple of $i$ zeros.
\end{definition}

\begin{example}
\[
0\prec(0,1)\prec(1,0)\prec(0,2)\prec(1,1)\prec(2,0)\prec(0,3)\prec(1,2)
\]
Corresponds to
\[
1\prec[x_2]_1\prec[x_1x_2]_1\prec[x_2]_2\prec[x_1,x_2]_{1,1}\prec[x_1x_2]_1\prec[x_2]_3\prec[x_1,x_2]_{1,2}
\]
\end{example}

% \begin{lemma}\label{lemma: greater multiple polylog has greater derivatives}
% According to this total ordering, greater multiple polylogarithm has greater derivatives.
% \end{lemma}

% \begin{proof}
% Suppose some multiply polylogarithm corresponds to $(0,\dots,0,k,\dots)$, $k\geq1$, it is not hard to see that the greatest term in its differential would correspond to $(0,\dots,0,k-1,\dots)$.
% \end{proof}

% Now we may define a free graded Hopf algebra compatible with a contraction system.

% \begin{definition}
% Suppose $X^{(k)}=\bigsqcup_{n\geq0} X^{(k)}_n$ with $k$, $n$ are refer to as the depth and weight respectively, and $\left\{X^{(k)}\right\}_{k=1}^\infty$ is a contraction system, such that the contraction maps preserves the weight. Denote $X_n=\bigsqcup_{k\geq0}X^{(k)}_n$, $X=\bigoplus_{n\geq0}X_n$ and suppose $H=\mathbb Q\left[X\right]$ has a free graded Hopf algebra structure. In addition, we say $H$ is free contracted Hopf algebra with total ordering $\prec$ if is is a total ordering on $X$ satisfying that
% \begin{itemize}
% \item if $|a|<|b|$ where $|\cdot|$ means weight,
% \item or if $|a|=|b|$ and $\dep(a)<\dep(b)$ where $\dep$ means depth.
% \end{itemize}
% then $a\prec b$.
% \end{definition}

\subsection{K\"ahler differentials of $\mathbb H^{\Symb}$}

Just like normal multiple polylogarithms, we can define differentials of elements of $\mathbb H^{\Symb}$.

\begin{definition}
We define linear maps $\partial_i:\mathbb H^{\Symb}\to\Omega_{\mathbb H^{\Symb}/\mathbb Q}$ as in Proposition~\ref{prop: derivatives for multiple polylog}. Thinking of $d[x_i]_0$ as $d\log(x_i)=\frac{dx_i}{x_i}$ and $d[x_i]_1$ as $d\Li_1(x_i)=\frac{dx_i}{1-x_i}$. We have
\begin{equation}
\partial_i[x_1,\cdots,x_d]_{n_1,\cdots,n_d}=[x_1,\cdots,x_d]_{n_1,\cdots,n_i-1,\cdots,n_d}d[x_i]_0,\quad n_i\geq2
\end{equation}
\begin{equation}
\partial_d[x_1,\cdots,x_d]_{n_1,\cdots,1}=-[x_1,\cdots,x_{d-1}x_d]_{n_1,\cdots,n_{d-1}}d[x_d]_1
\end{equation}
\begin{multline}
\partial_1[x_1,\cdots,x_n]_{1,\cdots,n_d}=([x_1x_2,\cdots,x_d]_{n_2,\cdots,n_d}-[x_2,\cdots,x_d]_{n_2,\cdots,n_d})d[x_1]_1\\
-[x_1x_2,\cdots,x_d]_{n_2,\cdots,n_d}d[x_1]_0
\end{multline}
\begin{multline}
\partial_i[x_1,\cdots,x_d]_{n_1,\cdots,1,\cdots,n_d}=-[x_1,\cdots,x_ix_{i+1},\cdots,x_d]_{n_1,\cdots,\widehat{1},\cdots,n_d}d[x_i]_0\\+([x_1,\cdots,x_ix_{i+1},\cdots,x_d]_{n_1,\cdots,\widehat{1},\cdots,n_d}-[x_1,\cdots,x_{i-1}x_i,\cdots,x_d]_{n_1,\cdots,\widehat{1},\cdots,n_d})d[x_i]_1
\end{multline}
The differential $d:\mathbb H^{\Symb}\to\Omega_{\mathbb H^{\Symb}/\mathbb Q}$ is defined as the sum $\sum_i\partial_i$.
\end{definition}

The following result is due to the author (not included in~\cite{ZDHZ_HopfAlgebrasOfMultiplePolylogarithmsAndHolomorphicOneForms}). It is useful in that it shows that there are no formal relations among generators. It can be thought of as an analog of the classical theorem which says polynomials with zero differential must be constants.

\begin{lemma}\label{lem: df=0 => f=const}
Consider the differential $d:\mathbb H^{\Symb}[\pi i]\to\Omega_{\mathbb H^{\Symb}[\pi i]/\mathbb Q[\pi i]}$, if $df=0$, then $f\in\mathbb Q[\pi i]$ must be a constant.
\end{lemma}

\begin{proof}
The proof can be divided into several steps.
\begin{enumerate}
\item If we write $f=f_0+(\pi i)f_1+\cdots$, where $f_i\in\mathbb H^{\Symb}$, then $df=0\iff df_i=0,\forall i$. So we may assume $f\in\mathbb H^{\Symb}$.
\item Let $\mathbb H^{\Symb}_{|\mathbf n|\geq2}\subset \mathbb H^{\Symb}$ be the subalgebra generated by regular symbols of weight no less than 2, and denote for short $[\mathbf x]_0^{\mathbf p}=[x_1]_0^{p_1}\cdots[x_d]_0^{p_d}$, $[\mathbf x]_1^{\mathbf q}=\displaystyle\prod_{1\leq j\leq k\leq d}[x_{j\to k}]_1^{q_{jk}}$. Write $f$ as
\[
f=\sum_{p_1,\cdots,p_d,q_1,\cdots,q_d\geq0}f_{\mathbf p,\mathbf q}[\mathbf x]_0^{\mathbf p}[\mathbf x]_1^{\mathbf q},\qquad f_{\mathbf p,\mathbf q}\in \mathbb H^{\Symb}_{|\mathbf n|\geq2}
\]
We will show that $f\in\mathbb H^{\Symb}_{|\mathbf n|\geq2}$. Pick a term with maximal $|\mathbf p|+|\mathbf q|\neq0$. We may assume either $p_i\neq0$ for some $i$ or $q_{jk}\neq0$ for some $j<k$. If we denote $\mathbf p'=(p_1,\cdots,p_i-1,\cdots,p_d)$, $\mathbf q'=(q_{12},\cdots,q_{jk}-1,\cdots,q_{d(d+1)})$, then we have either
\begin{align*}
0=df=df_{\mathbf p,\mathbf q}[\mathbf x]_0^{\mathbf p}[\mathbf x]_1^{\mathbf q}+(p_if_{\mathbf p,\mathbf q}d[x_i]_0+df_{\mathbf p',\mathbf q})[\mathbf x]_0^{\mathbf p'}[\mathbf x]_1^{\mathbf q}+\cdots
\end{align*}
or
\begin{align*}
0=df=df_{\mathbf p,\mathbf q}[\mathbf x]_0^{\mathbf p}[\mathbf x]_1^{\mathbf q}+(q_{jk}f_{\mathbf p,\mathbf q}d[x_{j\to k}]_1+df_{\mathbf p,\mathbf q'})[\mathbf x]_0^{\mathbf p}[\mathbf x]_1^{\mathbf q'}+\cdots
\end{align*}
This implies $df_{\mathbf p,\mathbf q}=0$, so $f_{\mathbf p,\mathbf q}$ is a nonzero constant by induction on weight. But then $p_if_{\mathbf p,\mathbf q}d[x_i]_0+df_{\mathbf p',\mathbf q}$ and $q_{jk}f_{\mathbf p,\mathbf q}d[x_{j\to k}]_1+df_{\mathbf p,\mathbf q'}$ cannot be zero.
\item Recall the ordering $\prec$ in~\ref{sec: multiple polylog ordering}. If we denote the greatest variable of $f$ by $L$, then we can write $f=L^mf_m+L^{m-1}f_{m-1}+\cdots+f_0$, $f_m\neq0,m\geq1$. Hence we get
\[
0=df=L^{m}df_{m}+L^{m-1}(mf_{m}dL+df_{m-1})+\cdots+df_0
\]
Again, by induction on the weight, $df_m=0$ implies that $f_m$ is a nonzero constant. Let us show that $mf_mdL+df_{m-1}$ cannot be zero. For this, we only need to show that $dL$ is not equal to $dg$ for any $g\in\mathbb H^{\Symb}_{|\mathbf n|\geq2}$ with variables strictly less than $L$. Now suppose $L$ corresponds to
\[
(\cdots,\overset{\substack{i_{d-1}\text{-th}\\\downarrow}}{n_{d-1}},0,\cdots,0,\overset{\substack{i_d\text{-th}\\\downarrow}}{n_d},0,\cdots,0)
\]
and $L'$ corresponds to
\[
(\cdots,n_{d-1},0,\cdots,0,n_d-1,0,\cdots,0)
\]
Then $dL$ contains term $L'd[x_{i_{d-1}\to i_{d+1}}]_1$ or $L'd[x_{i_{d-1}\to i_{d+1}}]_0$ which cannot be a term in $dg$.
\end{enumerate}
\end{proof}

\begin{remark}
Multiple polylogarithms satisfy stuffle relations~\cite{Zhao_MultipleZetaFunctionsMultiplePolylogarithmsAndTheirSpecialValues} such as
\begin{equation}\label{eq: shuffle relation for Li_{1,1}}
\Li_{1,1}(x_1,x_2)+\Li_{1,1}(x_2,x_1)+\Li_2(x_1x_2)-\Li_1(x_1)\Li_1(x_2)=0
\end{equation}
so one should have
\[
d([x_1,x_2]_{1,1}+[x_2,x_1]_{1,1}+[x_1x_2]_{2}-[x_1]_1[x_2]_1)=0
\]
At first glance, this might seem contradictory, however, this won't be an issue since symbols like $[x_2,x_1]_{1,1}$ with reversed order in $x$'s do not exist in $\overline{\mathbb H}$. This Lemma allows us to regard $\mathbb H^{\Symb}$ as a minimally generated Hopf algebra with symbolic multiple polylogarithmic generators, without assuming any relations.
\end{remark}

The differential of the generating series $[x_1,\dots,x_n|t_1,\dots,t_n]$ is straightforward to compute, and is given by:
\begin{equation}\label{eq: differential of the generating series}
\begin{aligned}
d[x_1,\dots,x_n|&t_1,\dots,t_n]=[x_1,\dots,x_n|t_1,\dots,t_n]\left(\sum_{k=1}^nd [x_k]_0t_k\right)\\
&+[x_2,\dots,x_n|t_2,\dots,t_n]d[x_1]_1\\&+\sum_{k=2}^n[x_1,\dots,x_{k-1}x_k,\dots,x_n|t_1,\dots,t_{k-1},t_{k+1},\dots,t_n)d[x_k]_1\\
&-\sum_{k=1}^{n-1}[x_1,\dots,x_kx_{k+1},\dots,x_n|t_1,\dots,t_{k-1},t_{k+1},\dots,t_n]\left(d[x_k]_1+d[x_k]_0\right)
\end{aligned} 
\end{equation}


\subsection{Morphism from iterated integrals to $\mathbb H^{\Symb}$}

In this section, we give a proof of Theorem~\ref{thm: Hbar is a Hopf algebra} and Theorem~\ref{thm: H is a Hopf algebra} using Goncharov's coproduct~\ref{eq:GoncharovCoproductFormula} on plane trees. Note that this is different than the proof in~\cite{Goncharov_GaloisSymmetriesOfFundamentalGroupoidsAndNoncommutativeGeometry} because our proof doesn't assume shuffle relations among the generators of $\overline{\mathbb H}^{\Symb}$. We achieve this by constructing a Hopf algebra morphism $\Phi$ from a certain Hopf subalgebra of $\mathcal I(S)$ to $\overline{\mathbb H}^{\Symb}$.

% In~\cite{Goncharov_GaloisSymmetriesOfFundamentalGroupoidsAndNoncommutativeGeometry}, Goncharov defined the generating series of iterated integrals in $\mathcal I(S)$ given that $S$ contains a distinct element $0$.
% \begin{multline}
% I(a_0;a_1,\cdots,a_d;a_{d+1}|t_0,\cdots,t_d)=\\
% \sum_{n_0,\cdots,n_d\geq0}I(a_0;0^{n_0},a_1,0^{n_1},\cdots,a_d,0^{n_d};a_{d+1})t_0^{n_0}\cdots t^{n_d}
% \end{multline}
% We define a map $\Gamma:\mathcal I(S)\to \mathcal I(S)$ that un-shuffles the iterated integrals
% \begin{multline}
% I(a_0;a_1,\cdots,a_d;a_{d+1}|t_0,\cdots,t_d)\mapsto\\
% \sum_{0\leq p\leq d} I(a_0;a_1,\cdots,a_p;0|t_0,\cdots,t_d)I(0;a_{p+1},\cdots,a_d;a_{d+1}|t_0,\cdots,t_d)
% \end{multline}

% \begin{example}
% \begin{multline}
% \Gamma(I(a_0;a_1,a_2;a_3))=I(0;a_1,a_2;a_3)+I(a_0;a_1;0)I(0;a_2;a_3)+I(a_0;a_1,a_2;0)
% \end{multline}
% \end{example}

Out of all iterated integrals, we are particularly interested in those that can be turned in to standard multiple polylogarithms. This is formalized as follows.

\begin{definition}\label{def: ISymb}
Let $S=\{0,1\}\cup\{x_{i\to j}^{-1}\}_{i<j}$. Define $\mathbb I^{\Symb}$ is defined to be the Hopf subalgebra of $\mathcal I(S)$ generated by elements of the forms
\begin{align*}
&I(0;0^{n_0-1},x_{i_1\to m}^{-1},0^{n_1-1},\cdots,x_{i_d\to m}^{-1},0^{n_d-1};x_{i_{d+1}\to m}^{-1}),\\
&I(x_{i_1\to m}^{-1};0^{n_1-1},\cdots,x_{i_d\to m}^{-1},0^{n_d-1};0),\quad I(0;0^{n_d-1},x_{i_d\to m}^{-1},\cdots,0^{n_1-1};x_{i_1\to m}^{-1}),\\
&I(x_{i_1\to m}^{-1},0^{n_1-1},\cdots,x_{i_d\to m}^{-1},0^{n_d-1};x_{i_{d+1}\to m}^{-1})
\end{align*}
for any $1\leq i_1<\cdots<i_d< i_{d+1}\leq m$ and $n_k\in\mathbb Z_{\geq1}$. Note $x_{i_{d+1}\to m}^{-1}$ could be 1 if $i_{d+1}=m$.
\end{definition}

Finally, we are ready to construct the Hopf algebra morphism $\Phi:\mathbb I^{\Symb}\to\overline{\mathbb H}^{\Symb}$. Its definition is based on the properties of iterated integrals.

\begin{definition}\cite{ZDHZ_HopfAlgebrasOfMultiplePolylogarithmsAndHolomorphicOneForms}
Suppose $a_i\neq 0$ and denote $[1]_0=0$. Then $\Phi:\mathbb I^{\Symb}\to\overline{\mathbb H}^{\Symb}$ is defined on the generating series as follows
\begin{enumerate}
\item 
\begin{equation}
\begin{aligned}
\Phi(I(0;&a_1,\cdots,a_m;a_{m+1}|t_0,\cdots,t_m))=\\
&(-1)^m\exp([a_{m+1}]_0t_0)\left[\frac{a_2}{a_1},\cdots,\frac{a_{m+1}}{a_m}\middle|t_1-t_0,\cdots,t_m-t_0\right]
\end{aligned}
\end{equation}
\item 
\begin{multline}
\Phi(I(a_0;a_1,\cdots,a_m;0|t_0,\cdots,t_m))=(-1)^m\Phi(I(0;a_m,\cdots,a_1;a_0|-t_m,\cdots,-t_1))
\end{multline}
\item 
\begin{multline}
\Phi(I(a_0;\cdots;a_{m+1}|t_0,\cdots,t_m))=\prod_{p=0}^k\Phi(I(a_{i_p};a_{i_p+1},\cdots,a_{j_p};0|t_{i_p},\cdots,t_{j_p}))\\
\Phi(I(0;a_{j_p+1},\cdots,a_{i_{p+1}-1};a_{i_{p+1}}|t_{j_p},\cdots,t_{i_{p+1}-1}))
\end{multline}
\end{enumerate}
Or equivalently, $\Phi$ is defined on the generators by
\begin{multline}\label{eq: Phi(I(0;0^n0-1,...))}
\Phi(I(0;0^{n_0-1},a_{1},0^{n_1-1},\cdots,a_{m},0^{n_m-1};a_{m+1}))=\sum_{i_0+\cdots+i_m=n_0-1}(-1)^{n_0+i_0+m-1}\\
\dfrac{[a_{m+1}]_0^{i_0}}{i_0!}\binom{n_1+i_1-1}{n_1-1}\cdots\binom{n_m+i_m-1}{n_m-1}\left[\frac{a_{2}}{a_{1}},\cdots,\frac{a_{m+1}}{a_{m}}\right]_{n_1+i_1,\cdots,n_m+i_m}
\end{multline}
\begin{multline}
\Phi(I(a_{0};0^{n_0-1},\cdots,a_{m},0^{n_m-1};0))=(-1)^{n_0+\cdots+n_m-1}\Phi(I(0;0^{n_m-1},a_{m},\cdots,0^{n_0-1};a_{0}))
\end{multline}
\begin{multline}
\Phi(I(a_{0};0^{n_0-1},\cdots,a_{m},0^{n_m-1};a_{m+1}))=\\
\sum_{k=0}^m\sum_{p+q=n_k>1}\Phi(I(a_{0};\cdots,a_{k},0^{p-1};0))\Phi(I(0,0^{q-1},a_{k+1},\cdots;a_{m+1}))
\end{multline}
\end{definition}

\begin{example}
\[
\Phi(I(0;0^n;x_1^{-1}))=\frac{(-1)^n}{n!}[x_1]_0
\]
\[
\Phi(I(0;x_{i_1\to i_{d+1}}^{-1},0^{n_1-1},\cdots,x_{i_d\to i_{d+1}}^{-1},0^{n_d-1};1))=(-1)^d[x_{i_1\to i_2},\cdots,x_{i_d\to i_{d+1}}]_{n_1,\cdots,n_d}
\]
\begin{align*}
\Phi(I(0;0,&0,(x_1x_2)^{-1},0;x_2^{-1}))=-\frac{1}{2}[x_2]_0^2[x_1]_2-[x_2]_0[x_1]_3-[x_1]_4
\end{align*}
\begin{align*}
\Phi((x_1x_2)^{-1};0,0;x_2^{-1})=\frac{1}{2}[x_2]_0^2-[x_1x_2]_0[x_2]_0+\frac{1}{2}[x_1x_2]_0^2
\end{align*}
\end{example}

\begin{theorem}\label{thm: Phi is Hopf algebra morphism}\cite{ZDHZ_HopfAlgebrasOfMultiplePolylogarithmsAndHolomorphicOneForms}
$\Phi$ is a Hopf algebra homomorphism.
\end{theorem}

\begin{proof}
The proof can be found in~\cite{ZDHZ_HopfAlgebrasOfMultiplePolylogarithmsAndHolomorphicOneForms}, Proposition 6.12.
\end{proof}

Since $\Phi$ preserves coproduct, immediately we have

\begin{corollary}\label{cor: Hbar is coassociative}
The coproduct on $\overline{\mathbb H}^{\Symb}$ is coassociative.
\end{corollary}

To prove Theorem~\ref{thm: H is a Hopf algebra}, we only need to show that coproduct commutes with $\INV$. This fact is straightforward yet lengthy and tedious. We put its proof in the next section.

\subsection{Commutativity of the coproduct and $\INV$}

\begin{theorem}\label{thm: coproduct commutes with INV}
The map $\INV: \overline{\mathbb H}^{\Symb}\to\mathbb H^{\Symb}$ is a homomorphism of Hopf algebras, i.e. $\Delta_{\mathbb H}\circ\INV=\INV\circ\Delta_{\overline{\mathbb H}}$.
\end{theorem}

\begin{proof}
$\Delta\circ\INV=\INV\circ\Delta$ clearly holds for the regular terms, so we consider only inverse terms. Assume this holds for lower depth (the depth one case $\INV\circ\Delta[y^{-1}|-t]=\Delta\circ\INV[y^{-1}|-t]$ is elementary). By the definition of $\INV$~\eqref{eq: INV map} one has  
$\INV [y_d^{-1},\dots,y_1^{-1}|-t_d,\dots,-t_1] = \sum \INV A_i$ with all $A_i$ of lower depth. By induction, we thus have
\begin{equation}
 \Delta \circ \INV [y_d^{-1}\cdots,y_1^{-1}|-t_d,\cdots,-t_1] = \sum \Delta \circ \INV A_i = \sum \INV \circ \Delta A_i,
\end{equation}
so it suffices to show that $\sum \INV \circ \Delta A_i = \INV \circ \Delta [y_d^{-1},\dots,y_1^{-1}|-t_d,\dots,-t_1] $. Rearranging the terms this is equivalent to:
\begin{equation}\label{eq:Claim}
\begin{aligned}
0=\INV\circ\Delta\left(\sum_{j=0}^d\right.&(-1)^j[y_j^{-1},\cdots,y_1^{-1}|-t_j,\cdots,-t_1][y_{j+1},\cdots,y_d|t_{j+1},\cdots,t_d]\\
&+\sum_{j=1}^d\frac{(-1)^j}{t_j}[y_{j-1}^{-1},\cdots,y_1^{-1}|-t_{j-1},\cdots,-t_1][y_{j+1},\cdots,y_d|t_{j+1},\cdots,t_d]\\
&-\sum_{j=1}^d\frac{(-1)^{j}}{t_j}[y_{j-1}^{-1},\cdots,y_1^{-1}|t_j-t_{j-1},\cdots,t_j-t_1]\\
&\phantom{sdfsdfsfsdfsf}\exp([y_{1\to d+1}]_0t_j)[y_{j+1},\cdots,y_d|t_{j+1}-t_{j},\cdots,t_d-t_j]\left.\vphantom{\sum_{j=0}^d(-1)}\right).
\end{aligned}
\end{equation}

We must prove~\eqref{eq:Claim}. We write the right hand side of~\eqref{eq:Claim} as $\INV\circ\Delta(A+B-C)$ and rewrite $\exp([X]_0t)$ as $X^t$. In what follows we use some natural notational simplifications, e.g.~we write $y^{-1}_{j,\dots,i}$ instead of $(y_j^{-1},y_{j-1}^{-1},\dots,y_i^{-1})$ and $t_k-t_{j,\dots,i}$ instead of $(t_k-t_j,t_k-t_{j-1},\dots,t_k-t_i)$.
We first rewrite $\Delta(B)$ as
\begin{align*}
\sum_{r=1}^d&\frac{(-1)^r}{t_r}\Delta[y_{r-1,\dots,1}^{-1}|-t_{r-1,\dots,1}]\Delta[y_{r+1,\dots,d}|t_{r+1,\dots,d}]\\
=&\sum_{r=1}^d\frac{(-1)^r}{t_r}\sum_{1=i_0\leq j_0<\dots<i_q\leq r<i_{q+1}<\dots<i_{k+1}=d+1}[y_{i_{q-1}\to i_q,\dots,i_0\to i_1}^{-1}|-t_{j_{q-1},\dots,j_0}]\\
&\hphantom{\sum_{r=1}}[y_{i_{q+1}\to i_{q+2},\dots,i_k\to i_{k+1}}|t_{j_{q+1},\dots,k}]\otimes[y_{r-1,\dots,i_q}^{-1}|-t_{r-1,\dots,i_q}][y_{r+1,\dots,i_{q+1}-1}|t_{r+1,\dots,i_{q+1}-1}]\\
&\hphantom{\sum_{r=1}}\prod_{p=0}^{q-1}(-1)^{j_p-i_{p+1}+1}y_{i_p\to i_{p+1}}^{t_{j_p}}[y_{j_p+1,\dots,i_{p+1}-1}|t_{j_p+1,\dots,i_{p+1}-1}-t_{j_p}][y^{-1}_{j_p-1,\dots,i_p}|t_{j_p}-t_{j_p-1,\dots,i_p}]\\
&\hphantom{\sum_{r=1}}\prod_{p=q+1}^k(-1)^{j_p-i_p}y_{i_p\to i_{p+1}}^{t_{j_p}}[y_{j_p-1,\dots,i_p}^{-1}|t_{j_p}-t_{j_p-1,\dots,i_p}][y_{j_p+1,\dots,i_{p+1}-1}|t_{j_p+1,\dots,i_{p+1}-1}-t_{j_p}],
\end{align*}

which simplifies to

\begin{equation}
\begin{aligned}
&\sum_{1=i_0\leq j_0<\dots<i_q\leq j_q<i_{q+1}<\dots<i_{k+1}=d+1}\frac{(-1)^{j_q}}{t_{j_q}}[y_{i_{q-1}\to i_q,\dots,i_0\to i_1}^{-1}|-t_{j_{q-1},\dots,j_0}]\\
&[y_{i_{q+1}\to i_{q+2},\dots,i_k\to i_{k+1}}|t_{j_{q+1},\dots,j_k}]\otimes[y_{j_q-1,\dots,i_q}^{-1}|-t_{j_q-1,\dots,i_q}][y_{j_q+1,\dots,i_{q+1}-1}|t_{j_q+1,\dots,i_{q+1}-1}](-1)^{i_q+q+1}\\
&\prod_{0\leq p\leq k,p\neq q}(-1)^{j_p-i_p}y_{i_p\to i_{p+1}}^{t_{j_p}}[y_{j_p-1,\dots,i_p}^{-1}|t_{j_p}-t_{j_p-1,\dots,i_p}][y_{j_p+1,\dots,i_{p+1}-1}|t_{j_p+1,\dots,i_{p+1}-1}-t_{j_p}].
\end{aligned}
\end{equation}
Similarly, $\Delta(C)$ can be simplified to
\begin{equation}
\begin{aligned}
&\sum_{1\leq i_0\leq j_0<\dots<i_q\leq j_q<i_{q+1}<\dots<i_{k+1}=d+1}\frac{(-1)^{j_q}}{t_{j_q}}[y^{-1}_{i_{q-1}\to i_q,\dots,i_0\to i_1}|t_{j_q}-t_{j_{q-1},\dots,j_0}]y_{1\to d+1}^{t_{j_q}}\\
&[y_{i_{q+1}\to i_{q+2},\dots,i_k\to i_{k+1}}|t_{j_{q+1},\dots,j_k}-t_{j_q}]\otimes[y_{j_q-1,\dots,i_q}^{-1},t_{j_q}-t_{j_q-1,\dots,i_q}]y_{1\to d+1}^{t_{j_q}}\\
&[y_{j_q+1,\dots,i_{q+1}-1}|t_{j_q+1,\dots,i_{q+1}-1}-t_{j_q}](-1)^{i_q+q+1}\\
&\prod_{0\leq p\leq k,p\neq q}(-1)^{j_p-i_p}y_{i_p\to i_{p+1}}^{t_{j_p}-t_{j_q}}[y_{j_p-1,\dots,i_p}^{-1}|t_{j_p}-t_{j_p-1,\dots,i_p}][y_{j_p+1,\dots.i_{p+1}-1}|t_{j_p+1,\dots.i_{p+1}-1}-t_{j_p}]
\end{aligned}
\end{equation}
Finally, $\Delta(A)$ equals
\begin{equation}
\begin{aligned}
\sum_{r=0}^d&(-1)^r\Delta[y_{r,\dots,1}^{-1}|-t_{r,\dots,1}]\Delta[y_{r+1,\dots,d}|t_{r+1,\dots,d}]\\
=&\sum_{r=0}^d(-1)^r\sum_{1=i_0\leq j_0<\cdots<i_q\leq r+1\leq i_{q+1}<\cdots<i_{k+1}=d+1}[y_{i_{q-1}\to i_q,\dots,i_0\to i_1}^{-1}|-t_{j_{q-1},\dots,j_0}]\\
&[y_{i_{q+1}\to i_{q+2},\dots,i_k\to i_{k+1}}|t_{j_{q+1},\dots,j_k}]\otimes[y_{r,\dots,i_q}^{-1}|-t_{r,\dots,i_q}][y_{r+1,\dots,i_{q+1}-1}|t_{r+1,\dots,i_{q+1}-1}]\\
&\prod_{p=0}^{q-1}(-1)^{j_p-i_{p+1}+1}y_{i_p\to i_{p+1}}^{t_{j_p}}[y_{j_p+1,\dots,i_{p+1}-1}|t_{j_p+1,\dots,i_{p+1}-1}-t_{j_p}][y_{j_p-1,\dots,i_p}^{-1}|t_{j_p}-t_{j_p-1,\dots,i_p}]\\
&\prod_{p=q+1}^k(-1)^{j_p-i_p}y_{i_p\to i_{p+1}}^{t_{j_p}}[y_{j_p-1,\dots,i_p}^{-1}|t_{j_p}-t_{j_p-1,\dots,i_p}][y_{j_p+1,\dots,i_{p+1}-1}|t_{j_p+1,\dots,i_{p+1}-1}-t_{j_p}],
\end{aligned}
\end{equation}
which simplifies to
\begin{equation}\label{eq: Induction hypothesis usage}
\begin{aligned}
&\sum_{1=i_0\leq j_0<\cdots<i_q\leq i_{q+1}<\cdots<i_{k+1}=d+1}[y_{i_{q-1}\to i_q,\dots,i_0\to i_1}^{-1}|-t_{j_{q-1},\dots,j_0}]\\
&\hspace*{4pc}[y_{i_{q+1}\to i_{q+2},\dots,i_k\to i_{k+1}}|t_{j_{q+1},\dots,j_k}]\bigotimes(-1)^q\\
&\hspace*{4pc}\left(\sum_{i_q\leq r+1\leq i_{q+1}}(-1)^{r-i_q+1}[y_{r,\dots,i_q}^{-1}|-t_{r,\dots,i_q}][y_{r+1,\dots,i_{q+1}-1}|t_{r+1,\dots,i_{q+1}-1}]\right)\\
&\hspace*{4pc}\prod_{0\leq p\leq k,p\neq q}(-1)^{j_p-i_p}y_{i_p\to i_{p+1}}^{t_{j_p}}[y_{j_p-1,\dots,i_p}^{-1}|t_{j_p}-t_{j_p-1,\dots,i_p}][y_{j_p+1,\dots,i_{p+1}-1}|t_{j_p+1,\dots,i_{p+1}-1}-t_{j_p}].
\end{aligned}
\end{equation}
We split the sum into two parts depending on whether or not $i_q=i_{q+1}$:
\begin{equation}
\sum_{1=i_0\leq j_0<\cdots<i_q<i_{q+1}<\cdots<i_{k+1}=d+1},\quad\sum_{1=i_0\leq j_0<\cdots<i_q=i_{q+1}<\cdots<i_{k+1}=d+1}
\end{equation}
We then apply $\INV$ and use induction on the bracket of~\eqref{eq: Induction hypothesis usage}. The first sum becomes
\begin{equation}
\begin{aligned}
&\INV\left\{\sum_{1=i_0\leq j_0<\cdots<i_q< i_{q+1}<\cdots<i_{k+1}=d+1}[y_{i_{q-1}\to i_q,\dots,i_0\to i_1}^{-1}|-t_{j_{q-1},\dots,j_0}]\right.\\
&[y_{i_{q+1}\to i_{q+2},\dots,i_k\to i_{k+1}}|t_{j_{q+1},\dots,j_k}]\bigotimes(-1)^q\\
&\left(-\sum_{i_q\leq r\leq i_{q+1}-1}\frac{(-1)^{r-i_q+1}}{t_r}[y_{r-1,\dots,i_q}^{-1}|-t_{r-1,\dots,i_q}][y_{r+1,\dots,i_{q+1}-1}|t_{r+1,\dots,i_{q+1}-1}]\right.\\
&+\left.\sum_{i_q\leq r\leq i_{q+1}-1}\frac{(-1)^{r-i_q+1}}{t_r}[y_{r-1,\dots,i_q}^{-1}|t_r-t_{r-1,\dots,i_q}]y_{i_q\to i_{q+1}}^{t_r}[y_{r+1,\dots,i_{q+1}-1}|t_{r+1,\dots,i_{q+1}-1}-t_r]
\right)\\
&\left.\prod_{0\leq p\leq k,p\neq q}(-1)^{j_p-i_p}y_{i_p\to i_{p+1}}^{t_{j_p}}[y_{j_p-1,\dots,i_p}^{-1}|t_{j_p}-t_{j_p-1,\dots,i_p}][y_{j_p+1,\dots,i_{p+1}-1}|t_{j_p+1,\dots,i_{p+1}-1}-t_{j_p}]\right\}.
\end{aligned}
\end{equation}
This equals
\begin{equation}
\begin{aligned}
&=-\INV\left\{\sum_{1=i_0\leq j_0<\cdots<i_q\leq j_q<i_{q+1}<\cdots<i_{k+1}=d+1}\frac{(-1)^{j_q-i_q+q+1}}{t_{j_q}}[y_{i_{q-1}\to i_q,\dots,i_0\to i_1}^{-1}|-t_{j_{q-1},\dots,j_0}]\right.\\
&[y_{i_{q+1}\to i_{q+2},\dots,i_k\to i_{k+1}}|t_{j_{q+1},\dots,j_k}]\otimes[y_{j_q-1,\dots,i_q}^{-1}|-t_{j_q-1,\dots,i_q}][y_{j_q+1,\dots,i_{q+1}-1}|t_{j_q+1,\dots,i_{q+1}-1}]\\
&\left.\prod_{0\leq p\leq k,p\neq q}(-1)^{j_p-i_p}y_{i_p\to i_{p+1}}^{t_{j_p}}[y_{j_p-1,\dots,i_p}^{-1}|t_{j_p}-t_{j_p-1,\dots,i_p}][y_{j_p+1,\dots,i_{p+1}-1}|t_{j_p+1,\dots,i_{p+1}-1}-t_{j_p}]\right\}\\
&+\INV\left\{\sum_{1=i_0\leq j_0<\cdots<i_q\leq j_q<i_{q+1}<\cdots<i_{k+1}=d+1}\frac{(-1)^{j_q-i_q+q+1}}{t_{j_q}}[y_{i_{q-1}\to i_q,\dots,i_0\to i_1}^{-1}|-t_{j_{q-1},\dots,j_0}]\right.\\
&[y_{i_{q+1}\to i_{q+2},\dots,i_k\to i_{k+1}}|t_{j_{q+1},\dots,j_k}]\otimes[y_{j_q-1,\dots,i_q}^{-1}|t_{j_q}-t_{j_q-1,\dots,i_q}][y_{j_q+1,\dots,i_{q+1}-1}|t_{j_q+1,\dots,i_{q+1}-1}-t_{j_q}]\\
&\left.y_{i_q\to i_{q+1}}^{t_{j_q}}\prod_{\substack{0\leq p\leq k\\p\neq q}}(-1)^{j_p-i_p}y_{i_p\to i_{p+1}}^{t_{j_p}}[y_{j_p-1,\dots,i_p}^{-1}|t_{j_p}-t_{j_p-1,\dots,i_p}][y_{j_p+1,\dots,i_{p+1}-1}|t_{j_p+1,\dots,i_{p+1}-1}-t_{j_p}]\right\},
\end{aligned}
\end{equation}
which we write as $-\INV(T_1)+\INV(T_2)$. The second sum becomes
\begin{equation}
\begin{aligned}
&\INV\left\{\sum_{1=i_1\leq j_1<\dots<i_{k+1}=d+1}\Bigg(\right.\\
&\sum_{0\leq q\leq k}(-1)^q[y_{i_q\to i_{q+1},\dots,i_1\to i_2}^{-1}|-t_{j_q,\dots,j_1}][y_{i_{q+1}\to i_{q+2},\dots,i_k\to i_{k+1}}|t_{j_{q+1},\dots,j_k}]\Bigg)\\
&\left.\otimes\prod_{1\leq p\leq k}(-1)^{j_p-i_p}y_{i_p\to i_{p+1}}^{t_{j_p}}[y_{j_p-1,\dots,i_p}^{-1}|t_{j_p}-t_{j_p-1,\dots,i_p}][y_{j_p+1,\dots,i_{p+1}-1}|t_{j_p+1,\dots,i_{p+1}-1}-t_{j_p}]\right\}\\
&=\INV\left\{\sum_{1=i_1\leq j_1<\dots<i_{k+1}=d+1}\Bigg(\right.\\
&-\sum_{1\leq q\leq k}\frac{(-1)^q}{t_{j_q}}[y_{i_{q-1}\to i_q,\dots,i_1\to i_2}^{-1}|-t_{j_{q-1},\dots,j_1}][y_{i_{q+1}\to i_{q+2},\dots,i_k\to i_{k+1}}|t_{j_{q+1},\dots,j_k}]\\
&+\sum_{1\leq q\leq k}\frac{(-1)^q}{t_{j_q}}[y_{i_{q-1}\to i_q,\dots,i_1\to i_2}^{-1}|t_{j_q}-t_{j_{q-1},\dots,j_1}]y_{1\to d+1}^{t_{j_q}}[y_{i_{q+1}\to i_{q+2},\dots,i_k\to i_{k+1}}|t_{j_{q+1},\dots,j_k}-t_{j_q}]\Bigg)\\
&\left.\otimes\prod_{1\leq p\leq k}(-1)^{j_p-i_p}y_{i_p\to i_{p+1}}^{t_{j_p}}[y_{j_p-1,\dots,i_p}^{-1}|t_{j_p}-t_{j_p-1,\dots,i_p}][y_{j_p+1,\dots,i_{p+1}-1}|t_{j_p+1,\dots,i_{p+1}-1}-t_{j_p}]\right\}.
\end{aligned}
\end{equation}
This equals
\begin{equation}
\begin{aligned}
&=-\INV\left\{\sum_{1=i_0\leq j_0<\dots<i_{k+1}=d+1}\frac{(-1)^q}{t_{j_q}}[y_{i_{q-1}\to i_q,\dots,i_0\to i_1}^{-1}|-t_{j_{q-1},\dots,j_0}][y_{i_{q+1}\to i_{q+2},\dots,i_k\to i_{k+1}}|t_{j_{q+1},\dots,j_k}]\right.\\
&\left.\otimes\prod_{0\leq p\leq k}(-1)^{j_p-i_p}y_{i_p\to i_{p+1}}^{t_{j_p}}[y_{j_p-1,\dots,i_p}^{-1}|t_{j_p}-t_{j_p-1,\dots,i_p}][y_{j_p+1,\dots,i_{p+1}-1}|t_{j_p+1,\dots,i_{p+1}-1}-t_{j_p}]\right\}\\
&+\INV\left\{\sum_{1=i_0\leq j_0<\dots<i_{k+1}=d+1}\frac{(-1)^q}{t_{j_q}}[y_{i_{q-1}\to i_q,\dots,i_0\to i_1}^{-1}|t_{j_q}-t_{j_{q-1},\dots,j_0}]y_{1\to d+1}^{t_{j_q}}\right.\\
&[y_{i_{q+1}\to i_{q+2},\dots,i_k\to i_{k+1}}|t_{j_{q+1},\dots,j_k}-t_{j_q}]\\
&\left.\otimes\prod_{0\leq p\leq k}(-1)^{j_p-i_p}y_{i_p\to i_{p+1}}^{t_{j_p}}[y_{j_p-1,\dots,i_p}^{-1}|t_{j_p}-t_{j_p-1,\dots,i_p}][y_{j_p+1,\dots,i_{p+1}-1}|t_{j_p+1,\dots,i_{p+1}-1}-t_{j_p}]\right\}
\end{aligned}
\end{equation}
which we write as $-\INV(T_3)+\INV(T_4)$. We note that 
\begin{equation}
    \Delta(B)=T_1,\qquad \Delta(C)=T_4,\qquad T_2=T_3.
\end{equation}
This implies that $\INV\circ\Delta(A+B-C)=0$, proving the claim.
\end{proof}

\section{Associated one-forms of multiple polylogarithms}\label{sec: one-forms of multiple polylogarithms}

\subsection{Motivation}

Recall Goncharov's definition of $\mathcal B_n(F)$ group as $\mathbb Z[\mathbb P^1_F]/R_n(F)$, where generators $[a]_n$ may be viewed as $\Li_n(a)$ and $R_n(F)$ is the subgroup generated by all functional relations between polylogarithms. These $\mathcal B_n(F)$ groups make up the Bloch complex~\eqref{eq: Bloch complex}
\[
\mathcal B_n(F)\xrightarrow{\delta_n}\mathcal B_{n-1}(F)\otimes F^\times\xrightarrow{\delta_{n-1}}\mathcal B_{n-1}(F)\otimes \textstyle\bigwedge^2F^\times\to\cdots\xrightarrow{\delta_2}\textstyle\bigwedge^nF^\times
\]
The differentials are
\begin{equation}\label{eq: differentials for Bloch complex}
\begin{aligned}
&\delta_n([a])=[a]\otimes a,\quad \delta_2([a])\otimes b= a\wedge (1-a)\wedge b\\
&\delta_k([a])\otimes b=[a]\otimes a\wedge b, \quad\forall 2<k<n
\end{aligned}
\end{equation}
Goncharov conjectured that the $i$-th cohomology group of this complex is rationally isomorphic to $H^i_{\mathcal M}(F,\mathbb Q(n))$.

In~\cite{Zickert_HolomorphicPolylogarithmsAndBlochComplexes}, Zickert considered lifted polylogarithms which are functions from $\widehat{\mathbb C}$ to $\mathbb C/\frac{(2\pi i)^n}{(n-1)!}\mathbb Z$ defined by
\begin{equation}
\widehat{\mathcal L}_n(u,v)=\sum_{r=0}^{n-1}\frac{(-1)^r}{r!}\Li_{n-r}(e^u)u^r-\frac{(-1)^n}{n!}u^{n-1}v
\end{equation}
He then constructed groups $\widehat{\mathcal B}_n(\mathbb C)=\mathbb Z[\widehat{\mathbb C}]/\widehat R_n(\mathbb C)$, where generators $[(u,v)]_n$ represent $\widehat{\mathcal L}_n(u,v)$ and $\widehat R_n(\mathbb C)$ is generated by functional relations among $\widehat{\mathcal L}_n$. These $\widehat{\mathcal B}$ groups forms a chain complex $\widehat\Gamma(\mathbb C,n)$
\begin{equation}\label{eq: lifted Bloch complex for C}
\widehat{\mathcal B}_n(\mathbb C)\xrightarrow{\widehat\delta_n}\widehat{\mathcal B}_{n-1}(\mathbb C)\otimes\widehat{\mathbb C}\xrightarrow{\widehat\delta_{n-1}}\widehat{\mathcal B}_{n-2}(\mathbb C)\otimes\textstyle\bigwedge^2\widehat{\mathbb C}\to\cdots\xrightarrow{\widehat\delta_2}\textstyle\bigwedge^n\widehat{\mathbb C}
\end{equation}
where the differentials are
\begin{equation}\label{eq: differentials for lifted Bloch complex}
\begin{aligned}
&\widehat\delta_n([(u,v)])=[(u,v)]\otimes u,\quad \widehat\delta_2([(u,v)])\otimes a= u\wedge v\wedge a\\
&\widehat\delta_k([(u,v)])\otimes a=[(u,v)]\otimes u\wedge a, \quad\forall 2<k<n
\end{aligned}
\end{equation}
Zickert conjectures that its $i$-th cohomology group is integrally isomorphic to $H^i_{\mathcal M}(\mathbb C,\mathbb Z(n))$. In particular, he argued that $\widehat{\mathcal L}_n$, viewed as a map from $\ker\widehat\delta_n$ to $\mathbb C/(2\pi i)^n\mathbb Z$, should correspond to the cycle map (see~\cite{Bloch_AlgebraicCyclesAndTheBeilinsonConjectures})
\begin{equation}\label{eq: cycle map}
b_n: CH^n(\Spec\mathbb C,2n-1)\to H^1_{\mathcal D}(\Spec\mathbb C,\mathbb Z(n))
\end{equation}

Since the Chow group $CH^n(X,2n-i)$ isomorphic to $H^i_{\mathcal M}(X,\mathbb Z(n))$ for an algebraic manifold $X$ (see~\cite{Voevodsky_MotivicCohomologyGroupsAreIsomorphicToHigherChowGroupsInAnyCharacteristic}), and the Deligne cohomology $H^i_{\mathcal D}(\Spec\mathbb C,\mathbb Z(n))$ is isomorphic to $\mathbb C/(2\pi i)^n\mathbb Z$. Therefore $b_n$ can be regarded as a map from $H^i_{\mathcal M}(\Spec\mathbb C,\mathbb Z(n))$ to $\mathbb C/(2\pi i)^n\mathbb Z$.

These lifted polylogarithms have well-defined differential one-forms in $\Omega^1(\widehat{\mathbb C})$:
\begin{equation}\label{eq: dL_n}
w_n(u,v)=d\widehat{\mathcal L}_n=(-1)^n\frac{n-1}{n!}u^{n-2}(udv-vdu)
\end{equation}
If we treat coordinates $u,v$ as indeterminates in the polynomial ring $\mathbb C[u,v]$, then $w_n(u,v)\in\Omega^1_{\mathbb C[u,v]/\mathbb C}$. It is tempting to define lifted multiple polylogarithms $\widehat{\mathcal L}_{n_1,\cdots,n_d}$ such that
\begin{enumerate}[i.]
\item They are functions from $\widehat S_d(\mathbb C)$ to $\mathbb C/\frac{(2\pi i)^n}{(n-1)!}\mathbb Z$, where $n=n_1+\cdots+n_d$.
\item They can be expressed as sums and products of multiple polylogarithms of lower weights.
\item Their differentials $d\widehat{\mathcal L}_{n_1,\cdots,n_d}$ are in $\Omega^1_{\mathbb C[\{u_i, v_{j,k}\}]/\mathbb C}$.
\end{enumerate}
Unfortunately, this cannot be done. In fact, Zickert defined $\widehat{\mathcal L}_{1,1}$, $\widehat{\mathcal L}_{2,1}$, $\widehat{\mathcal L}_{1,2}$, $\widehat{\mathcal L}_{1,1,1}$ (unpublished), but that the best one can do for $\widehat{\mathcal L}_{3,1}$ is such that
\begin{equation}
d\widehat{\mathcal L}_{3,1}=w_{3,1}-w_2(u_2,v_2)\widehat{\mathcal L}_2(u_{1,2},v_{1,2}),\quad dw_{3,1}=w_2(u_2,v_2)\wedge w_2(u_{1,2},v_{1,2})
\end{equation}
Here $u_{1,2}$ stands for $u_1+u_2$. Note that $w_{3,1}$ is neither exact (not the differential of $\widehat{\mathcal L}_{3,1}$) nor closed. This suggests a Hodge structure hidden behind these lifted multiple polylogarithms, similar to Zhao's description of variation of Hodge structures of multiple polylogarithms. We will tackle this problem in Chapter 4.

These one-forms $w_{n_1,\cdots,n_d}$ can be obtained from the symbols of the multiple polylogarithm $\Li_{n_1,\cdots,n_d}$~\cite{ZDHZ_HopfAlgebrasOfMultiplePolylogarithmsAndHolomorphicOneForms}. By first applying Gangl's projection map $P$ in Definition~\ref{def: projection map P} and then the map
\begin{equation}\label{eq: symmetrized symbol to one-form}
a_1\otimes\cdots\otimes a_n\to\frac{(-1)^{n+1}}{(n-1)!}a_2\cdots a_nda_1
\end{equation}
These one-forms have nice combinatorial properties. We will discuss them in the next section and later on in Chapter 4.

\subsection{One-forms}

\begin{definition}
Let $H$ be a connected graded Hopf algebra. Consider the \textit{one-forms} map $w_f:T^*H_1\to\Omega^1_{\mathbb Q[H_1]/\mathbb Q}$
\begin{equation}\label{eq: w_f}
w_f(a_1\otimes\cdots\otimes a_n)=\frac{(-1)^{n+1}}{n!}\sum_{1\leq i\leq n}(-1)^{i-1}\binom{n-1}{i-1}a_1\cdots da_i\cdots a_n
\end{equation}
\end{definition}

\begin{theorem}\label{thm: one-form map factor through Gangl's projection map P}
$w_f$ is equal to the composition of $P$ in Definition~\ref{def: projection map P} and map in~\eqref{eq: symmetrized symbol to one-form}.
\end{theorem}

\begin{proof}
The proof can be found in~\cite{ZDHZ_HopfAlgebrasOfMultiplePolylogarithmsAndHolomorphicOneForms} Lemma 4.3 with $\Pi$ as $P$.
\end{proof}

% The associated one-forms are symmetric under the symmetrization

% \begin{definition}
% The \textit{symmetrization} of one-forms is a map $R:\Omega^1_{\mathbb Q[H_1]/\mathbb Q}\to \Omega^1_{\mathbb Q[H_1]/\mathbb Q}$ defined by
% \begin{equation}
% R(a_1\cdots a_{n-1}da_n)=a_1\cdots a_{n-1}da_n-\frac{1}{n}d(a_1\cdots a_n)
% \end{equation}
% \end{definition}

% \begin{proposition}
% $R$ is an idempotent with $\ker R=d\mathbb Q[H_1]$ consists of exact differential forms, and it also induce a projection map $R(\Omega^1)\xrightarrow{d}R(\Omega^1)\wedge R(\Omega^1)$.
% \end{proposition}

% \begin{remark}

% \end{remark}

\begin{definition}\label{def: one-forms}
The one form map $w:\mathbb H^{\Symb}\to\Omega_{\mathbb Q[u_i,v_{j,k}]/\mathbb Q}$ is defined to be the composite of the symbol map $\Delta_{1,\cdots,1}:\mathbb H^{\Symb}\to T^*\mathbb H^{\Symb}_1$, $w_f:T^*\mathbb H^{\Symb}_1\to\Omega_{\mathbb Q[\mathbb H^{\Symb}_1]/\mathbb Q}$, and  $w_s:\Omega_{\mathbb Q[\mathbb H^{\Symb}_1]/\mathbb Q}\to\Omega_{\mathbb Q[\{u_i,v_{j,k}\}]/\mathbb Q}$. Where $w_s$ substitute $[x_i]_0$ into $u_i$ and $[x_{j\to k+1}]_1$ into $-v_{j,k}$.
\end{definition}

\begin{example}
\begin{equation}\label{eq: w_n}
w([x_1]_n)=\frac{(-1)^n}{n!}u_1^{n-2}(u_1dv_1-v_1du_1),\quad n\geq2
\end{equation}
\begin{multline}
w([x_1,x_2]_{1,1})=-\frac{1}{2} u_1 dv_{1,2}+\frac{1}{2} v_{1,2} du_1+\frac{1}{2} v_1 dv_{1,2}-\frac{1}{2} v_2 dv_{1,2}\\
-\frac{1}{2}
   v_{1,2} dv_1+\frac{1}{2} v_{1,2} dv_2-\frac{1}{2} v_1 dv_2+\frac{1}{2} v_2 dv_1
\end{multline}
Note that\eqref{eq: w_n} corresponds to~\eqref{eq: dL_n} up to a constant $(n-1)$.
\end{example}

\section{Free contraction Hopf algebras}

We will construct several different Hopf algebras at the end of this section, and justifying the Hopf algebra structure of each one individually would require significant effort. Instead, we introduced the notion of a contraction system, which captures the essence of the coproduct~\ref{eq:GoncharovCoproductFormula}. Consider maps $(i_1,i_2,\cdots,i_{d+1})$ that contracts variables
\begin{equation}
\begin{aligned}
S_n(\mathbb C)&\xrightarrow{(i_1,i_2,\cdots,i_{d+1})} S_d(\mathbb C)\\
(y_1,y_2,\cdots,y_n)&\mapsto\textstyle\left(\prod\limits_{i_1\leq r<i_2}y_r,\prod\limits_{i_2\leq r<i_3}y_r,\cdots,\prod\limits_{i_d\leq r<i_{d+1}}y_r\right)
\end{aligned}
\end{equation}
Then the pullback
\begin{equation}\label{eq: contraction property on multiple polylog}
(i_1,i_2,\cdots,i_{d+1})^*\Li_{n_1,\cdots,n_d}(y_1,\cdots,y_n)=\Li_{n_1,\cdots,n_d}(y_{i_1\to i_2},\cdots,y_{i_d\to i_{d+1}})
\end{equation}
appears in the second term of the tensor products in the coproduct formula~\ref{eq:GoncharovCoproductFormula}.

\subsection{Contraction System}

A contraction system is a category that formalizes the contraction property described in~\eqref{eq: contraction property on multiple polylog}. In fact, it is equivalent to a semisimplical category with $X_1=\{*\}$. However, we retain the term ``contraction system'' because it is more suggestive of the context of our work. Hopf algebras in Definition~\ref{def: free contraction Hopf algebra} are called contraction algebras due to this.

Let's denote $\mathscr C$ as the category where
\begin{itemize}
\item The objects are sets of the form $[n]=\{0,1,\cdots,n\}$ for $n\geq1$.
\item A morphism from $[n]$ to $[d]$ is given by a tuple $\mathbf i=(i_1,\cdots,i_{d+1})$, $1\leq i_1<\cdots<i_{d+1}\leq n+1$.
\item The identity morphism for each object $[n]$ is $\mathbf 1=(1,2,\cdots,n+1)$
\item and composition of $\mathbf j=(j_1,\cdots,j_{k+1}):[d]\to[k]$ and $\mathbf i$ is $\mathbf j\circ\mathbf i=(i_{j_1},\cdots,i_{j_{k+1}})$.
\end{itemize}

\begin{example}\label{ex: contraction system X}
Let $\mathcal X=\bigcup_d\mathcal X^d$ be the set of continuous products of $x$ symbols where $\mathcal X^d=\{(x_{i_1\to i_2},\cdots,x_{i_d\to i_{d+1}})\}$. Then $\mathcal X$ forms a contraction system with contractions
\[
\mathcal X^d\xrightarrow{(j_1,\cdots,j_{k+1})}\mathcal X^k,\quad (x_{i_1\to i_2},\cdots,x_{i_d\to i_{d+1}})\mapsto(x_{i_{j_1}\to i_{j_2}},\cdots,x_{i_{j_k}\to i_{j_{k+1}}})
\]
\end{example}

With this notation, \eqref{eq:GoncharovCoproductFormula} becomes
\begin{equation}\label{eq:GoncharovCoproductFormulaWithContractions}
\begin{aligned}
\Delta([\mathbf y|\mathbf t])=\sum&[(i_1,\cdots,i_{d+1})\mathbf y|t_{j_1},\dots,t_{j_k}]\bigotimes\prod_{\alpha=0}^k(-1)^{j_\alpha-i_\alpha}\exp([(i_{\alpha},i_{\alpha+1})\mathbf y]_0t_{j_\alpha})\\
&[((i_\alpha,i_\alpha+1,\cdots,j_\alpha)\mathbf y)^{-1}|t_{j_\alpha}-t_{j_\alpha-1},\dots, t_{j_\alpha}-t_{i_\alpha}]\\
&[(j_\alpha+1,j_\alpha+2,\cdots,i_{\alpha+1})\mathbf y|t_{j_\alpha+1}-t_{j_\alpha},\dots,t_{i_{\alpha+1}-1}-t_{j_\alpha}].\\
\end{aligned}
\end{equation}
The sum is over all instances of $1=i_0\leq j_0<i_1\leq j_1<\dots <i_k\leq j_k<i_{k+1}=d+1$.

\begin{example}
We can turn the maps in Example~\ref{ex: contraction system X} into maps between schemes
\[
\left\{S_n=\Spec\mathbb Z\left[x_i^{\pm1},\left(\textstyle\prod\limits_{j\leq r<k}x_r-1\right)^{-1}\right]\right\}
\]
where $S_n\xrightarrow{(i_1,\cdots,i_{d+1})} S_d$ is given by $x_p\mapsto\prod\limits_{i_p\leq r<i_{p+1}}x_r$.
\end{example}

We would like to interpret the sets in both examples as functors, this leads to a formal definition of a contraction system.

\begin{definition}
A \textit{contraction system} in $\mathscr D$ is defined as a functor from $\mathscr C$ to $\mathscr D$. A \textit{morphism between contraction systems} is a natural transformation.
\end{definition}

\begin{example}
Given a field $F$, all morphisms between contraction systems $\mathcal X\to\{S_n(F)\}$ are in one-to-one correspondence with tuples $\{(a_i)_{i\in\mathbb Z_{>0}}, a_i\in F\}$ determined by $\{x_i\to a_i\}_{i\in\mathbb Z_{>0}}$
\end{example}

\begin{example}\label{ex: SnHat(F)}
Suppose $F$ is a field and $\pi:E\to F^*$ is a torsion free extension of $F^*$ by $\mathbb Z$. Similar to the construction of $\widehat S_d(\mathbb C)$ (where $\pi=\exp:\mathbb C\to\mathbb C^*$), we can construct a contraction system $\{\widehat S_n(F)\}$ as
\begin{equation}
\widehat S_n(F)=\left\{\left(\{u_i\}_{1\leq i\leq n},\{v_{j,k}\}_{1\leq j\leq k\leq n}\right)\in E^{n+\binom{n+1}{2}}\middle|\pi\left(\textstyle\sum_{r=j}^k u_r\right)+\pi(v_{j,k})=1,\forall j\leq k\right\}
\end{equation}
and the contraction maps are
\begin{equation}
(i_1,\cdots,i_{d+1})\left(\{u_p\}_{p},\{v_{j,k}\}_{j\leq k}\right)=\left(\left\{\textstyle\sum_{r=i_q}^{i_{q+1}-1}u_r\right\}_q,\left\{v_{i_j,i_{k+1}-1}\right\}_{j\leq k}\right)
\end{equation}
\end{example}

\subsection{Free contraction Hopf algebra}

In this subsection, we want to construct $\overline{\mathbb H}$ and $\mathbb H$ as functors from the category of contraction systems to the category of graded Hopf algebras.

\begin{definition}\label{def: free contraction Hopf algebra}\cite{ZDHZ_HopfAlgebrasOfMultiplePolylogarithmsAndHolomorphicOneForms}
For any contraction system $\mathcal A$, we define $\overline{\mathbb H}(\mathcal A)$ to be the free algebra generated by regular symbols $[\alpha]_{n_1,\cdots,n_d}$ and inverted symbols $[\alpha^{-1}]_{n_d,\cdots,n_1}$, where $\alpha\in\mathcal A^d, \mathbf n=(n_1,\cdots,n_d)\in\mathbb Z^d_{>0}$ and $[\alpha]_0, [\alpha^{-1}]_0$, $\alpha\in \mathcal A^1$ modulo relations
\[
[(i_1,i_2)\mathbf\alpha]_{0} + [(i_2,i_3)\mathbf\alpha]_{0} = [(i_1,i_3)\mathbf\alpha]_{0},\forall\alpha\in\mathcal A^d, d\geq2,\quad[\alpha^{-1}]_0=-[\alpha]_0
\]
Here $d$ is referred to as the depth. We also denote $\mathbb H(\mathcal A)$ as the subalgebra of $\overline{\mathbb H}(\mathcal A)$ generated solely by regular symbols $[\alpha]_{n_1,\cdots,n_d}$, $[\alpha]_0$.
\end{definition}

Similar to~\eqref{eq: INV map}, we can also define $\INV:\overline{\mathbb H}(\mathcal A)\to\overline{\mathbb H}(\mathcal A)$ which fixes the regular symbols and acts on inverted symbols inductively as
\begin{equation}
\begin{aligned}
&\INV([\mathbf y^{-1}|-t_d,\cdots,-t_1])\\
&=\sum_{j=0}^{d-1}(-1)^{d-1+j}\INV([((1,\cdots,j+1)\mathbf y)^{-1}|-t_j,\cdots,-t_1])[(j+1,\cdots,d+1)\mathbf y|t_{j+1},\cdots,t_d]\\
&+\sum_{j=1}^d\frac{(-1)^{d-1+j}}{t_j}\INV([((1,\cdots,j)\mathbf y)^{-1}|-t_{j-1},\cdots,-t_1])[(j+1,\cdots,d+1)\mathbf y|t_{j+1},\cdots,t_d]\\
&+\sum_{j=1}^d\bigg(\frac{(-1)^{d+j}}{t_j}\INV([((1,\cdots,j)\mathbf y)^{-1}|t_j-t_{j-1},\cdots,t_j-t_1])\\
&\qquad\qquad\exp([(1,d+1)\mathbf y]_0t_j)[(j+1,\cdots,d+1)\mathbf y|t_{j+1}-t_{j},\cdots,t_d-t_j]\bigg)
\end{aligned}
\end{equation}

\begin{proposition}
The coproduct $\Delta$ given by~\eqref{eq:GoncharovCoproductFormulaWithContractions} and
\[
\Delta([\alpha^{\pm}]_0)=1\otimes[\alpha]_0+[\alpha^{\pm}]_0\otimes1,\quad\Delta([\alpha]^{\pm}_1)=1\otimes[\alpha^{\pm}]_1+[\alpha^{\pm}]_1\otimes1
\]
on any $\overline{\mathbb H}(\mathcal A)$ and $\mathbb H(\mathcal A)$ define Hopf algebra structures.
\end{proposition}

\begin{proof}
The proof is straightforward. See~\cite{ZDHZ_HopfAlgebrasOfMultiplePolylogarithmsAndHolomorphicOneForms}, Theorem 2.15.
\end{proof}

\begin{example}
With $\mathcal X$ defined in Example~\ref{ex: contraction system X}, it is not hard to see that $\overline{\mathbb H}(\mathcal X)=\overline{\mathbb H}^{\Symb}$, $\mathbb H(\mathcal X)=\mathbb H^{\Symb}$.
\end{example}

We call these $\overline{\mathbb H}(\mathcal A)$ and $\mathbb H(\mathcal A)$ \textit{free contraction Hopf algebras}. And we have the following proposition.

\begin{proposition}
$\overline{\mathbb H}$, $\mathbb H$ are functors from the category of contraction systems to the category of graded Hopf algebras.
\end{proposition}

% \begin{definition}
% We can define an evaluation map
% \[
% \ev:\mathbb H^{\Symb}\times\mathcal A\to\mathbb H(\mathcal A),\qquad (\mathbf i[x_1,\cdots,x_n]_{\mathbf n}, \alpha)\mapsto[\mathbf i(\alpha)]_{\mathbf n}
% \]
% Sometimes it is also easy to use currying $\ev_{\alpha}(\mathbf i[x_1,\cdots,x_n]_{\mathbf n})=[\mathbf i(\alpha)]_{\mathbf n}$
% \end{definition}

\begin{example}
Suppose $\mathcal A$ is the contraction system $\{S_n(F)\}$, then the corresponding free contraction Hopf algebras are denoted by $\overline{\mathbb H}^{\Symb}(F)$, $\mathbb H^{\Symb}(F)$. The corresponding free contraction Hopf algebra of $\{\widehat S_n(F)\}$ in Example~\ref{ex: SnHat(F)} is denoted $\widehat{\mathbb H}^{\Symb}_E(F)$ or $\widehat{\mathbb H}^{\Symb}(F)$ for short. Specifically, if $E=F=\mathbb C$ and $\pi:\mathbb C\to\mathbb C^*$ is the exponential map, we get the free contraction Hopf algebra as $\widehat{\mathbb H}^{\Symb}(\mathbb C)$.
\end{example}

More generally, we can construct sheaves of Hopf algebras over manifolds.

\begin{example}
Let $M$ be a smooth complex manifold and $U$ be an open subset of $M$, then holomorphic maps from $U$ to $\widehat S_n(\mathbb C)$ forms a contraction system where the contraction maps $\Omega^0(U,\widehat S_n(\mathbb C))\to\Omega^0(U,\widehat S_d(\mathbb C))$ are simply induced by contraction maps $\widehat S_n(\mathbb C)\to\widehat S_d(\mathbb C)$. This way we obtain a sheaf $\widehat{\mathbb H}^{\Symb}_M$ of Hopf algebras on $M$.
\end{example}

\section{Motivic complex}

In this section, we  assume $0\to\mathbb Z\to E\xrightarrow{\pi} F^*\to0$ is a torsion free $\mathbb Z$ extension, and construct candidates for motivic complexes in Question~\ref{Fundamental questions} ii. and Question~\ref{Fundamental questions LHat} iii.. And we formulate the conjecture that relates motivic cohomology and singular cohomology through our construction.

\subsection{The $\mathbb L^{\Symb}$ complex}

Recall that according to Definition~\ref{def: Lie coalgebra}, any connected graded Hopf algebra modulo products defines a Lie coalgebra, it is natural to define Lie coalgebras
\begin{equation}
\begin{aligned}
\mathbb L^{\Symb}&:=\mathbb H^{\Symb} / (\mathbb H^{\Symb}_{>0}\cdot \mathbb H^{\Symb}_{>0})\\
\widehat{\mathbb L}^{\Symb}(F)&:=\widehat{\mathbb H}^{\Symb}(F) /\left (\widehat{\mathbb H}^{\Symb}(F)_{>0}\cdot \widehat{\mathbb H}^{\Symb}(F)_{>0}\right)\\
\widehat{\mathbb L}^{\Symb}_M&:=\widehat{\mathbb H}^{\Symb}_M/\left(\widehat{\mathbb H}^{\Symb}_{M\quad>0}\cdot\widehat{\mathbb H}^{\Symb}_{M\quad>0}\right)
\end{aligned}
\end{equation}
The one form map $\mathbb H^{\Symb}\to\Omega^1_{\mathbb Q[\{u_i,v_{j,k}\}]/\mathbb Q}$ induces a chain map from its Chevalley-Eilenberg complex $\bigwedge^*\mathbb L^{\Symb}$ to the de Rham complex $\Omega^*_{\mathbb Q[\{u_i,v_{j,k}\}]/\mathbb Q}$.

\begin{remark}
Even though $\mathbb H^{\Symb}$, $\widehat{\mathbb H}^{\Symb}(F)$ and $\widehat{\mathbb H}^{\Symb}_M$ are defined over $\mathbb Q$, but $\mathbb L^{\Symb}$, $\widehat{\mathbb L}^{\Symb}(F)$ and $\widehat{\mathbb L}^{\Symb}_M$ are defined over $\mathbb Z$.
\end{remark}

\begin{theorem}\cite{ZDHZ_HopfAlgebrasOfMultiplePolylogarithmsAndHolomorphicOneForms}\label{thm: LSymb to de Rham}
The one-form map induces a chain map from $\bigwedge^*\mathbb L^{\Symb}$ to $\Omega^*_{\mathbb Q[\{u_i,v_{j,k}\}]/\mathbb Q}$, i.e. the following diagram commutes
\begin{center}
\begin{tikzcd}
\mathbb L^{\Symb} \arrow[d, "w"] \arrow[r, "\delta"] & \textstyle\bigwedge^2\mathbb L^{\Symb} \arrow[d, "w\wedge w"] \arrow[r, "1\wedge\delta-\delta\wedge1"] & \textstyle\bigwedge^3\mathbb L^{\Symb} \arrow[d, "w\wedge w\wedge w"] \arrow[r] & \cdots \\
\Omega^1 \arrow[r, "d"]                       & \Omega^2 \arrow[r, "d"]                                                                             & \Omega^3 \arrow[r, "d"]                                                  & \cdots
\end{tikzcd}
\end{center}
\end{theorem}

\begin{proof}
First we show that $w_f$ defined in~\eqref{eq: w_f} induces a chain map $w_f:\bigwedge^*L(T(H_1))\to\Omega^*$, where $L(T(H_1))=\frac{T(H_1)}{T(H_1)_{>0}\cdot T(H_1)_{>0}}$. This is Proposition 4.4 in~\cite{ZDHZ_HopfAlgebrasOfMultiplePolylogarithmsAndHolomorphicOneForms}.

On the other hand, $\Delta_{1,\cdots,1}$ induces a chain map $L(H)$ to $L(T(H_1))$ by Proposition~\ref{prop: Hopf algebra morphism induce Lie coalgebra morphism}, where $L(H)=\frac{H}{H_{>0}\cdot H_{>0}}$. Composing both chain maps, we obtain a chain map from $\bigwedge^*\mathbb L^{\Symb}$ to $\Omega^*_{\mathbb Q[\{u_i,v_{j,k}\}]/\mathbb Q}$.
\end{proof}

\subsection{Main conjectures}

In~\cite{Zickert_HolomorphicPolylogarithmsAndBlochComplexes}, Zickert defined the lifted Bloch complex $\widehat\Gamma(F,n)$
\begin{equation}\label{eq: lifted Bloch complex}
\widehat{\mathcal B}_n(F)\xrightarrow{\widehat\delta_1}\widehat{\mathcal B}_{n-1}(F)\otimes E\xrightarrow{\widehat\delta_2}\widehat{\mathcal B}_{n-2}(F)\otimes\textstyle\bigwedge^2 E\to\cdots\to\widehat{\mathcal B}_{2}(F)\otimes\textstyle\bigwedge^{n-2} E\xrightarrow{\widehat\delta_{n-1}}\textstyle\bigwedge^n E
\end{equation}
with differentials given by~\ref{eq: differentials for lifted Bloch complex}. Assuming $\pi:E\to F^*$ is an $\mathbb Z$ extension. He conjectured that the $i$-th cohomology group of this complex is integrally isomorphic to $H^i_{\mathcal M}(F,\mathbb Z(n))$.

We wish to generalize the Bloch complex~\eqref{eq: Bloch complex} and the lifted Bloch complex~\eqref{eq: lifted Bloch complex} to include generators representing multiple polylogarithms and lifted multiple polylogarithms. Similar to the definition of $\mathcal B$ and $\widehat{\mathcal B}$ groups, we define $\mathbb L_n(F)$ and $\widehat{\mathbb L}_n(F)$ to be $\mathbb L_n^{\Symb}(F)/R_n(F)$ and $\widehat{\mathbb L}_n^{\Symb}(F)/\widehat R_n(F)$ respectively. Where $R_n(F)$ is inductively defined similar to Goncharov~\cite{GoncharovMotivicGalois}, which can be found in~\cite{ZDHZ_TheLieCoalgebraOfMultiplePolylogarithms}, section 3. And $\widehat R_n(F)$ is yet unknown, but it should be constructed similar to that in~\cite{Zickert_HolomorphicPolylogarithmsAndBlochComplexes}, and it is expected to be generated by all functional relations between lifted multiple polylogarithms of weight $n$.

Their Chevalley-Eilenberg complexes $\bigwedge^*\mathbb L(F)$ and $\bigwedge^*\widehat{\mathbb L}(F)$ will be referred to as the motivic complex and lifted motivic complex. We conjecture that they compute rational and integral motivic cohomology, respectively.

\begin{conjecture}
\[
H^i\Big(\left(\textstyle\bigwedge^*\mathbb L(F)\right)_n\Big)_{\mathbb Q}\cong H_{\mathcal M}^i(F,\mathbb Q(n))_{\mathbb Q},\quad H^i\Big(\left(\textstyle\bigwedge^*\widehat{\mathbb L}(F)\right)_n\Big)\cong H_{\mathcal M}^i(F,\mathbb Z(n))
\]
We use subscript $\mathbb Q$ in the first isomorphism to indicate that this only conjectured to be true rationally.
\end{conjecture}

$\widehat{\mathbb L}^{\Symb}_M$ also defines a sheaf of Lie coalgebras over a complex manifold $M$. Theorem~\ref{thm: LSymb to de Rham} induces a sheaf map $\bigwedge^*\widehat{\mathbb L}^{\Symb}_M\to\Omega^*_M$ defined locally by $\widehat{\mathbb L}^{\Symb}_M(U)\to\Omega_M^1(U)$ as pulling back of one-form map $w$. For example, for $f=(f_1,f_2,f_3,f_4,f_5)\colon U\to \widehat S_2$, that
\begin{multline}
w([f]_{1,1})=\frac{1}{2}(-f_1^*u_1 f_5^*d v_{1,2}+ f_5^*v_{1,2}f_1^*d u_1+ f_3^*v_1 f_5^*d v_{1,2}\\-f_4^*v_2f_5^*d v_{1,2}-f_5^*v_{1,2}f_3^*d v_1+f_5^*v_{1,2}f_4^*d v_2-f_3^*v_1f_4^*d v_2+f_4^*v_2f_3^*d v_1)\in\Omega^1_M(U).
\end{multline}
Assume $R_M$ is the subsheaf of $\widehat{\mathbb L}^{\Symb}_M$ generated by functional relations of lifted multiple polylogarithms pullback to $M$. Then $\bigwedge^*\widehat{\mathbb L}_M\to\Omega^*_M$ is well defined. Since one forms are differentials of lifted multiple polylogarithms modulo products of lower weight functions, and the differential of any functional relation should be zero. 

\begin{conjecture}\label{conj: motivic cohomology and singular cohomology}\cite{ZDHZ_HopfAlgebrasOfMultiplePolylogarithmsAndHolomorphicOneForms}
If $M$ is an algebraic manifold, we conjecture the following diagram commutes
\begin{center}
\begin{tikzcd}
{H^i(M,(\bigwedge^*\widehat{\mathbb L}_M)_n)} \arrow[r]              & {H^i(M,\Omega_M^*)} \arrow[d, "\cong"] \\
{H^i_{\mathcal M}(M,\mathbb Z(n))} \arrow[u, dashed] \arrow[r] & {H^i(M,\mathbb C)}
\end{tikzcd}
\end{center}
Recall from Definition~\ref{def: graded Lie coalgebra} that $(\bigwedge^*\widehat{\mathbb L}_M)_n$ is the degree $n$ part of the complex. The top arrow is induced by $\bigwedge^\bullet\widehat{\mathbb L}_M\to\Omega^\bullet$, the bottom arrow is the realization functor from integral motivic cohomology to singular cohomology, the right arrow is the isomorphism between de Rham cohomology and singular cohomology~\cite{Grothendieck_OnTheDeRhamCohomologyOfAlgebraicVarieties}, and the left arrow is yet unclear and requires further investigation.
\end{conjecture}
