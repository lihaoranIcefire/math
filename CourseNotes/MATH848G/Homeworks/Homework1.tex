\documentclass[../main.tex]{subfiles}

\begin{document}

\textbf{1.} \par
Consider Hilbert space $(\mathcal{H}_s,\langle,\rangle)$, where $\displaystyle\mathcal{H}_s=\left\{(c_m)_{m\in\mathbb{Z}^m}\middle\vert\sum_{m\in\mathbb{Z}^n}|c_m|^2(1+|m|^2)^s<\infty\right\}$, with the obvious vector space structure and $\displaystyle\langle (c_m),(d_m)\rangle=\sum_{m\in\mathbb{Z}^n}c_m\overline{d_m}(1+|m|^2))^s$, then fourier transform $\mathcal{F}:H^s(\mathbb{T}^n)\rightarrow\mathcal{H}_s$, $f\mapsto (\widehat{f}(m))$ is an isometry, thus it suffices to prove $\mathcal{H}_{s_2}\Rightarrow\mathcal{H}_{s_1}$ is compact \par
Suppose $\left\{(c_{\alpha m})\in \mathcal{H}_{s_2}\right\}$ is bounded set, namely, $\displaystyle\sum_{m\in\mathbb{Z}^n}|c_{\alpha m}|^2(1+|m|^2)^{s_2}\leq M$ for some $M$, $\forall k\in\mathbb{N}$, there exists $N_k$, for $|m|\geq N_k$, we have $(1+|m|^2)^{s_1-s_2}<\dfrac{1}{M2^{k+2}}$, and for $|m|\leq N_k$, $(c_{\alpha m})_{|m|\leq N_k}$ is a bounded set, thus having a convergent subsequence such that any $(c_{\alpha m}),(c_{\beta m})$ in that sequence satisfy $\displaystyle\sum_{|m|\leq N_k}|c_{\alpha m}-c_{\beta m}|^2(1+|m|^2)^{s_1}<\dfrac{1}{2^{k+1}}$, then by repeating this process, each time we get a subsequence out of the previous one, taking the diagonal sequence, we get a sequence $(c_{km})$ where for $|m|\geq N_k$, $(1+|m|^2)^{s_1-s_2}<\dfrac{1}{M2^{k+2}}$, and $\forall p,q\geq k$, $\displaystyle\sum_{|m|\leq N_k}|c_{pm}-c_{qm}|^2(1+|m|^2)^{s_1}<\dfrac{1}{2^{k+1}}$, then \par
$\displaystyle\sum_{m\in\mathbb{Z}^n}|c_{pm}-c_{qm}|^2(1+|m|^2)^{s_1}\leq\sum_{|m|\leq N_k}|c_{pm}-c_{qm}|^2(1+|m|^2)^{s_1}+\sum_{|m|\geq N_k}|c_{pm}-c_{qm}|^2(1+|m|^2)^{s_2}(1+|m|^2)^{s_1-s_2}<\dfrac{1}{2^{k+1}}+\dfrac{1}{M2^{k+2}}\sum_{m\in\mathbb{Z}^n}|c_{pm}-c_{qm}|^2(1+|m|^2)^{s_2}<\dfrac{1}{2^k}$
Therefore $(c_{km})$ is a convergent sequence in $\mathcal{H}_{s_1}$ \par
\textbf{2.} \par
If $\theta$ is an irrational number and $\alpha$ is a rational number, then $m_1+\theta m_2+\alpha\neq0$, $\forall(m_1,m_2)\in\mathbb{Z}^2$ \par
Since $\displaystyle\widehat{f}(m)=\dfrac{1}{(2\pi)^n}\int_{\mathbb{T}^n}f(z)z^{-m}\dfrac{dz}{z}$, hence \par
$\displaystyle\widehat{D_jf}(m)=\dfrac{1}{(2\pi)^n}\int_{\mathbb{T}^n}z_j\dfrac{\partial}{\partial z_j}f(z)z^{-m-1}dz=\dfrac{1}{(2\pi)^n}\int_{\mathbb{T}^n}\left[\dfrac{\partial}{\partial z_j}f(z)z_jz^{-m-1}-f(z)\dfrac{\partial}{\partial z_j}(z_jz^{-m-1})\right]dz=$ \par
$\dfrac{1}{(2\pi)^n}\int_{\mathbb{T}^n}\dfrac{\partial}{\partial z_j}f(z)z_jz^{-m-1}dz-\dfrac{m_j}{(2\pi)^n}\int_{\mathbb{T}^n}f(z)z^{-m-1}dz=m_j\widehat{f}(m)$, hence by Plancherel theorem
$\|\widehat{f}(m)\|^2=\|\widehat{Df}\|^2=\|\widehat{D_1f}+\theta\widehat{D_2f}+\alpha \widehat{f}(m)\|^2=\sum_{m\in\mathbb{Z}^n}|\widehat{D_1f}(m)+\theta\widehat{D_2f}(m)+\alpha \widehat{f}(m)|^2=\sum_{m\in\mathbb{Z}^n}|m_1\widehat{f}+\theta m_2\widehat{f}+\alpha \widehat{f}(m)|^2=\sum_{m\in\mathbb{Z}^n}|m_1+\theta m_2+\alpha|^2|\widehat{f}(m)|^2$, thus $D$ is injective \par
On the other hand, we want $\widehat{f}(m)$ to be not rapidly decreasing while $(m_1+\theta m_2+\alpha)\widehat{f}(m)$ is, consider Liouville number $\ell=\sum_{k=0}^{\infty}2^{-k!}$ which is a transcendental number with the property that $\left|\ell - \sum_{k=0}^{n}2^{-k!}\right|=\left|\sum_{k=n+1}^{\infty}2^{-k!}\right|\leq2^{1-(n+1)!}\leq\left(2^{-n!}\right)^n$, let $\dfrac{a_n}{b_n}:=\sum_{k=0}^{n}2^{-k!}$, then $a_n$ is an odd number whereas $b_n$ is an even number with $\left|\ell - \dfrac{a_n}{b_n}\right|\leq\dfrac{1}{b_n^n}$, thus if we choose $\alpha=\dfrac{1}{2}, \theta=\ell, m_1=\dfrac{a_n-1}{2}, m_2=\dfrac{b_n}{2}$, then $|m_1+\theta m_2+\alpha|=|m_2|\left|\theta + \dfrac{2m_1+1}{2m_2}\right|\leq\dfrac{C}{|m_2|^{n-1}}$, and as $m_2$ is big enough, we would have $\dfrac{1}{|m_2|}\leq\dfrac{C}{\left(1+|m|^2\right)^{\frac{1}{2}}}$, hence $|m_1+\theta m_2+\alpha|\leq\dfrac{C}{\left(1+|m|^2\right)^{\frac{n-1}{2}}}$, for such choices, make $\widehat{f}(m)=1$ and all others $0$, then $\widehat{f}(m)$ is not rapidly decreasing while $(m_1+\theta m_2+\alpha)\widehat{f}(m)$ is \par

\end{document}