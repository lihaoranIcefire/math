\documentclass[../main.tex]{subfiles}

\begin{document}

\textbf{1.} \par
\textbf{(i)} \par
Since $d_0(f)=df=\dfrac{\partial f}{\partial x}dx+\dfrac{\partial f}{\partial y}dy$, $d_1^*(fdx\wedge dy)=-\star d\star(fdx\wedge dy)=-\star df=-\star\left(\dfrac{\partial f}{\partial x}dx+\dfrac{\partial f}{\partial y}dy\right)=\dfrac{\partial f}{\partial y}dx-\dfrac{\partial f}{\partial x}dy$, the symbol(or in this case the principal symbol) of $D$ is $\begin{pmatrix}
i\xi_1 & i\xi_2 \\
i\xi_2 & -i\xi_1
\end{pmatrix}=i\begin{pmatrix}
\xi_1 & \xi_2 \\
\xi_2 & -\xi_1
\end{pmatrix}$ is invertible off the zero section, hence $D$ is elliptic \par
Alternatively, $d_0(f)=df=\dfrac{\partial f}{\partial z}dz+\dfrac{\partial f}{\partial\bar z}d\bar z$, $d_1^*(fdz\wedge d\bar z)=i\dfrac{\partial f}{\partial z}dz-i\dfrac{\partial f}{\partial \bar z}d\bar z$, thus the symbol(principal symbol) of $D$ would be $\begin{pmatrix}
\xi & i\xi \\
\bar\xi & -i\bar\xi
\end{pmatrix}$ which is invertible off the zero section, thus $D$ is elliptic \par
\textbf{(ii)} \par

\textbf{(iii)} \par
$D^*=d_0^*+d_1$, thus $D^*D=(d_0^*+d_1)(d_0+d_1^*)=d_0^*d_0+d_0^*d_1^*+d_1d_0+d_1d_1^*=d_0^*d_0+(d_1d_0)^*+d_1d_0+d_1d_1^*=d_0^*d_0+d_1d_1^*$, then $\langle D(f+\omega),D(f+\omega)\rangle=\langle D^*D(f+\omega),(f+\omega)\rangle=\langle d_0^*d_0f,f\rangle+\langle d_1d_1^*\omega,\omega\rangle=\langle d_0f,d_0f\rangle+\langle d_1^*\omega,d_1^*\omega\rangle$, thus $D(f+\omega)=0\Leftrightarrow d_0f=d_1^*\omega=0$, $\ker D=\ker d_0\oplus\ker d_1^*\cong H_{dR}^0(S^2,\mathbb R)\oplus H_{dR}^2(S^2,\mathbb R)\cong H^0(S^2,\mathbb R)\oplus H^2(S^2,\mathbb R)$, but the cokernel of $D$ is just $H_{dR}^1(S^2,\mathbb R)\cong H^1(S^2,\mathbb R)=0$, therefore $\mathrm{ind}D=\dim\ker D-\dim\mathrm{coker}D=2$ \par
\textbf{2.} \par




\end{document}