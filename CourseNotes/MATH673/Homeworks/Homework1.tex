\documentclass[../main.tex]{subfiles}

\begin{document}

\begin{customproblem}\textbf{2.5.3}
Define $$\phi(r):=\dfrac{1}{|\partial B(0,r)|}\int_{\partial B(0,r)}u(x)dS+\int_{B(0,r)}\Phi(x)f(x)dx-\Phi(r)\int_{B(0,r)}f(x)dx $$Where $\Phi(x)=\dfrac{1}{n(n-2)|B(0,1)|}\dfrac{1}{|x|^{n-2}}$, then 
\[
\begin{aligned}
\phi'(r)
&= \dfrac{r}{n}\dfrac{1}{|B(0,r)|}\int_{B(0,r)}\Delta u(x)dx+ \int_{\partial B(0,r)}\Phi(x)f(x)dS \\
&\quad -\Phi'(r)\int_{B(0,r)}f(x)dx-\Phi(r)\int_{\partial B(0,r)}f(x)dS \\
&= -\dfrac{r}{n}\dfrac{1}{|B(0,r)|}\int_{B(0,r)}f(x)dx+\dfrac{r}{n}\dfrac{1}{|B(0,r)|}\int_{B(0,r)}f(x)dx \\
&\quad +\int_{\partial B(0,r)}\left(\Phi(x)-\Phi(r)\right)f(x)dS \\
&= 0
\end{aligned}
\]
\[
\begin{aligned}
\phi(s)
&=u(0)+\dfrac{1}{|\partial B(0,s)|}\int_{\partial B(0,s)}\left(u(x)-u(0)\right)dS+\int_{B(0,s)}\Phi(x)f(x)dx-\Phi(s)\int_{B(0,s)}f(x)dx \\
&:=u(0)+I_{1}+I_{2}+I_{3}
\end{aligned}
\]
\[
|I_{1}|\leq \dfrac{1}{|\partial B(0,s)|}\int_{\partial B(0,s)}\left|u(x)-u(0)\right|dS\rightarrow 0, \,s\rightarrow 0
\]
By the continuity of $u$
\[
|I_{2}|\leq \left\Vert f\right\Vert_{L^{\infty}}\int_{B(0,s)}\Phi(x)dx=\left\Vert f\right\Vert_{L^{\infty}}\int_{B(0,s)}\Phi(x)dx=\dfrac{s^{2}\left\Vert f\right\Vert_{L^{\infty}}}{2(n-2)}\rightarrow 0, \,s\rightarrow 0
\]
\[
|I_{3}|\leq \Phi(s)\left\Vert f\right\Vert_{L^{\infty}}\int_{B(0,s)}dx=\left\Vert f\right\Vert_{L^{\infty}}\int_{B(0,s)}\Phi(x)dx=\dfrac{s^{2}\left\Vert f\right\Vert_{L^{\infty}}}{n(n-2)}\rightarrow 0, \,s\rightarrow 0
\]
Thus
\[
\begin{aligned}
u(0)&=\lim_{s\rightarrow 0}\phi(s)=\lim_{s\rightarrow r}\phi(s) \\
&=\dfrac{1}{|\partial B(0,r)|}\int_{\partial B(0,r)}g(x)dS+\int_{B(0,r)}\left(\Phi(x)-\Phi(r)\right)f(x)dx
\end{aligned}
\]
\end{customproblem}

\begin{customproblem}\textbf{2.5.4}
\textbf{a)}\par
Since 
\[
\dfrac{d}{dr}\left\{\dfrac{1}{|\partial B(x,r)|}\int_{\partial B(x,r)}v(y)dS\right\}=\dfrac{r}{n}\dfrac{1}{|B(x,r)|}\int_{B(x,r)}\Delta v(y)dy \geq 0
\]
Thus 
\[
v(x)\leq\dfrac{1}{|\partial B(x,r)|}\int_{\partial B(x,r)}v(y)dS
\]
and
\[
\begin{aligned}
\dfrac{1}{|B(x,r)|}\int_{B(x,r)}v(y)dy
&= \dfrac{1}{|B(x,r)|}\int_{0}^{r}\left(\int_{\partial B(x,s)}v(y)dS\right)ds \\
&\geq v(x)\dfrac{1}{|B(x,r)|}\int_{0}^{r}|\partial B(x,s)|ds \\
&=v(x)
\end{aligned}
\]
\textbf{b)}\par
Suppose $v$ attains at $x_{0}\in U$ its maximum $\underset{\overline{U}}{\mathrm{max}}\,v > \underset{\partial U}{\mathrm{max}}\,v$, consider the connected component $U_{0}$ which contains $x_{0}$,  then by $a)$ we know that the set $\{x\in U_{0}\:v(x)=\underset{\overline{U}}{\mathrm{max}}\,v\}$ is both open and closed, thus $\underset{\partial U_{0}}{\mathrm{max}}\,v=\underset{\overline{U}}{\mathrm{max}}\,v$ which is a contradiction. \par
\textbf{c)}\par
Since $\phi$ is convex, $\phi''(x)\geq 0$, thus
\[
\Delta v(x)=\Delta\phi(u(x))=\phi'(u(x))\Delta u+\phi''(u(x))|\nabla u|^{2}=\phi''(u(x))|\nabla u|^{2}\geq 0
\]
Hence $v$ is subharmonic. \par
\textbf{d)}\par
\[
\begin{aligned}
\Delta v &= \Delta |\nabla u|^{2} =2\sum_{i,j}u_{x_{i}x_{j}}^{2}+2\nabla u\cdot\nabla(\Delta u)\geq 0
\end{aligned}
\]
By \textbf{a)} we know $v$ is subharmonic
\end{customproblem}

\begin{customproblem}\textbf{2.5.5}
Define $M:=\underset{B(0,1)}{\mathrm{max}}\,|f|$ and $v:=u+\dfrac{M}{2n}|x|^{2}$, then we have $\Delta v=\Delta u+M=M-f\geq 0$ in $B^{0}(0,1)$, thus $v$ is subharmonic, maximum principle still holds, hence \[
\underset{B(0,1)}{\mathrm{max}}\,u \leq\underset{B(0,1)}{\mathrm{max}}\,v=\underset{\partial B(0,1)}{\mathrm{max}}\,v\leq \underset{\partial B(0,1)}{\mathrm{max}}\,|g|+\dfrac{M}{2n}=\underset{B(0,1)}{\mathrm{max}}\,|v|\leq \underset{\partial B(0,1)}{\mathrm{max}}\,|g|+\dfrac{1}{2n}\underset{B(0,1)}{\mathrm{max}}\,|f|
\]Similary, we could consider $\left\{\begin{matrix}
-\Delta(-u)=-f\\ 
-u=-g
\end{matrix}\right.$, then we would have $$-\underset{B(0,1)}{\mathrm{min}}\,u=\underset{B(0,1)}{\mathrm{max}}\,(-u)\leq\underset{\partial B(0,1)}{\mathrm{max}}\,|g|+\dfrac{1}{2n}\underset{B(0,1)}{\mathrm{max}}\,|f| $$
Therefore, we have $$\underset{B(0,1)}{\mathrm{max}}\,|u|\leq\underset{\partial B(0,1)}{\mathrm{max}}\,|g|+\dfrac{1}{2n}\underset{B(0,1)}{\mathrm{max}}\,|f|\leq\underset{\partial B(0,1)}{\mathrm{max}}\,|g|+\underset{B(0,1)}{\mathrm{max}}\,|f|$$ 
\end{customproblem}

\begin{customproblem}\textbf{2.5.6}
Using Poisson's formula and the fact that $u$ is harmonic, we have \[
\begin{aligned}
u(x)
&=\dfrac{r^{2}-|x|^{2}}{|\partial B(0,1)| r}\int_{\partial B(0,r)}\dfrac{u(y)}{|x-y|^{n}}dS \\
&\leq \dfrac{r^{2}-|x|^{2}}{|\partial B(0,1)| r}\int_{\partial B(0,r)}\dfrac{u(y)}{(r-|x|)^{n}}dS \\
&= \dfrac{r^{2}-|x|^{2}}{|\partial B(0,1)| r}\dfrac{1}{(r-|x|)^{n}}\int_{\partial B(0,r)}u(y)dS \\
&= \dfrac{r^{n-2}(r+|x|)}{(r-|x|)^{n-1}}u(0)
\end{aligned}
\]Similarly, we have\[
\begin{aligned}
u(x)
&=\dfrac{r^{2}-|x|^{2}}{|\partial B(0,1)| r}\int_{\partial B(0,r)}\dfrac{u(y)}{|x-y|^{n}}dS \\
&\geq \dfrac{r^{2}-|x|^{2}}{|\partial B(0,1)| r}\int_{\partial B(0,r)}\dfrac{u(y)}{(r+|x|)^{n}}dS \\
&= \dfrac{r^{2}-|x|^{2}}{|\partial B(0,1)| r}\dfrac{1}{(r+|x|)^{n}}\int_{\partial B(0,r)}u(y)dS \\
&= \dfrac{r^{n-2}(r-|x|)}{(r+|x|)^{n-1}}u(0)
\end{aligned}
\]Thus $\dfrac{r^{n-2}(r-|x|)}{(r+|x|)^{n-1}}u(0) \leq u(x) \leq \dfrac{r^{n-2}(r+|x|)}{(r-|x|)^{n-1}}u(0)$
\end{customproblem}

\begin{customproblem}\textbf{2.5.7}
Since $K(x,y)\in C^{\infty}(B^{0}(0,1))$, thus $$u(x):=\int_{\partial B(0,r)}K(x,y)g(y)dS(y)\in C^{\infty}(B^{0}(0,1)) $$
Also, $$\Delta u(x) = \int_{\partial B(0,r)}\Delta_{x}K(x,y)g(y)dS(y) = 0 $$Since $\Delta_{x}K(x,y)=0$, when $y\in \partial B(0,r)$ \par
By taking $u\equiv 1$ we get $\displaystyle{\int_{\partial B(0,r)}K(x,y)dS(y)=1}$ \par
Since $g\in C(\partial B(0,r))$, $|g|\leq M$ for some $M>0$, and $\forall \epsilon > 0, \exists \delta > 0$, such that $\left|g(y)-g(x^{0})\right|\leq\epsilon, \forall y\in \partial B(0,r)\cap B(x^{0},\delta)$, hence, when $|x-x^{0}|<\dfrac{\delta}{2}$, we have $|x-y|\geq|x^{0}-y|-|x-x^{0}|>\dfrac{\delta}{2}, \forall y\in\partial B(0,r)- B(x^{0},\delta) $, hence
\[
\begin{aligned}
\left|u(x)-g(x^{0})\right|
&=\left|\int_{\partial B(0,r)}K(x,y)\left(g(y)-g(x^{0})\right)dS(y)\right| \\
&\leq \left|\int_{\partial B(0,r)\cap B(x^{0},\delta)}K(x,y)\left(g(y)-g(x^{0})\right)dS(y)\right| \\
&\quad + \left|\int_{\partial B(0,r)- B(x^{0},\delta)}K(x,y)\left(g(y)-g(x^{0})\right)dS(y)\right| \\
&\leq \epsilon\int_{\partial B(0,r)\cap B(x^{0},\delta)}K(x,y)dS(y) + 2M\int_{\partial B(0,r)- B(x^{0},\delta)}K(x,y)dS(y) \\
&\leq \epsilon\int_{\partial B(0,r)}K(x,y)dS(y) + \dfrac{2M\left(r^{2}-|x|^{2}\right)}{|\partial B(0,1)|r}\int_{\partial B(0,r)}\dfrac{2^{n}}{\delta^{n}} dS(y)  \\
&= \epsilon + \dfrac{2^{n+1}Mr^{n-2}\left(r^{2}-|x|^{2}\right)}{\delta^{n}}
\end{aligned}
\]Thus $\displaystyle{\underset{x\in B^{0}(0,r)}{\underset{x\rightarrow x^{0}}{\lim}}\,u(x)=g(x^{0})}$
\end{customproblem}

\begin{customproblem}\textbf{1}[January 2010]
\textbf{a)} \par
Suppose $u,u_{1}$ are two bounded solution, then $\forall \epsilon>0, \exists R>0$, such that $\forall r > R$
$$u(x)-\epsilon\ln|x| \leq u_{1}(x) \leq u(x)+\epsilon\ln|x|$$
on $\partial B(0,r)$, using maximum principle for $u(x)-\epsilon\ln|x|-u_{1}(x)$ and $u_{1}(x)-u(x)-\epsilon\ln|x|$ on $B(0,r)-B^{0}(0,1)$ \par
Thus the inequality above holds for any $\epsilon>0$ and $|x|>1$, let $\epsilon\rightarrow 0$, we have $u=u_{1}$ \par
\textbf{b)} \par
$u\equiv 1$ and $u(x)=\dfrac{1}{|x|}$ are both bounded solutions with $f\equiv 1$ \par
One additional condition that ascertain the uniqueness of the solution could be $\underset{x\rightarrow \infty}{\lim} u(x) = 0$, in this case we would have $$u(x)-\epsilon \leq u_{1}(x) \leq u(x)+\epsilon$$ On $\partial B(0,r)$ for sufficiently large $r$
\end{customproblem}

\begin{customproblem}\textbf{1}[January 2005]
\textbf{Lemma:} If $\displaystyle{g=\sum_{k=0}^{\infty}a_{k}z^{k}}$ is a holomorphic fuction on $\mathbb{C}$, then
\[
\int\int_{B(0,R)}|f'(z)|^{2}dxdy=\sum_{k=0}^{\infty}|a_{k}|^{2}\dfrac{\pi R^{2k+2}}{k+1}
\]
\textbf{Proof:} Using the fact that $$\int\int_{B(0,R)}z^{k}\overline{z}^{l}dxdy=
\left\{\begin{matrix}
\dfrac{\pi R^{2k+2}}{k+1}, \,k=l \\ 
0, \qquad\quad k\neq l
\end{matrix}\right.
$$
\textbf{Method 1:} \par
Since $u$ is harmonic on $\mathbb{R}^{2}$, there exists a holomorphic function $f$ on $\mathbb{C}$, such that $\mathrm{Re}f=u$, then $|\nabla u|=|f'|$, hence we have $\displaystyle{
\int\int_{\mathbb{R}^{2}}|\nabla u|^{2}dxdy=\int\int_{\mathbb{C}}|f'(z)|^{2}dxdy < \infty
}$, according to the lemma above, we have $f'\equiv 0$, hence $f$ is a constant, so is $u$ \par
\textbf{Method 2:} \par
According to \textbf{2.5.4 d)}, we know that $|\nabla u|^{2}$ is subharmonic, thus 
$$|\nabla u(x)|^{2} \leq \dfrac{1}{|B(x,r)|}\int_{B(x,r)}|\nabla u(y)|^{2}dy \leq\dfrac{1}{|B(x,r)|}\int_{\mathbb{R}^{2}}|\nabla u(y)|^{2}dy$$Which tends to $0$ as $r$ tends  to $0$, thus $\nabla u \equiv 0$, $u$ is a constant
\end{customproblem}

\begin{customproblem}\textbf{1}[August 2004]
Assume $f\not\equiv  0$, since $f\in C^{1}(\mathbb{R}^{n})$, there is some $B(x,\epsilon)$, such that $f$ is positive(negative) on it, consider $v\equiv 1$, then we have$$
\begin{aligned}
0 &=\int_{\partial B(x,\epsilon)}v\dfrac{\partial u}{\partial n} dS = \int_{B(x,\epsilon)}v\Delta udx + \int_{B(x,\epsilon)}\nabla v\cdot\nabla udx \\
&= \int_{B(x,\epsilon)}\nabla v\cdot\nabla udx - \int_{B(x,\epsilon)}vfdx \\
&= -\int_{B(x,\epsilon)}fdx <(>) 0
\end{aligned}
$$ Which is a contradiction, Thus $f\equiv 0$
\end{customproblem}

\end{document}