\documentclass[main]{subfiles}

\begin{document}

\section{In troduction to higher K theory}

Now that we've seen an \textbf{ad hoc} construction of $K_0,K_1,K_2$, and negative K-groups, it is time to study how one could get all the K-groups at once, so that they fit together as that they fit together as a unit. Quillen's idea, which won him the Fields medal in 1972, was to construct not a sequence of groups, but a \textbf{topological space}, or actually a \textbf{spectrum}, so that $K_i$ is just the $i$-homotopy group of this space. In other words, out of a ring $R$ (or later, an abelian category or exact category or Waldhausen category) we want a space $K(R)$ so that $K_i(R)=\pi_i(K(R))$. Of course, certain key conditinos have to be satisfied:
\begin{enumerate}[label=\alph*)]
\item The association $R\mapsto K(R)$ should be functorial.
\item Given an ideal $I\subseteq R$, we should get a fibration $K(R,I)\to K(R)\to K(R/I)$, so that the induced long exact sequence of homotopy groups has the form
\[
\cdots\to K_i(R,I)\to K_i(R)\to K_i(R/I)\xrightarrow{\partial}K_{i-1}(R,I)\to\cdots
\]
\item The groups $K_0,K_1,K_2$ should agree with the ones we've already defined. To motivate the simplest construction, the Quillen $+$-construction, recall that topological K-theory $K^{\text{top}}$ is given by homotopy classes of maps $X\to K_0(\mathbb C)\times B\GL(\mathbb C)^{\text{top}}$
\end{enumerate}

\begin{theorem}
\[K_i(F_q)=
\begin{cases}
Z,i=0\\
cyclic of order q^k-1, i=2k-1\\
0, i even \geq2
\end{cases}
\]
More precisely, $K(F_q)$ is homotopy equivalent to the homotopy fiber of the map $BU\xrightarrow{\psi^q-1}BU$, where $\psi^q$ is the Adams operation
\end{theorem}

Idea: Given a matrix with entries in $\bar F_q$, which is roughly analogues to $\mathbb C$, it has entries in $F_q$ iff it is fixed under Frobenius map. Once you make this precise, $K_i(F_q)$ falls out from the homotopy sequence since $\pi_i(BU)=\begin{cases}
Z, i even\\
0, i odd
\end{cases}$, and $\psi^q-1$ on $\pi_{2k}$ acts by $q^k$, and then consider the long exact sequence of this fibration

\subsection{Motivation}

We've talked a lot about defining K-theory. But how do you actually compute it in practice? For motivation, it helps to look at the easier case of topological K-theory. There we had characteristic classes which provide a lot of information. We also had the splitting principle, which, you may recall, says that "one can pretend every vector bundle is a direct sum of line bundles." Using these, one can define the Chern character $\Ch:K^0(X)\to\oplus_iH^{2i}(X,\mathbb Q)$ for $X$ a finite CW-complex by $\Ch(L_1\oplus\cdots\oplus L_n)=\sum_je^{c_1(L_j)}$, for $L_1,\cdots,L_n$ line bundles with characteristic classes $c_1(L_j)\in H^2(X,\mathbb Z)$ and with
\[
e^{c_1(L_j)}=1+c_1(L_j)+\frac{1}{2}c_1(L_j)^2+\cdots
\]
The sum is finite since $c_1(L_j)$ is nilpotent. The splitting principle and naturality imply that $\Ch$ is well-defined for all vector bundles, and in fact for all formal differences of vector bundles, since we can define $\Ch(E-F)=\Ch(E)-\Ch(F)$

\begin{theorem}
$\Ch$ is a ring homomorphism $K^0(X)\to H^{\text{even}}(X,\mathbb Q)$ and its kernel is a finite nilpotent ideal. Furthermore, by Bott periodicity, it extends to a homomorphism of graded rings $K^*(X)\to H^*(X,\mathbb Q)$, with finite kernel. It's an isomorphism after tensoring $K^*(X)$ with $\mathbb Q$. So cohomology almost completely computes topological K-theory. And there is a spectral sequence $H^p(X,K^q(\text{pt})\Rightarrow K^{p+q}(X)$. This suggests that having the right cohomology theories should give most of the information about algebraic K-theory as well, and indeed this is the case. There are two main theories that apply:
\begin{itemize}
\item Cyclic homology and its variants, especially topological cyclic homology TC
\item Motivic (co)homology for schemes
\end{itemize}
\end{theorem}

\subsection{Motivic cohomology}

For $X$ a smooth quasi-projective scheme over a field $k$, there are complexes of sheaves $\mathbb Z(j),j\geq0$ over $X$ and the motivic cohomology of $X$ is $H^{n,j}(X,\mathbb Z)$ is defined to be $H^n(X,\mathbb Z(j))$. Note the bigrading. These groups are only nonzero for $j\geq0$, and $\mathbb Z(0)=\mathbb Z$, $\mathbb Z(1)=\mathbb G_m[-1]$. They are related to the usual Chow groups by $\CH^i(X)=H^{2i}(X,\mathbb Z(i))$. There is a spectral sequence (for $q$ even)
\[
E^{p,q}_2=H^p(X,\mathbb Z(-q/2))\Rightarrow K_{-p-q}(X)
\]
For $X=\Spec k$, $H^{j,j}(X)=K^M_j(k)$, the Milnor K-theory ($\mathbb Z$ for $j=0$ and generated by $k^\times$ with the Steinberg relations for $j\geq1$). So all of this gives considerable information about K-theory of fields. There is also the Suslin-Nestorenko Theorem: $K^M_n(k)\to K_n(k)$ followed by the Chern class $c_{n,n}:K_n(k)\to H^{n,n}(\Spec k,\mathbb Z)=K_n^M(k)$ is multiplication by $(-1)^{n-1}(n-1)!$, and in particular, is an injection modulo torsion of order involving only primes $\leq n-1$.

Motivic cohomology thus turns out to be a very powerful tool for computing K-theory of schemes

Some results proved with this technology: Milnor conjecture, proved by Voevodsky, Bloch-Kato conjecture, proved by Voevodsky, Rost, et al. Information on Quillen-Lichtenbaum conjecture

\end{document}