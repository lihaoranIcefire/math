\documentclass[a4paper,10pt]{article}
\usepackage{My_math_package}

\begin{document}

\begin{definition}
A \textit{discrete valuation} on a field $F$ is $v:F\to\mathbb Z\cup\{\infty\}$ satisfying
\begin{enumerate}
\item $v(x)=\infty\iff x=0$
\item $v(x+y)\geq\min(v(x),v(y))$, equality holds iff $v(x)\neq v(y)$
\item $v(xy)=v(x)+v(y)$
\end{enumerate}
A valuation is when replacing $\mathbb Z$ by some totally ordered abelian group(called the value group). $v$ is trivial if $v(F^\times)=0$
\end{definition}

\begin{definition}
A \textit{discrete valuation ring} is an integral domain $R$ with a discretely valued fraction field $F=\Frac(R)$ such that $R=\{x\in F|v(x)\geq0\}$. In this case $m=\{x\in F|v(x)>0\}$ is the unique prime ideal in $R$, and $m=(\pi)$, we call $\pi$ the \textit{uniformizer}, and for $x\in F$, $v(x)$ is the maximal power of $\pi$ in $a$.
\end{definition}

\begin{definition}
Pick some $0<c<1$, we can define norm $|\cdot|:F\to\mathbb R_{\geq0}$ as $|x|=c^{v(x)}$, this defines a metric topology(\textcolor{blue}{which can be shown to be independent of the choice of $c$})
\end{definition}

\begin{example}
$\mathbb Z_{(p)}\subseteq \mathbb Q$ with the $p$-adic valuation, $c=p^{-1}$
\end{example}

\begin{definition}
By completing the metric topology, we obtain a complete discretely valued field $\widehat F$ with extended norm $|\cdot|$, and the closure of $R$ in $F$ is denoted by $\widehat R$ which can be proven is $\widehat R=\varprojlim R/m^n$
\end{definition}

\begin{example}
$k[t]_{(t)}\subseteq k(t)$ with $t$-adic valuation, $\widehat{k(t)}=k((t))$ is the field of formal Laurent series, $\widehat{k[t]_{(t)}}=k[[t]]$ is the ring of formal power series
\end{example}

\begin{definition}
Consider finite separable extension $E/F$, then $x\in E$, $|x|_E=|N_{E/F}(x)|_F^{\frac{1}{|[E:F]|}}$. If $F$ was complete wrt $|\cdot|_F$, then $E$ is complete wrt $|\cdot|_E$ and has a valuation $w:E^\times\to\mathbb Z$ satisfying
\begin{enumerate}
\item The valuation ring of $w$ is $S$ the integral closure of $R$ in $E$
\item $\exists e$, such that $\forall x\in R$, $w(x)=ev(x)$, and $[E:F]=ef$, where $f=[S/m':R/m]$
\end{enumerate}
\end{definition}

Our setting: If $F$ is complete, $E/F$ finite and separable is said to be unramified if $f=[E:F]$ and $(S/m')/(R/m)$ is separable

\begin{theorem}
Let $N_F$ be the category with objects unramified extensions $E/F$ and morphisms are $E\to E'$ over $F$. $R_F$ be the category of finite separable extension $R/m$ with $R/m$-algebra morphism. Consider the restion functor $\res:N_F\to R_F$, $E/F\rightsquigarrow (S/m')/(R/m)$, and $(E_1\to E_2)\rightsquigarrow(S_1/m_1'\xrightarrow{f^r}S_2/m_2')$ where $f^r$ is defined as follows: for $x\in S_1/m_1'$, let $y\in S_1$ be a lift, and let $f^r(x)$ be the image of $f(y)$ in $S_2/m_2'$. This functor is an equivalence of categories. So $\Hom_F(E,E)=\Hom_{R/m}(S/m',S/m')$ have isomorphic Galois groups
\end{theorem}

\begin{example}
$F=\mathbb Q_p$, $E/F$ unramified with $[E:F]=n$, then $E=F(\xi_{p^n-1})$, and $\Gal(E/F)$ is cyclic
\end{example}

Now assume $k$ is infinite, $v:k^\times\to\mathbb Z$ is a valuation, $k$ is complete with respect to associated absolute values. $R$ is the valuation ring.

\begin{lemma}
$D/k$ is a central division algebra with $\deg D=d$, then $\exists$ valution $w:D^\times\mathbb Z$. And $w$ is uniquely determined by restricting on any field $F\subseteq D$ containing $k$ by $w(x)=v_F(x)=\frac{1}{[F:k]}v(N_{F/k}(x))$. Moreover $w(x)=\frac{1}{d}v(\Nrd(x))$ and this valuation gives $D$ the structure of a normed $k$-vector space
\end{lemma}

\begin{proof}
For a fixed $F\subseteq D$, $N_{D/k}(x)=\Nrd(x)^d$, so $D$ viewed as a left $F$-vector space is $D\cong F^{\oplus m}$ where $m=\frac{d^2}{[F:k]}$. since for $x\in F$, $\Nrd(x)^d=N_{D/k}(x)=N_{F/k}(x)^m$, $mv(N_{F/k}(x))=dv(\Nrd(x))\Rightarrow\frac{1}{[F:k]}v(N_{F/k}(x))=\frac{1}{d}v(\Nrd(x))=w(x)$. Now let $x,y\in D^\times$ such that $x+y\in D^\times$, $F=k(x^{-1}y)\subseteq D$, we know $w$ is a group homomorphism(since $\Nrd$ is a group homomorphism), we have
\[
w(1+x^{-1}y)=v_F(1+x^-1{y})\geq\min(v_F(1),v_F(x^{-1}y))=\min(v_F(1),v_F(x^{-1}y))\Rightarrow w(x+y)\geq\min(w(x),w(y))
\]
So $w$ is a valuation
\end{proof}

\begin{theorem}\label{15:20-04/20/2022}
If $R/m$ is a perfect field, then any CSA/k is split by an unramified extension of $k$
\end{theorem}

\begin{proof}
We will show that for any $L/k$ unramified, any $A\in\text{CSA}/k$ is split by another unramified $E/L$. We prove this by induction on the index. If $A/L$ has index one, then we can just take $E=L$. Now suppose the statement holds for all $L/k$ unramified, $\ind_L(A)\leq d$, by Lemma~\ref{17:21-4/19/2022}, $A$ contains an unramified $E/L$ with $[E:L]>1$, then $\ind_E(A)$ becomes less
\end{proof}

\begin{lemma}\label{17:21-4/19/2022}
If $R/m$ is perfect, $L/k$ is a unramified and $D$ is a division algebra over $L$ with $\ind_L(D)>1$, then $D$ contains an unramified $E/L$ with $[E:L]>1$
\end{lemma}

\begin{proof}
Suppose not, then for every field $L\subseteq F\subseteq D$, the residue field extension $(S/m')/(R/m)$ ($S\subseteq F$ valution ring with $m'\subseteq S$ maximal ideal) is trivial. Pick $x\in D^\times$, $\pi\in D^\times$ such that $w(\pi)=1$. By induction we can write $y=a_0+a_1\pi+a_2\pi^2+\cdots+a_{n-1}\pi^{n-1}+b_n\pi^n$ which is a Cauchy sequence, so $y\in R(\pi)$
\end{proof}

By Theorem~\ref{15:20-04/20/2022}, we know
\[
\Br(k)=\varinjlim_{E/k\text{ (finite) unramified}}\Br(E/k)=\varinjlim_{E/k\text{ (finite) unramified}}H^2(\Gal(E/k),E^\times)
\]
And we have a split short exact sequence of $G=\Gal(E/k)$-modules
\begin{center}
\begin{tikzcd}
1 \arrow[r] & S^\times \arrow[r] & E^\times \arrow[r, "v"] & \mathbb Z \arrow[r] \arrow[l, "1\mapsto\pi", bend left] & 0
\end{tikzcd}
\end{center}
So there is a split exact sequence for all $i\geq0$
\[
0\to H^i(G,S^\times)\to H^i(G,E^\times)\to H^i(G,\mathbb Z)\to0
\]

\begin{proposition}
$H^i(G,S^\times)=0$ for all $i\geq1$
\end{proposition}

\begin{proof}
Since $G$ is cyclic, for all $j\geq1$, $H^{2j-1}(G,S^\times)=H^1(G,S^\times)=1$ by Hilbert 90. $H^{2j}(G,S^\times)=H^{2}(G,S^\times)=(S^\times)^G/N_G(S^\times)$, here recall the norm map $N_G(x)=\prod_{g\in G}gx$. First consider commutative diagram
\begin{center}
\begin{tikzcd}
S^\times \arrow[r, two heads] \arrow[d, "N_G", two heads] & (S/m')^\times \arrow[d, "N_G", two heads] \\
R^\times=(S^\times)^G \arrow[r, two heads]                & (R/m)^\times                             
\end{tikzcd}
\end{center}
The right map is surjective since $\Gal((S/m')(R/m))$ induce action of $(S/m')^\times$, we know $H^2(G,(S,m')^\times)=(S/m')^\times/N_G(S/m')^\times$, we want to show that the left map is surjective. For $x\in R^\times$, we can find $z_1\in S^\times$ such that $N_G(z_1)$ has the same image as $x$ in $(R/m)^\times$, so $\frac{x}{N_G(z_1)}=1+u\pi$ for some $u\in R$
Define
\[
U_E^{(j)}=1+m'^j,\quad U_k^{(j)}=1+m^j
\]
Consider commutative diagram
\begin{center}
\begin{tikzcd}
U_E^{(j)} \arrow[r] \arrow[d, "N_G", two heads] & S/m' \arrow[d, "N_G", two heads] \\
U_k^{(j)} \arrow[r]                             & R/m                             
\end{tikzcd}
\end{center}
The right map(called the trace map $x\mapsto \sum_{g\in G}gx$) is surjective by lemma~\ref{15:56-01/20/2022}, so $\frac{1}{N_G(z_j)}\frac{x}{N_G(z_1\cdots z_{j-1})}=\frac{x}{N_G(z_1\cdots z_j)}=1+u_j\pi^j$. Note that $\{\prod_{i=1}^nz_i\}$ is Cauchy in $E^\times$, if we take $z$ to be the limit, then $\frac{x}{N_G(z)}=1$
\end{proof}

\begin{lemma}\label{15:56-01/20/2022}
$H^i(G,S/m')=0$ for all $i\geq1$
\end{lemma}

\begin{proof}
Suppose $R/m=\mathbb F_q$, $S/m'=\mathbb F_{q^n}$ for some $n$. Consider $\mathbb F_{q^n}$ as an abelian group, since $G$ is cyclic of order $n$, $\mathbb F_{q^n}=\frac{\mathbb F_q[t]}{(t^n-1)}=\mathbb F_qG\cong M^G_{\{e\}}(\mathbb F_q)\cong\Hom_{\text{Ab}}(\mathbb ZG,\mathbb F_q)$, then apply Shapiro's lemma~\ref{Shapiro's lemma}.
\end{proof}

\begin{lemma}
The short exact sequence of $\mathbb ZG$ modules(as trivial modules)
\[
0\to \mathbb Z\to\mathbb Q\to\mathbb Q/\mathbb Z
\]
Induces isomorphisms $H^{i-1}(G,\mathbb Q/\mathbb Z)\xrightarrow{\delta^i}H^i(G,\mathbb Z)$
\end{lemma}

\begin{proof}
We have $H^^i(G,\mathbb Q)\xrightarrow{\res}H^i(\{e\},\mathbb Q)=0\xrightarrow{\cor}H^i(G,\mathbb Q)$ for $i\geq1$ is multiplying by $|G|$ which is an isomorphism, hence $H^i(G,\mathbb Q)=0$
\end{proof}

\begin{corollary}
There are isomorphisms
\begin{center}
\begin{tikzcd}
{H^2(G,E^\times)} \arrow[r, "\cong", shift left] & {H^2(G,\mathbb Z^\times)} \arrow[l, "\cong", shift left] & {H^1(G,\mathbb Q/\mathbb Z)} \arrow[l, "\cong"]
\end{tikzcd}
\end{center}
\end{corollary}

\begin{theorem}
If $f$ is a 1-cocycle, then $f(gh)=f(g)+g(f(h))=f(g)+f(h)$ is a group homomorphism, so as $\mathbb ZG$ modules, $H^1(G,\mathbb Q/\mathbb Z)=\Hom_{\text{Ab}}(G,\mathbb Q/\mathbb Z)$. Let $H=\ker f$, $F=E^H$, $\chi=\bar f:G/H\to\frac{1}{[F;k]}\mathbb Z/\mathbb Z$, so it can be identified with
\[
\{(F/k,\chi),k\leq F\leq E\text{ cyclic Galois, and }\chi:\frac{\Gal(E/k)}{\Gal(F/k)}\xrightarrow{\cong}\frac{1}{[F:k]}\mathbb Z/\mathbb Z\subseteq\mathbb Q/\mathbb Z\}
\]
And in our case, we have
\begin{center}
\begin{tikzcd}
{\frac{1}{[F:k]}\mathbb Z/\mathbb Z} \arrow[r, "\cong"] \arrow[rrrd, "{1\mapsto[(\chi,\pi)]}"'] & {\Hom(G,\mathbb Q/\mathbb Z)} \arrow[r, "\cong"] & {H^1(G,\mathbb Q/\mathbb Z)} \arrow[r, "\cong"] & {H^2(G,\mathbb Z)} \arrow[r, "\cong"] & {H^2(G,E^\times)} \arrow[d, "{[\alpha]\mapsto[(E/k,\alpha)]}"] \arrow[ld, "\delta^2"] \\
                                                                                                &                                                  &                                                 & \Br(E/k)                              & \Br(E/k) \arrow[l, "{[A]\mapsto[A^{\text{op}}]}"]                                    
\end{tikzcd}
\end{center}
If $L/E/k$ unramified, $G'=\Gal(L/k)$ and $\chi:G'\to\frac{1}{[L:k]}\mathbb Z/\mathbb Z$ such that $\overline{\chi'}:\frac{\Gal(L/k)}{\Gal(E/L)}\xrightarrow{\cong}\frac{1}{[E:k]}\mathbb Z/\mathbb Z$ is canonically $\chi$, then
\begin{center}
\begin{tikzcd}
{\frac{1}{[E:k]}\mathbb Z/\mathbb Z} \arrow[r, "\cong"] \arrow[d, hook] & \Br(E/k) \arrow[d, hook] \\
{\frac{1}{[L:k]}\mathbb Z/\mathbb Z} \arrow[r, "\cong"]                 & \Br(E/k)                
\end{tikzcd}
\end{center}
Hence $\Br(k)=\cong\varinjlim\Br(E/k)\cong\mathbb Q/\mathbb Z$
\end{theorem}

\begin{corollary}
If $A$ is a CSA/$k$, then $\ind(A)=\exp(A)$
\end{corollary}

\begin{proof}
Since $\Br(k)=\mathbb Q/\mathbb Z$, the algebra $[A]$ generates a subgroup of size $\exp(A)=d$, so $[A]$ is the image of $\frac{1}{d}\cong H^2(\Gal(E_d/k),E_d^\times)$, where $E_d$ is the unique unramified extension of degree $d$, so $d=\exp(A)|\ind(A)|d\Rightarrow\exp(A)=\ind(A)=d$
\end{proof}

\begin{corollary}
Every CSA/$k$ is a cyclic
\end{corollary}

\begin{proof}
If $A$ is a CSA/$k$, $\deg(A)=n$, and $\ind(A)=d$, then $d=\ind(A)|\deg(A)=n$, so $A$ is split by $E_d$, Hence $A$ is split by $E_d\subseteq E_n$, so $E_n\subseteq A$ is the maximal subfield of $A$.
\end{proof}

\begin{example}
$k=\mathbb F_3((t))$, $E=\mathbb F_3((t))[x]/(x^3-x-1)\not\cong\mathbb F_{27}((t))$ \textcolor{blue}{which is actually a infinite degree extension}. $A=[1,t)_{\mathbb F_3((t))}$ has non zero class in $\Br(k)\Rightarrow\Br(\mathbb F_3((t)))\neq0$, $A'=[1,t)_{\mathbb F_3(t)}$, $A'\otimes_{\mathbb F_3(t)}\mathbb F((t))\cong A$
\end{example}

\paragraph{Setting:}
\begin{itemize}
\item $k$ is a field.
\item $C/k$ is a smooth geometrically connected, projective curve.
\item $C_F=C\times_kF$ for any field extension $F/k$, denote $\pi:C_F\to C$.
\item $E/k$ is a finite Galois extension with $G=\Gal(E/k)$. $\forall\sigma\in G$
\begin{center}
\begin{tikzcd}
C_E \arrow[r, "\id\times\sigma"] \arrow[rd, "\pi"] & C_E \arrow[d, "\pi"] \\
                                       & C                   
\end{tikzcd}
\end{center}
defines a $G$-action on $C_E$.
\item $E(C_E)=k(C)\otimes_kE$ is the function field of $C_E$.
\item $\Div(C_E)=\bigoplus_{x\in C_E^{(1)}}\mathbb Zx$ is the group of divisors of $C_E$.
\item $\Pic(C_E)$ is the Picard group of $C_E$.
\end{itemize}

We have the following short exact sequence
\begin{equation}\label{14:08-04/27/2022}
1\to E^\times\to E(C_E)^\times\xrightarrow{\operatorname{div}}\Div(C_E)\xrightarrow{D\mapsto\mathcal O(D)}\Pic(C_E)\to0
\end{equation}


\begin{lemma}
There is an isomorphism of $\mathbb ZG$-modules
\[
\Div(C_E)=\bigoplus_{x\in C^{(1)}}\left(\bigoplus_{y\in\pi^{-1}(x)}\mathbb Zy\right)\cong\bigoplus_{x\in C^{(1)}}M^G_{G_y}(\mathbb Z)
\]
Here $G_y$ is the stabilizer of $y$
\end{lemma}

\begin{proof}
Look at the fiber over $x$, we may identify $\mathbb ZG$-modules
\[G/G_y=\{\sigma_1,\cdots,\sigma_r\}\rightarrow \pi^{-1}(x)=\{y_1,\cdots,y_r\},\quad \sigma_i\mapsto\sigma_iy\]
 then we can define isomorphism
\begin{align}
M^G_{G_y}(\mathbb Z)=\Hom(\mathbb ZG,\mathbb Z)\to\bigoplus_{i=1}^r\mathbb Zy_i,\quad f\mapsto\sum_{i=1}^rf(\sigma_i)y_i
\end{align}
\end{proof}

\begin{example}
\begin{align*}
\pi^{-1}(x)&=C_E\times_C\Spec k(x)=E\times_kC\times_C\Spec k(x)=E\times_k\Spec k(x)\\
&=\Spec(k(x)\otimes_kE)=\Spec\left(k(x)\otimes_k\frac{k[t]}{(f(t))}\right)\cong\Spec\frac{k(x)[t]}{f(t)}\\
&=\Spec\left(\prod_{i=1}^rk(x)[t]/(h_i(t))\right)=\coprod_{i=1}^r\Spec\frac{k(x)[t]}{(h_i(t))}
\end{align*}
\end{example}

\begin{example}
$k=\mathbb Q$, $E=\mathbb Q(i,\sqrt[4]{2})$, $k(x)\cong\mathbb Q(\sqrt2)$
\[
\Gal(E/k)=\langle\sigma,\tau|\tau^2=e,\sigma^4=e,\tau\sigma\tau=\sigma^{-1}\rangle,\begin{cases}
\sigma(\sqrt[4]{2}=i\sqrt[4]{2})\\
\sigma(i)=i
\end{cases},\begin{cases}
\tau(\sqrt[4]{2}=\sqrt[4]{2})\\
\tau(i)=-i
\end{cases}
\]
\[
E=\mathbb Q[x]/(x^8-4x^6+8x^4-4x^2+1)
\]
\[
k(x)\otimes_kE=\mathbb Q(\sqrt2)[x]/(x^8-4x^6+8x^4-4x^2+1)=\frac{\mathbb Q(\sqrt2)[x]}{(h_1(x))}\times\frac{\mathbb Q(\sqrt2)[x]}{(h_2(x))}
\]
Here $h_1(x)=x^4+(-\sqrt2-2)x^2-2\sqrt2+3$, $h_2(x)=x^4+(\sqrt2-2)x^2+2\sqrt2+3$. The stabilizer is $H=\langle e,\sigma^2,\tau,\sigma^2\tau\rangle$
\end{example}

\begin{definition}
The \textit{residue map} $r^i_x(y)$ is defined to be
\begin{align*}
H^i(G,E(C_E)^\times)&\to H^i(G,\Div(C_E))\cong\bigoplus_{x\in C^{(1)}}H^i(G,M^G_{G_y}(\mathbb Z))\cong\bigoplus_{x\in C^{(1)}}H^i(G_y,\mathbb Z)\\
&\cong\bigoplus_{x\in C^{(1)}}H^{i-1}(G_y,\mathbb Q/\mathbb Z)\to H^{i-1}(G_y,\mathbb Q/\mathbb Z)
\end{align*}
\end{definition}

\begin{lemma}
For any $y,z\in\pi^{-1}(x)$, the residue maps $r_x^i(y),r_x^i(z)$ differ by a canonical isomorphism
\end{lemma}

\begin{proof}
There exists $\sigma\in G$ such that $\sigma y=z$, and $G_z=\sigma G_y\sigma^{-1}$
\end{proof}

\begin{lemma}
The sequence
\begin{center}\label{13:06-04/29/2022}
\begin{tikzcd}
{H^i(G,E(E_E)^\times)} \arrow[rrr, "\bigoplus_{x\in C^{(1)}}r^i_x(y)"] &  &  & {\bigoplus_{x\in C^{(1)}}H^{i-1}(G_y,\mathbb Q/\mathbb Z)} \arrow[rrr, "\sum_{x\in C^{(1)}}d_y\cor^{G_y}_G"] &  &  & {H^{i-1}(G,\mathbb Q/\mathbb Z)}
\end{tikzcd}
\end{center}
is a complex for $i\geq2$, here $d_y=[E(y):E]$ is the degree of $y$
\end{lemma}

\begin{proof}
We can split \eqref{14:08-04/27/2022} into two exact sequences
\begin{equation}\label{00:30-04/48/2022}
1\to E^\times\to E(C_E)^\times\to E(C_E)^\times/E^\times\to1
\end{equation}
\begin{equation}\label{00:31-04/48/2022}
1\to E(C_E)^\times/E^\times\to\Div(C_E)\to\Pic(C_E)\to0
\end{equation}
Hence we get commutative diagram with the help of Lemma~\ref{08:59-04/28/2022}
\begin{center}
\begin{tikzcd}
{H^{i}(G,E(C_E)^\times/E^\times)} \arrow[r] & {H^i(G,\Div(C_E))} \arrow[r] \arrow[d,"\cong"]                 & {H^i(G,\Pic(C_E))} \arrow[d, "\cong"]                  \\
{H^i(G,E(C_E))^\times} \arrow[u] \arrow[r]  & {\bigoplus_{x\in C^{(1)}}H^i(G_y,\mathbb Z)} \arrow[r] & {H^{i-1}(G,\mathbb Q/\mathbb Z)\cong H^i(G,\mathbb Z)}
\end{tikzcd}
\end{center}
Looking at a single summand, the map reads
\[
H^i(G,\Div(C_E))\to H^i(G,M^G_{G_y}(\mathbb Z))\to H^{i}(G_y,\mathbb Z)\to H^i(G,M^G_{G_y}(\mathbb Z))\xrightarrow{\phi_*}H^i(G,\mathbb Z)
\]
Recall $\phi:M^G_{G_y}(\mathbb Z)=\Hom_G(\mathbb Z[G],\mathbb Z)\to \mathbb Z$ is defined via $\phi(f)=\sum_{s\in G/G_y}sf(s^{-1}e)$
\end{proof}

\begin{lemma}\label{08:59-04/28/2022}
There is a commutative exact ladder
\begin{center}
\begin{tikzcd}
0 \arrow[r] & \Div^0(C_E) \arrow[r] \arrow[d] & \Div(C_E) \arrow[r, "\deg"] \arrow[d] & \mathbb Z \arrow[r] \arrow[d,equal] & \mathbb Z/d\mathbb Z \arrow[r] \arrow[d,equal] & 0 \\
0 \arrow[r] & \Pic^0(C_E) \arrow[r]           & \Pic(C_E) \arrow[r]                   & \mathbb Z \arrow[r]           & \mathbb Z/d\mathbb Z \arrow[r]           & 0
\end{tikzcd}
\end{center}
Here $\deg\left(\sum n_yy\right)=\sum n_y[E(y):E]$
\end{lemma}

\begin{proof}
Hartshorne chapter II, corollary 6.10
\end{proof}

\begin{example}
$C=\mathbb P_k^1$, $E(C_E)=E(t)^\times$, $G=\Gal(E/k)=\Gal(E(t)/k(t))$, then
\[H^2(G,E(C_E)^\times)\cong H^2(\Gal(E(t)/k(t)),E(t)^\times)\cong\Br(E(t)/k(t))\]
And our complex is
\begin{center}
\begin{tikzcd}
\Br(E(t)/k(t)) \arrow[r] & {\bigoplus_{x\in(\mathbb P_k^1)^{(1)}}H^1(G_y,\mathbb Q/\mathbb Z)} \arrow[r] & {H^1(G,\mathbb Q/\mathbb Z)}
\end{tikzcd}
\end{center}
Let $p(t)\in E(t)^\times$, assume $E/k$ is cyclic with $\chi:\Gal(E/k)\to\mathbb Z/n\mathbb Z$. Then $[(\chi,p(t))^{op}]$ is given by the 2 cocycle $\alpha:G\times G\to E(t)^\times$, $\alpha(\sigma^i,\sigma^j)=1$ if $i+j<n$, $p(t)$ otherwise.

The residue map just takes $\operatorname{div}(\alpha)=0$, $[\infty]-[0]$
\end{example}

\begin{theorem}
Assume $C=\mathbb P_k^1$ such that $H^2(G,E^\times)=H^3(G,E^\times)=0$ (This happens if $k$ is a finite field), then \eqref{13:06-04/29/2022} is exact
\end{theorem}

\begin{proof}
$\Pic(\mathbb P^1_E)\xrightarrow[\cong]{\deg}\mathbb Z$, then we get a splitting in \eqref{00:31-04/48/2022} with $1\mapsto[\infty]$.

so we get
\begin{center}
\begin{tikzcd}
0 \arrow[r] & {H^2(G,E(C_E)^\times/E^\times)} \arrow[r]  & {H^2(G,\Div(C_E))} \arrow[r]                                         & {H^2(G,\mathbb Z)} \arrow[r]                              & 0 \\
0 \arrow[r] & {H^2(G,E(C_E)^\times)} \arrow[u] \arrow[r] & {\bigoplus H^1(G_y,\mathbb Q\mathbb Z)} \arrow[u, "\cong"] \arrow[r] & {H^1(G,\mathbb Q/\mathbb Z)} \arrow[r] \arrow[u, "\cong"] & 0
\end{tikzcd}
\end{center}
\end{proof}

\begin{lemma}
For any curve $C/k$ as the in the set up
\[
H^2(G,E(C_E)^\times)\cong H^2(\Gal(E(C_E)/k(C)),E(C_E)^\times)\cong\Br(E(C_E)/k(C))
\]
Moreover, if $k$ is perfect
\[
\Br(k(C))=\varinjlim_{E/k\text{ finite Galois}}\Br(E(C_E)/k(C))=\varinjlim_{E/k\text{ finite Galois}}H^2(\Gal(E(C_E)/k(C)),E(C_E)^\times)
\]
\end{lemma}

\begin{proof}
The restriction
\[
\Gal(E(C_E)/k(C))\to\Gal(E/k), \varphi\mapsto\varphi|_E
\]
is an isomorphism. This gives the first part. Suppose $k<E<F$ are all finite Galois, then we have inclusion
\[
\Br(E(C_E)/k(C))\to\Br(F(C_F)/k(C))
\]
And all these map to $\Br(k(C))$, surjective since any CSA/$k$, say $A$ is split over $\bar k(C_{\bar k})$ by Tsen's theorem~\ref{}. This proves the moreover part
\end{proof}

\begin{example}
$\Br(\mathbb F_3(t))\neq0$, Take $k=\mathbb F_3$, $E=\mathbb F_{27}$, $C=\mathbb P^1_k$, we get split exact sequence
\begin{center}
\begin{tikzcd}
0 \arrow[r] & \Br(\mathbb F_{27}(t)/\mathbb F_3(t)) \arrow[r] & {\bigoplus H^1(G_y,\mathbb Q/\mathbb Z)} \arrow[r] & {H^1(G,\mathbb Q/\mathbb Z)} \arrow[r] \arrow[l, bend left] & 0
\end{tikzcd}
\end{center}
So $\Br(\mathbb F_{27}(t)/\mathbb F_3(t))\cong\bigoplus_{x\in\mathbb A^1_k}H^1(G_y,\mathbb Q/\mathbb Z)=\bigoplus_{x\in\mathbb A^1_k,y^3-y-1\text{ doesn't split in }k(x)}\mathbb Z/3\mathbb Z$

For every point $x\in \mathbb A^1_k$ with $k(x)=\mathbb F_3$, the local ring $\mathcal O_{\mathbb A^1,x}$ can be completed to $\widehat{\mathcal O_{\mathbb A^1,x}}$ with fraction field $\mathbb F_3((\lambda))$, where $\lambda$ is a uniformizer of $\mathcal O_{\mathbb A^1,x}$, the map $O_{\mathbb A^1,x}\to\widehat{O_{\mathbb A^1,x}}$ induces $F(t)\to F((\lambda))$, giving $\Br(\mathbb F_{27}(t)/\mathbb F_3(t))\to\Br(\mathbb F_{27}((\lambda))/\mathbb F_3((\lambda)))\cong \mathbb Z/3\mathbb Z$
\end{example}

\end{document}