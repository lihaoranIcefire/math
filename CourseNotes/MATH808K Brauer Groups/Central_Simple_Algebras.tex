\documentclass[a4paper,10pt]{article}
\usepackage{My_math_package}

\begin{document}

\begin{definition}
A $k$-algebra $A$ is a \textit{simple} if the only two-sided ideals of $A$ is $0$ and $A$. A $k$\textit{central algebra} is a finite $k$-algebra with $Z(A)=k$
\end{definition}

\begin{example}
$M_n(k)$ as a left module is not simple, for example, consider left ideal  generated by $\begin{pmatrix}1&0\\1&0\end{pmatrix}$
\end{example}

\begin{example}
\begin{enumerate}
\item $M_n(k)$
\item Quaternion algebra $A/k$, $F/k$ is a quadratic field extension such that $A$ splits over $F$, i.e. $A\otimes_kF\cong M_2(F)$, thus $Z(A)\otimes_k F\subseteq Z(A\otimes_kF)=Z(M_2(F))=F$, which implies that $Z(A)=k$. (Similarly, if $I\subseteq A$ is a two-sided ideal, then $I\otimes_kF$ is either $M_2(F)$ or $0$, which implies that $I=A$ or $I=0$ respectively)
\end{enumerate}
\end{example}

\begin{proposition}
$A,B$ are finite $k$-algebras, then $Z(A\otimes_kB)=Z(A)\otimes_kZ(B)$
\end{proposition}

\begin{proposition}
$A$ is $CSA/k$, $B$ finite $k$-alg, then two-sided ideals of $A\otimes_kB$ are of the form $A\otimes_kJ$ where $J=I\cap(1\otimes_kB)$ is a two-sided ideal in $B$
\end{proposition}

\begin{example}
If $D=(a,b)/k$, $D'=(a,b')/k$, then $D\otimes_kD'\cong(a,bb')\otimes_kM_2(k)$
\end{example}

\begin{example}
$D$ is a $k$-central division algebra, then $M_n(k)\otimes_kD\to M_n(D)$ $(a_{ij})\otimes d\mapsto(a_{ij}d)$ is an iso, in particular $M_n(k)\otimes M_m(k)\to M_{nm}(k)$ is the Kronecker product
\end{example}

\begin{theorem}[Wedderburn's theorem]\label{Wedderburn's theorem}
$A$ is a finite simple $k$-alg, then $A\cong M_n(D)$ for some division $k$-alg and some $n$, $D$ is unique up to iso, $n$ is unique. Moreover, if $A$ is $k$-central, so is $D$
\end{theorem}

As a corollary of a Theorem~\ref{Wedderburn's theorem}, $\dim_k(A)=n^2$, so we can define

\begin{definition}
The the \textit{degree} of a simple $k$-algebra is the square root of its dimension
\end{definition}

\begin{definition}
We say the a CSA/$k$ $A$ is split if $A\cong M_n(k)$ for some $n$, we say $A$ is a $F$ split, or split by $F$ for a field extension $F/k$ if $A\otimes_kF\cong M_n(F)$ for some $n$
\end{definition}

The notion of a CSA is really irrelevant with respect to base change, as the following proposition asserts

\begin{proposition}
$A$ is a finite $k$-algebra, $F/k$ field extension. Then $A$ is a CSA/k iff $A\otimes_kF$ is a CSA/F for all $F$(for some $F$)
\end{proposition}

\begin{corollary}
$A$ is a CSA/$k$, then there is a finite field extension $F/k$ where $A$ splits
\end{corollary}

\begin{proof}
Let $\varphi:A\otimes_k\bar k\to M_n(\bar k)$ be an isomorphism, pick a basis $\{a_i\}$ of $A$, let $\varphi(a_i)=\sum c_{ij}e_j$ for $c_{ij}\in\bar k$. Let $F=k(c_{ij})$, hten we can define $\psi:A\otimes_kF\to M_n(F)$, $\psi(a_i)=\sum c_{ij}e_j$, since $\psi\otimes_F\bar k=\varphi$, and since $\bar k$ is a faithfully flat $F$-module, we find $\psi$ is an isomrophism
\end{proof}

\begin{lemma}
If $D$ is a finite division $\bar k$ algebra, then $D\cong \bar k$
\end{lemma}

\begin{theorem}
A finite $k$-algebra $A$ is CSA/k iff $A\otimes_k\bar k\cong M_n(\bar k)$
\end{theorem}

\begin{proof}

\end{proof}

\end{document}