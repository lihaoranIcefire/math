\documentclass[../main.tex]{subfiles}

\begin{document}

\begin{definition}
$C^1$ function $f:\Omega\to\mathbb C$ is \textit{holomorphic} if $\bar\partial f=0$. Denote the set of all holomorphic functions on $\Omega$ as $A(\Omega)$
\end{definition}

\begin{lemma}
If $f$ is holomorphic, then $\displaystyle\int_{\partial \Omega}fdz=0$
\end{lemma}

\begin{proof}
\[\int_{\partial \Omega}fdz=\int_\Omega d(fdz)=\int_\Omega\bar\partial f\wedge dz=0\]
\end{proof}

\begin{theorem}[Poincar\'e-Lelong formula]\label{Poincare-Lelong formula}
Since $\Delta=\partial_x^2+\partial_y^2=4\partial_z\partial_{\bar z}=4\partial_{\bar z}\partial_z$, $dz\wedge d\bar z=-2idx\wedge dy=-2id\mu$. In the distributional sense, $-\dfrac{\log r}{2\pi}=-\dfrac{1}{4\pi}\log(x^2+y^2)$ is the fundamental solution of Laplacian equation in dimension $2$, i.e. $\Delta\log(x^2+y^2)=4\pi\delta$, we have
\[\Delta\log|z|^2dz\wedge d\bar z=4\pi\delta dz\wedge d\bar z\Leftrightarrow\bar\partial\partial\log|z|^2=2\pi i\delta dx\wedge dy\]
\end{theorem}

\begin{note}
$\partial\log|z|^2=\partial\log(z)+\partial\log(\bar z)=\dfrac{dz}{z}$ is integrable around $0$
\end{note}

\begin{proof}
We prove a slightly general result. For any $\phi\in C^\infty_c(\Omega)$, by definition we have
\begin{align*}
\iint_{\Omega}\phi\bar\partial\partial\log|z-w|^2&=-\iint_{\Omega}\bar\partial\phi\wedge\partial\log|z-w|^2 \\
&=-\lim_{\epsilon\to0}\iint_{|z-w|\geq\epsilon}\bar\partial\phi\wedge\partial\log|z-w|^2 \\
&=-\lim_{\epsilon\to0}\iint_{|z-w|\geq\epsilon} d\left(\phi\partial\log|z-w|^2\right) \\
&=\lim_{\epsilon\to0}\oint_{|z-w|=\epsilon} \phi\partial\log|z-w|^2 \\
&=\lim_{\epsilon\to0}\oint_{|z-w|=\epsilon} \frac{\phi}{z-w}dz \\
&=2\pi i\phi(w)
\end{align*}
\end{proof}

\begin{theorem}[Cauchy's formula]\label{Cauchy's formula}
If $f\in C^1(\overline\Omega)$, then
\[f(w)=\frac{1}{2\pi i}\iint_{\Omega}\frac{\partial_{\bar z}f dz\wedge d\bar z}{z-w}+\frac{1}{2\pi i}\int_{\partial \Omega}\frac{f}{z-w}dz\]
\end{theorem}

\begin{proof}
By Poincar\'e-Lelong formula \ref{Poincare-Lelong formula}, we have
\begin{align*}
f(w)&=\frac{1}{2\pi i}\iint_\Omega f\bar\partial\partial\log|z-w|^2 \\
&=-\frac{1}{2\pi i}\iint_\Omega \bar\partial f\wedge\partial\log|z-w|^2 + \frac{1}{2\pi i}\int_{\partial \Omega}f\partial\log|z-w|^2 \\
&=\frac{1}{2\pi i}\iint_{\Omega}\frac{\partial_{\bar z}f dz\wedge d\bar z}{z-w}+\frac{1}{2\pi i}\int_{\partial \Omega}\frac{f}{z-w}dz
\end{align*}
\end{proof}

\begin{corollary}
If $f\in C^1(\overline\Omega)\cap A(\Omega)$, then by Cauchy's formula \ref{Cauchy's formula}, we know
\[f(w)=\frac{1}{2\pi i}\int_{\partial \Omega}\frac{f(z)}{z-w}dz\]
Which is $C^\infty$ in $w$
\[f^{(n)}(w)=\frac{n!}{2\pi i}\int_{\partial \Omega}\frac{f(z)}{(z-w)^{n+1}}dz\]
\end{corollary}

\begin{corollary}[Cauchy's estimate]\label{Cauchy's estimate}
For $K\subseteq\Omega$ compact, there are constants $C_n$ such that for any $f\in A(\Omega)$
\[\sup_{z\in K}|f^{(n)}(z)|\leq C_n\|f\|_{L^1(\Omega)}\]
\end{corollary}

\begin{proof}
Consider a bump function $\chi$ with $\supp\chi\subseteq\Omega$ and $\chi\equiv1$ on $K$, then for any $w\in K$
\begin{align*}
f(w)&=\frac{1}{2\pi i}\iint_{\Omega}\frac{\partial_{\bar z}(\chi f) dz\wedge d\bar z}{z-w}+\frac{1}{2\pi i}\int_{\partial \Omega}\frac{\chi f}{z-w}dz \\
&=\frac{1}{2\pi i}\iint_{\Omega}\frac{(\partial_{\bar z}\chi) f dz\wedge d\bar z}{z-w} \\
&=\frac{1}{2\pi i}\iint_{\Omega\setminus K}\frac{(\partial_{\bar z}\chi) f dz\wedge d\bar z}{z-w}
\end{align*}
$\dfrac{\partial_{\bar z}\chi}{z-w}$ can be bounded on $\Omega\setminus K$
\end{proof}

\begin{corollary}
$A(\Omega)\subseteq C(\Omega)$ is closed, thus a Fr\'echet space
\end{corollary}

\begin{proof}
Suppose $\{f_j\}\subseteq A(\Omega)$ converges to $f$ in $C(\Omega)$, but since
\[f_j(w)=\frac{1}{2\pi i}\int_{\partial \Omega}\frac{f_j(z)}{z-w}dz\]
$f(w)=\displaystyle\dfrac{1}{2\pi i}\int_{\partial \Omega}\frac{f(z)}{z-w}dz$ which implies $\bar\partial f=0$
\end{proof}

\begin{theorem}[Montel's theorem]\label{Montel's theorem}
Suppose $\{f_i\}\subseteq A(\Omega)$ are uniformly bounded on each compact subset, then there is a subsequence $f_{i_k}$ uniformly converges on compact subsets
\end{theorem}

\begin{proof}
For $K\subseteq\Omega$ compact, by Cauchy's estimate \ref{Cauchy's estimate}, $f_j$ are Lipschitz with the same $C_k$, by Ascoli-Arzela theorem, $f_j$ are equicontinuous, thus have convergent subsequence, and then use diagonal argument by exhaust $\Omega$ with compact subsets $K$
\end{proof}

\begin{theorem}[Riemann extension theorem]\label{Riemann extension theorem}
$E\subseteq\Omega$ is a discrete subset, $f\in A(\Omega\setminus E)$, and $f$ is bounded around each point in $E$, then $f$ can be extended to a unique $\tilde f\in A(\Omega)$
\end{theorem}

\begin{proof}
For $z_0\in E$, suppose such $\tilde f$ exists, then by Cauchy's formula \ref{Cauchy's formula}, for any $w\in D(z_0,r)$
\[\tilde f(w)=\frac{1}{2\pi i}\int_{\partial D(z_0,r)}\frac{f(z)}{z-w}dz\]
Thus we just take this as a definition, then
\begin{align*}
\tilde f(w)-f(w)&=\frac{1}{2\pi i}\int_{\partial D(z_0,r)}\frac{f(z)}{z-w}dz-\frac{1}{2\pi i}\int_{\partial D(w,\epsilon)}\frac{f(z)}{z-w}dz \\
&=\frac{1}{2\pi i}\int_{\partial D(z_0,\epsilon)}\frac{f(z)}{z-w}dz
\end{align*}
Which can be show to arbitrarily small as $\epsilon\to0$
\begin{center}
\begin{tikzpicture}
\draw (0,0) circle (5);
\draw (2,2) circle (0.5);
\draw (0,0) circle (0.5);
\filldraw (0,0) circle (0.02);
\filldraw (2,2) circle (0.02);
\node at (0,0) [below] {$z_0$};
\node at (2,2)[below] {$w$};
\end{tikzpicture}
\end{center}
\end{proof}

\begin{theorem}\label{d bar theorem}
If $\alpha=g(z)d\bar z$ is a smooth $(0,1)$-form on $\Omega$, then there exists $u\in C^\infty(\Omega)$ such that $\bar\partial u=\alpha$
\end{theorem}

\begin{proof}
suppose such a $u$ exists, then by Cauchy's formula \ref{Cauchy's formula}
\[u(w)=\frac{1}{2\pi i}\iint_{\Omega}\frac{g(z) dz\wedge d\bar z}{z-w}+\frac{1}{2\pi i}\int_{\partial \Omega}\frac{u(z)}{z-w}dz\]
Since $\bar\partial\displaystyle\int_{\partial \Omega}\frac{u(z)}{z-w}dz=0$. This motivates us to first assume $\alpha$ has compact support, and define
\[u(w)=\frac{1}{2\pi i}\iint_{\Omega}\frac{ g(z)dz\wedge d\bar z}{z-w}\]
Then
\[u(w+\zeta)=\frac{1}{2\pi i}\iint_{\Omega}\frac{g(z)dz\wedge d\bar z}{(z-\zeta)-w}=\frac{1}{2\pi i}\iint_{\Omega}\frac{g(z+\zeta)dz\wedge d\bar z}{z-w}\]
Hence
\begin{align*}
\partial_{\bar w} u(w)&=\frac{1}{2\pi i}\iint_{\Omega}\frac{\partial_{\bar z} g(z)dz\wedge d\bar z}{z-w} \\
&=\frac{1}{2\pi i}\iint_{\Omega}\partial \log|z-w|^2\wedge\bar\partial g \\
&=g(w)
\end{align*}
Therefore $\bar\partial u=\alpha$. In general, consider a compact exhaustion $\Omega=\displaystyle\bigcup_{i}K_i$, where $\hat K_i= K_i$, $K_i\subset\subset\overset{\circ}{K_{i+1}}$, ensured by Corollary \ref{Compact exhaustion of a domain}, let $\chi_i$ be a cutoff function such that $\chi_i\equiv1$ on $K_i$ and $\supp\chi_i\subseteq\overset{\circ}{K_{i+1}}$, then there exists $f_i$ such that $\bar\partial f_i=\chi_i\alpha$, by Runge's theorem \ref{Runge's theorem}, there exists $h_i\in\mathcal O(K_i)$ such that $\|f_{i+1}-f_i-h_i\|_{K_i}<\dfrac{1}{2^i}$. Now define
\[ u_N=f_1+\sum_{k=1}^{N}(f_{k+1}-f_k-h_k)=f_{N+1}-\sum_{k=1}^{N}h_N\]
Converges uniformly on compact subsets to $u$, and $\partial u_N=\alpha$ on $K_i$ for any $i\leq N$
\end{proof}

\end{document}