\documentclass[../main.tex]{subfiles}

\begin{document}

Cousin problems: Let $X$ be a complex manifold, $\{U_\alpha\}$ a locally finite open covering of $X$ \\
I. Given $\{m_\alpha\}$ locally meromorphic functions such that $m_\alpha-m_\beta\in A(U_\alpha\cap U_\beta)$, find a global meromorphic function $m$ on $X$ such that $m-m_\alpha$ is holomorphic on each $U_\alpha$ \\
II. Given $\{m_\alpha\}$ locally meromorphic functions, $m_\alpha\not\equiv0$, such that $m_\alpha m_\beta^{-1}\in A(U_\alpha\cap U_\beta)$, find a global meromorphic function $m$ on $X$ such that $mm_\alpha^{-1}$ and $m^{-1}m_\alpha$ are holomorphic on each $U_\alpha$ \\
These generalize the Mittag-Leffler and Weierstrass theorems in one dimension

Solutio to Cousin I for Stein manifold
Set $u_{\alpha\beta}=m_\alpha-m_\beta$. Then $\{u_{\alpha\beta}\}$ is a 1-cocucle of holoomorphic funtions. It suffices to find $u_\alpha\in A(U_\alpha)$ such that $u_{\alpha\beta}=u_\alpha-u_\beta$. For then $m_\alpha-u_\alpha$ is a well-defined global meromorphic funtion. Let $\{\psi_\alpha\}$ be a partition of unity. Set $h_\alpha=\sum_\beta u_{\alpha\beta}\psi_\beta$. Then $u_{\alpha\beta}=h_\alpha-h_\beta$. $\{\bar\partial h_\alpha\}$ defines a global $(0,1)$-form on $X$. Since $X$ is Stein, we can find a function $v$ so that $\bar\partial v=\bar\partial h_\alpha$, i.e. $v-h_\alpha$ is holomorphic on each $U_\alpha$. Set $u_\alpha=h_\alpha-v$

\begin{definition}
Given $\xi\in\mathbb C^n$, let $\mathcal O_\xi$ denote the ring of germs of holomorhic functions at $\xi$. $\mathcal O_\xi$ is a commutative algebra over $\mathbb C$ and $\cong$ the ring of convergent power series at $\xi$. Clearly, $\mathcal O^{(n)}_0$ is an integral domain. Its function field $\mathcal M_0$ is the field of germs of meromorphic functions at $0$. $\mathcal O_0$ is a local ring with residue field $\mathbb C$, the unique maximal ideal $\mathfrak m_0$ consists of germs of functions vanishing at $0$
\end{definition}

\begin{definition}
$p\in O^{(n-1)}_0[z_n]$ is called a Weierstass polynomial if it is representaed by a funciton ofthe form
\[p(w,z_n)=z_n^k+a_1(w)z_n^{k-1}+\cdots+a_k(w)\]
where $a_i\in \mathfrak m_0^{(n-1)}$
\end{definition}

\begin{theorem}[Weierstrass preparation theorem]
Suppose $f$ is regular of order $k$ in $z_n$, i.e. $f(0,z_n)$ has a zero of order $k$ at $z_n=0$ (In general, we can rotate). Then there exists a unique Weierstrass polynomial $p$ such that $f=up$ for a unit $u$
\end{theorem}

\begin{proof}
By Rouch\'e's theorem, on a small neighborhood of $0\in\mathbb C^n$, $z_n\mapsto f(w,z_n)$ has $k$-zeros $\xi_1(w),\cdots,\xi_k(w)$, counting multiplicity. By the residue theorem, for each $r=1,2,\cdots$ and fixed $w$
\[\sum_{j=1}^k\xi_j(w)^r=\frac{1}{2\pi i}\int_{|\xi|=\delta}\zeta^r\frac{f'(w,\zeta)}{f(w,\zeta)}d\zeta\]
From this, the elementary symmetric polynomials in the $\xi_j(w)$ are holomorphic in $w$. Hence
\[p(w,z_n)=\prod_{j=1}^k(z_n-\xi_j(w))\]
is a Weierstrass polynomial. Now let $u=f/p$. This is holomorphic by the Riemann extension theorem
\end{proof}

\begin{theorem}[Weierstrass division theorem]
Let $p\in\mathcal O^{(n-1)}_0[z_n]$ be a Weierstrass polynomial of degree $k$. Then any $f\in\mathcal O^{(n)}_0$ can be written uniquely as $f=gp+r$, where $r\in\mathcal O^{(n-1)}_0[z_n]$ is a polynomial of degree $<k$. If $f\in \mathcal O^{(n-1)}_0[z_n]$, then so is $g$
\end{theorem}

\begin{proof}

\end{proof}

\begin{lemma}
A Weierstrass polynomial $h\in\mathcal O^{n-1}_0[z_n]$ is reducible over $\mathcal O^{(n)}_0$ iff it is reducible over $\mathcal O^{n-1}_0[z_n]$. In this case, all factors are Weierstrass polynomials, modulo units
\end{lemma}

\begin{proof}

\end{proof}

\begin{theorem}
$\mathcal O^{(n)}_0$ is a UFD
\end{theorem}

\begin{proof}

\end{proof}

\begin{theorem}[Oka]
$\mathcal O^{(n)}_0$ is a Noetherian local ring
\end{theorem}

\begin{proof}
By Hilbert basis theorem, $\mathcal O^{(n-1)}_0[z_n]$ is Noetherian by induction. Let $\mathfrak a\subseteq\mathcal O^{(n)_0}$ be an ideal, and assume $g\in\mathfrak a$ is a regular in $z_n$. Up to units, we may assume it is a Weierstrass polynoimal Let $g_1,\cdots,g_m$ generate $\mathfrak a\cap\mathcal O^{(n-1)}_0[z_n]$. We claim the $g,g_1,\cdots,g_m$ generate $\mathfrak a$. Let $f\in\mathfrak a$. By WD write $f=$
\end{proof}

We will be interested in $\mathcal O_0$ modules $\mathcal E_0$

Since $\mathcal O_0$ is Noetherian, a finitely generated module is finitely presented, i.e. there is an exact sequence (This is actually called \textit{syzygy}\index{Syzygy})
\[\mathcal O^{\oplus s}_0\to\mathcal O^{\oplus r}_0\to\mathcal E_0\to0\]
$\mathcal O^{\oplus s}_0$ are the finitely many relations

Let $\Omega\subseteq\mathbb C^n$ be a domain, and $F^1,\cdots, F^r\in A(\Omega)^{\oplus p}$. For $\xi\in\Omega$, let $\mathcal E_\xi$ be the $\mathcal O_\xi$ module generated by $F^1_\xi,\cdots,F^r_\xi$. Let $\mathcal R_\xi$ be the $\mathcal O_\xi$ module of relations
\[\mathcal R_\xi=\{\}\]
Since $\mathcal O_\xi$ is Noetherian, $\mathcal R_\xi$ is finitely generated submodule of $\mathcal O_\xi^{\oplus r}$ for every $\xi\in\Omega$
Question: Can we find finitely many functions

We now rephrase Oka's theorem

\begin{theorem}
Given, then for each there is an open set and such that
\[(\mathcal O^{(n)}_U)^{\oplus r}\xrightarrow\lambda(\mathcal O^{(n)}_U)^{\oplus p}\xrightarrow\mu(\mathcal O^{(n)}_U)^{\oplus q}\]
is exact on $U$, here $r$ might depend on the choice of $\xi_0$
\end{theorem}

Let $\mathcal O^{(n-1)}_{0,d}[z_n]$ denote the polynomials of degree at most $d$. This is a finitely generated $\mathcal O^{(n-1)}_0$ submodule of $\mathcal O^{(n-1)}_0[z_n]$

First assume $q=1$
\begin{lemma}
Suppose $\lambda,\mu$ have entries in $\mathcal O^{(n-1)}_0[z_n]$ of degree $l$ and $m$, respectively. Assume that one of the entries of $\mu$ is regular in $z_n$. Then we have a sequence
\begin{equation}\label{Lemma for oka's theorem 1}
\mathcal (O^{(n-1)}_{0,d-l}[z_n])^{\oplus r}\xrightarrow\lambda\mathcal (O^{(n-1)}_{0,d}[z_n])^{\oplus p}\xrightarrow\mu\mathcal (O^{(n-1)}_{0,d-l}[z_n])
\end{equation}
as well as
\begin{equation}\label{Lemma for oka's theorem 2}

\end{equation}
If $d\geq m$ and is an exact suquence of modules, then is an exact sequence of modules
\end{lemma}

\begin{proof}
Assume $\mu=$
\end{proof}

Consequence of Oka's theorem: coherent analytic sheaves on $X$ form an abelian category. This is not true for holomorphic vector bundles

\begin{proposition}
If $\phi:\mathcal E\to\mathcal F$ is a homomorphism between coherent analytic sheaves, then $\ker\phi$ and $\coker\phi$ are coherent
\end{proposition}

\begin{proposition}
Suppose we have a short exact sequence of analytic sheaves
\[0\to\mathcal S\to\mathcal E\to\mathcal Q\to0\]
If any two of the three sheaves are coherent, then so is the third
\end{proposition}

\end{document}