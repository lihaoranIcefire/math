\documentclass[main]{subfiles}

\begin{document}

Wirthinger presentaiton

cup product

Cellular cross product: $X,Y$ CW complexes, $X\times Y$ is also a CW complex with characteristic maps $\phi_\alpha\times\psi_\beta$

diagonal approximation

Poincare duality

$H_k(M,\partial M)\cong H^{n-k}(M)$, $H_k(M)\cong H^{n-k}(M,\partial M)$

Define map $C_k\to c^{n-k}$, taking $\sigma$ to 1, but not 

D is a chain map up to sign, so fine for Z/2, and the sign works exactly when M is orientable

\begin{theorem}[Freedman's theorem]
Closed, simply connected, oriented 4-manifold up to homeomorphism
to
symmetric, bilinear, non-degenerate form up to basis change
is surjective, and at most 2 to 1
Moreover, if one consider only smooth manifolds, this map is 1 to 1, but no longer surjective
In other words, smooth, simply connected, closed, oriented 4-manifolds are characterized by their cup product pairing
\end{theorem}

\begin{theorem}["The half lives, half dies" theorem]
$M$ compact manifold of dimension $2n+1$ with boundary of dimension $2n$. Then $\rank H_n(\partial M)$ is even and $\frac{1}{2}\rank H_n(\partial M)=\rank(i_*:H_n(\partial M)\to H_n(M))=\rank(i^*:H^n(M)\to H_n(\partial M))$
\end{theorem}

\begin{theorem}
The homotopy group functor preserves products. If $X=\prod X_\alpha$, then $\pi_n(X)=\prod_\alpha\pi_n(X_\alpha)$
\end{theorem}

\begin{example}
Since $I^0\times I\to\mathbb R$, $(s,t)\mapsto t/s$, $(I^0,I)$ doesn't have the HEP
\end{example}

\begin{lemma}[Compression lemma]
$A\subseteq X,B\subseteq Y$ are CW complexes, for each $k$-cell $\sigma\in X-A$, if $\pi_k(Y,B)=0$, any map $f:(X,A)\to(Y,B)$ is homotopic to a map with image in $B$
\end{lemma}

\begin{theorem}[Whitehead theorem]
$X,Y$ are CW complexes, if $f:X\to Y$ induces isomorphism on each $\pi_n$, then $f$ is a homotopy
\end{theorem}

\begin{proof}
Replace $f:X\to Y$ by the mapping cylinder $f:X\to M_f$, and then apply compression lemma
\end{proof}

$S^k\to S^n$ can be homotopied to a smooth map, which is locally linear, thus can miss a point

\begin{lemma}[Extension lemma]
$f:X\to Y$ such that $f|_{S^k}\to Y$ is null homotopic, then $f$ can be extended to $X\cup e^k\to Y$
\end{lemma}

\begin{theorem}[CW approximation]
Assume
\end{theorem}

\begin{theorem}
An $n$-connected CW model for $X$ is an $n$-connected CW complex $Z$ and a map $f:Z\to X$ such that $f_*:\pi_n(Z)\to\pi_n(X)$ are isomorphisms for $k>n$. Such a model always exists and is unique up to isomorphism
\end{theorem}

\begin{theorem}
If $A\subseteq X$ is a CW complex, An $n$-connected CW model for $(X,A)$ is an $n$-connected CW pair $(Z,A)$ and $f:(Z,A)\to(X,A)$, such that $f|_A=id$, and $\pi_i(Z,A)\to\pi_i(X,A)$ are iso for $i>n$. Such exists and unique up to iso
\end{theorem}

\begin{theorem}
For arbitrary $(X,A)$, there exists an $n$-connected CW model unique up to iso
\end{theorem}

\begin{proof}
Let $Z_0$ be a CW approximation of $A$, $(Z,Z_0)$ be an n-connected CW model for $(M,Z_0)$, M mapping cylider of $Z_0\to A\to X$, then $(M,Z_0)$ is an n-connected CW approximation for $(X,A)$, then by LES and five lemma
\end{proof}

\begin{theorem}
$n$-connected CW models is functorial, i.e. For any $n$-connected model $(Z,A)$ of $(X,A)$ and $n'$-connected model $(Z',A')$ of $(X',A')$, and $g:(X,A)\to (X',A')$, there is a unique $h:(Z,A)\to(Z',A')$ making the diagram commute
\end{theorem}

Postnikov tower: Construction: Fix $n$, the idea is to kill all higher homotopy groups by attaching cells, then by extension lemma, $X\hookrightarrow X_n$ extend to a map $X_{n+1}\to X_n$

Excision fails for homotopy groups
$\pi_4(S^3\wedge S^3)=\pi_4(S^3\times S^3)=Z_2\times Z_2$. $S^3/S^2=S^3\wedge S^3$
$Z_2=\pi_$

\begin{theorem}
C connected, (A,C) m connected, $(B,C)$ n connected, \pi_i(A,C)\to \pi_i(X,B) is an iso if $i<m+n$ and surjective if $i=m+n$
\end{theorem}

\begin{theorem}[Freudenthal's suspension theorem]

\end{theorem}

Ring structure for stable homotopy: $f:S^{i+k}\to S^k$, $g:S^{j+k}\to S^k$, get $S^{i+j+k}\to S^{k}$

If $G$ acts properly discontinuously on a contractible CW complex $X$, then $X/G$ is a $K(G,1)$

Application of Hurewicz theorem in 3 manifold topology: M is a closed, oriented, simply connected 3 manifold. then $\pi_1=0$, H_1=[\pi_1,\pi_i]=0, H^1=0. By Poincare \pi_1=H_2=H^1=0, \pi_3=H_3=Z. Poincare conjecture says M homeomorphic to S^3

\begin{example}
If $\pi_1(A)$ is non trivial, Hurewicz theorem may fail, (S^2\vee S^1,S^1) has different \pi_2, H_2
\end{example}

\begin{exapmole}[representalbe functor for G bundles]
Top be the category of top spaces up to homotopy equivalence, G top group, F:Top\to Sets, X\mapsto G-bundles over X, the BG is a representable functor
\end{example}

Omega spectra

\begin{definition}
An $\Omega$-spectrum is a sequence of $K_n$ of CW complexes with weak homotopy equivalences $K_n\to\Omega K_{n+1}$
\end{definition}

\begin{remark}
The loop space of a CW complex is again a CW complex
\end{remark}

\begin{theorem}
$h^n(X)=\langle X,K_n\rangle$ defines base pointed cohomology theory of CW complexes
\end{theorem}

\begin{example}[Eilenberg-MacLane specturm]

\end{example}

\begin{example}[Complex K theory]
Let $U=\bigcup U(n)$, one can show that there is a weak homotopy equivalence $U\to\Omega^2U$, hence one get a spectrum by setting $K_i=\begin{cases}
\Omega U &i\text{ even} \\
U &i\text{ odd} \\
\end{cases}$. The complex cohomology theory is periodic(Bott periodicity). $h^0(X)=K_0(X)$ is the Grothendieck group. $h^1(X)=<X,U>=<X,\Omega^2 U>=<\Sigma X,\Omega U>=K_0(\Sigma X)$
\end{example}

\begin{example}[Real K theory]
$O \to \Omega^8O$ is a weak homotopy equivalence
\end{example}

\begin{definition}
A spectrum is a sequence $K_n$ with maps $\Sigma K_n\to K_{n+1}$
\end{definition}

\begin{example}[Sphere spectrum]
$K_n=S^n$, $SK_n\cong S^{n+1}$. More generally, fix $Y$, can define $K_n=\Sigma^nY$
\end{example}

\begin{theorem}
$h_i(X)=\varinjlim\pi_{i+n}(X\wedge K_n)$, $\pi_{i+n}(X\wedge K_n)\xrightarrow S\pi_{i+n+1}(\Sigma(X\wedge K_n))=\pi_{i+n+1}(X\wedge\Sigma K_n)\to\pi_{i+n+1}(X\wedge K_{n+1})$ is a reduced base pointed homology theory
\end{theorem}

\begin{proof}
Use Puppe sequence again, note that $\pi_{i+n}(\Sigma A\wedge K_{n})=\pi_{i+n}(\Sigma (A\wedge K_{n}))=\pi_{i+n}(A\wedge\Sigma K_{n})$
\end{proof}

\begin{theorem}
Any homology theory with $H_*(X)=\varinjlim_p H_*(X^p)$ is induced by a spectrum
\end{theorem}

\begin{example}[Stable homotopy groups]
$\pi_{i+n}(X\wedge S^n)=\pi_{i+n}(S^nX)$
\end{example}

\begin{definition}
A cohomology operation is a natural transformation $H^n(-;G)\to H^n(-;G)$
\end{definition}

\begin{example}[Power operation on $H^1(-;\mathbb Z)$]

\end{example}

spectral sequences

$F_pC_*(X)=C_*(X^p)$, E^0_{p,q}=C_{p+q}(X^p,X^{p-1})=C_{p+q}(X^p)/C_{p+q}(X^{p-1})$, $E^1_{p,q}=H_{p+q}(C_*(X^p,X^{p-1}))=H_{p+q}(X^p,X^{p-1})$

\end{document}