\documentclass[../main.tex]{subfiles}

\begin{document}

\textbf{Hatcher 1.1.6.} \par
If $X$ is path-connected, for any $[u]\in [S^{1},X]$, $u$ is a representative of $[u]$, denote $x_{1}:=u(0)$, then there is a path $\gamma$ connecting $x_{0}$ and $x_{1}$ with $\gamma(0)=x_{0}$ and $\gamma(1)=x_{1}$, then $[\gamma\cdot u\cdot\overline{\gamma}]\in \pi_{1}(X,x_{0})$, define $H:S^{1}\times I\rightarrow X$ where
\[
H(s,t)=\left\{\begin{matrix}
\gamma\left( t+3s(1-t) \right), \, s\in[0,\dfrac{1}{3}]\\ 
u(3s-1), \, s\in[\dfrac{1}{3},\dfrac{2}{3}]\\ 
\gamma\left( t+3(1-s)(1-t) \right), \, s\in[\dfrac{2}{3},1]
\end{matrix}\right.
\]
Then $H$ is a homotopy between $\gamma\cdot u\cdot\overline{\gamma}\simeq H(s,0)\simeq H(s,1)\simeq u$, thus $\gamma\cdot u\cdot\overline{\gamma}$ is also a representative of $[u]$, thus $\Phi$ is surjective. \par
If $[f]$ and $[g]$ are conjugate in $\pi_{1}(X,x_{0})$, assume $\gamma\cdot g\cdot\overline{\gamma} \simeq f$ for some $\gamma\in\pi_{1}(X,x_{0})$ with some path homotopy $H$, then $\Phi([f])=\Phi([\gamma\cdot g\cdot\overline{\gamma}])=\Phi([g])$, the second equality holds using the same construction above. \par
Conversely, suppose $\Phi([f])=\Phi([g])$, then there is a homotopy $H$ between $f$ and $g$, let $\gamma(t) = H(0,t)$, then $F:I\times I\rightarrow X$
\[
F(s,t)=\left\{\begin{matrix}
\gamma\left(3st\right), \, s\in[0,\dfrac{1}{3}]\\ 
H\left(3s-1,t\right), \, s\in[\dfrac{1}{3},\dfrac{2}{3}]\\ 
\gamma\left(3(1-s)t\right), \, s\in[\dfrac{2}{3},1]
\end{matrix}\right.
\]
shows that $
\gamma\cdot g\cdot\overline{\gamma} \simeq f$, thus $f$ and $g$ are conjugate \par
\textbf{Hatcher 1.1.12.} \par
Suppose $f\in \pi_{1}(S^{1})$ is a generator, and a homomorphism $\alpha: \pi_{1}(S^{1})\rightarrow \pi_{1}(S^{1})$  with $\alpha(f)=f^{n}$, then define $\varphi: S^{1}\rightarrow S^{1}$ as $\phi(e^{2\pi \mathrm{i}\theta}) = e^{2\pi \mathrm{i}n\theta}$, then we would have $\varphi_{*}=\alpha$ \par
\textbf{1.} \par
Define $\Psi:\pi_{1}(X,x_{0})\times\pi_{1}(Y,y_{0})\rightarrow \pi_{1}\left(X\times Y,(x_{0},y_{0})\right)$, $\Psi([f],[g])=[(f,g)]$, to check $\Psi$ is well-defined, suppose $f^{'}\simeq f, g^{'}\simeq g$, and $F,G$ are homotopies between $f^{'},f$ and $g^{'},g$ respectively, then $(F,G)$ is a homotopy between $(f^{'},g^{'})$ and \((f,g)\), also, $\Psi\left(([f_{1}],[g_{1}])([f_{2}],[g_{2}])\right)=\Psi\left([f_{1}f_{2},g_{1}g_{2}]\right)=[(f_{1}f_{2}),(g_{1}g_{2})]=[(f_{1},g_{1})][(f_{2},g_{2})]$, thus $\Psi$ is a homomorphism, similarly, define $\Phi: \pi_{1}\left(X\times Y,(x_{0},y_{0})\right) \rightarrow \pi_{1}(X,x_{0})\times\pi_{1}(Y,y_{0}), \Phi[(h^{1},h^{2})]=([h^{1}],[h^{2}])$ for $h=(h^{1},h^{2})$, to check $\Psi$ is well-defined, suppose $h^{'}\simeq h$ by $H$, then $h^{1'}\simeq h^{1},h^{2'}\simeq h^{2}$by $H^{1},H^{2}$, also, $\Phi(h_{1}h_{2})=\Phi\left([h_{1}^{1},h_{1}^{2}][h_{2}^{1},h_{2}^{2}]\right)=\Phi\left([h_{1}^{1}h_{1}^{2},h_{2}^{1}h_{2}^{2}]\right)=[h_{1}^{1}h_{1}^{2},h_{2}^{1}h_{2}^{2}]=[h_{1}^{1},h_{1}^{2}][h_{2}^{1},h_{2}^{2}]=\Phi(h_{1})\Phi(h_{2})$, thus $\Phi$ is a homomorphism, $\Psi\circ\Phi=1,\Phi\circ\Psi=\mathbbm{1}$, hence $\Psi$ is bijective thus an isomorphism. \par
\textbf{2.} \par
Since $r\circ i=\mathrm{id}_{A}$, $r_{*}i_{*}=\mathbbm{1}_{*}$, thus $i_{*}$ is injectivce. \par
\textbf{3.} \par
By problem 1 we have $\pi_{1}(D^{2}\times S^{1}) \cong \mathbb{Z}$ and $\pi_{1}(S^{1}\times S^{1}) \cong \mathbb{Z}\times \mathbb{Z}$, there is no injective map from $\pi_{1}(S^{1}\times S^{1})$ to $\pi_{1}(D^{2}\times S^{1})$, since if there is such a injective homomorphism, we would have $(1,0)\mapsto n, (0,1)\mapsto m $ with $n\neq m$, but then \((-m,n)\mapsto 0\) which has the same image as $(0,0)$, but $(-m,n)\neq (0,0)$, thus we have reached a contradiction, then by problem 2 we know that there is no such retraction.

\end{document}