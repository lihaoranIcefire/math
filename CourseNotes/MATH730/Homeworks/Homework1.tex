\documentclass[../main.tex]{subfiles}

\begin{document}

\textbf{1.} \par\noindent
Suppse $\tau$ is a topology on $X$ with exactly 7 elements. Since $\mathscr{P}(X)$ has exactly 8 elements, and $\varnothing,X \in\tau$. Thus, without loss of generality, we could assume either $\tau = \mathscr{P}(X)-\{3\}$ or $\tau = \mathscr{P}(X)-\{2,3\}$ \par\noindent
Case I: $\tau = \mathscr{P}(X)-\{3\}$ \par\noindent
$ \{1,3\},\{2,3\}\in\tau $ which implies $ \{3\} = \{1,3\}\cap \{2,3\} \in \tau $, that is a contradiction. \par\noindent
Case II: $\tau = \mathscr{P}(X)-\{2,3\}$ \par\noindent
$ \{2\},\{3\}\in\tau $ which implies $ \{2,3\} = \{2\}\cup \{3\} \in \tau $, that is also a contradiction. \par\noindent
Therefore, it isn't a topology on $X$ with exactly 7 elements. \par\par\noindent
\textbf{2.} \par\noindent
Suppose $\{U_{\alpha}\}_{\alpha\in A}$ is an open cover of $f(X)$, then $\{f^{-1}(U_{\alpha})\}_{\alpha\in A}$ is an open cover of $X$ since $f$ is continuous, $X$ has a finite subcover $\{f^{-1}(U_{\alpha_{1}}),\cdots,f^{-1}(U_{\alpha_{n}})\}$ since $X$ is compact, hence $f(X)$ also has a finite subcover $\{U_{\alpha_{1}},\cdots,U_{\alpha_{n}}\}$, thus $f(X)$ is also compact. \par\par\noindent
\textbf{3.} \par\noindent
Consider the following commutative diagram
\begin{center}
\begin{tikzcd}
& D^{n} \arrow[r,"h"] \arrow[d,"q"] \arrow[dr,"g"]
& \mathbb{R}^{n+1} \\
& D^{n}/S^{n-1} \arrow[r,"f"]
& S^{n} \arrow[u,hook,"j"]
\end{tikzcd}
\end{center}
\par\noindent
Where $h$ is given explicitly by
$$ h(x) = \left(\cos{\left(\pi|x|\right)},\dfrac{x}{|x|}\sin{\left(\pi|x|\right)}\right) $$
$h$ is continuous, so is $g$ since $h=jg$, so is $f$ since $g=fq$.
Since $D^{n}$ is compact, so is $D^{n}/S^{n-1}=q(D^{n})$, since $S^{n}$ is Hausdorff and $f$ is bijective, thus $f$ is a homeomorphism. Therefore, $D^{n}/S^{n-1}\cong S^{n}$ \par\par\noindent
\textbf{4.} \par\noindent
The pushout of 
\begin{center}
\begin{tikzcd}
\mathbb{R}-\{0\} \arrow[r,hook] \arrow[d,hook]
& \mathbb{R} \arrow[d] \\
\mathbb{R} \arrow[r]
& X
\end{tikzcd}
\end{center}
is not Hausdorff. \par\noindent
Consider the two origins $O_{1}$ and $O_{2}$ of X, then any two neighborhoods of them would intersect, thus $X$ cannot be Hausdorff. \par\par\noindent
\textbf{5.} \par\noindent
Pushout $X\cup_{A}Y$ is constructed by $X\sqcup Y / \sim$, where $\sim$ is the equivalence relationship generated by binary relationship $f(a)\sim g(a)$, hence $\forall x\neq x'\in X$, suppose $\iota_{X}(x)=\iota_{X}(x')$, then $ x = f(a_{1}) \sim g(a_{1}) = g(a_{2}) \sim \cdots \sim f(a_{n}) = x' $, however, since g is injective, $a_{1}=a_{2}$, thus $x=x'$, which is a contradiction. Therefore, $\iota_{X}$ is injective. \par\par\noindent
\textbf{6.} \par\noindent
Consider the trivial topology on $\mathbb{R}$, then a neighborhood of any point can only be $\mathbb{R}$ itself, thus it is not Hausdorff. \par\par\noindent
\textbf{7.} \par\noindent
\textbf{(a)} \par\noindent
$\forall \epsilon>0$, let $\delta=\min\left(1,\dfrac{\epsilon}{1+|x_{0}|+|y_{0}|}\right)$, then $\forall (x,y)\in(x_{0}-\delta,x_{0}+\delta)\times(y_{0}-\delta,y_{0}+\delta)$, $|xy-x_{0}y_{0}|\leq|x||(y-y_{0})|+|(x-x_{0})||y_{0}|\leq(|x_{0}|+\delta)\delta+\delta|y_{0}|=\delta(\delta+|x_{0}|+|y_{0}|)\leq\delta(1+|x_{0}|+|y_{0}|) < \epsilon$, hence $f$ is continuous. \par\noindent
\textbf{(b)} \par\noindent
Consider $x_{0}=y_{0}=-1$, then $(0,\infty)$ is a neighborhood of $x_{0}y_{0}$, however, any neighborhood $(a,\infty)\times(b,\infty)$ of $(x_{0},y_{0})$ with $a<-1,b<-1$, we have $f(0,0)=0\notin (0,\infty)$, therefore, $f$ isn't continuous.

\end{document}