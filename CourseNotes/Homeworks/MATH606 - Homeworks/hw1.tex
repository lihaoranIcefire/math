\documentclass{article}
\usepackage[utf8]{inputenc}
\usepackage{amsmath}
\usepackage{amsfonts}
\usepackage{mathrsfs}
\usepackage{amssymb}
\usepackage{tikz}
\usetikzlibrary{cd}

\title{MATH606 homework 1}
\author{Haoran Li}
\date{}

\setlength{\parindent}{0cm}

\begin{document}

\maketitle

\textbf{1.} \par
First not that
\[
\Lambda \subseteq \Lambda' \quad \Leftrightarrow \quad 
\begin{pmatrix}
\omega_{1}\\ 
\omega_{2}
\end{pmatrix}
=A
\begin{pmatrix}
\omega_{1}'\\ 
\omega_{2}'
\end{pmatrix}
,A \in \mathrm{M}(2,\mathbb{Z})
\]
Hence we have
\[
\Lambda = \Lambda' \quad \Leftrightarrow \quad 
\begin{pmatrix}
\omega_{1}\\ 
\omega_{2}
\end{pmatrix}
=A
\begin{pmatrix}
\omega_{1}'\\ 
\omega_{2}'
\end{pmatrix}
,
\begin{pmatrix}
\omega_{1}'\\ 
\omega_{2}'
\end{pmatrix}
=B
\begin{pmatrix}
\omega_{1}\\ 
\omega_{2}
\end{pmatrix}
,A,B \in \mathrm{M}(2,\mathbb{Z})
\]
Which is equivalent to \(A\in \mathrm{GL}(2,\mathbb{Z})\)

\textbf{2.} \par
Since $\mathbb{C}$ is the universal cover of \(\mathbb{C}/\Lambda'\), $f\circ\pi:\mathbb{C}\rightarrow \mathbb{C}/\Lambda'$ has a lift $F:\mathbb{C}\rightarrow \mathbb{C}$, and locally we have $F=\pi'|_{V}^{-1}\circ f\circ\pi|_{U}$, thus $F$ is holomorphic

\begin{center}
\begin{tikzcd}
&\mathbb{C} \arrow[r,"F"] \arrow[d,"\pi"]
& \mathbb{C} \arrow[d,"\pi'"] \\
& \mathbb{C}/\Lambda \arrow[r,"f"]
& \mathbb{C}/\Lambda'
\end{tikzcd}
\end{center}

Fix \(\omega\in \Lambda\), since $\pi(z+\omega)=\pi(z)$ for any $z\in \mathbb{C}$, we have $F(z+\omega)-F(z)\in\Lambda'$, hence  \(F(z+\omega)-F(z)\) is a continuous function of $z$ but $\Lambda'$ is discrete, thus $F(z+\omega)-F(z)\equiv C_{\omega}$, where $C_{\omega}\in\Lambda'$ is a constant. Then $F'(z+\omega)=F'(z)$ which shows $F':\mathbb{C}\rightarrow \mathbb{C}$ is doubly periodic function, thus induces $G:\mathbb{C}/\Lambda\rightarrow \mathbb{C}$ with $F=G\circ\pi$

\begin{center}
\begin{tikzcd}
&\mathbb{C} \arrow[r,"F'"] \arrow[d,"\pi"]
& \mathbb{C} \\
& \mathbb{C}/\Lambda \arrow[ur,"G"]
\end{tikzcd}
\end{center}

Thus $G$ must be a constant, so is $F'$, therefore $F$ has the form $F(z)=\alpha z+\beta$. Then for any $\omega\in \Lambda$, we have $F(\omega)-F(0)=\alpha\omega\in\Lambda'$, thus $\alpha\Lambda\subset\Lambda'$
If $f$ is biholomorphic, then $\pi'\circ F=f\circ\pi\Rightarrow \pi\circ F^{-1}=f^{-1}\circ\pi'$, which implies $\left\{\begin{array}{rl}
\alpha\Lambda\subset\Lambda' \\
\alpha^{-1}\Lambda'\subset\Lambda
\end{array}\right. \Rightarrow \alpha\Lambda=\Lambda'$

\begin{center}
\begin{tikzcd}
&\mathbb{C} \arrow[d,"\pi"]
& \mathbb{C} \arrow[l,"F^{-1}"] \arrow[d,"\pi'"] \\
& \mathbb{C}/\Lambda
& \mathbb{C}/\Lambda' \arrow[l,"f^{-1}"]
\end{tikzcd}
\end{center}

Conversely, if \(\alpha\Lambda=\Lambda'\), $\pi\circ F^{-1}$ is doubly periodic and induce $f^{-1}$, hence $f$ is biholomorphic \par
\textbf{3.} \par
Suppose $\Lambda=\mathbb{Z}\omega_{1}+\mathbb{Z}\omega_{2}, \mathrm{Im}\left(\dfrac{\omega_{2}}{\omega_{1}}\right)>0$, define $\Lambda'=\mathbb{Z}+\mathbb{Z}\tau$, where $\tau=\dfrac{\omega_{2}}{\omega_{1}}$, we have $\omega_{1}\Lambda'=\Lambda$, thus $X$ and $X(\tau)$ are biholomorphic. \par
$X(\tau)$ and $X(\tau')$ are biholomorphic if and only if $\begin{pmatrix}
\tau'\\ 
1
\end{pmatrix}
=\alpha A
\begin{pmatrix}
\tau\\ 
1
\end{pmatrix}, \alpha\in \mathbb{C}-\{0\}, A\in \mathrm{SL}(2,\mathbb{Z})$
If $X(\tau)$ and $X(\tau')$ are biholomorphic, then $\mathbb{Z}+\mathbb{Z}\tau'=\Lambda'=\alpha\Lambda=\mathbb{Z}\alpha+\mathbb{Z}\alpha\tau$ for some $\alpha\in \mathbb{C}-\{0\}$, thus $\begin{pmatrix}
\tau'\\ 
1
\end{pmatrix}
=
A
\begin{pmatrix}
\alpha\tau\\ 
\alpha
\end{pmatrix}
=\alpha A
\begin{pmatrix}
\tau\\ 
1
\end{pmatrix}$, for some $A\in \mathrm{SL}(2,\mathbb{Z})$, the other direction is easy. \par
\textbf{4.} \par
Assume it is not the case, then $\exists \delta>0$ and $c\in\mathbb{C}$ such that $B(c,\delta) \cap f(X') = \varnothing$, then consider non-constant holomorphic function $\dfrac{1}{f-c}:X'\rightarrow \mathbb{C}$, then it is bounded since $\left| \dfrac{1}{f-c} \right| \leq \delta^{-1}$, by Riemann's removable singularity theorem, we can extend \(\dfrac{1}{f-c}\) to a non-constant holomorphic function $g:X\rightarrow \mathbb{C}$, but since $X$ is compact, $g$ should be a constant, that is a contradiction. \par
\textbf{5.} \par
$f'(z)=\dfrac{1}{2}\left(1-\dfrac{1}{z^{2}}\right)$ when $z\neq0$, thus $1,-1$ are branch points. \par
Consider the chart $({P}^{1}-\{0\},\varphi)$ with $\varphi(z)=\dfrac{1}{z}$

\begin{center}
\begin{tikzcd}
\mathbb{C} \arrow[r,"f"] \arrow[dr]
& \mathbb{P}^{1}-\{0\} \arrow[d,"\varphi"] \\
& \mathbb{C}
\end{tikzcd}
\end{center}

Thus $g(z)=\varphi\circ f(z)=\dfrac{z}{2(z^{2}+1)}$, $g'(z)=\dfrac{1-z^{2}}{2(z^{2}+1)}$, hence \(0\) is not a branch point.

\end{document}