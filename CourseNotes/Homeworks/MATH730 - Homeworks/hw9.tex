\documentclass[12pt]{article}
\usepackage[left=2cm, right=2cm, top=2cm]{geometry}
\usepackage[utf8]{inputenc}
\usepackage{amsmath}
\usepackage{dsfont}
\usepackage{bbm}
\usepackage{faktor}
\usepackage{amsfonts}
\usepackage{mathrsfs}
\usepackage{amssymb}
\usepackage{pgfplots,tikz}
\usetikzlibrary{decorations.markings}
\usetikzlibrary{cd}

\newcommand{\bigslant}[2]{{\raisebox{.2em}{$#1$}\left/\raisebox{-.2em}{$#2$}\right.}}

\title{MATH730 hw9}
\author{Haoran Li}
\date{}

\setlength{\parindent}{0cm}

\begin{document}

\maketitle
\textbf{1.} \par
\vspace{13cm}
Thus the standard form is $eje^{-1}j^{-1}ihi^{-1}h^{-1}$ which is $M_2$ \par
\textbf{2.} \par
\textbf{(a)} \par
$F:G\rightarrow\mathrm{Set}$ is a functor, then $F(*)=S$ is a set, $F(g):S\rightarrow S$ is a map, also, $F(e)=\mathrm{id}_S, F(gh)=F(g)F(h)$, thus $F$ is precisely a group action and vice versa \par
\textbf{(b)} \par
$F:G\rightarrow\mathrm{Vec}$ is a functor, then $F(*)=V$ is a $K$-vector space, $F(g):V\rightarrow V$ is a linear transformation, also, $F(e)=\mathrm{id}_V, F(gh)=F(g)F(h)$, thus $F$ is precisely a group representation and vice versa \par
\textbf{(c)} \par
Define $F:\mathrm{Set}\rightarrow\mathrm{Group}$ by sending set $S$ to the free group $F(S)$ generated by $S$, sending map $\sigma:S\rightarrow T$ to $F(\sigma):F(S)\rightarrow F(T)$ by sending the corresponding generators, $F(\mathrm{id}_S)=\mathrm{id}_{F(S)},F(\mu\sigma)=F(\mu)F(\sigma)$, thus $F$ is a functor \par
\textbf{(d)} \par
If $F:\mathbb{P}(\{0,1\})\rightarrow \mathrm{Top}$ is a functor, denote $F(\varnothing)=A,F(\{0\})=X,F(\{1\})=Y,F(\{0,1\})=Z$ and $F(\varnothing\rightarrow\{0\})=f,F(\varnothing\rightarrow\{1\})=g,F(\{0\}\rightarrow\{0,1\})=\alpha,F(\{1\}\rightarrow\{0,1\})=\beta$,$F(\varnothing\rightarrow\{0,1\})=\alpha\cdot f=\beta\cdot g$, then the following diagram commutes \par
\begin{center}
\begin{tikzcd}
& A \arrow[r,"f"] \arrow[d,"g"]
& X \arrow[d,"\alpha"]\\
& Y \arrow[r,"\beta"]
& Z 
\end{tikzcd}
\end{center}
This induce a pushout \par
\begin{center}
\begin{tikzcd}
& A \arrow[r, "f"] \arrow[d, "g"]
& X \arrow[d, "\iota_X"] \arrow[ddr,bend left, "\alpha"] \\
& Y \arrow[r, "\iota_Y"] \arrow[drr,bend right, "\beta"]
& X\cap_A Y \arrow[dr, dotted, "h" description] \\
& & & Z
\end{tikzcd}
\end{center}
\textbf{3.} \par
\textbf{(a)} \par
Define composition as $(f',g')\circ(f,g)=(f'\circ f,g'\circ g)$ then it is associative \par
\[
\begin{aligned}
&(f'',g'')\circ[(f',g')\circ(f,g)]=(f'',g'')\circ(f'\circ f,g'\circ g)=(f''\circ(f'\circ f),g'\circ(g'\circ g)) \\
&=((f''\circ f')\circ f,(g'\circ')\circ g)=(f''\circ f',g''\circ g)\circ(f,g)=[(f'',g'')\circ(f',g')]\circ(f,g)
\end{aligned}
\]
For any object $(C,D)$ in $\mathscr{C}\times\mathscr{D}$, we have identity $(1_C,1_D):(C,D)\rightarrow(C,D)$, then $\forall (f,g):(C,D)\rightarrow(C',D')$, $(f',g'):(C',D')\rightarrow(C,D)$, we have
\[
(f,g)\circ(1_c,1_D)=(f\circ 1_C,g\circ 1_D)=(f,g), (1_C,1_D)\circ (f',g')=(1_C\circ f',1_D\circ g')=(f',g')
\]
Therefore, $\mathscr{C}\times\mathscr{D}$ is indeed a category \par
\textbf{(b)} \par
The group operation $F:G\times G\rightarrow G, F(g_1,g_2)=g_1g_2$ won't be a functor, if so, then $g_1g_2g_1'g_2'=F(g_1g_2)F(g_1'g_2')=F[(g_1,g_2)(g_1',g_2')]=F(g_1g_1',g_2g_2')=g_1g_1'g_2g_2'\Rightarrow g_2g_1'=g_1'g_2$ so that $g_1'$ and $g_2$ are commutative which is not true in general \par
\textbf{4.} \par\
$h$ induces $[h(x,\cdot)]=[h_x]: h_0(x)\rightarrow h_1(x)$, such that the following diagram commutes \par
\begin{center}
\begin{tikzcd}
& h_0(x) \arrow[r,"{[h_x]}"] \arrow[d,"{[h_0\circ\gamma]}"]
& h_1(x) \arrow[d,"{[h_1\circ\gamma]}"]\\
& h_0(x') \arrow[r,"{[h_{x'}]}"] 
& h_1(x')
\end{tikzcd}
\end{center}
Where $[\gamma]\in\pi_1(X,x,x')$, thus $h$ induces a natural transformation between the induced functors $(h_0)_*$ and $(h_1)_*$
\textbf{5.} \par
\textbf{(a)} \par
Reflexivity: $h_n=0$ is a chain homotopy between $f$ and $f$ since $h_{n-1}\partial_n^C+\partial_{n+1}^Dh_n=0=f_n-f_n$ \par
Symmetry: If $h_n$ is a homotopy between $f$ and $g$, then $-h_n$ is a chain homotopy between $g$ and $f$ since $-h_{n-1}\partial_n^C-\partial_{n+1}^Dh_n=0=-(g_n-f_n)=f_n-g_n$ \par
Transitivity: If $\alpha_n$ is a chain homotopy between $f$ and $g$, $\beta_n$ is a chain homotopy between $g$ and $h$, then 
\[
\begin{aligned}
h_n-f_n
&=(h_n-g_n)+(g_n-f_n) \\
&=(\beta_{n-1}\partial_n^C+\partial_{n+1}^D\beta_n)+(\alpha_{n-1}\partial_n^C+\partial_{n+1}^D\alpha_n) \\
&=(\beta_{n-1}+\alpha_{n-1})\partial_n^C+\partial_{n+1}^D(\beta_n+\alpha_n)
\end{aligned}
\]
Thus $\beta_n+\alpha_n$ is a chain homotopy between $f$ and $h$ \par
\textbf{(b)} \par
\[
\begin{aligned}
\partial^2
\begin{pmatrix}
c_0 \\
c_1 \\
c_2 
\end{pmatrix}
&=\partial
\begin{pmatrix}
\partial(c_0)+(-1)^n c_2\\
\partial(c_1)+(-1)^{n+1} c_2 \\
\partial(c_2)
\end{pmatrix} \\
&=
\begin{pmatrix}
\partial\left(\partial(c_0)+(-1)^n c_2\right)+(-1)^{n-1}\partial(c_2)\\
\partial\left(\partial(c_1)+(-1)^{n+1} c_2\right)+(-1)^n\partial(c_2) \\
\partial^2(c_2)
\end{pmatrix} \\
&=
\begin{pmatrix}
\partial^2(c_0)+(-1)^n\partial(c_2)+(-1)^{n-1}\partial(c_2)\\
\partial^2(c_1)+(-1)^{n+1} \partial(c_2)+(-1)^n\partial(c_2) \\
\partial^2(c_2)
\end{pmatrix} \\
&=
\begin{pmatrix}
0\\
0\\
0
\end{pmatrix}
\end{aligned}
\]
\textbf{(c)} \par
Define $\iota_0(c)
=\begin{pmatrix}
c \\
0 \\
0
\end{pmatrix}
,\iota_1(c)
=\begin{pmatrix}
0 \\
c \\
0
\end{pmatrix}$, then
\[
\partial\iota_0(c)=
\partial\begin{pmatrix}
c \\
0 \\
0
\end{pmatrix}
=\begin{pmatrix}
\partial(c) \\
0 \\
0
\end{pmatrix}
=\iota_0\partial(c)
\]
\[
\partial\iota_1(c)=
\partial\begin{pmatrix}
0 \\
c \\
0
\end{pmatrix}
=\begin{pmatrix}
0 \\
\partial(c) \\
0
\end{pmatrix}
=\iota_1\partial(c)
\]
\textbf{(d)} \par
Define $\xi_n: C_n\rightarrow (C_*\otimes D_*)_{n+1}$ by $
\xi_n(c)
=\begin{pmatrix}
0\\
0 \\
(-1)^n c
\end{pmatrix}
$, then
\[
\begin{aligned}
(\partial\xi_n+\xi_{n-1}\partial)(c)
&=\partial\begin{pmatrix}
0\\
0 \\
(-1)^n c
\end{pmatrix}
+\begin{pmatrix}
0 \\
0 \\
(-1)^{n-1} \partial(c)
\end{pmatrix} \\
&=\partial\begin{pmatrix}
(-1)^{n+1}(-1)^n c\\
(-1)^{n+2}(-1)^n c \\
(-1)^n \partial(c)
\end{pmatrix}
+\begin{pmatrix}
0 \\
0 \\
(-1)^{n-1} \partial(c)
\end{pmatrix} \\
&=\begin{pmatrix}
-c \\
c \\
0
\end{pmatrix}
\end{aligned}
\]
Thus 
\[
\begin{aligned}
\left(\partial(h\xi_n)+(h\xi_{n-1})\partial\right)(c)
&=h(\partial\xi_n+\xi_{n-1}\partial)(c) \\
&=h\begin{pmatrix}
-c \\
c \\
0
\end{pmatrix}
=h\begin{pmatrix}
0 \\
c \\
0
\end{pmatrix}
-h\begin{pmatrix}
c \\
0 \\
0
\end{pmatrix} \\
&=h\circ\iota_1(c)-h\circ\iota_0(c)
\end{aligned}
\]
Thus $h\xi$ is the corresponding chain homotopy between $h\circ\iota_0$ and $h\circ\iota_1$ \par

\end{document}