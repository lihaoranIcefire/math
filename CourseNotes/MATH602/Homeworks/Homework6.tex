\documentclass[../main.tex]{subfiles}

\begin{document}

\begin{customexercise}{3.4.1}
Show that if $p$ is prime, there are exactly $p$ equivalence classes of extensions of $\mathbb Z/p$ by $\mathbb Z/p$ in $\mathbf {Ab}$: the split extension and the extensions
\[0\to\mathbb Z/p\xrightarrow{p}\mathbb Z/p^2\xrightarrow i\mathbb Z/p\to0\quad(i=1,2,\cdots,p-1)\]
\end{customexercise}

\begin{proof}
Suppose $0\to\mathbb Z/p\to A\to\mathbb Z/p\to0$ is exact, then
\[\dfrac{A}{\mathbb Z/p}\cong\mathbb Z/p\Rightarrow|A|=|\mathbb Z/p|^2=p^2\Rightarrow A=\mathbb Z/p^2\text{ or }A=\mathbb Z/p\times\mathbb Z/p\]
If $A=\mathbb Z/p\times\mathbb Z/p$, and $0\to\mathbb Z/p\xrightarrow{\begin{pmatrix}
a \\
b
\end{pmatrix}}\mathbb Z/p\xrightarrow{\begin{pmatrix}
c&d
\end{pmatrix}}\mathbb Z/p\to 0$ is exact, then $\begin{pmatrix}
a \\
b
\end{pmatrix}$, $\begin{pmatrix}
c \\
d
\end{pmatrix}$ are perpendicular in $\mathbb F_p^2$, thus $\begin{pmatrix}
c \\
d
\end{pmatrix}=\lambda\begin{pmatrix}
-b \\
a
\end{pmatrix}$ for some $\lambda\in\mathbb F_p^\times$, and $\begin{pmatrix}
a \\
b
\end{pmatrix}$ can be any nonzero vector in $\mathbb F_p^2$. We have $\lambda(av-bu)=1$ for some $u,v$ such that
\begin{center}
\begin{tikzcd}[ampersand replacement=\&]
0 \arrow[r] \& \mathbb Z/p \arrow[d,equal] \arrow[r, "{\begin{pmatrix}1 \\ 0\end{pmatrix}}"] \& \mathbb Z/p\times\mathbb Z/p \arrow[r, "{\begin{pmatrix} 0&1 \end{pmatrix}}"] \arrow[d, "{\begin{pmatrix}a&u \\ b&v\end{pmatrix}}"', ,"\cong"] \& \mathbb Z/p \arrow[r] \arrow[d,equal] \& 0 \\
0 \arrow[r] \& \mathbb Z/p \arrow[r, "{\begin{pmatrix}a \\ b\end{pmatrix}}"']           \& \mathbb Z/p\times \mathbb Z/p \arrow[r, "{\begin{pmatrix} -\lambda b&\lambda a \end{pmatrix}}"']                                  \& \mathbb Z/p \arrow[r]           \& 0
\end{tikzcd}
\end{center}
If $A=\mathbb Z/p^2$, $0\to\mathbb Z/p\xrightarrow{p}\mathbb Z/p^2\xrightarrow i\mathbb Z/p\to0$, $i=1,2,\cdots,p-1$ are all the possible exact sequences. Suppose the following diagram commutes
\begin{center}
\begin{tikzcd}
0 \arrow[r] & \mathbb Z/p \arrow[d] \arrow[r, "p"] & \mathbb Z/p^2 \arrow[d, "k"] \arrow[r, "i"] & \mathbb Z/p \arrow[d] \arrow[r] & 0 \\
0 \arrow[r] & \mathbb Z/p \arrow[r, "p"']          & \mathbb Z/p^2 \arrow[r, "j"']               & \mathbb Z/p \arrow[r]           & 0
\end{tikzcd}
\end{center}
$kp\equiv p\mod p^2\Leftrightarrow p(k-1)\equiv0\mod p^2\Rightarrow k\equiv1\mod p\Rightarrow jk\equiv i\mod p$, thus $0\to\mathbb Z/p\xrightarrow{p}\mathbb Z/p^2\xrightarrow i\mathbb Z/p\to0$, $i=1,2,\cdots,p-1$ are nonequivalent
\end{proof}

\begin{customexercise}{3.5.1}
Let $\{A_i\}$ be a tower in which the maps $A_{i+1}\to A_i$ are inclusions. We may regard $A=A_0$ as a topological group in which the sets $a+A_i(a\in A,i\geq0)$ are open sets. Show that $\varprojlim A_i=\bigcap A_i$ is zero iff $A$ is Hausdorff. Then show that $\varprojlim\textstyle^1A_i=0$ iff $A$ is complete in the sense that every Cauchy sequence has a limit, not necessarily unique. Hint: Show that $A$ is Hausdorff and complete iff $A\cong\varprojlim(A/A_i)$
\end{customexercise}

\begin{proof}
Consider $\varprojlim(A/A_i)\cong\left\{(\cdots,a_i,\cdots)\in \prod A/A_i\middle|a_j\equiv a_i\mod A_i, \forall j\geq i\right\}$, $A\xrightarrow\phi\varprojlim(A/A_i)$, $a\mapsto(\cdots,a\mod A_i,\cdots)$. Let's show $\phi$ is injective iff $A$ is Hausdorff, $\phi$ is surjective iff $A$ is complete \par
Note that $\phi$ is injective $\Leftrightarrow$ $\varprojlim A_i=\bigcap A_i=0$. If $\bigcap A_i=0$, then for any $a\neq b\in A$, $a-b\in A_{i-1}\setminus A_{i}$ for some $i$, then $a+A_{i}\cap b+A_i=\varnothing$, otherwise $a+c=b+d$ for some $c,d\in A_i$, but then $a-b=d-c\in A_i$ which is a contradiction, hence $A$ is Hausdorff. If $A$ is Hausdorff, for any $a\neq0$, $a\notin 0+A_i=A_i$ for some $i$, thus $\bigcap A_i=0$ \par
If $\phi$ is surjective, suppose $\{a_n\}$ converges, then there is a subsequence $\{a_{n_i}\}$ such that $a_{n_j}-a_{n_i}\in A_i$ for any $j\geq i$, i.e. $(\cdots,a_{n_i},\cdots)\in \varprojlim(A/A_i)$, then there exists $a\in A$ such that $a\equiv a_{n_i}\mod A_i$. If $A$ is complete, for any $(\cdots,a_i,\cdots)\in\varprojlim(A/A_i)$, $\{a_i\}$ is a Cauchy sequence since $a_j-a_i\in A_i$ for any $j\geq i$, there exists a limit $a\in A$ such that $a-a_i\in A_i$, thus $a$ is a preimage \par
Use snake lemma on the following commutative diagram
\begin{center}
\begin{tikzcd}
            &                               & A \arrow[d] \arrow[r]       & \varprojlim(A/A_i) \arrow[d] \arrow[llddd, dashed] &   \\
0 \arrow[r] & \prod A_i \arrow[r] \arrow[d] & \prod A \arrow[r] \arrow[d] & \prod A/A_i \arrow[r] \arrow[d]                    & 0 \\
0 \arrow[r] & \prod A_i \arrow[r] \arrow[d] & \prod A \arrow[r]           & \prod A/A_i \arrow[r]                              & 0 \\
            & \varprojlim\textstyle^1A_i    &                             &                                                    &  
\end{tikzcd}
\end{center}
Thus $\varprojlim\textstyle^1A_i=0$ iff $\phi$ is surjective
\end{proof}

\begin{customexercise}{3.5.2}
Show that $\varprojlim\textstyle^1A_i=0$ if $\{A_i\}$ is a tower of finite abelian groups, or a tower of finite dimensional vector spaces over a field
\end{customexercise}

\begin{proof}
If $\{A_i\}$ is a tower of finite abelian groups, or a tower of finite dimensional vector spaces over a field, then $A_i=0$ for $i$ big enough, hence $\{A_i\}$ satisfies trivial Mittag-Leffler condition, thus $\varprojlim\textstyle^1A_i=0$
\end{proof}

\begin{customexercise}{3.5.4}
Let $C$ be a second quadrant double complex with exact rows, and let $B^h_{pq}$ be the image of $d^h:C_{pq}\to C_{p-1,q}$. Show that $H_{p+q}Tot(T_{-p}C)\cong H_q(B^h_{p*},d^v)$. Then let $b=d^h(a)$ be an element of $B_{pq}^h$ representing a cycle $\xi$ in $H_{p+q}Tot(T_{-p}C)$ and show that the image of $\xi$ in $H_{p+q}Tot(T_{-p-1}C)$ is represented by $d^v(a)\in B^h_{p+1,q-1}$. This provides an effective method for calculating $H_*Tot(C)$
\end{customexercise}

\begin{proof}
Since $d^vx_{p,q}+d^hx_{p+1,q-1}=0\Rightarrow d^vd^hx_{p,q}=-d^hd^vx_{p,q}=(d^h)^2x_{p+1,q-1}=0$, $C_{p,q}\to B_{p,q}^h$ defines a map $Z_{p+q}Tot(T_{-p}C)\xrightarrow{\phi_q} Z_q(B^h_{p*},d^v)$. For any $d^vd^hx_{p,q}=0$, by diagram chasing, we can find element in $Z_{p+q}Tot(T_{-p}C)$ with $(p,q)$ entry $x_{p,q}$, thus $\phi_q$ is surjective. Again by some diagram chasing, we can show that $\phi_q^{-1}(B_q(B^h_{p*},d^v))=B_{p+q}Tot(T_{-p}C)$. Hence $H_{p+q}Tot(T_{-p}C)\cong H_q(B^h_{p*},d^v)$ \par
Suppose an representative in of $\xi$ in $Z_{p+q}Tot(T_pC)$ has $a$ in $(p,q)$ entry and $c$ in $(p+1,q-1)$ entry, then the image of $\xi$ has a representative in $Z_{p+q}Tot(T_{-p-1}C)$ having $c$ in $(p+1,q-1)$ entry, thus the image of $\xi$ in $H_{p+q}Tot(T_{-p-1}C)$ is represented by $d^v(a)=-d^h(c)$
\begin{center}
\begin{tikzcd}
  & {} \arrow[d]                    & {} \arrow[d]                  & {} \arrow[d]                    & {} \arrow[d]                    &              \\
0 & {B^h_{p,2}} \arrow[l] \arrow[d] & {C_{p,2}} \arrow[d] \arrow[l] & {C_{p+1,2}} \arrow[d] \arrow[l] & {C_{p+2,2}} \arrow[d] \arrow[l] & {} \arrow[l] \\
0 & {B^h_{p,1}} \arrow[l] \arrow[d] & {C_{p,1}} \arrow[d] \arrow[l] & {C_{p+1,1}} \arrow[d] \arrow[l] & {C_{p+2,1}} \arrow[d] \arrow[l] & {} \arrow[l] \\
0 & {B^h_{p,0}} \arrow[l] \arrow[d] & {C_{p,0}} \arrow[d] \arrow[l] & {C_{p+1,0}} \arrow[d] \arrow[l] & {C_{p+2,0}} \arrow[d] \arrow[l] & {} \arrow[l] \\
  & 0                               & 0                             & 0                               & 0                               &             
\end{tikzcd}
\end{center}
\end{proof}

\begin{customexercise}{3.5.5}(Pullback)
Let $\to\leftarrow$ denote the poset $\{x,y,z\}$, $x<z$ and $y<z$, so that $\displaystyle\underset{\to\leftarrow}{\lim}A_i$ is the pullback of $A_x$ and $A_y$ over $A_z$. Show that $\displaystyle\underset{\to\leftarrow}{\lim}^1A_i$ is the cokernel of the difference map $A_x\times A_y\to A_z$ and that $\displaystyle\underset{\to\leftarrow}{\lim}^n=0$ for $n\neq0,1$

\end{customexercise}

\begin{proof}
Consider the construction in Vista 3.5.12
\begin{center}
\begin{tikzcd}
A_x \arrow[r, "f"] & A_z & A_y \arrow[l, "g"']
\end{tikzcd}
\end{center}
$C_0=A_x\times A_y\times A_z$, $C_1=A_z\times A_z$, $C_n=0$ for $n\geq2$. $d^0=(p_xf,p_yg)$, $d^1=(p_z,p_z)$, where $p_x,p_y,p_z$ are projections of $A_x\times A_y\times A_z$ onto each factor, thus we have a cochain complex
\[0\to A_x\times A_y\times A_z\xrightarrow{\begin{pmatrix}
f&0&-1 \\
0&g&-1
\end{pmatrix}}A_z\times A_z\to0\]
The image being $B_1=\left\{\begin{pmatrix}
fa-c \\
gb-c
\end{pmatrix}\middle|(a,b,c)\in A_x\times A_y\times A_z\right\}$ \par
Consider surjection $A_z\times A_z\xrightarrow{\begin{pmatrix}
1 &-1
\end{pmatrix}}A_z$, the preimage of $\mathrm{im}d=\left\{fa-gb\middle|(a,b)\in A_x\times A_y\right\}$ which is the image of the difference map $A_x\times A_y\xrightarrow dA_z$ is precisely $B_1$. Hence $\displaystyle\underset{\to\leftarrow}{\lim}^1A_i=\mathrm{coker}(C_0\to C_1)\cong\mathrm{coker}d$ \par
It is clear that $\displaystyle\underset{\to\leftarrow}{\lim}^n=0$ for $n\neq0,1$
\end{proof}

\begin{customexercise}{5.1.1}
Suppose that the double complex $E$ consists solely of the two columns $p$ and $p-1$. Fix $n$ and set $q=n-p$, so that an element of $H_n(T)$ is represented by an element $(a,b)\in E_{p-1,q+1}\times E_{pq}$. Show that we have calculated the homology of $T=Tot(E)$ up to extension in the sense that there is a short exact sequence
\[0\to E^2_{p-1,q+1}\to H_{p+q}(T)\to E^2_{pq}\to0\]
\end{customexercise}

\begin{proof}
Consider $E^1_{p-1,q+1}\to H_{p+q}T$ induced by $E_{p-1,q+1}\hookrightarrow T_{p+q}$, if $\bar a\mapsto0$ with $a\in E_{p-1,q+1}$ being a representative, then $a=d''b+d'c$ for some $(b,c)\in E_{p-1,q+2}\times E_{p,q+1}$, thus $\bar a\in E^1_{p-1,q+1}$ is the image of $\bar c\in E^1_{p,q+1}$, therefore $E^2_{p-1,q+1}=\mathrm{coker}(E^1_{p-1,q+1}\to H_{p+q}T)$ \par
Consider $H_{p+q}T\to E^1_{p,q}$ induced by $T_{p+q}\to E_{p,q}$, for any $(a,b)\in E_{p-1,q+1}\times E_{p,q}$, $d'b+d''a=0$, thus $\bar b$ maps to zero in $E^1_{p-1,q}$. On the other hand, if $\bar b\in E^1_{p,q}$ mapsto zero in $E^1_{p-1,q}$, then $d'b=d''a$, then $(-a,b)$ is a preimage of $\bar b$ under $Z_{p+q}T\to E^1_{p,q}$, therefore $E^2_{p,q}\cong\ker(H_{p+q}T\to E^1_{p,q})$
\end{proof}

\begin{customexercise}{5.2.1}(2 columns)
Suppose that a spectral sequence converging to $H_*$ has $E_{pq}^2=0$ unless $p=0,1$. Show that there are exact sequences
\[0\to E^2_{0n}\to H_n\to E^2_{1,n-1}\to0\]
\end{customexercise}

\begin{proof}
Since $E^2_{p,q}=0$ unless $p=0,1$, $E^2_{p,q}=E^\infty_{p,q}$, $F_{-1}H_n=0$, $E^2_{0,n}=\dfrac{F_{0}H_n}{F_{-1}H_n}=F_{0}H_n$, $E^2_{1,n-1}=\dfrac{F_{1}H_n}{F_{0}H_n}$ and $F_1H_n=H_n$, thus we have exact sequences
\[0\to E^2_{0n}\to H_n\to E^2_{1,n-1}\to0\]
\end{proof}

\begin{customexercise}{5.2.2}(2 rows)
Suppose that a spectral sequence converging to $H_*$ has $E_{pq}^2=0$ unless $q=0,1$. Show that there is a long exact sequence
\[\cdots H_{p+1}\to E^2_{p+1,0}\xrightarrow dE^2_{p-1,1}\to H_p\to E^2_{p0}\xrightarrow dE^2_{p-2,1}\to H_{p-1}\cdots\]
\end{customexercise}

\begin{remark}
If a spectral sequence is not bounded, everything is more complicated, and there is no uniform terminology in the literature. For exacmple, a filtration in [CE] is "regular" if for each $n$ there is an $N$ such that $H_n(F_pC)=0$ for $p<N$, and all filtrations are exhaustive. In [MacH] exhaustive filtrations are called "convergent above". In [EGA, $0_{\text{III}}$(11.2)] even the definition of spectral sequence is different, and "regular" spectral sequences are not only convergent but also bounded below. In what follows, we shall mostly follow the terminology of Bourbaki [BX, p175]
\end{remark}

\begin{proof}
Since $E_{pq}^2=0$ unless $q=0,1$, $E^3_{p,q}=E^\infty_{p,q}$, $E^3_{p,0}=\dfrac{F_pH_p}{F_{p-1}H_p}=\ker(E^2_{p,0}\to E^2_{p-2,1})$, $E^3_{p,1}=\dfrac{F_pH_{p+1}}{F_{p-1}H_{p+1}}=\mathrm{coker}(E^2_{p+2,0}\to E^2_{p,1})$, thus $F_{n-2}H_n=0$, $F_nH_n=H_n$, we have exact sequences
\[0\to\mathrm{coker}(E^2_{p+1,0}\to E^2_{p-1,1})=\dfrac{F_{p-1}H_p}{F_{p-2}H_p}\to H_p\to\dfrac{F_pH_p}{F_{p-1}H_p}=\ker(E^2_{p,0}\to E^2_{p-2,1})\to0\]
Splice these together we get
\[\cdots H_{p+1}\to E^2_{p+1,0}\xrightarrow dE^2_{p-1,1}\to H_p\to E^2_{p0}\xrightarrow dE^2_{p-2,1}\to H_{p-1}\cdots\]
\end{proof}

\end{document}