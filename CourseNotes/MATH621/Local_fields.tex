\documentclass[main]{subfiles}

\begin{document}

A discrete valued field $(K,v)$ is a surjective(normalized and exclude trivial valuations) function $v:K\to\mathbb Z\cup\{+\infty\}$ satisfying
\begin{enumerate}
\item $v(x)=+\infty \iff x=0$
\item $v(xy)=v(x)+v(y)$ is a group homomorphism $K^\times\to\mathbb Z$
\item $v(x+y)\geq\min(v(x),v(y))$
\end{enumerate}
Subring $\mathcal O_K=\{x\in K|v(x)\geq0\}$ of $K$ with $\Frac(\mathcal O_K)=K$ and that it is a valuation ring(in fact a DVR, i.e. a PID with unique non-zero prime ideal), with the unique non-zero prime ideal $m_K=\{x\in K|v(x)>0\}$, generated by the uniformizer $\pi$ such that $v(\pi)=1$. $\mathcal O_K^\times=\{x\in K|v(x)=0\}$. For any $x\in K$, there is a unique $n$ such that $\pi^{-n}x\in\mathcal O_K^\times$, $n=v(x)$, i.e. $v$ can be recovered from $\mathcal O_K$, in fact, all discrete valuations $v$ on $K$ corresponds to DVR's $\mathcal O\subseteq K$ whose fraction field is $K$. $k=\mathcal O_K/m$ is the residue field, then is a natural topology on $K$, pick $\alpha\in\mathbb R$, $0<\alpha<1$, define absolute value on $K$, $K^\times\to\mathbb R_{>0}$, $x\mapsto\alpha^{v(x)}=|x|$. Discrete valuations correspond to non-Archimedean absolute values whose image is a discrete subgroup of $\mathbb R^\times/\sim$, making $K$ into a metric space, whose topology is independent of $\alpha$. In fact, $\mathcal O_K$ is open, and $m^n$, $n\geq1$ form an open neighborhood basis of $0$

\begin{theorem}[Ostrowski's theorem]\label{Ostrowski's theorem}
Every non-trivial absolute value on $\mathbb Q$ is equivalent to either the real absolute value $||_\infty$ or $p$-adic absolute values $||_p$. A field that is complete with respect to an Archimedean absolute value is(topologically and algebraically) $\mathbb R$ or $\mathbb C$
\end{theorem}

\begin{note}
An absolute value is a norm with $|xy|=|x||y|$. Two absolute values $||,||_*$ are equivalent if $||_*=||^c$ for some $c>0$. The trivial absolute value is $|0|=\infty$ and 0 otherwise
\end{note}

\begin{example}
$K=\mathbb Q,v=v_p,\mathcal O_K=\mathbb Z_{(p)}, m=p\mathbb Z_{(p)}$ is locally profinite. $K=\mathbb Q_p,v=v_p,\mathcal O_K=Z_p$ is profinite
\end{example}

\begin{definition}
$(R,m)$ is a local ring, its completion $\hat R=\displaystyle\varprojlim_{n}R/m^n$. $\hat R$ is also local with unique max ideal $\hat m=\ker(\hat R\to R/m)$, $R\to \hat R$ is a natural local ring homomorphism, induce $R/m^n\cong \hat R/\hat m^n$, thus $\hat{\hat R}\cong\hat R$
\end{definition}

We call $(K,v)$ complete if $\mathcal O_K$ is complete as a local ring. and this is true iff $K$ is complete in the metric sense

\begin{definition}
A non-archimedean local field is a discrete valued field $(K,v)$ which is complete and has finite residue field. Archimedean local fields are $\mathbb Q,\mathbb C$ by Theorem \ref{Ostrowski's theorem}
\end{definition}

We use local fields to mean just non-archimedean local fields. In this case, $\mathcal O_K/m^n$ are discrete by exact sequences, so $\mathcal O_K$ is profinite, thus compact, $K$ is locally profinite. Conversely, if $(K,v)$ is a discrete valued field such that $K$ is locally profinite(suffices to show that $K$ is locally compact), then $(K,v)$ is a local field

$(K,v)$ is a discretely valued field, we have $\hat{\mathcal O_K}$ as a DVR with $\pi$ again as the uniformizer, let $\hat K$ to be the field of fractions with a natural valuation $\hat v$, and $K$ has dense image in $\hat K$. So if $(K,v)$ has finite residue field, the completion is a local field

\begin{example}
$\mathbb F_q(t)$, $v_t$ valuation, with residue field $\mathbb F_q$, valuation ring $\mathbb F_q[t]$, and max ideal $t\mathbb F_q[t]$, the completion is the Laurent series in $t$, $v_t$ gives the order of zero or pole, with valuation ring $\mathbb F[[t]]$
\end{example}

$K$ is a local field

Structure of $(K,+)$ and $(K^\times,\times)$

$K^\times\cong\mathbb Z\times\mathcal O_K^\times$ as topological groups, $\mathbb Z$ with discrete topology

$U=\mathcal O_K^\times$, $U_n=1+m_K^n$, then $U\supseteq U_1\supseteq\cdots$ form an open subgroup neighborhood basis of $1$, $U/U_1\cong k^\times$, $U_n/U_{n+1}\cong m_K^n/m_K^{n+1}\cong k$, $x\mapsto x-1$

$U_1$ is the unique pro-$p$ Sylow subgroup of $U$ since $|U_n/U_{n+1}|=p$ and $|U/U_1|=p-1$ is coprime to $p$

$U$ is profinite since $U$ is compact(closed in $\mathcal O_K$)

\begin{remark}
If $K$ is a local field of characteristic 0, then for sufficiently large $n$(so that the series converge), $m_K^n\cong U_n$, $x\mapsto e^x$
\end{remark}

\begin{corollary}
In this case, every finite index subgroup of $K^\times$ is open
\end{corollary}

\begin{proof}
Suppose $H$ is of finite index $j$ in $K^\times$, then $(K^\times)^j\subseteq H$. Fix uniformizer $\pi$, $(K^\times)^j\cong\pi^{j\mathbb Z}\times U^j$. $U^j\supseteq U_1^j\supseteq\cdots$, only need to show $U_n^j$ is open for some $n$, for $n$ large enough, $U_n\cong m_K^n\cong\mathcal O_K$, so $U_n^j\cong j\mathcal O_K\overset{open}{\subseteq}\mathcal O_K$
\end{proof}

Teichmuller lift:

\begin{fact}
The surjective homomrophism $\mathcal O_K^\times\to k^\times$($k$ is the residue field which is finite) has a unique multiplicative section $[]:k^\times\to\mathcal O_K^\times$. Moreover, $[x]=\varinjlim_ny_n^{p^n}$, $y_n\in\mathcal O_K^\times$ is an arbitrary lift of $\sqrt{p^n}{x}\in k^\times$
\end{fact}

\begin{example}
$K=\mathbb Q_5$, $[\bar4]=-1\in\mathcal O_K^\times=\mathbb Z_5^\times$
\end{example}

\begin{fact}
$\forall x\in K^\times$, $\exists_1(a_n)_{n\geq v(x)}$ such that $a_n\in\{0\}\cup[k^\times]$, $x=\sum_{n\geq v(x)}\pi^n a_n$
\end{fact}

Warning: $x=sum \pi^na_n$, $y=sum \pi^nb_n$, $x+y=x=sum \pi^n(a_n+b_n)$ is not the canonical choice

Finite extensions of local fields

\begin{theorem}[Serre II.2]
$(K,v)$ is a complete discretely valued field, $E/K$ is a separable field extension, $\exists_1 w$ on $E$ and $\exists_1 e\in\mathbb Z_{\geq1}$ such that $\forall x\in K$($e$ is the ramification index), $w(x)=ev(x)$. Moreover, $(E,w)$ is complete, $k_E/k_K$ is a finite extension of degree $f=[E:K]/e$
\end{theorem}

\begin{remark}
$w(y)=v(N_{E/K}(y))/f$, $\forall y\in E$
\end{remark}

$\Rightarrow$ Every finite extension of a local field has canonical structure of a local field itself. In the future, when we talk about finite extensions of local fields $E/K$, it's always assumed that the local field structure on $E$ is obtained from $K$ in this way

\begin{fact}
Every local field is either a finite extension of $\mathbb Q_p$ or $\mathbb F_q([t])$, Laurent series
\end{fact}

\end{document}