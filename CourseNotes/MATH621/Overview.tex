\documentclass[main]{subfiles}

\begin{document}

Class field theory(CFT) is the study of abelian extensions of global and local fields

\begin{definition}
A global field is a finite separable extension of $\mathbb Q$ or function field of a geometrically smooth curve over $F_q$. A local field is a finite extension of $\mathbb Q_p$ or function field of $F_q((t))$
\end{definition}

we want to understand abelian extensions of $K$ in terms of an invariant of $K$: the \textit{idele class group}\index{idele class group}(some generalization of $\Cl(O_K)$)
\[C_K=\begin{cases}
\mathbb A^\times_k/K^\times, K\text{ global} \\
K^\times, K\text{ local}
\end{cases}\]

Why do we care?

Power reciprocity: The Legendre symbol $(n/p)=\begin{cases}
1&\text{$n$ is a square $\mod p$}\\
-1&\text{otherwise}
\end{cases}$

Quadratic reciprocity: For $p,q$ distinct odd primes
$(p/q)(q/p)=1$ if one of $p,q\equiv1\mod4$, -1 if $p\equiv q\equiv 3\mod4$, or more succinctly
$(p/q)(q/p)=(-1)^{\frac{(p-1)(q-1)}{4}}$. Also $(-1/p)=(-1)^{\frac{p-1}{2}}$, $(2/p)=(-1)^{\frac{p^2-1}{8}}$


Class field theory is a vast and conceptual generalization of this, it put quadratic reciprocity into context. CFT $\Rightarrow$ higher power reciprocity, e.g. cubic reciprocity: 2,7 are not cubic powers mod 61

A classical problem: $p=x^2+ny^2$, when a prime $p$ can be written as above

If $n=1$, this holds iff $p\equiv 1\mod 4$ or $p=2$ iff $p$ splits in $\mathbb Q(\sqrt{-1})$. CFT gives a complete solution to this for all $n$. See D. Cox "Primes of the form $x^2+ny^2$"

If $n=14$, then this holds iff $(-14/p)=1$ and $(x^2+1)^2-8$ has root mod $p$

$K=\mathbb Q(\sqrt{-14})$, then this holds iff $p=\bar PP$ splits in $K$ and $P$ is principle iff(by CFT) $p=\bar P P$ splits in $K$ and $P$ splits in the Hilbert class field of $K$($H/K$ some specific finite abelian extension) iff $p$ splits in $H$

"Reciprocity": whether a prime is principal is related to whether it splits in certain abelian extensions

"Class field": $K$ is a number field, a modulus of $K$ is a formal symbol $\mathfrak m=v_1^{e_1}\cdots v_k^{e_k}$. $v_i$'s are places of $K$, and $e_i\in\mathbb Z_{\geq0}$, satisfying: Complex places $v$ don't appear in $\mathfrak m$, for real places $v$, $e_i=0,1$. e.g. $K=\mathbb Q$

Here a place is an equivalence class of absolute values, two absolute values are equivalent if they differ by a positive real power

The \textit{ray class group}\index{ray class group} $Cl_m$ is generated by fractional ideals coprime to $\mathfrak m$ modulo principal ideals generated by $f\in K^\times, f\equiv1\mod \mathfrak m$. In the number field case, $\Cl_m$ is finite which is no longer true for global fields with char>0

The usual class group $\Cl(\mathcal O_K)$ is where $\mathfrak m=1$

Fact: $\Cl_m$ is always finite abelian

The narrow class group, corresponds to $\mathfrak m$ is the product of all real places. This is the quotient group of all fractional ideals modulo those principal ideals generated $x \in K^\times$ such that $v(x) > 0$ for all real places $v$, i.e., the totally positive $x \in K^\times$

CFT: there is a finite abelian ext $K_m/K$ called \textit{ray class field}\index{ray class field} of $\mathfrak m$

Example: $m=1, K_m$ is the hilbert class field

$K_m$ is uniquely characterized among finite abelian(Galois) extension over $K$ such that whether prime $p$ of $K$ splits in $K_m$ iff whether $p$ has trivial image in $Cl_m$, for all $p$ coprime to $m$

\begin{theorem}[Generalized Kronecker-Weber theorem]
Every finite abelian extension $E/K$ is contained in $K_m/K$ for sufficiently large $\mathfrak m$, one can choose $\mathfrak m$ such that its members are precisely the places of $K$ that ramify in $E$
\end{theorem}

\begin{example}
If $v_1,\cdots,vk$ are the achmedian places of $K$ that ramify in $E$, and $p_1\cdots,p_k$ are unramified places, then $m=v_1,\cdots,v_k,p^{e_1},\cdots,p_k^{e_k}$ for suff large $e_1,\cdots,e_k, E\subseteq K_m$
\end{example}

\textit{Artin isomorphism}\index{Artin isomorphism} $\Psi:\Cl_m\to \Gal(K_m/K)$ has a concrete formula, $p\mapsto (p,K_m/K)$(well-definedness is nontrivial, called the Artin reciprocity). for every $p$ coprime to $m$, know $p$ is unramifed in $K_m$, 

Recall: $E/K$ is a finite Galois extension of global fields, suppose $p$ is a prime of $K$ that is unramifed in $E$, then $\forall P|p$, the arithmetic Frobenius element $\sigma=(P,E/K)$(Artin symbol) in $\Gal(E/K)$ is characterized by $\sigma$ stablizes $P$, the action of $\sigma on k(P)$ as $x\mapsto x^q, q=|k(p)|$. $\{(P,E/K)|P\}$ runs through the primes of $P|p$ is a conjugacy class in $\Gal(E/K)$ called $(p,E/K)$. If $\Gal(E/K)$ is abelian, then $(p,E/K)$ is an element

Fact: For $p$ of $K$ unramified in $E$, $(p,E/K)=\{1\}$ iff $p$ splits in $E$

The theory of ray CFT + Artin iso $\Psi$ + K-W theorem  gives the ideal theoretic formulation of global CFT

Adelic formulation in terms of $\mathbb A^\times_k/K^\times$ is cleaner. And it's easier to see functoriality in $K$ that way

\end{document}