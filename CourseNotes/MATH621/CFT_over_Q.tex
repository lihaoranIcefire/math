\documentclass[main]{subfiles}

\begin{document}

Application: Chebotarev density theorem

$E/K$ is a finite Galois extension of number fields, $G=\Gal(E/K)$, forall $p$ prime in $K$, unramified in $E$

\begin{theorem}
$\forall$ conjugacy classes $C$ in $G$ the set of $p$ of $K$ such that $(p,E/K)=C$ has density $|C|/|G|$ among all primes of $K$. In particular, there are infinitely many such primes $p$
\end{theorem}

applications in global Galois representation by density

consequence: $p$ splits iff $(p,E/K)=\{1\}$, such primes constitute $1/|G|$ of all primes, thus infinitely many

\begin{theorem}[Dirichlet theorem: primes in arithemetic progression]
if $a,b\in \mathbb Z$, $(a,b)=1$, exists infinitely many primees $p$ in the arithemetic progression $a+b\mathbb Z$, e.g. $a=1,b=4$, infinitely primes $\equiv1\mod4$
\end{theorem}

More application of CFT

Artin $L$ funcitons: Let $E/K$ be a finite Galois extension of number fields, $\rho:\Gal(E/K)\to GL_n(\mathbb C)$, S finite primes including all ramified primes of $K$, define $L(\rho,s)$ for $\operatorname{re}(s)>>0$, CFT $\Rightarrow$ meoromorphic extension to $\mathbb C$

Conjecture(Artin):..

\begin{theorem}[Grumwold-Wang, local-global behavior of number fields, local-global principle for quadratic forms]
A non-degenerate quadratic form over a number field $K$ represents 0 over K(it=0 has sol over $K$) iff it represents 0 over $K_v$ for all places $v$ of $K$
\end{theorem}

CFT is the $\GL_1$ case of the Langlands program

CFT for $\mathbb Q$(given by cyclotomic extension of $\mathbb Q$)

Review of cyclotomic extesions of $\mathbb Q$

$K$ is a general field, $m\in\mathbb Z_{>0}$, the $m$-th cyclotomic extension of $K$ is $K(\mu_m)$, $\mu_m$ is the roots of unity in $\bar K$, all roots would be simple, and is a cyclic group $\cong (\mathbb Z/m)^\times$ under multiplication, generator is primitive $m$-th root of 1, denoted $\zeta_m$. $K(\mu_m)=K(\zeta_m)$, by definition also the splitting field of $x^m-1$ over $K$, thus Galois. Obeservation: $\Gal(K(\zeta_m))$ embeds into $(\mathbb Z/m\mathbb Z)^\times$, $\sigma\mapsto a, \sigma(\zeta_m)\zeta_m^a$, so the extension is abelian. Cyclotomic polynomials are $\Phi_m(x)=\prod_\zeta(x-\zeta)\in \mathbb Z[x]$, $\zeta$ runs primitive $m$-th roots of unity in $\mathbb C$

$\Phi_1=x-1$, $\Phi_2=x+1$, $\Phi_m=\frac{x^m-1}{\prod_{d|m,d<m}\Phi_d(x)}$. $K(\zeta_m)/K$ is the spliting filed of $\Phi_m$. $\deg(\Phi_m)=\phi(m)=|(\mathbb Z/m\mathbb Z)^\times|$

$\alpha:\Gal(K(\zeta_m)/K)\to (\mathbb Z/m)^\times$ is an isomorphism iff $\Phi_m$ is irreducible

\begin{theorem}[Gauss]
$\Phi_m$ is irreducible in $\mathbb Q[x]$
\end{theorem}

\begin{proof}
Gauss's lemma, reduce to factorization mod $p$
\end{proof}

\begin{fact}[L washington sec2]
\begin{enumerate}
\item $\mathcal O_{\mathbb Q(\zeta_m)}=\mathbb Z[\zeta_m]\cong \mathbb Z[x]/\Phi_m(x)$
\item assume $m\equiv2\mod 4$(if $m\equiv2\mod4$, then $\phi(m)=\phi(m/2)\Rightarrow\mathbb Q(\zeta_m)=\mathbb Q(\zeta_{m/2})$)
prime $p$ of $\mathbb Q$ is unramified in $\mathbb Q(\zeta_m)$ iff $p\nmid m$
\item formula for $\disc \mathbb Q(\zeta_m)$
\end{enumerate}
\end{fact}

\begin{lemma}
$\forall p\nmid m$, $p \in(\mathbb Z/m)^\times$ is $(p,\mathbb Q(\zeta_m)/\mathbb Q)$ the Frobenius element in $\Gal(\mathbb Q(\zeta_m)/\mathbb Q)$
\end{lemma}

\begin{proof}
Only need to prove $\sigma$ fix $P$ for $P|p$ \\
Recall: Suppose $E/K$ is finite separable extension of number fields, there is a way to explicitly factorize a prime $p$ of $K$ inside $E$(for almost all $p$), write $E=K(\alpha)$ such that $\alpha\in\mathcal O_E$, $\mathcal O_K[\alpha]\subseteq\mathcal O_E$ and $\mathcal O_k[\alpha]\otimes_{\mathcal O_E}E=E(\mathcal O_K[\alpha]$ is an order in $\mathcal O_E$). Conductor: $f=\{x\in O_E|xO_E\subseteq O_K[\alpha]\}$, largest ideal of $\mathcal O_E$ that lies inside $\mathcal O_K[\alpha]$ \\
Fact: for $p$ prime of $K$, coprime to $f$, $p\mathcal O_E=\prod_{i=1}^gP_i^{e_i}$, $f(x)$ is the minimal polynomial of $\alpha$ in $\mathcal O_K[x]$, factorize over $k(p)=\mathcal O_K/p$, $\prod_{i=1}^gf_i^{e_i}$, $f_i$ irreducible in $k(p)[x]$, $P_i$ is a lift of $f_i(\alpha)\mathcal O_E+p\mathcal O_E$ \\
$\mathcal O_E=\mathbb Z[\zeta_m]$, $\min(zeta_m/\mathbb Q)=\Phi_m$, $p\mathcal O_E=\prod P_i^{e_i}$, $\Phi_m$ in $F_p[x]$ factor as $\prod f_i^{e_i}$, $P_i$ is a lift of $f_i(\zeta_m)$. Suppose $p$ sends $P_i$ to $P_j$, $i\neq j$, but $p$ send $P_i$ to lift of $f_i(\zeta_m^p)=h(\zeta_p)$, $h(\zeta_p)=(\tilde f(\zeta_m))^p$ in $F_p$ implies $B_j\subseteq B_i$, contradiction!
\end{proof}

\begin{theorem}
For $\mathbb Q$, a modulus is a symbol $\mathfrak m=\infty m$ or $\mathfrak m=m$ for some $m\in \mathbb Z_{>0}$, $Cl_{\mathfrak m}$ is the group of fractional ideals of $\mathbb Q$ coprime to $m$/principle ideals generated by $x\in \mathbb Q^\times$ such that $x$ coprime to $m$, $x\equiv1\mod m$, $x>0$ if $\mathfrak m=\infty\cdot m$
\end{theorem}

\begin{exercise}
When $\mathfrak m=\infty\cdot m$, then we have an iso $(\mathbb Z/m)^\times\to Cl_m$, $\forall p\nmid m$, the ray class group, $p\mapsto$ the class of the prime ideal $(p)$
iso $(\mathbb Z/m)^\times/\{\pm1\}\to Cl_m$, $\mathfrak m=m$, $\forall p\nmid m$, $p\mapsto$ the class of the prime ideal $(p)$
\end{exercise}

$E_m=\mathbb Q(\zeta_m)$, think of this as $Cl_{\infty m}\to\Gal(\mathbb Q(\zeta_m)/\mathbb Q)$

This means that $\mathbb Q(\zeta_m)/\mathbb Q$ is the ray class field

Recall: this is also characterized by the following property: for almost all primes p, p splits in the ray class field iff $p$ is trivial in the ray class group

for $p\nmid m$, $p$ splits in $m$ iff $(p,E_m/\mathbb Q)=1$ iff $p$ is trivial in $Cl_{\infty m}$

Note if $E\subseteq E_m$, then $E/\mathbb Q$ is again finte abelian extesion, and the composition $Cl_\infty m\to Gal(E_m/\mathbb Q)\xrightarrow{\text{restriction}} Gal(E/\mathbb Q)$

Theorem[Kronecker-Weber, proof given by Hilbert, later simplified]
Every finite abelian extension $E/\mathbb Q$ is over some $\mathbb Q(\zeta_m)$, can also choose $m$ to be divisible only by those $p$ that ramify in $E$

there is an elementary proof for the analogous result for $\mathbb Q_p$(local Kronecker-Weber, see Larry's book). Theorem true for all $\mathbb Q_p$ imply $\mathbb Q$(any non-trivial $E/\mathbb Q$ cannot be unramified everywhere)

CFT for $\mathbb Q$ is basically: $\Psi$ and Kronecker-Weber theorem

Elegant proof for quadratic reciprocity: $p\neq q$ odd primes, and $q\equiv1\mod 4$, then $(p/q)=(q/p)$

there is a unique index 2 subgroup of $(\mathbb Z/q)^\times$, i.e. there is a unique quadratic extension $K$ of $\mathbb Q$ inside $E_q$, we know that $q$ is the only finite prime that ramifies in $E_q$, thus $q$ is the only finite prime that ramifies in $K$, thus $K=\mathbb Q(\sqrt{q})(K\neq Q(\sqrt{-q})$ since $q\equiv1\mod4, \disc=q$, not $4q$(2 also ramifies))

One can explicitly construct $\sqrt{q}$ inside $Q(\zeta_q)$(Gauss, see exercise in the notes). $(p/q)=1$ iff $p$ represents a square in $(\mathbb Z/q)^\times\cong\Gal(E_q/\mathbb Q)$ iff $p$ lies in the kernel of $\Gal(E_q/\mathbb Q)\to \Gal(K/\mathbb Q)$ iff $(p,E_q/\mathbb Q)\to1$ iff $(p,K/\mathbb Q)=1$ iff $p$ splits in $K$. $\mathcal O_K=\mathbb Z[\frac{1+\sqrt{q}}{2}]$ contain $\mathbb Z[\sqrt{q}]$ with conductor $2\mathcal O_K$ iff $X^1-q$ splits mod $p$ iff $(q/p)=1$

\begin{exercise}
Try other cases: prove $(p/q)=-(q/p)$ if $p\equiv q\equiv3\mod4$
\end{exercise}

\end{document}