\documentclass{beamer}
\setbeamercovered{}

\usepackage{bookman}
\usepackage{newtxtext}
\usepackage{amssymb}
\usepackage{amsmath}
\usepackage{amsfonts}
\usepackage{mathrsfs}
\usepackage{mathtools}
\usepackage{hyperref}
\usepackage{pifont}
\usepackage{systeme}
\usepackage{subcaption}
\usepackage{bm} % Bold
\usepackage{bbm} % Can use \mathbbm{1}
\usepackage{ifthen} % Use if then else
\usepackage{shuffle} % Use shuffle product
\usepackage{minted} % For listing codes
\usepackage{tikz}
\usetikzlibrary{calc}
\usepackage{pgfplots}

\makeatletter
\newcommand*{\Relbarfill@}{\arrowfill@\Relbar\Relbar\Relbar}
% \newcommand*{\relbarfill@}{\arrowfill@\relbar-\relbar}
\newcommand*{\xequal}[2][]{\ext@arrow 0055\Relbarfill@{#1}{#2}}
% \newcommand*{\xdash}[2][]{\ext@arrow 0055\relbarfill@{#1}{#2}}
\makeatother

\newlength{\simlength}

\newcommand{\xsim}[1]{
\settowidth{\simlength}{$#1$}
\mathrel{\overset{#1}{\resizebox{\simlength}{\height}{\sim}}}
}

\DeclareMathOperator{\Nul}{Nul}
\DeclareMathOperator{\Col}{Col}
\DeclareMathOperator{\Row}{Row}
\DeclareMathOperator{\Rank}{Rank}
% \DeclareMathOperator{\}{}

\mode<presentation> {
    \usetheme{Madrid}
    \usecolortheme{beaver}
    \usefonttheme{professionalfonts} 
    \setbeamertemplate{navigation symbols}{}
    \setbeamertemplate{caption}[numbered]
}

\theoremstyle{definition}
\newtheorem{proposition}[theorem]{Proposition}
\newtheorem{construction}[theorem]{Construction}
\newtheorem{deduction}[theorem]{Deduction}
\newtheorem{exercise}[theorem]{Exercise}
\newtheorem{conjecture}[theorem]{Conjecture}

\theoremstyle{remark}
\newtheorem*{remark}{Remark}
\newtheorem*{recall}{Recall}
\newtheorem*{question}{Question}
\newtheorem*{answer}{Answer}

\newcounter{saveenumi}
\newcommand{\seti}{\setcounter{saveenumi}{\value{enumi}}}
\newcommand{\conti}{\setcounter{enumi}{\value{saveenumi}}}
\DeclareMathOperator{\Span}{Span}
\resetcounteronoverlays{saveenumi}

\AtBeginSection[]{
    \begin{frame}
    \centering
    \begin{beamercolorbox}[center, shadow=true, rounded=true]{title}
    \usebeamerfont{title}\insertsectionhead\par%
    \end{beamercolorbox}
    \end{frame}
}

\title{MATH240 - Introduction to Linear Algebra}
\author{Haoran Li}
\institute[UMD]{University of Maryland, College Park}
\date{Summer 2024}


\begin{document}

\maketitle

\section{Lecture 11 - Null spaces, Column spaces, Row spaces, Rank and Nullity}

\begin{frame}[t]
\begin{definition}
Suppose $V$ is a vector space, $\{\bm v_1,\bm v_2,\cdots,\bm v_n\}\subseteq V$ is a set of linearly independent vectors that span $V$, we define the \textcolor{blue}{dimension}\index{dimension} of $V$ to be $\dim V=n$.
\end{definition}
\end{frame}

\begin{frame}[t]
\begin{definition}
Suppose $A$ is a $m\times n$ matrix, we define the \textcolor{blue}{null space}\index{null space} of $A$ to be $\Nul A=\{\mathbf x\in\mathbb R^n|A\mathbf x=\mathbf0\}$. Note that the solution set to linear system $A\mathbf x=\mathbf0$ is the intersection of $m$ hyperplanes (one for each homogeneous equation) that pass through the origin.
\end{definition}
\pause
\begin{example}
$A=\begin{bmatrix}
1&-3&2\\
-5&12&-1
\end{bmatrix}$, the to find the $\Nul A$ is equivalent to solve $A\mathbf x=\mathbf 0$\\
\scalebox{0.7}{
$\begin{bmatrix}
A&\mathbf 0
\end{bmatrix}\xsim{R2\rightarrow R2+5R1}\begin{bmatrix}
1&-3&2&0\\
0&-3&9&0
\end{bmatrix}\xsim{R2/(-3)}\begin{bmatrix}
1&-3&2&0\\
0&1&-3&0
\end{bmatrix}\xsim{R1\rightarrow R1+3R2}\begin{bmatrix}
1&0&-7&0\\
0&1&-3&0
\end{bmatrix}$
}
Hence the solution set is $\begin{cases}
x_1=7x_3\\
x_2=3x_3\\
x_3\text{ is free}
\end{cases}$, in parametric form, $\begin{bmatrix}
x_1\\x_2\\x_3
\end{bmatrix}=x_3\begin{bmatrix}
7\\3\\1
\end{bmatrix}$, which describes a line in $\mathbb R^3$ that passes through the origin, and this line is the intersection of planes $x_1-3x_2+2x_3=0$ and $-5x_1+12x_2-x_3=0$
\end{example}
\end{frame}

\begin{frame}[t]
\begin{center}
\begin{tikzpicture}
\definecolor{color1}{rgb}{255,0,0}
\definecolor{color2}{rgb}{0,0,255}
\definecolor{color3}{rgb}{255,0,255}
\def\XMAX{2.5};\def\YMAX{2.5};\def\ZMAX{4.5};
\begin{scope}[blend mode=multiply]
\draw [color1, fill=color1!20] plot (-2,-2,-2)--(-2,2,4)--(2,2,2)--(2,-2,-4)--cycle;
\node at (2,-2,-4) [below right]{$x_1-3x_2+2x_3=0$};
\draw [color2, fill=color2!20] plot (-2,-7/6,-4)--(-2,-1/2,4)--(2,7/6,4)--(2,1/2,-4)--cycle;
\node at (2,1/2,-4) [above]{$-5x_1+12x_2-x_3=0$};
\end{scope}
\draw [purple, ultra thick](-2,-6/7,-2/7)--(2,6/7,2/7);
\draw[->] (-\XMAX,0,0)--(\XMAX,0,0) node[right]{$x_1$};
\draw[->] (0,-\YMAX,0)--(0,\YMAX,0) node[above]{$x_2$};
\draw[->] (0,0,-\ZMAX)--(0,0,\ZMAX) node[below left]{$x_3$};
\end{tikzpicture}
\end{center}

\begin{remark}
As discussed in Example~\ref{15:10-06/23/2022}, in general, the solution set of $A\mathbf x=\mathbf b$ is not a subspace of $\mathbb R^n$ unless $\mathbf b=\mathbf0$. And in fact, any subspace of $\mathbb R^n$ is the null space for some $m\times n$ matrix $A$, i.e. the intersection of hyperplanes passing through the origin
\end{remark}
\end{frame}

\begin{frame}[t]
\begin{definition}
Suppose $A=\begin{bmatrix}
\mathbf a_1&\mathbf a_2&\cdots&\mathbf a_n
\end{bmatrix}$ is an $m\times n$ matrix, then the \textcolor{blue}{column space}\index{column space} (denote as $\Col A$) is the subspace $\Span\{\mathbf a_1,\mathbf a_2,\cdots,\mathbf a_n\}$ in $\mathbb R^m$. Suppose $A=\begin{bmatrix}
R1\\R2\\\vdots\\Rm
\end{bmatrix}$, then the \textcolor{blue}{row space}\index{row space} (denote as $\Row A$) is the subspace spaned by row vectors $\Span\{R1,R2,\cdots, Rm\}$ in $\mathbb R^n$ written horizontally.
\end{definition}
\pause
\begin{theorem}
Row reductions preserve $\Nul A, \Row A$, but not $\Col A$. However, row reductions preserve linear dependences of the column vectors.
\end{theorem}
\pause
\begin{remark}
Suppose $A\mathbf x=\mathbf0$ is some linear dependence of the column vectors of $A$, after elementary row reduction $A\sim EA$, $(EA)\mathbf x=E(A\mathbf x)=E\mathbf0=\mathbf0$ is again the same linear dependence of columns of $EA$. In other words, the linear dependence of columns of a matrix is preserved by row equivalence.
\end{remark}
\end{frame}

\begin{frame}[t]
\begin{theorem}
Suppose $A$ is a $m\times n$ matrix, $A\sim U$ is of RREF form
\begin{itemize}
\item The direction vectors in the solution set of $\begin{bmatrix}
U&\mathbf0
\end{bmatrix}$ in parametric vector form gives a basis for $\Nul A$. Note that $\dim\Nul A=$ the number of free variables.
\item A basis for $\Col A$ could be the set of pivot columns in $A$. Note that $\dim\Col A=$ the number of pivots.
\item A basis for $\Row A$ could be the set of non-zero row vectors in $U$ (Or any REF of $A$). Note that $\dim\Row A=$ the number of pivots.
\end{itemize}
\end{theorem}
\end{frame}

\begin{frame}[t]
\begin{definition}
$\dim\Nul A$ is also name the \textcolor{blue}{nullity}\index{nullity} of $A$. The number of pivots of $A$ (which is equal to both $\dim\Col A$ and $\dim\Row A$) is called the \textcolor{blue}{rank}\index{rank} of $A$
\end{definition}
\pause
\begin{theorem}[Rank-Nullity theorem]\label{16:38-06/24/2022}
Notice that the number of columns in $A$ (say a $m\times n$ matrix) is equal to the number of free variables and the number of pivot columns, thus we have
\begin{center}
$n=$ nullity + rank
\end{center}
\end{theorem}
\end{frame}

\begin{frame}[t]
\begin{example}
$A=\begin{bmatrix}
1&-1&1\\
2&0&-1\\
1&1&-2
\end{bmatrix}\sim\begin{bmatrix}
1&-1&1\\
0&2&-3\\
0&0&0
\end{bmatrix}\sim\begin{bmatrix}
1&0&-\frac{1}{2}\\
0&1&-\frac{3}{2}\\
0&0&0
\end{bmatrix}$, which is an REF and an RREF respectively\pause. There is only one free variable $x_3$, so the nullity is 1\pause, and the 1st, 2nd columns are pivot columns, so the rank is 2\pause. We see that Theorem~\ref{16:38-06/24/2022} holds as $3=1+2$\pause, and
\[
\Nul A=\Span\left\{\begin{bmatrix}
\frac{1}{2}\\\frac{3}{2}\\1
\end{bmatrix}\right\}
\]\pause
\[
\Col A=\Span\left\{\begin{bmatrix}
1\\2\\1
\end{bmatrix},\begin{bmatrix}
-1\\0\\1
\end{bmatrix}\right\}
\]\pause
$\Row A=\Span\left\{\begin{bmatrix}
1&0&-\frac{1}{2}
\end{bmatrix},\begin{bmatrix}
0&1&-\frac{3}{2}
\end{bmatrix}\right\}$\pause or $\Span\left\{\begin{bmatrix}
1&-1&1
\end{bmatrix},\begin{bmatrix}
1&2&-3
\end{bmatrix}\right\}$
\end{example}
\end{frame}

\begin{frame}[t]
\begin{exercise}
$A=\begin{bmatrix}
1&2&0&4&5\\
0&0&5&-7&8\\
0&0&0&0&-9\\
0&0&0&0&0
\end{bmatrix}\sim\begin{bmatrix}
1&2&0&4&0\\
0&0&1&-\frac{7}{5}&0\\
0&0&0&0&1\\
0&0&0&0&0
\end{bmatrix}$\pause
Note that here we have 2 free variables $x_2,x_4$, so the nullity is 2\pause, and the 1st, 3rd, 5th columns are pivot columns, so the rank is 3\pause. We see that Theorem~\ref{16:38-06/24/2022} holds as $5=2+3$\pause, and
\[
\Nul A=\Span\left\{\begin{bmatrix}
-2\\1\\0\\0\\0
\end{bmatrix},\begin{bmatrix}
-4\\0\\\frac{7}{5}\\1\\
\end{bmatrix}\right\},\qquad \Col A=\Span\left\{\begin{bmatrix}
1\\0\\0\\0
\end{bmatrix},\begin{bmatrix}
0\\5\\0\\0
\end{bmatrix},\begin{bmatrix}
5\\8\\-9\\0
\end{bmatrix}\right\}
\]\pause
\[
\Row A=\Span\left\{\begin{bmatrix}
1&2&0&4&5
\end{bmatrix},\begin{bmatrix}
0&0&5&-7&8
\end{bmatrix},\begin{bmatrix}
0&0&0&0&-9
\end{bmatrix}\right\}
\]
\end{exercise}
\end{frame}

\begin{frame}[t]
\begin{question}
If you have a set $\mathcal S$ of vectors in $\mathbb R^m$, how do you find a subset of $\mathcal S$ that is a basis for $\Span\{\mathcal S\}$ (i.e. remove linear dependences)?
\end{question}
\pause
\begin{answer}
Collect these vectors as the column vectors of a matrix, and then find its columns space.
\end{answer}
\end{frame}

\begin{frame}[t]{Exercise}
Suppose $\mathbf v_1=\begin{bmatrix}
1\\2\\1
\end{bmatrix}$, $\mathbf v_2=\begin{bmatrix}
-1\\0\\1
\end{bmatrix}$, $\mathbf v_3=\begin{bmatrix}
1\\-1\\-2
\end{bmatrix}$. Find a basis for $\Span\{\mathbf v_1, \mathbf v_2, \mathbf v_3\}$.
\end{frame}

\begin{frame}[t]{Exercise}
Suppose $A=\begin{bmatrix}
1&-7&0&6&5\\
0&0&1&-2&-3\\
-1&7&-4&2&7
\end{bmatrix}$. Find $\Nul A,\Row A,\Col A$, and the nullity, rank of $A$.
\end{frame}

% \begin{frame}[t]
% \end{frame}

% \begin{frame}[t]
% \end{frame}

\end{document}