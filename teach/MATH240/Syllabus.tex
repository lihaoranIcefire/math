\documentclass[a4paper,10pt]{article}
\usepackage{hyperref}

\begin{document}

\begin{center}
\large
\textbf{MATH240 - Introduction to Linear Algebra - Syllabus} \\
\vspace{0.5cm}
\normalsize
University of Maryland, College park - Summer 2024\\
\end{center}

\noindent
\textbf{NOTE:} This syllabus is specific to Section WB11 of MATH240. Each instructor sets his/her own policies. Please consult your instructor’s syllabus if you are in a different section.

\section{Contact Information}
\textbf{Instructor:} Haoran(Harrison) Li\\
\textbf{Email:} haoranli@umd.edu\\
\textbf{Office Hours:} Wednesday 12:30-2:30pm with zoom link 7636616227

\section{ELMS/Canvas}
ELMS/Canvas will be used extensively throughout the course. This syllabus, along with assignments and grades will be posted in ELMS. You will submit some of your assignments, through ELMS (via Gradescope). Our ELMS course page will essentially function as the course website. To access ELMS, go to \url{http://myelms.umd.edu}, and log in using your UMD username and password.

\section{Lecture Notes}
We shall follow the Lecture Notes that is under Files in ELMS. The reference is \textit{Linear Algebra and its Applications} by Lay, and McDonald, 6th edition. A day-to-day schedule is posted in ELMS.

\section{Class Format}
The class meets on M/Tu/W/Th/F 11:00am - 12:22pm. You are \textbf{expected} to attend every class, though attendance will not be formally taken. There will be review and practice sessions during lectures.

\section{Lecture Videos}
The lecture videos can be found under ``Zoom/Cloud Recordings" in ELMS. Please consider watching them if you miss a class or if you want to review a topic.

% \section{Homework}
% Homework is essential to the learning process in any math class. Homework problems will be assigned for each section of the textbook that we cover, but they will not be collected or graded. Mathematics is learned by doing, not by watching. You need to do the homework problems if you want to succeed. The homework problem list will be posted in ELMS.

\section{Online Assignments}
There will be 10 assignments within Gradescope that \textbf{will be graded}. They will be posted days before they are due. Some problems will require you to submit your handwritten work electronically. There will be late penalties and you are welcome to reach out to me if you are having technical issues with the online submissions.

\section{MATLAB Projects}
There will be two MATLAB projects. They will be due at \textbf{11:59 pm} on the following dates.
\begin{itemize}
\item \textbf{MATLAB Project 1 – due Sunday, June 23rd}.
\item \textbf{MATLAB Project 2 – due Sunday, July 14th}.
\end{itemize}
% At the time each project is posted, we likely will not yet have discussed all the topics covered in that project. It may be wise to keep looking at the questions and do them as you’re able. \\
% The assignment must be completed within a MATLAB script. \textbf{You will then publish your script within MATLAB to produce a clean readable PDF file that shows all of your MATLAB commands as well as the output produced by your commands.} The published script must be submitted via Gradescope.

\section{Exams}
There will be two one-hour midterm exams and a final exam, given during lectures on the following dates:
\begin{itemize}
\item \textbf{Exam 1 – Friday, June 14th, 11:00 - 12:00pm}
\item \textbf{Exam 2 – Tuesday, July 2nd, 11:00 - 12:00pm}
\item \textbf{Final Exam - Friday, July 19th, 11:00 – 1:00 pm.}
\end{itemize}
In the event of an excused absence (see the link in “University Policies” below for a description of what constitutes an excused absence), the student must notify the instructor in advance (or as soon as possible in the case of illness or emergency). Additionally, \textbf{the student must provide proper documentation justifying the absence in advance.}

\section{Calculators}
Calculators and all electronic devices may \textbf{NOT} be used during exams.

\section{Grades}
A final grade will be determined based on the following breakdown:
\begin{itemize}
\item \textbf{Online Assignments} – 22\% (2.2\% each)
\item \textbf{MATLAB Projects} – 8\% (4\% each)
\item \textbf{Midterm Exams} – 40\% (20\% each)
\item \textbf{Final Exam} – 30\%
\end{itemize}
Listed below are the lower cutoffs for final letter grades:
\begin{center}
\begin{tabular}{|c|c|}
\hline
Letter Grade & Percent Score\\
\hline
\texttt{A+} & at least 97\%\\
\hline
\texttt{A} & at least 93\%\\
\hline
\texttt{A-} & at least 90\%\\
\hline
\texttt{B+} & at least 87\%\\
\hline
\texttt{B} & at least 83\%\\
\hline
\texttt{B-} & at least 80\%\\
\hline
\texttt{C+} & at least 77\%\\
\hline
\texttt{C} & at least 73\%\\
\hline
\texttt{C-} & at least 70\%\\
\hline
\texttt{D} & at least 60\%\\
\hline
\texttt{F} & below 60\%\\
\hline
\end{tabular}
\end{center}
I reserve the right to \textbf{decrease} the lower cutoffs if I feel that it is appropriate based on the overall class performance. For example, this may happen if the average scores for the exams are much lower than I expected. This is \textbf{curving}, which is not something you should count on happening. If it does, it will happen at the end of the semester, after all grades have been computed. Individual exams or assignments will not be curved throughout the semester. I will \textbf{never} increase the cutoffs in the table, which means that curving cannot hurt your grade.

\section{University Policies}
General policies relevant to Undergraduate Courses are found here:
\begin{center}
\url{http://ugst.umd.edu/courserelatedpolicies.html}
\end{center}
Topics that are addressed in these various policies include academic integrity, student and instructor conduct, accessibility and accommodations, attendance and excused absences, grades and appeals, copyright and intellectual property.

\section{Disability Support}
The University provides accommodations for students with disabilities. To receive accommodations, students must first register with the campus’ Accessibility and Disability Service (ADS) and have their disabilities documented by ADS. The ADS office then prepares an Accommodation Letter for course instructors regarding needed accommodations. Students are responsible for presenting this letter to their instructors by the end of the drop/add period.

\end{document}